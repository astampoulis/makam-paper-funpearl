\section{In which our heroes reflect on structural
recursion}\label{in-which-our-heroes-reflect-on-structural-recursion}

\heroADVISOR{} Your pattern matching encoding looks good! You seem to be
getting the hang of this. How about we do something challenging then?
Say, type synonyms?

\heroSTUDENT{} Type synonyms? You mean, introducing type definitions like
\texttt{type\ natpair\ =\ nat\ *\ nat}? That does not seem particularly
tricky.

\heroADVISOR{} I think we will face a couple of interesting issues with it,
the main one being how to do \emph{structural recursion} in a nice way.
But first, let me write out the necessary pen-and-paper rules, so that
we are on the same page. We'll do top-level type definitions, so let's
add a top-level notion of programs \(c\) and a well-formedness judgment
`\(\vdash c \; \text{wf}\)' for them. (We could do modules instead of
just programs, but I feel that would derail us a little.) We also need
an additional environment \(\Delta\) to store type definitions:

\vspace{-1.2em}\begin{mathpar}
\inferrule[WfProgram-Main]{\emptyset; \; \Delta \vdash e : \tau}
          {\Delta \vdash (\text{main} \; e) \; \text{wf}}

\inferrule[WfProgram-Type]{\Delta, \; \alpha = \tau \vdash c \; \text{wf}}
          {\Delta \vdash (\texttt{type} \; \alpha = \tau \; ; \; c) \; \text{wf}}

\inferrule[Typeof-Conv]
          {\Gamma; \Delta \vdash e : \tau \\\\ \Delta \vdash \tau =_\delta \tau'}
          {\Gamma; \Delta \vdash e : \tau'}

\inferrule[TypEq-Def]
          {\alpha = \tau \in \Delta}
          {\Delta \vdash \alpha =_\delta \tau}
\cdots
\end{mathpar}

\heroSTUDENT{} Right, we will need the conversion rule, so that we identify
types up to expanding their definitions; that's
\(\delta\)-equality\ldots{} And I see you haven't listed out all the
rules for that, but those are mostly standard.

\heroADVISOR{} Still, there are quite a few of those rules. Want to give
transcribing this to Makam a try?

\heroSTUDENT{} Yes, I got this. I'll add a new \texttt{typedef} predicate; I
will only use it for local assumptions, to correspond to the \(\Delta\)
context of type definitions. I will also do the well-formed program
rules:

\begin{verbatim}
typedef : (NewType: typ) (Definition: typ) \ensuremath{\to} prop.

program : type. 
main : term \ensuremath{\to} program. 
lettype : (Definition: typ) (A_Program: typ \ensuremath{\to} program) \ensuremath{\to} program.

wfprogram : program \ensuremath{\to} prop.
wfprogram (main E) \ensuremath{:\!-} typeof E T.
wfprogram (lettype T A_Program) \ensuremath{:\!-}
  (a:typ \ensuremath{\to} typedef a T \ensuremath{\to} wfprogram (A_Program a)).
\end{verbatim}

\noindent
Well, I can do the conversion rule and the type-equality judgment
too\ldots{}. I will name that \texttt{typeq}. I'll just write the one
rule for now, which should be sufficient for a small example:

\begin{verbatim}
typeq : (T1: typ) (T2: typ) \ensuremath{\to} prop.
typeof E T \ensuremath{:\!-} typeof E T', typeq T T'.
typeq A T \ensuremath{:\!-} typedef A T.

wfprogram (lettype (arrow onat onat) (fun a \ensuremath{\Rightarrow}
          (main (lam a (fun f \ensuremath{\Rightarrow} (app f ozero)))))) ?
>> (Complete silence)
\end{verbatim}

\heroADVISOR{} Time to \texttt{Ctrl-C} out of the infinite loop?

\heroSTUDENT{} Oh. Oh, right. I guess we hit a case where the proof-search
strategy of Makam fails to make progress?

\heroADVISOR{} Correct. The loop happens when the new \texttt{typeof} rule
gets triggered: it has \texttt{typeof\ E\ T\textquotesingle{}} as a
premise, but the same rule still applies to solve that goal, so the rule
will fire again, and so on. Makam just does depth-first search right
now; until my friend implements a more sophisticated search strategy, we
need to find another way to do this.

\heroSTUDENT{} I see. I guess we should switch to an algorithmic type system
then.

\heroADVISOR{} Yes. Fortunately we can do that with relatively painless edits
and additions. Consider this: we only need to use the conversion rule in
cases where we already know something about the type \texttt{T} of an
expression \texttt{E}, but the typing rules require that \texttt{E} has
a type \texttt{T\textquotesingle{}} of some other form. That was the
case above -- for \texttt{E\ =\ f}, we knew that \texttt{T\ =\ a}, but
the typing rule for \texttt{app} required that
\texttt{T\textquotesingle{}\ =\ arrow\ T1\ T2} for some \texttt{T1},
\texttt{T2}.

\heroSTUDENT{} Oh. In that case we could try \emph{not} propagating the
concrete type information we have? We could then use the conversion rule
to check that the type we end up with matches what we expect.

\heroADVISOR{} Exactly. So we need to change the rule you wrote to apply only
in the case where \texttt{T} starts with a concrete constructor, rather
than when it is an uninstantiated unification variable. We will then
check whether the resulting type \texttt{T\textquotesingle{}} is equal
to \texttt{T}, using our \texttt{typeq} predicate.

\heroSTUDENT{} Is that even possible? Is there a way in \foreignlanguage{greek}{λ}Prolog to tell
whether something is a unification variable?

\heroADVISOR{} There is! Most Prolog dialects have a predicate that does that
-- it's usually called \texttt{var}. In Makam it is called
\texttt{refl.isunif}, the \texttt{refl} namespace prefix standing for
\emph{reflective} predicates. So, here's a second attempt, where I'm
also using logical
negation\footnote{Makam follows \citet{kiselyov05backtracking} closely in terms of the semantics for logical if-then-else and logical negation.}:

\begin{verbatim}
typeof E T \ensuremath{:\!-} not(refl.isunif T), typeof E T', typeq T T'.
\end{verbatim}

\heroSTUDENT{} Interesting. But wouldn't this lead to an infinite loop, too?
After all, \texttt{typeq} is going to be reflexive when we add all rules
-- so we could end up in the same situation as before.

\heroADVISOR{} Correct: for every proof of
\texttt{typeof\ E\ T\textquotesingle{}} through the other rules, a new
proof using this rule will be discovered, which will lead to another
proof for it, etc. One fix is to make sure that this rule is only used
once at the end, if typing using the other rules fails. But for now,
let's do a trick to side-step this issue -- let's check that \texttt{T}
and \texttt{T\textquotesingle{}} are not identical:

\begin{verbatim}
typeof E T \ensuremath{:\!-} not(refl.isunif T), typeof E T', typeq T T', not(eq T T').
\end{verbatim}

By the way, \texttt{eq} is a standard-library predicate that simply
attempts unification of the two arguments:

\begin{verbatim}
eq : A \ensuremath{\to} A \ensuremath{\to} prop. eq X X.
\end{verbatim}

\heroSTUDENT{} If we ever made a paper submission out of this, some reviewers
would not be happy about this \texttt{typeof} rule. But sure. Oh, and we
should add the conversion rule for \texttt{typeof\_patt}, but that's
almost the same as for terms. (\ldots{}) I'll do \texttt{typeq} next.

\begin{verbatim}
typeq A T' \ensuremath{:\!-} typedef A T, typeq T T'.
typeq T' A \ensuremath{:\!-} typedef A T, typeq T T'.
\end{verbatim}

\heroADVISOR{} I like how you added the symmetric rule, but\ldots{} this is
subtle, but if \texttt{A} is a unification variable, we don't want to
unify it with an arbitrary synonym. So we need to check that \texttt{A}
is concrete
somehow\footnote{Though not supported in Makam, an alternative to this in other languages based on higher-order logic programming would be to add a \texttt{mode (i o)} declaration for \texttt{typedef}, so that \texttt{typedef A T} would fail if \texttt{A} is not concrete.}:

\begin{verbatim}
typeq A T' \ensuremath{:\!-} not(refl.isunif A), typedef A T, typeq T T'.
typeq T' A \ensuremath{:\!-} not(refl.isunif A), typedef A T, typeq T T'.
\end{verbatim}

\heroSTUDENT{} I see what you mean. OK, I'll continue on to the rest of the
cases\ldots{}.

\begin{verbatim}
typeq (arrow T1 T2) (arrow T1' T2') \ensuremath{:\!-}
  typeq T1 T1', typeq T2 T2'.
typeq (arrowmany TS T) (arrowmany TS' T') \ensuremath{:\!-}
  map typeq TS TS', typeq T T'.
\end{verbatim}

\heroADVISOR{} Writing boilerplate is not fun, is it?

\heroSTUDENT{} It is not. I wish we could just write the first two rules that
you wrote; they're the important ones, after all. All the others just
propagate the structural recursion through. Also, whenever we add a new
constructor for types, we'll have to remember to add a \texttt{typeq}
rule for it\ldots{}.

\heroADVISOR{} Right. Let's just use some magic instead.

\begin{scenecomment}
(Roza changes the type definition of \texttt{typeq} to \texttt{typeq : [Any] (T1: Any) (T2: Any) \ensuremath{\to} prop},
and adds a few lines:)
\end{scenecomment}

\begin{verbatim}
typeq A T' \ensuremath{:\!-} not(refl.isunif A), typedef A T, typeq T T'.
typeq T' A \ensuremath{:\!-} not(refl.isunif A), typedef A T, typeq T T'.
typeq T T' \ensuremath{:\!-} structural_recursion @typeq T T'.
\end{verbatim}

\heroSTUDENT{} \ldots{} What just happened. Is \texttt{structural\_recursion}
some special Makam trick I don't know about yet?

\heroADVISOR{} Indeed. There is a little bit of trickery involved here, but
you will see that there is much less of it than you would expect, upon
close reflection. \texttt{structural\_recursion} is just a normal
standard-library predicate like any other; it essentially applies a
polymorphic predicate ``structurally'' to a term. Its implementation
will be a little special of course. But let's just think about how you
would write the rest of the rules of \texttt{typeq} generically, to
perform structural recursion.

\heroSTUDENT{} OK. Well, when looking at two \texttt{typ}s together, we have
to make sure that their constructors are the same and also that any
\texttt{typ}s they contain as arguments are recursively
\texttt{typeq}ual. So something like this:

\begin{verbatim}
typeq (Constructor Arguments) (Constructor Arguments') \ensuremath{:\!-}
  map typeq Arguments Arguments'.
\end{verbatim}

\heroADVISOR{} Right. Note, though, that the types of arguments might be
different than \texttt{typ}. So even if we start comparing two types at
the top level, we might end up having to compare two lists of types that
they contain -- imagine the case for \texttt{arrowmany} for example.

\heroSTUDENT{} I see! That's why you edited \texttt{typeq} to be polymorphic
above; you have extended it to work on \emph{any type} (of the
metalanguage) that might contain a \texttt{typ}.

\heroADVISOR{} Exactly. Now, the list of \texttt{Arguments} -- can you come up
with a type for them?

\heroSTUDENT{} We can use the GADT of heterogeneous lists for them; not all
the arguments of each constructor need to be of the same type!

\begin{verbatim}
typenil : type. typecons : (T: type) (TS: type) \ensuremath{\to} type.
hlist : (TypeList: type) \ensuremath{\to} type.
hnil : hlist typenil. hcons : T \ensuremath{\to} hlist TS \ensuremath{\to} hlist (typecons T TS).
\end{verbatim}

\heroADVISOR{} Great! We will need a heterogeneous \texttt{map} for these
lists too. We'll need a polymorphic predicate as an argument, since
we'll have to use it for \texttt{Arguments} of different types:

\begin{verbatim}
hmap : [TS] (P: forall A (A \ensuremath{\to} A \ensuremath{\to} prop)) (XS: hlist TS) (YS: hlist TS) \ensuremath{\to} prop.
hmap P hnil hnil.
hmap P (hcons X XS) (hcons Y YS) \ensuremath{:\!-} forall.call P X Y, hmap P XS YS.
\end{verbatim}

\noindent
As I mentioned before, the rank-2 polymorphism support in Makam is quite
limited, so you have to use \texttt{forall.call} explicitly to
instantiate the polymorphic \texttt{P} predicate accordingly and call
it.

\heroSTUDENT{} Let me try out an example of that:

\begin{verbatim}
hmap @eq (hcons 1 (hcons "foo" hnil)) YS ?
>> Yes:
>> YS := hcons 1 (hcons "foo" hnil).
\end{verbatim}

\noindent
Looks good enough. So, going back to our generic rule -- is there a way
to actually write it? Maybe there's a reflective predicate we can use,
similar to how we used \texttt{refl.isunif} before to tell if a term is
an uninstantiated unification variable?

\heroADVISOR{} Exactly -- there is \texttt{refl.headargs}. It relates a
concrete term to its decomposition into a constructor and a list of
arguments\footnote{Other versions of Prolog have predicates toward the same effect; for example, SWI-Prolog \citep{wielemaker2012swi} provides `\texttt{compound\_{}name\_{}arguments}', which is quite similar.}.
This does not need an extra-logical feature save for
\texttt{refl.isunif}, though: we could define \texttt{refl.headargs}
without any special support, if we maintained a discipline whenever we
add a new constructor, roughly like this:

\begin{verbatim}
refl.headargs : (Term: TermT) (Constr: ConstrT) (Args: hlist ArgsTS) \ensuremath{\to} prop.

arrowmany : (TS: list typ) (T: typ) \ensuremath{\to} typ.
refl.headargs Term Constructor Args \ensuremath{:\!-}
  not(refl.isunif Term), eq Term (arrowmany TS T),
  eq Constructor arrowmany, eq Args (hcons TS (hcons T hnil)).
\end{verbatim}

\heroSTUDENT{} I see. I think I can write the generic rule for \texttt{typeq}
now then!

\begin{verbatim}
typeq T T' \ensuremath{:\!-}
  refl.headargs T Constr Args,
  refl.headargs T' Constr Args',
  hmap @typeq Args Args'.
\end{verbatim}

\heroADVISOR{} That looks great! Simple, isn't it? You'll see that there are a
few more generic cases that are needed, though. Should we do that? We
can roll our own reusable \texttt{structural\_recursion} implementation
-- that way we will dispel all magic from its use that I showed you
earlier! I'll give you the type; you fill in the first case:

\begin{verbatim}
structural_recursion : [Any] 
  (RecursivePred: forall A (A \ensuremath{\to} A \ensuremath{\to} prop))
  (X: Any) (Y: Any) \ensuremath{\to} prop.
\end{verbatim}

\heroSTUDENT{} Let me see. Oh, so, the first argument is a predicate -- are we
doing this in open-recursion style? I see. Well, I can adapt the case I
just wrote above.

\begin{verbatim}
structural_recursion Rec X Y \ensuremath{:\!-}
  refl.headargs X Constructor Arguments,
  refl.headargs Y Constructor Arguments',
  hmap Rec Arguments Arguments'.
\end{verbatim}

\heroADVISOR{} Nice. Now, here you assume that \texttt{X} and \texttt{Y} are
both concrete terms. What happens when \texttt{X} is concrete and
\texttt{Y} isn't, or the other way around? Hint: you can use this
\texttt{happly} predicate, to apply a list of arguments to a
constructor, and thus reconstruct a term:

\begin{verbatim}
happly : [Constr Args Terms] Constr \ensuremath{\to} hlist Args \ensuremath{\to} Term \ensuremath{\to} prop.
happly Constr hnil Constr.
happly Constr (hcons A AS) Term \ensuremath{:\!-} happly (Constr A) AS Term.
\end{verbatim}

\heroSTUDENT{} How about this? This way, we will decocompose the concrete
\texttt{X}, perform the transformation on the \texttt{Arguments}, and
then reapply the \texttt{Constructor} to get the result for \texttt{Y}.

\begin{verbatim}
structural_recursion Rec X Y \ensuremath{:\!-}
  refl.headargs X Constructor Arguments,
  hmap Rec Arguments Arguments',
  happly Constructor Arguments' Y.
\end{verbatim}

\heroADVISOR{} That is exactly right. You need the symmetric case too but
that's entirely similar. Also, there is another type of concrete terms
in Makam: meta-level functions! It does not make sense to destructure
functions using \texttt{refl.headargs}, so it fails in that case, and we
have to treat them specially:

\begin{verbatim}
structural_recursion Rec (X : A \ensuremath{\to} B) (Y : A \ensuremath{\to} B) \ensuremath{:\!-}
  (x:A \ensuremath{\to} structural_recursion Rec x x \ensuremath{\to}
    structural_recursion Rec (X x) (Y x)).
\end{verbatim}

\heroSTUDENT{} Ah, I see! Here you \emph{are} actually relying on the
\texttt{typecase} aspect of ad-hoc polymorphism, right? To check if
\texttt{X} and \texttt{Y} are of the meta-level function type.

\heroADVISOR{} Exactly. And you know what, that's all there is to it! So,
we've minimized the boilerplate, and we won't need any adaptation when
we add a new constructor -- even if we make use of all sorts of new and
complicated types.

\heroSTUDENT{} That's right: we do not need to do anything special for the
binding forms we defined, like \texttt{bindmany}\ldots{}. quite a payoff
for a small amount of code! But, wait, isn't
\texttt{structural\_recursion} missing a case: what happens if both
\texttt{X} and \texttt{Y} are uninstantiated unification variables?

\heroADVISOR{} You are correct, it would fail in that case. But in my
experience, it's better to define how to handle unification variables as
needed, in each new structurally recursive predicate. In this case, we
should never get into that situation based on how we have defined
\texttt{typeq}.

\begin{scenecomment}
(Roza and Hagop try out a few examples and convince themselves that this works OK and no endless loops happen when things don't typecheck correctly.)
\end{scenecomment}
