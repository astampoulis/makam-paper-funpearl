\section{In which we get a premonition of things to
come}\label{in-which-we-get-a-premonition-of-things-to-come}

\identNormal

\emph{Chapter 3} serves as a tutorial to \lamprolog/Makam, showing the
basic usage of the language to encode the static and dynamic semantics
of the simply typed lambda calculus. \emph{Chapter 4} explores the
question of how to implement multiple-variable binding, culminating in a
reusable polymorphic datatype. \emph{Chapters 5 and 6} present a novel
account of how GADTs are directly supported in \lamprolog thanks to the
presence of ad-hoc polymorphism and showcase their use for accurate
encodings of mutually recursive definitions and pattern matching.
\emph{Chapter 7} describes a novel way to define operations by
structural recursion in \lamprolog/Makam while only giving the essential
cases, motivating them through the example of encoding a simple
conversion rule. The following chapters make use of the presented
features to implement polymorphism and algebraic datatypes
(\emph{Chapter 8}), heterogeneous staging constructs with contextual
typing (\emph{Chapter 9}) and Hindley-Milner type generalization
(\emph{Chapter 10}). We then summarize and compare to related work.

\identDialog
