Let us proceed to add type synonyms:

\begin{verbatim}
type_synonym : dbind typ T typ \ensuremath{\to} (typeconstructor T \ensuremath{\to} program) \ensuremath{\to} program.

type_synonym_info : typeconstructor T \ensuremath{\to} dbind typ T typ \ensuremath{\to} prop.

wfprogram (type_synonym Syn Program') :-
  (t:(typeconstructor T) \ensuremath{\to}
   type_synonym_info t Syn \ensuremath{\to}
   wfprogram (Program' t)).
\end{verbatim}

Simple enough. How to typecheck them though? We need something like the
conversion rule:

\begin{displaymath}
\inferrule{
  \Gamma \vdash e : \tau \\ \tau =_{\delta} \tau'
}{
  \Gamma \vdash e : \tau'
}
\end{displaymath}

Here \(=_{\delta}\) means equality up to expanding type synonyms.

We will need a type equality predicate:

\begin{verbatim}
teq : typ \ensuremath{\to} typ \ensuremath{\to} prop.
\end{verbatim}

A naive attempt at the conversion rule would be:

\begin{verbatim}
typeof E T :- typeof E T', teq T T'.
\end{verbatim}

However, it is easy to see that this rule leads to divergence: it does a
recursive call to itself.

We can do a bit better. We only need to use the conversion rule in cases
where we already know something about the type \texttt{T} of the
expression, but our typing rules do not match that type. In
bi-directional typing parlance, instead of analyzing the type \texttt{T}
of the expression \texttt{E}, we want to synthesize the type starting
from a new meta-variable \texttt{T\textquotesingle{}}, and then check
that the two types are equal using \texttt{teq}. So we need to change
our rule to only apply in the case where \texttt{T} starts with a
concrete type constructor, rather than when it is an uninstantiated
meta-variable.

It is typical for a logic programming language to have a predicate that
only succeeds when a specific term is uninstantiated (usually called
\texttt{var}). In Makam this is called \texttt{refl.isunif} -- the
\texttt{refl} namespace prefix standing for the fact that we call these
kinds of predicates ``reflective'', as they give us extra-logical
information about the form of a term (sometimes referred to as
``meta-predicates'' in Prolog parlance). Our second attempt thus looks
as follows:

\begin{verbatim}
typeof E T :- not(refl.isunif T), typeof E T', teq T T'.
\end{verbatim}

Upon further consideration, we see that this rule leads to an infinite
loop as well: since \texttt{teq} should be reflective, for every proof
of \texttt{typeof\ E\ T\textquotesingle{}} through the other rules, a
new proof using this rule will be discovered, which will lead to another
proof for it, etc. One way to fix it is to make sure that this rule is
only used once at the end, if typing using the other rules fails:

\begin{verbatim}
typeof, typeof_cases, typeof_conversion : term \ensuremath{\to} typ \ensuremath{\to} prop.
typeof E T :-
  if (typeof_cases E T)
  then success
  else (typeof_conversion E T).
typeof_cases (app E1 E2) T' :-
  typeof E1 (arrow T1 T2),
  typeof E2 T1.
...
typeof_conversion E T :-
  not(refl.isunif T), typeof_cases E T', teq T T'.
\end{verbatim}

However, this would require changing every typing rule we had. Instead,
we can do a trick, to force the rule to only fire once for each
expression \texttt{E}, remembering the fact that we have used the rule
already:

\begin{verbatim}
already_in : [A] A \ensuremath{\to} prop.
typeof E T :-
  not(refl.isunif T),
  not(already_in (typeof E)),
  (already_in (typeof E) \ensuremath{\to} typeof E T'),
  teq T T'.
\end{verbatim}

Also, we need to make sure that we also take the conversion rule into
account for patterns:

\begin{verbatim}
typeof (P : patt A B) S' S T :-
  not(refl.isunif T),
  not(already_in (typeof P)),
  (already_in (typeof P) \ensuremath{\to} typeof P S' S T'),
  teq T T'.
\end{verbatim}

Now let's go and define the actual \texttt{teq} predicate. A first
approach would be to just write out each case individually:

\begin{verbatim}
teq (arrow T1 T2) (arrow T1' T2') :- teq T1 T1', teq T2 T2'.
teq (product TS) (product TS') :- map teq TS TS'.
teq (arrowmany TS T) (arrowmany TS' T') :- teq T T', map teq TS TS'.
teq nat nat.
teq (forall T) (forall T') :- (x:typ \ensuremath{\to} teq x x \ensuremath{\to} teq (T x) (T' x)).
teq (tconstr TC Args) (tconstr TC Args') :- map teq Args Args'.
teq (tconstr TC Args) T' :-
  type_synonym_info TC Syn,
  applymany Syn Args T,
  teq T T'.
teq T' (tconstr TC Args) :-
  type_synonym_info TC Syn,
  applymany Syn Args T,
  teq T' T.
\end{verbatim}

Only the two last cases are important; the rest is boilerplate that
performs structural recursion through the type. Can we do better than
that?

Let us ruminate on a possible solution. We want to handle the case where
we have a constructor applied to a number of arguments generically, so
roughly something like:

\begin{verbatim}
teq (Constructor Arguments) (Constructor Arguments') :-
  map teq Arguments Arguments'.
\end{verbatim}

What we mean here, taking the \texttt{arrow} rule as an example, is that
\texttt{Constructor} would match with \texttt{arrow}, and
\texttt{Arguments} would get instantiated with the list of arguments of
the constructor. One thing to be careful about though is that the types
of arguments are not all the same. As a result, instead of a homogeneous
list, we need a heterogeneous list. This is simple to represent using
the existential type, \texttt{dyn}:

\begin{verbatim}
dyn : type.
dyn : A \ensuremath{\to} dyn.
\end{verbatim}

So the type of \texttt{Arguments} should be \texttt{list\ dyn} rather
than \texttt{list\ typ}. The type of \texttt{teq} will need to be
changed, so that we can apply it for any different type, rather than
just \texttt{typ}:

\begin{verbatim}
teq : [A] A \ensuremath{\to} A \ensuremath{\to} prop.
\end{verbatim}

Furthermore, since we have a heterogeneous list, we need a \texttt{map}
that uses polymorphic recursion: it needs take a polymorphic function as
an argument, so that it can be used at different types for different
elements of the list.

\begin{verbatim}
dyn.map : (forall A. [A] A \ensuremath{\to} A \ensuremath{\to} prop) \ensuremath{\to} list dyn \ensuremath{\to} list dyn \ensuremath{\to} prop.
\end{verbatim}

This is in contrast to a type like
\texttt{{[}A{]}\ (A\ \ensuremath{\to}\ A\ \ensuremath{\to}\ prop)\ \ensuremath{\to}\ list\ dyn\ \ensuremath{\to}\ list\ dyn\ \ensuremath{\to}\ prop},
which would instantiate the type \texttt{A} to the type of the first
element of the list, making further applications to different types
impossible.

Makam currently does not provide higher-rank types as the above type
suggests -- nor do any \lamprolog implementations that we are aware of.
Instead, it provides a way to side-step this issue, through a predicate
that replaces polymorphic type variables with fresh variables, allowing
it to be instantiated with new types. This is called \texttt{dyn.call},
and \texttt{dyn.map} can be defined in terms of that:

\begin{verbatim}
dyn.call : [B] (A \ensuremath{\to} A \ensuremath{\to} prop) \ensuremath{\to} B \ensuremath{\to} B \ensuremath{\to} prop.
dyn.map : (A \ensuremath{\to} A \ensuremath{\to} prop) \ensuremath{\to} list dyn \ensuremath{\to} list dyn \ensuremath{\to} prop.
dyn.map P [] [].
dyn.map P (HD :: TL) (HD' :: TL') :- dyn.call P HD HD', dyn.map P TL TL'.
\end{verbatim}

(\texttt{dyn.call} is itself defined in terms of a more fundamental
predicate \texttt{dyn.duphead} that creates a fresh version of a single
polymorphic constructor with fresh type variables.)

Based on these, the only thing missing is a way to actually check
whether a term is a ground term that can be decomposed into a
constructor and a list of arguments. Makam provides this in the form of
the \texttt{refl.headargs} predicate:

\begin{verbatim}
refl.headargs : B \ensuremath{\to} A \ensuremath{\to} list dyn \ensuremath{\to} prop.
\end{verbatim}

(Other Prolog implementations also provide predicates towards the same
effect; for example, SWI-Prolog provides
\texttt{compound\_name\_arguments} which is quite similar. Though such
predicates are not typical in other \lamprolog implementations, they
should not be viewed as a hack: we could always define these within the
language if we maintained a discipline, where we added a rule to
\texttt{refl.headargs} for every constructor that we introduce. For
example:

\begin{verbatim}
arrowmany : list typ \ensuremath{\to} typ \ensuremath{\to} typ.
refl.headargs (arrowmany TS T) [arrowmany, [dyn TS, dyn T]].
\end{verbatim}

Maybe taking extra care to check that we are not instantiating a
unification by using \texttt{refl.isunif}.)

We are now ready to proceed to defining the boilerplate generically. We
will do this as a reusable higher-order predicate for structural
recursion, that we will use to implement \texttt{teq}. We will define it
in open recursion style, providing the predicate to use on recursive
calls as an argument:

\begin{verbatim}
structural_recursion : [B] (A \ensuremath{\to} A \ensuremath{\to} prop) \ensuremath{\to} B \ensuremath{\to} B \ensuremath{\to} prop.

structural_recursion Rec X Y :-
  refl.headargs X Constructor Arguments,
  dyn.map Rec Arguments Arguments',
  refl.headargs Y Constructor Arguments'.
\end{verbatim}

We also need to handle built-in types, such as the meta-level
\texttt{int} and \texttt{string} types, in case they are used as an
argument with other constructors:

\begin{verbatim}
structural_recursion Rec (X : string) (X : string).
structural_recursion Rec (X : int) (X : int).
\end{verbatim}

And last, we need to handle the case of the meta-level function type as
well:

\begin{verbatim}
structural_recursion Rec (X : A \ensuremath{\to} B) (Y : A \ensuremath{\to} B) :-
  (x:A \ensuremath{\to} structural_recursion Rec x x \ensuremath{\to} structural_recursion Rec (X x) (Y x)).
\end{verbatim}

We are done! Now we can define \texttt{teq} using
\texttt{structural\_recursion}, through an auxiliary predicate called
\texttt{teq\_aux}. We only need to define the non-trivial cases for it,
using \texttt{structural\_recursion} for the rest, while tying the open
recursion knot at the same time:

\begin{verbatim}
teq_aux : [A] A \ensuremath{\to} A \ensuremath{\to} prop.

teq T T' :- teq_aux T T'.

teq_aux T T' :-
  structural_recursion teq_aux T T'.

teq_aux (tconstr TC Args) T' :-
  type_synonym_info TC Synonym,
  applymany Synonym Args T,
  teq_aux T T'.

teq_aux T' (tconstr TC Args) :-
  type_synonym_info TC Synonym,
  applymany Synonym Args T,
  teq_aux T' T.
\end{verbatim}

Other than minimizing the boilerplate, the great thing about using
\texttt{structural\_recursion} is that no adaptation needs to be done
when we add any new constructor to our \texttt{typ} datatype -- even if
that uses new types that we have not defined before. For example, we did
not have to take any special provision to handle types we defined
earlier such as \texttt{dbind} -- everything works out thanks to the
reflective predicates we are using. (Mention something about the
expression problem?)

The one form of terms that \texttt{structural\_recursion} does not
handle are uninstantiated unification variables. We find that leaving
that as something that we handle whenever we define a new predicate that
uses \texttt{structural\_recursion} works fine. In this case,
\texttt{teq} is only supposed to be used with ground terms, so it is
fine if we fail when we encounter a unification variable.

Let's try an example out:

\begin{verbatim}
ascribe : term \ensuremath{\to} typ \ensuremath{\to} term.
typeof (ascribe E T) T :- typeof E T.

wfprogram (
  (type_synonym (dbindnext (fun a \ensuremath{\Rightarrow} dbindbase (product [a, a])))
  (fun bintuple \ensuremath{\Rightarrow} 
  
  main (lam (tconstr bintuple [product [nat, nat]])
            (fun x \ensuremath{\Rightarrow} 
    case_or_else x
    (patt_tuple [patt_tuple [patt_wild, patt_wild], patt_tuple [patt_wild, patt_wild]])
    (dbindbase (tuple []))
    (tuple [])
  ))
))) ?
>> Yes.
\end{verbatim}

Let's make sure we don't diverge on type error:

\begin{verbatim}
ascribe : term \ensuremath{\to} typ \ensuremath{\to} term.
typeof (ascribe E T) T :- typeof E T.

wfprogram (
  (type_synonym (dbindnext (fun a \ensuremath{\Rightarrow} dbindbase (product [a, a])))
  (fun bintuple \ensuremath{\Rightarrow} 
  
  main (lam (tconstr bintuple [product [nat, nat]])
            (fun x \ensuremath{\Rightarrow} 
    case_or_else x
    (patt_tuple [patt_tuple [patt_wild], patt_tuple [patt_wild, patt_wild]])
    (dbindbase (tuple []))
    (tuple [])
  ))
))) ?
>> Impossible.
\end{verbatim}
