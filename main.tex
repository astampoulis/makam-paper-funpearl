\documentclass[format=acmlarge,review,anonymous]{acmart}\settopmatter{printfolios=true}

\usepackage[utf8]{inputenc}
\usepackage[greek,english]{babel}
\usepackage{booktabs}
\usepackage{subcaption}
\usepackage{alltt}
\usepackage{xspace}
\usepackage{mathpartir}
\usepackage{indentfirst}
\usepackage{hyperref}
\usepackage{color}
\usepackage{fancyvrb}

\bibliographystyle{shared/ACM-Reference-Format}
\citestyle{acmauthoryear}   %% For author/year citations

\makeatletter\if@ACM@journal\makeatother
%% Journal information (used by PACMPL format)
%% Supplied to authors by publisher for camera-ready submission
\acmJournal{PACMPL}
\acmVolume{1}
\acmNumber{1}
\acmArticle{1}
\acmYear{2017}
\acmMonth{1}
\acmDOI{10.1145/nnnnnnn.nnnnnnn}
\startPage{1}
\else\makeatother
%% Conference information (used by SIGPLAN proceedings format)
%% Supplied to authors by publisher for camera-ready submission
\acmConference[PL'17]{ACM SIGPLAN Conference on Programming Languages}{January 01--03, 2017}{New York, NY, USA}
\acmYear{2017}
\acmISBN{978-x-xxxx-xxxx-x/YY/MM}
\acmDOI{10.1145/nnnnnnn.nnnnnnn}
\startPage{1}
\fi


\begin{document}

\title{Prototyping a Functional Language using Higher-Order Logic Programming}
\subtitle{A Functional Pearl on learning the ways of \lamprolog/Makam}

\author{Antonis Stampoulis}
\affiliation{
  \institution{Originate Inc.}
  \city{New York}
  \state{New York}
}
\email{antonis.stampoulis@gmail.com}

\author{Adam Chlipala}
\affiliation{
  \department{CSAIL}
  \institution{MIT}
  \city{Cambridge}
  \state{Massachusetts}
}
\email{adamc@csail.mit.edu}

%% -- Macro definitions
\newcommand\TODO[0]{\textbf{TODO}}
\newcommand\lamprolog[0]{\foreignlanguage{greek}{λ}Prolog\xspace}
\newcommand\fomega[0]{F$\omega$\xspace}
\renewenvironment{verbatim}{\begin{quote}\begin{alltt}}{\end{alltt}\end{quote}}
\newenvironment{codequote}{\begin{quote}\begin{alltt}}{\end{alltt}\end{quote}}
\newcommand\hide[1]{}
\newcommand\tightlist[0]{\itemsep1pt\parskip0pt\parsep0pt}
\renewcommand\thesection{\textbf{CHAPTER \arabic{section}}}
\renewcommand\thesubsection{\textbf{SECTION \arabic{section}.\arabic{subsection}}}
\newcommand\hero[1]{\textit{#1}.}
\newcommand\heroSTUDENT[0]{\hero{STUDENT}}
\newcommand\heroADVISOR[0]{\hero{ADVISOR}}
\newenvironment{scenecomment}{\em\noindent}{}

% =====
% used for highlighting from pandoc
\DefineVerbatimEnvironment{Highlighting}{Verbatim}{commandchars=\\\{\}}
\newenvironment{Shaded}{\begin{quote}}{\end{quote}}
\newcommand{\KeywordTok}[1]{\textcolor[rgb]{0.00,0.44,0.13}{\textbf{{#1}}}}
\newcommand{\DataTypeTok}[1]{\textcolor[rgb]{0.56,0.13,0.00}{{#1}}}
\newcommand{\DecValTok}[1]{\textcolor[rgb]{0.25,0.63,0.44}{{#1}}}
\newcommand{\BaseNTok}[1]{\textcolor[rgb]{0.25,0.63,0.44}{{#1}}}
\newcommand{\FloatTok}[1]{\textcolor[rgb]{0.25,0.63,0.44}{{#1}}}
\newcommand{\CharTok}[1]{\textcolor[rgb]{0.25,0.44,0.63}{{#1}}}
\newcommand{\StringTok}[1]{\textcolor[rgb]{0.25,0.44,0.63}{{#1}}}
\newcommand{\CommentTok}[1]{\textcolor[rgb]{0.38,0.63,0.69}{\textit{{#1}}}}
\newcommand{\OtherTok}[1]{\textcolor[rgb]{0.00,0.44,0.13}{{#1}}}
\newcommand{\AlertTok}[1]{\textcolor[rgb]{1.00,0.00,0.00}{\textbf{{#1}}}}
\newcommand{\FunctionTok}[1]{\textcolor[rgb]{0.02,0.16,0.49}{{#1}}}
\newcommand{\RegionMarkerTok}[1]{{#1}}
\newcommand{\ErrorTok}[1]{\textcolor[rgb]{1.00,0.00,0.00}{\textbf{{#1}}}}
\newcommand{\NormalTok}[1]{{#1}}
% =====

\begin{abstract}
\TODO{} Need to write an abstract.
\end{abstract}

%% 2012 ACM Computing Classification System (CSS) concepts
%% Generate at 'http://dl.acm.org/ccs/ccs.cfm'.
%% \begin{CCSXML}
%% <ccs2012>
%% <concept>
%% <concept_id>10011007.10011006.10011008</concept_id>
%% <concept_desc>Software and its engineering~General programming languages</concept_desc>
%% <concept_significance>500</concept_significance>
%% </concept>
%% <concept>
%% <concept_id>10003456.10003457.10003521.10003525</concept_id>
%% <concept_desc>Social and professional topics~History of programming languages</concept_desc>
%% <concept_significance>300</concept_significance>
%% </concept>
%% </ccs2012>
%% \end{CCSXML}

%% \ccsdesc[500]{Software and its engineering~General programming languages}
%% \ccsdesc[300]{Social and professional topics~History of programming languages}
%% End of generated code

\maketitle

{
  % Hanging Indent
  \setlength{\leftskip}{1em}
  \setlength{\parindent}{-1em}
  \setlength{\parskip}{3pt}

  % No Indent
  % \setlength{\parindent}{0em}
  
  \section{Where our heroes set on a road to prototype a type
system}\label{where-our-heroes-set-on-a-road-to-prototype-a-type-system}

\heroSTUDENT{} (\ldots{}) So yes, I think my next step should be writing a toy
implementation of the type system we have in mind, so that we can try
out some examples and see what works and what does not.

\heroADVISOR{} Definitely -- trying out examples will help us refine our
ideas, too.

\heroSTUDENT{} Let's see, though; we have an ML core, dependently typed
constructs, and contextual types like in
\citet{nanevski2008contextual}\ldots{} I guess I will need a few months?

\heroADVISOR{} That sounds like a lot. Why don't you use some kind of
metalanguage to implement the prototype?

\heroSTUDENT{} You mean a tool like PLT Redex \citep{felleisen2009semantics},
the K Framework \citep{rosu2010overview,ellison2009rewriting}, Spoofax
\citep{kats2010spoofax}, Needle \& Knot \citep{keuchel2016needle}, or
CRSX \citep{rose2011crsx}?

\heroADVISOR{} Sure, all fine choices. Though I do not think these frameworks
have been used to implement a type system as advanced as the one we have
in mind, or can handle all the binding constructs we will need\ldots{} I
was actually thinking we should use higher-order logic programming.

\heroSTUDENT{} Oh, so \foreignlanguage{greek}{λ}Prolog \citep{miller1988overview} or LF
\citep{pfenning1999system}.

\heroADVISOR{} Yes. Actually, a few years back I worked on this new
implementation of \foreignlanguage{greek}{λ}Prolog just for this purpose -- rapid prototyping of
languages. It's called Makam. It should be able to handle what we have
in mind, and we won't need more than a few hours.

\heroSTUDENT{} Sounds great! Anything I can read up on it?

\heroADVISOR{} Not much, unfortunately\ldots{} Let's just work on this
together, it'll be fun.

\begin{scenecomment}
(Our heroes install Makam from
\if@ACM@anonymous{\url{http://github.com/astampoulis/makam}}\else{--elided for blind
reviewing--}\fi\xspace
and figure out how to run the REPL.)
\end{scenecomment}

  
  \section{Where our heroes get the easy stuff out of the
way}\label{where-our-heroes-get-the-easy-stuff-out-of-the-way}

STUDENT. OK, let's just start with the simply typed lambda calculus to
see how this works. Let's define just the basics, application, lambda
abstraction and the arrow type.

ADVISOR. Right. We will first need to define the two meta-types for
these two sorts:

\begin{verbatim}
term : type.
typ : type.
\end{verbatim}

STUDENT. Oh, so \texttt{type} is the reserved keyword for the meta-level
kind of types, and we'll use \texttt{typ} for our object-level types?

ADVISOR. Exactly. And let's do the easy constructors first:

\begin{verbatim}
app : term \ensuremath{\to} term \ensuremath{\to} term.
arrow : typ \ensuremath{\to} typ \ensuremath{\to} typ.
\end{verbatim}

STUDENT. So we add constructors to a type at any point, we do not list
them out when we define it like in Haskell. But how about lambdas? I
have heard that \lamprolog supports higher-order abstract syntax, which
should make those really easy to add too, right?

ADVISOR. Yes, functions at the meta-level are parametric, so they
correspond exactly to single variable binding -- they cannot perform any
computation, and thus we do not have to worry about exotic terms. So
this works fine for Church-style lambdas:

\begin{verbatim}
lam : typ \ensuremath{\to} (term \ensuremath{\to} term) \ensuremath{\to} term.
\end{verbatim}

STUDENT. I see. And how about the typing judgement,
\(\Gamma \vdash e : \tau\) ?

ADVISOR. Let's add a predicate for that. Only, no \(\Gamma\), there is
an implicit context of assumptions:

\begin{verbatim}
typeof : term \ensuremath{\to} typ \ensuremath{\to} prop.
\end{verbatim}

STUDENT. Let me see if I can get the typing rule for application. I know
that in Prolog we write the conclusion of a rule first, and the premises
follow the \texttt{:-} sign. Does something like this work?

\begin{verbatim}
typeof (app E1 E2) T' :-
  typeof E1 (arrow T T'), typeof E2 T.
\end{verbatim}

ADVISOR. Yes! That's exactly right. Makam uses capital letters for
unification variables.

STUDENT. I will need help with the lambda typing rule though. What's the
equivalent of extending the context as in \(\Gamma, x : \tau\) ?

ADVISOR. Simple, we introduce a fresh constructor for terms, and a new
typing rule for it:

\begin{verbatim}
typeof (lam T1 E) (arrow T1 T2) :-
  (x:term \ensuremath{\to} typeof x T1 \ensuremath{\to} typeof (E x) T2).
\end{verbatim}

STUDENT. Hmm, so \texttt{x:term\ \ensuremath{\to}} introduces the fresh
constructor standing for the new variable, and
\texttt{typeof\ x\ T1\ \ensuremath{\to}} introduces the new assumption?
Oh, and we need to get to the body of the lambda function in order to
type-check it, that's why you do \texttt{E\ x}.

ADVISOR. Yes. Note that the introductions are locally scoped, so they
are only into effect for the recursive call to \texttt{typeof}.

STUDENT. Makes sense. So do we have a type checker already? Can we run
queries?

ADVISOR. We do! Observe:

\begin{verbatim}
typeof (lam _ (fun x \ensuremath{\Rightarrow} x)) T ?
>> Yes:
>> T := arrow T1 T1
\end{verbatim}

STUDENT. Cool! But wait, last time I implemented unification in my toy
STLC implementation it was easy to make it go into an infinite loop with
\(\lambda x. x x\). How does that work here?

ADVISOR. Well you were missing the occurs-check. \lamprolog unification
includes it:

\begin{verbatim}
typeof (lam _ (fun x \ensuremath{\Rightarrow} app x x)) T' ?
>> Impossible.
\end{verbatim}

STUDENT. Right. Cool, so what else can we do? How about adding tuples to
our language? Can we use something like a polymorphic list?

ADVISOR. Sure! \lamprolog has polymorphic types and higher-order
predicates:

\begin{verbatim}
list : type \ensuremath{\to} type.
nil : list A.
cons : A \ensuremath{\to} list A \ensuremath{\to} list A.

map : (A \ensuremath{\to} B \ensuremath{\to} prop) \ensuremath{\to} list A \ensuremath{\to} list B \ensuremath{\to} prop.
map P nil nil.
map P (cons X XS) (cons Y YS) :- P X Y, map P XS YS.
\end{verbatim}

STUDENT. Nice! So this should work:

\begin{verbatim}
tuple : list term \ensuremath{\to} term.
product : list typ \ensuremath{\to} typ.
typeof (tuple ES) (product TS) :-
  map typeof ES TS.
\end{verbatim}

ADVISOR. It does! And we can use syntactic sugar for \texttt{cons} and
\texttt{nil} too:

\begin{verbatim}
typeof (lam _ (fun x \ensuremath{\Rightarrow} lam _ (fun y \ensuremath{\Rightarrow} tuple [x, y]))) T ?
>> Yes:
>> T := arrow T1 (arrow T2 (product [T1, T2]))
\end{verbatim}

STUDENT. So how about evaluation? Can we write the big-step semantics
too?

ADVISOR. Why not? Let's add a predicate and do the two easy rules:

\begin{verbatim}
eval : term \ensuremath{\to} term \ensuremath{\to} prop.
eval (lam T F) (lam T F).
eval (tuple ES) (tuple VS) :- map eval ES VS.
\end{verbatim}

STUDENT. OK, let me try my hand at the beta-redex case. I'll just do
call-by-value. And I think in \(\lamprolog\) function application is
exactly capture-avoiding substitution, so this should be fine:

\begin{verbatim}
eval (app E E') V'' :-
  eval E (lam _ F), eval E' V', eval (F V') V''.
\end{verbatim}

ADVISOR. Exactly! See, I told you this would be easy!

  
  \section{Where our heroes add parentheses and discover how to do
multiple
binding}\label{where-our-heroes-add-parentheses-and-discover-how-to-do-multiple-binding}

\heroSTUDENT{} Still, I feel like we've been playing to the strengths of
\foreignlanguage{greek}{λ}Prolog\ldots{}. Yes, single-variable binding, substitutions, and so on
work nicely, but how about any other form of binding? Say, binding
multiple variables at the same time? We are definitely going to need
that for the language we have in mind. I was under the impression that
HOAS encodings do not work for that -- for example, I was reading
\citet{keuchel2016needle} recently and I remember reading something to
that end.

\heroADVISOR{} That's not really true; having first-class support for
single-variable binders should be enough. But let's try it out, maybe
adding multiple-argument functions for example -- I mean uncurried ones.
Want to give it a try?

\heroSTUDENT{} Let me see. We want the terms to look roughly like this:

\begin{verbatim}
lammany (fun x \ensuremath{\Rightarrow} fun y \ensuremath{\Rightarrow} tuple [y, x])
\end{verbatim}

For the type of \texttt{lammany}, I want to write something like this,
but I know this is wrong.

\begin{verbatim}
lammany : (list term \ensuremath{\to} term) \ensuremath{\to} term.
\end{verbatim}

\heroADVISOR{} Yes, that doesn't quite work. It would introduce a fresh
variable for \texttt{list}s, not a number of fresh variables for
\texttt{term}s. HOAS functions are parametric, too, which means we
cannot even get to the potential elements of the fresh \texttt{list}
inside the \texttt{term}.

\heroSTUDENT{} Right. So I don't know, instead we want to use a type that
stands for \texttt{term\ \ensuremath{\to}\ term},
\texttt{term\ \ensuremath{\to}\ term\ \ensuremath{\to}\ term}, and so on.
Can we write \texttt{term\ \ensuremath{\to}\ ...\ \ensuremath{\to}\ term}?

\heroADVISOR{} Well, not quite, but we have already defined something similar,
a type that roughly stands for \texttt{term\ *\ ...\ *\ term}, and we
did not need anything special for that\ldots{}

\heroSTUDENT{} You mean the \texttt{list} type?

\heroADVISOR{} Exactly. What do you think about this definition?

\begin{verbatim}
bindmanyterms : type.
bindnil : term \ensuremath{\to} bindmanyterms.
bindcons : (term \ensuremath{\to} bindmanyterms) \ensuremath{\to} bindmanyterms.
\end{verbatim}

\heroSTUDENT{} Hmm. That looks quite similar to lists; the parentheses in
\texttt{cons} are different. \texttt{nil} gets an extra \texttt{term}
argument, too\ldots{}

\heroADVISOR{} Yes\ldots{} So what is happening here is that \texttt{bindcons}
takes a single argument, introducing a binder; and \texttt{bindnil} is
when we get to the body and don't need any more binders. Maybe we should
name them accordingly.

\heroSTUDENT{} Right, and could we generalize their types? Maybe that will
help me get a better grasp of it. How is this?

\begin{verbatim}
bindmany : type \ensuremath{\to} type \ensuremath{\to} type.
body : Body \ensuremath{\to} bindmany Variable Body.
bind : (Variable \ensuremath{\to} bindmany Variable Body) \ensuremath{\to} bindmany Variable Body.
\end{verbatim}

\heroADVISOR{} This looks great! That is exactly what's in the Makam standard
library actually. And we can now define \texttt{lammany} using it -- and
our example term from before.

\begin{verbatim}
lammany : bindmany term term \ensuremath{\to} term.
lammany (bind (fun x \ensuremath{\Rightarrow} bind (fun y \ensuremath{\Rightarrow} body (tuple [y,x]))))
\end{verbatim}

\heroSTUDENT{} I see. That is an interesting datatype. Is there some reference
about it?

\heroADVISOR{} Not that I know of, at least where it is called out as a
reusable datatype -- though the construction is definitely part of PL
folklore. After I started using this in Makam, I noticed similar
constructions in the wild, for example in MTac \citep{ziliani2013mtac},
for parametric HOAS implementation of telescopes in Coq.

\heroSTUDENT{} Interesting. So how do we work with \texttt{bindmany}? What's
the typing rule like?

\heroADVISOR{} The rule is written like this, and I'll explain what goes into
it:

\begin{verbatim}
arrowmany : list typ \ensuremath{\to} typ \ensuremath{\to} typ.
typeof (lammany F) (arrowmany TS T) :-
  openmany F (fun xs body \ensuremath{\Rightarrow}
    assumemany typeof xs TS (typeof body T)).
\end{verbatim}

\heroSTUDENT{} Let me see if I can read this\ldots{} \texttt{openmany} somehow
gives you fresh variables \texttt{xs} for the binders, and the
\texttt{body} of the \texttt{lammany}; and then the
\texttt{assumemany\ typeof} part is what corresponds to extending the
\(\Gamma\) context with multiple typing assumptions?

\heroADVISOR{} Yes, and then we typecheck the \texttt{body} in that local
context that includes the fresh variables and the typing assumptions.
But let's do one step at a time. \texttt{openmany} is simple; we iterate
through the nested binders, introducing one fresh variable at a time. We
also substitute each bound variable for the current fresh variable, so
that when we get to the body, it only uses the fresh variables we
introduced.

\begin{verbatim}
openmany : bindmany A B \ensuremath{\to} (list A \ensuremath{\to} B \ensuremath{\to} prop) \ensuremath{\to} prop.
openmany (body Body) Q :- Q [] Body.
openmany (bind F) Q :-
  (x:A \ensuremath{\to} openmany (F x) (fun xs \ensuremath{\Rightarrow} Q (x :: xs))).
\end{verbatim}

\heroSTUDENT{} I see. I guess \texttt{assumemany} is similar, introducing one
assumption at a time?

\begin{verbatim}
assumemany : (A \ensuremath{\to} B \ensuremath{\to} prop) \ensuremath{\to} list A \ensuremath{\to} list B \ensuremath{\to} prop \ensuremath{\to} prop.
assumemany P [] [] Q :- Q.
assumemany P (X :: XS) (T :: TS) Q :- (P X T \ensuremath{\to} assumemany P XS TS Q).
\end{verbatim}

\heroADVISOR{} Yes, exactly! Just a note though -- \lamprolog typically does
not allow the definition of \texttt{assumemany}, where a non-concrete
predicate like \texttt{P\ X\ Y} is used as an assumption, because of
logical reasons. Makam is more lax, and so is ELPI, another recent
\lamprolog implementation, and allows this form statically, though there
are instantiations of \texttt{P} that will fail at run-time.

\heroSTUDENT{} I see. But we could just manually inline
\texttt{assumemany\ typeof} instead, so that's not a big problem, just
more verbose. But can I try our typing rule out?

\begin{verbatim}
typeof (lammany (bind (fun x \ensuremath{\Rightarrow} bind (fun y \ensuremath{\Rightarrow} body (tuple [y, x]))))) T ?
>> Yes:
>> T := arrowmany [T1, T2] (product [T2, T1]).
\end{verbatim}

Great, I think I got the hang of this. We could definitely add a
multiple-argument application construct \texttt{appmany} or define the
rules for \texttt{eval} for these. But that would be easy; we can do it
later. Something that worries me, though -- all these fancy higher-order
abstract binders, how do we \ldots{} make them concrete? Say, how do we
print them?

\heroADVISOR{} That's actually quite easy. We just add a concrete name to
them. A plain old \texttt{string}. Our typing rules etc. do not care
about it, but we could use it for parsing concrete syntax into our
abstract binding syntax, or for pretty-printing\ldots{}. Let's not get
into that for the time being, but let's just say that we could have
defined \texttt{bind} with an extra \texttt{string} argument,
representing the concrete name; and then \texttt{openmany} would just
ignore it.

\begin{verbatim}
bind : string \ensuremath{\to} (Var \ensuremath{\to} bindmany Var Body) \ensuremath{\to} bindmany Var Body.
\end{verbatim}

\heroSTUDENT{} Interesting. I would like to see more about this, but maybe
some other time. I thought of another thing that could be challenging:
mutually recursive \texttt{let\ rec}s?

\heroADVISOR{} Sure. Let's take this term for example:

\begin{verbatim}
let rec f = f_def and g = g_def in body
\end{verbatim}

If we write this in a way where the binding structure is explicit, we
would bind \texttt{f} and \texttt{g} simultaneously, and then write the
definitions and the body in that scope:

\begin{verbatim}
letrec (fun f \ensuremath{\Rightarrow} fun g \ensuremath{\Rightarrow} ([f_def, g_def], body))
\end{verbatim}

\heroSTUDENT{} I think I know how to do this then! How does this look?

\begin{verbatim}
letrec : bindmany term (list term * term) \ensuremath{\to} term.
\end{verbatim}

\heroADVISOR{} Exactly! Want to try writing the typing rules?

\heroSTUDENT{} Maybe something like this?

\begin{verbatim}
typeof (letrec XS_DefsBody) T' :-
  openmany XS_DefsBody (fun xs (Defs, Body) \ensuremath{\Rightarrow}
    assumemany typeof xs TS (map typeof Defs TS),
    assumemany typeof xs TS (typeof Body T')).
\end{verbatim}

\heroADVISOR{} Almost! The parser isn't clever enough to tell that the
predicate argument to \texttt{openmany} is, in fact, a predicate, so we
can't use the normal predicate syntax for it. We can use the syntactic
form \texttt{pfun} for writing anonymous predicates instead. Since this
will be a predicate, you are also able to destructure parameters like
you do here -- that doesn't work for normal functions in the general
case, since they need to treat arguments parametrically. And there is an
actual issue here: could you guess what it is? It has to do with which
free variables a unification variable is allowed to capture.

\heroSTUDENT{} Not really, but might have something to do with the fresh
variables that \texttt{openmany} introduces?

\heroADVISOR{} Yes. See, a unification variable is allowed to capture all the
free variables in scope at the point where it is introduced. By default,
all unification variables used in a rule get introduced when we check
whether the rule fires. But here we need to say explicitly that certain
unification variables need to be introduced when \texttt{openmany} gets
to use the \texttt{pfun} argument, and has therefore introduced all the
needed fresh variables. So we have to write the rule like this:

\begin{verbatim}
typeof (letrec XS_DefsBody) T' :-
  openmany XS_DefsBody (pfun [XS Defs Body] XS (Defs, Body) \ensuremath{\Rightarrow}
    assumemany typeof XS TS (map typeof Defs TS),
    assumemany typeof XS TS (typeof Body T')).
\end{verbatim}

\heroSTUDENT{} Ah, I see. So the \texttt{{[}XS\ Defs\ Body{]}} notation is
like existential quantification then.

\heroADVISOR{} Exactly.

\heroSTUDENT{} One thing I noticed with our representation of \texttt{letrec}
is that we have to be careful so that the number of binders matches the
number of definitions we give. Our typing rules disallow that, but I
wonder if there's a way to have a more accurate representation for
\texttt{letrec} that requires that to be the case?

\heroADVISOR{} Funny you should ask that\ldots{} Let me tell you something
surprising.

  
  \section{Where our heroes get out of the bind using ad-hoc
polymorphism}\label{where-our-heroes-get-out-of-the-bind-using-ad-hoc-polymorphism}

\heroSTUDENT{} No dependent types\ldots{} so what is the \foreignlanguage{greek}{λ}Prolog type system?
Is it some version of System F?

\heroADVISOR{} It is a subset of System F\(_\omega\), actually -- so, the
simply typed lambda calculus, plus prenex polymorphism, plus simple type
constructors of the form
\texttt{type\ *\ ...\ *\ type\ \ensuremath{\to}\ type}. The \texttt{prop}
sort of propositions is a bit special, since we can only add rules to
its inhabitants, but otherwise it is just a normal type.

\heroSTUDENT{} I see. So, that is quite similar to base Haskell -- without the
recent extensions with kind definitions \citep{yorgey2012giving} or
Type-in-Type \citep{weirich2013system} -- and it is still possible to
have dependently typed datatypes there through GADTs.

\heroADVISOR{} How so?

\heroSTUDENT{} Well, you can use empty datatypes to encode type-level data, by
exploiting the type parameters. For example, you could do type-level
natural numbers with \texttt{NatZ\ ::\ *} and
\texttt{NatS\ ::\ *\ \ensuremath{\to}\ *} and then use them as a
``dependent'' index for vectors.

\heroADVISOR{} Oh, right, I forgot about that. Well, we could do the same
thing in Makam then:

\begin{verbatim}
natZ : type.  natS : type \ensuremath{\to} type.
vector : type \ensuremath{\to} type \ensuremath{\to} type.
vnil : vector natZ A.
vcons : A \ensuremath{\to} vector N A \ensuremath{\to} vector (natS N) A.
\end{verbatim}

\heroSTUDENT{} Oh, so \foreignlanguage{greek}{λ}Prolog supports GADTs? Pattern matching propagates type
information and everything?

\heroADVISOR{} Well, it does not work quite the same way, but yes. The way it
works in \foreignlanguage{greek}{λ}Prolog is through \emph{ad-hoc polymorphism}: polymorphic type
variables can be unified at \emph{runtime} rather than at type-checking
time. So before performing term-level unification, type-level
unification is done, so uninstantiated type variables can be further
determined; we can thus ``learn'' and propagate extra type information
at runtime. So we could do \texttt{map} for vectors as follows:

\begin{verbatim}
vmap : [N] (A \ensuremath{\to} B \ensuremath{\to} prop) \ensuremath{\to} vector N A \ensuremath{\to} vector N B \ensuremath{\to} prop.
vmap P vnil vnil.
vmap P (vcons X XS) (vcons Y YS) :- P X Y, vmap P XS YS.
\end{verbatim}

\heroSTUDENT{} Interesting. What is the \texttt{{[}N{]}} notation in the type
of \texttt{vmap}?

\heroADVISOR{} Well, type arguments for propositions are parametric by
default, so that notation says that \texttt{N} is ad-hoc polymorphic --
rules can specialize it further. And the reason Makam requires this
annotation is that we are used to type arguments being parametric, and
we can catch some errors that way -- for example this erroneous
\texttt{foldl} is still well-typed if the type arguments are ad-hoc:

\begin{verbatim}
foldl : (B \ensuremath{\to} A \ensuremath{\to} B \ensuremath{\to} prop) \ensuremath{\to} B \ensuremath{\to} list A \ensuremath{\to} B \ensuremath{\to} prop.
foldl P S nil S.
foldl P S (cons HD TL) S'' :- P HD S S', foldl P S' TL S''.
\end{verbatim}

\heroSTUDENT{} Oh, the \texttt{HD} argument in the call to \texttt{P} should
be second instead of first -- and this definition would only work when
\texttt{A} and \texttt{B} are the same?

\heroADVISOR{} Precisely, but we would only find out at runtime, if it was
called with \texttt{A} and \texttt{B} being different.

\heroSTUDENT{} I see. So is type-level unification a standard \foreignlanguage{greek}{λ}Prolog feature,
or just Makam?

\heroADVISOR{} It's hard to say. I think it was part of the original design by
\citet{miller1988overview}, but I have not come upon any examples that
actively use it so far -- for example, the book by
\citet{miller2012programming} hardly mentions the feature, and the
standard \foreignlanguage{greek}{λ}Prolog implementation, Teyjus \citep{nadathur1999system}, has
a few issues related to polymorphic types that have not allowed me to
test it there.

\heroSTUDENT{} Well, let's see if it is actually useful for what we were
trying to do. Maybe we can use this feature for a better encoding of
\texttt{letrec}? We could do the \texttt{vector} equivalent of
\texttt{bindmany}, carrying a type argument for the number of binders,
so that we can reuse that for a \texttt{vector} of definitions:

\begin{verbatim}
dbind : type \ensuremath{\to} type \ensuremath{\to} type \ensuremath{\to} type. 
dbindbase : B \ensuremath{\to} dbind A natZ B.
dbindnext : (A \ensuremath{\to} dbind A N B) \ensuremath{\to} dbind A (natS N) B.
letrec : dbind term N (vector N term * term) \ensuremath{\to} term.
\end{verbatim}

\heroADVISOR{} That looks good, but I do not like this natural number trick --
if we could define a new \texttt{nat} kind and specify it in the type of
\texttt{dbind}, it would be fine, but as it stands\ldots{}. Well, here's
an idea. The ``dependent'' index could be the right type of
substitutions for the \texttt{n} variables we introduce, so an
\texttt{n}-tuple of terms, rather than \texttt{n} itself:

\begin{verbatim}
dbind : type \ensuremath{\to} type \ensuremath{\to} type \ensuremath{\to} type.
dbindbase : B \ensuremath{\to} dbind A unit B.
dbindnext : (A \ensuremath{\to} dbind A T B) \ensuremath{\to} dbind A (A * T) B.
\end{verbatim}

\heroSTUDENT{} That should work. Should we also define the equivalent of
\texttt{vector} with the same type of index?

\heroADVISOR{} Sure. We could just use the tuple type \texttt{T} and do
pattern-matching on whether it's equal to \texttt{A\ *\ B} or
\texttt{unit}, but constructor-based pattern matching reads much better.
Let's call it \texttt{subst} for substitutions:

\begin{verbatim}
subst : type \ensuremath{\to} type \ensuremath{\to} type.
nil : subst A unit.  cons : A \ensuremath{\to} subst A T \ensuremath{\to} subst A (A * T).
\end{verbatim}

\heroSTUDENT{} We are already using \texttt{nil} and \texttt{cons} for lists.
Should we call the constructors something else?

\heroADVISOR{} No, this works fine, and we can reuse the syntactic sugar for
them. Makam allows overloading for all constants. It takes statically
known type information into account for resolving variables and
disambiguating between them. Sometimes you have to do a type ascription,
but I find it works nicely in most cases.

\heroSTUDENT{} I see. So let me try my hand at writing the helper predicates
that we'll need for \texttt{dbind} and \texttt{subst}. How do these
look?

\begin{verbatim}
intromany : [T] dbind A T B \ensuremath{\to} (subst A T \ensuremath{\to} prop) \ensuremath{\to} prop.
applymany : [T] dbind A T B \ensuremath{\to} subst A T \ensuremath{\to} B \ensuremath{\to} prop.
openmany : [T] dbind A T B \ensuremath{\to} (subst A T \ensuremath{\to} B \ensuremath{\to} prop) \ensuremath{\to} prop.
assumemany : [T T'] (A \ensuremath{\to} B \ensuremath{\to} prop) \ensuremath{\to} subst A T \ensuremath{\to} subst B T' \ensuremath{\to} prop \ensuremath{\to} prop.
map : [T T'] (A \ensuremath{\to} B \ensuremath{\to} prop) \ensuremath{\to} subst A T \ensuremath{\to} subst B T' \ensuremath{\to} prop.
\end{verbatim}

\heroADVISOR{} These look fine -- they're quite similar to the ones for
\texttt{bindmany}. \texttt{assumemany} and \texttt{map} do not really
capture the relationship between \texttt{T} and
\texttt{T\textquotesingle{}}, which are the same tuples save for
\texttt{A}s being replaced by \texttt{B}s\ldots{} but we don't have to
complicate this further; I am sure we could capture that if needed with
another dependent construction.

\heroSTUDENT{} That's what I was thinking too. Let me see; I think most of
these are almost identical to what we had before.

\begin{verbatim}
intromany (dbindbase F) P :- P [].
intromany (dbindnext F) P :- (x:A \ensuremath{\to} intromany (F x) (pfun t \ensuremath{\Rightarrow} P (x :: t))).
...
\end{verbatim}

\begin{scenecomment}
(Our heroes copy-paste the code from before for the rest of the predicates,
changing `\texttt{bindbase}' to `\texttt{dbindbase}' and `\texttt{bindnext}' to `\texttt{dbindnext}'.)
\end{scenecomment}

\heroADVISOR{} Alright, I think we should be able to do \texttt{letrec} now!

\begin{verbatim}
letrec : dbind term T (subst term T * term) \ensuremath{\to} term.
typeof (letrec XS_DefsBody) T' :-
  openmany XS_DefsBody (pfun xs defsbody \ensuremath{\Rightarrow} [Defs Body]
    eq defsbody (Defs, Body),
    assumemany typeof xs TS (map typeof Defs TS),
    assumemany typeof xs TS (typeof Body T')).
\end{verbatim}

\heroSTUDENT{} What is \texttt{eq}?

\heroADVISOR{} Oh, that's just to force unification with
\texttt{(Defs,\ Body)} and get the elements of the tuple -- there's no
destructuring \texttt{pfun}. \texttt{eq} is just defined as:

\begin{verbatim}
eq : A \ensuremath{\to} A \ensuremath{\to} prop.  eq X X.
\end{verbatim}

\heroSTUDENT{} I see. Say, can we use the same dependency trick to do
patterns?

\heroADVISOR{} We should be able to\ldots{} the linearity is going to be a bit
tricky, but I am fairly confident that having explicit support in our
metalanguage just for single-variable binding is enough to model most
complicated forms of binding, when we also make use of polymorphism and
GADTs.

\heroSTUDENT{} Makes sense. Well, I think I have an idea for patterns: we can
have a type argument to keep track of what variables they introduce.
Since within a pattern we can only refer to a variable once\ldots{} no
actual binding needs to take place. But we can use the type argument to
bind the right number of pattern variables into the body of a branch.

\heroADVISOR{} That is true\ldots{}. One way I think about binding is that it
is just a way to introduce a notion of sharing into abstract syntax
trees, so that we can refer to the same thing a number of times. And
you're right that for patterns, the sharing happens from the side of the
pattern into the branch body, not within the pattern itself.

\heroSTUDENT{} Though there is some of that in dependent pattern matching,
where you can reuse a pattern variable and an exact matching takes place
rather than unification\ldots{}.

\heroADVISOR{} \ldots{}Right. But let's not worry about that right now; let's
just do simple patterns. So at the top level, a pattern will just have a
single ``tuple type'' argument with the variables it used. I am thinking
that for sub-patterns, we will need two arguments. One for all the
variables that \emph{can} be used, initially matching the type argument
of the top-level pattern; another argument, for the variables that
\emph{remain} to be used after this sub-pattern is traversed.

\heroSTUDENT{} I don't get that yet. Wait, let me first add natural numbers as
a base type so that we have a simple example.

\begin{verbatim}
nat : typ. zero : term. succ : term \ensuremath{\to} term.
typeof zero nat. typeof (succ N) nat :- typeof N nat.
eval zero zero. eval (succ E) (succ V) :- eval E V.
\end{verbatim}

\heroADVISOR{} Good idea. OK, so here's what I meant:

\begin{verbatim}
patt : type \ensuremath{\to} type \ensuremath{\to} type.
patt_var : patt (term * T) T.
patt_zero : patt T T.
patt_succ : patt T T' \ensuremath{\to} patt T T'.
\end{verbatim}

\heroSTUDENT{} Hmm. So, you said the first argument is what variables are
``available'' when we go into the sub-pattern, second is what we're
``left with''\ldots{} so in the variable case, we ``use up'' one
variable. In the zero case, we don't use any. And for successors, we
just propagate the variables.

\heroADVISOR{} Exactly. Could you do tuples?

\heroSTUDENT{} Let me see, I think I'll need a helper type for multiple
patterns\ldots{}.

\begin{verbatim}
pattlist : type \ensuremath{\to} type \ensuremath{\to} type.
nil : pattlist T T.
cons : patt T1 T2 \ensuremath{\to} pattlist T2 T3 \ensuremath{\to} pattlist T1 T3.
patt_tuple : pattlist T T' \ensuremath{\to} patt T T'.
\end{verbatim}

\heroADVISOR{} Exactly! And here's an interesting one: wildcards.

\begin{verbatim}
patt_wild : patt T T.
\end{verbatim}

\heroSTUDENT{} Oh, because that does not really introduce any pattern
variables that we can use. So if I understand this correctly, top-level
patterns should always use up all their variables -- they should end
with the second argument being \texttt{unit}, right?

\heroADVISOR{} Exactly, so this should be fine for a single-branch
pattern-match construct:

\begin{verbatim}
case_or_else : term \ensuremath{\to} patt T unit \ensuremath{\to} dbind term T term \ensuremath{\to} term \ensuremath{\to} term.
\end{verbatim}

\heroSTUDENT{} Let me parse that\ldots{} the first argument is the scrutinee,
the second is the pattern\ldots{} the third is the branch body, with the
pattern variables introduced. Oh, and the last argument is the
\texttt{else} case.

\heroADVISOR{} Right. And I think something like this should work for the
typing judgment. Let me write a few cases.

\begin{verbatim}
typeof : [T T' Ttyp T'typ] patt T T' \ensuremath{\to} subst typ T'typ \ensuremath{\to} subst typ Ttyp \ensuremath{\to} typ \ensuremath{\to} prop.
typeof patt_var S' (cons T S') T.
typeof patt_wild S S T.
typeof patt_zero S S nat.
typeof (patt_succ P) S' S nat :- typeof P S' S nat.
\end{verbatim}

\heroSTUDENT{} I see, so given a pattern and the types of the variables
following the sub-pattern, we produce the types of all the variables and
the type of the pattern itself. Makes sense. I'll do tuples:

\begin{verbatim}
typeof : [T T' Ttyp T'typ]
  pattlist T T' \ensuremath{\to} subst typ T'typ \ensuremath{\to} subst typ Ttyp \ensuremath{\to} list typ \ensuremath{\to} prop.
typeof (patt_tuple PS) S' S (product TS) :- typeof PS S' S TS.
typeof [] S S [].
typeof (P :: PS) S3 S1 (T :: TS) :- typeof PS S3 S2 TS, typeof P S2 S1 T.
\end{verbatim}

\heroADVISOR{} Looks good. Can you do the typing rule for the case statement?

\heroSTUDENT{} How does this look?

\begin{verbatim}
typeof (case_or_else Scrutinee Pattern Body Else) T' :-
  typeof Scrutinee T,
  typeof Pattern nil TS T,
  openmany Body (pfun xs body \ensuremath{\Rightarrow} assumemany typeof xs TS (typeof body T')),
  typeof Else T'.
\end{verbatim}

\heroADVISOR{} That's great! This was a little tricky, but still, not too bad.
Actually, I know of one thing that is surprisingly simple to do: the
evaluation rule. We just have to convert a pattern into a term, where we
replace the pattern variables with \emph{meta-level} unification
variables -- then we can just reuse meta-level unification to do the
actual pattern match!

\heroSTUDENT{} Oh, that would be nice. So not only do we get variable
substitutions for free, we also get unification for free in some cases!

\heroADVISOR{} Exactly. So something like this should work:

\begin{verbatim}
patt_to_term : [T T'] patt T T' \ensuremath{\to} term \ensuremath{\to} subst term T' \ensuremath{\to} subst term T \ensuremath{\to} prop.
patt_to_term patt_var X Subst (X :: Subst).
patt_to_term patt_wild _ Subst Subst.
patt_to_term patt_zero zero Subst Subst.
patt_to_term (patt_succ PN) (succ EN) Subst' Subst :- patt_to_term PN EN Subst' Subst.
\end{verbatim}

\heroSTUDENT{} I see, interesting! So in each rule we introduce the
unification variables that we need, like \texttt{X} for the variable
case, and store them in the substitution that we will use with the
pattern body.

\begin{scenecomment}
(Our heroes also write down the rules for multiple patterns and tuples, which are
available in the unabridged version of this story.)
\end{scenecomment}

\heroADVISOR{} We should be good to write the evaluation rule now.

\begin{verbatim}
eval (case_or_else Scrutinee Pattern Body Else) V :-
  patt_to_term Pattern TermWithUnifvars [] Unifvars,
  if (eq Scrutinee TermWithUnifvars)  (* reuse unification from the meta-language *)
  then (applymany Body Unifvars Body', eval Body' V)
  else (eval Else V).
\end{verbatim}

\heroSTUDENT{} I see! So, if meta-level unification is successful, we have a
match, and we substitute the instantiations we found for the pattern
variables into the body. But you are using if-then-else? We haven't used
that so far.

\heroADVISOR{} Oh yes, I forgot to mention that. It behaves as follows: when
there is at least one way to prove the condition, it proceeds to the
\texttt{then} branch, otherwise it goes to the \texttt{else} branch.
Pretty standard really. It is one thing that the Prolog cut statement,
\texttt{!}, is useful for, but cut introduces all sorts of trouble.
\citet{kiselyov05backtracking} is worth reading for alternatives to the
cut statement and the semantics of
\texttt{if}-\texttt{then}-\texttt{else} and \texttt{not} in logic
programming, and Makam follows that closely.

\heroSTUDENT{} Noted in my to-read list. But let us try pattern matching out!
How about predecessors for natural numbers? I'll write a query that
type-checks and evaluates a couple of cases.

\begin{verbatim}
(eq _PRED (lam _ (fun n \ensuremath{\Rightarrow} case_or_else n
  (patt_succ patt_var) (dbindnext (fun pred \ensuremath{\Rightarrow} dbindbase pred))
  zero)),
 typeof _PRED T,
 eval (app _PRED zero) PRED0, eval (app _PRED (succ (succ zero))) PRED2) ?
>> Yes:
>> T := arrow nat nat, PRED0 := zero, PRED2 := succ zero.
\end{verbatim}

\heroADVISOR{} Seems to be working fine!

  
  \section{More ML-like language}
  \section{Where our heroes break into song and add more ML
features}\label{where-our-heroes-break-into-song-and-add-more-ml-features}

\begin{scenecomment}
(Our heroes need a small break, so they work on a couple of features while improvising on a makam\footnote{Makam is the system of melodic modes used in traditional Arabic and Turkish music and in the Greek rembetiko, comprised of a set of scales, patterns of melodic development, and rules for improvisation.}. Roza is singing, and Hagop is playing the oud.)
\end{scenecomment}

\begin{verse}
``Explicit System F polymorphism is easy, at some point we'll do Hindley-Milner too. \\
Types are well-formed by construction, an extra `$\vdash \tau \; \text{wf}$' judgment we won't do.''
\end{verse}

\begin{verbatim}
forall : (typ \ensuremath{\to} typ) \ensuremath{\to} typ.
lamt : (typ \ensuremath{\to} term) \ensuremath{\to} term.
appt : term \ensuremath{\to} typ \ensuremath{\to} term.
typeof (lamt E) (forall T) :- (a:typ \ensuremath{\to} typeof (E a) (T a)).
typeof (appt E T) (TF T) :- typeof E (forall TF).
\end{verbatim}

\begin{verse}
``We are now adding top-level programs, to get into datatype declarations. \\
We would rather do modules, but those would need quite a bit of deliberation. \\
And we still have contextual types to do, those will require our full attention.''
\end{verse}

\begin{verbatim}
program : type.
wfprogram : program \ensuremath{\to} prop.

let : term \ensuremath{\to} (term \ensuremath{\to} program) \ensuremath{\to} program.
wfprogram (let E P) :- typeof E T, (x:term \ensuremath{\to} typeof x T \ensuremath{\to} wfprogram (P x)).

main : term \ensuremath{\to} program.
wfprogram (main E) :- typeof E _.
\end{verbatim}

\begin{center}\rule{0.5\linewidth}{\linethickness}\end{center}

\heroADVISOR{} I think we are ready to do polymorphic algebraic datatypes now.
We'll add a type for type constructors, like \texttt{list}, dependent on
their arity; and a type for the constructors of a datatype. Also a type
for constructor declarations, dependent on the number of constructors
they introduce:

\begin{verbatim}
typeconstructor : type \ensuremath{\to} type.
constructor : type.

ctor_declaration : type \ensuremath{\to} type.
nil : ctor_declaration unit.
cons : list typ \ensuremath{\to} ctor_declaration T \ensuremath{\to} ctor_declaration (constructor * T).
\end{verbatim}

\heroSTUDENT{} Oh, so each constructor takes multiple arguments. Great. So
datatype declarations would be something like this:

\begin{verbatim}
datatype_declaration : type \ensuremath{\to} type \ensuremath{\to} type.
datatype_declaration : 
  (typeconstructor Arity \ensuremath{\to} dbind typ Arity (ctor_declaration Ctors)) \ensuremath{\to}
  datatype_declaration Arity Ctors.

datatype :
  datatype_declaration Arity Ctors \ensuremath{\to}
  (typeconstructor Arity \ensuremath{\to} dbind constructor Ctors program) \ensuremath{\to} program.
\end{verbatim}

\heroADVISOR{} Right, so when declaring a datatype, we introduce a
\texttt{typeconstructor} variable so that we can refer to the type
recursively when we declare our constructors. And we also have access to
the right number of polymorphic variables, matching the \texttt{Arity}
of the constructor. I like how you split out the declaration of the type
itself from the ``rest of the program'' part, since this could become
unwieldy otherwise.

\heroSTUDENT{} That's what I thought too. And I see why you made the type
constructors carry their arities -- to keep types well-formed by
construction. In order to be able to actually refer to the type
constructors, though, don't we need a type former:

\begin{verbatim}
tconstr : typeconstructor T \ensuremath{\to} subst typ T \ensuremath{\to} typ.
\end{verbatim}

\heroADVISOR{} We do. Also keep in mind that in a richer type system, we
probably would need an extra kind-checking predicate. But this will do
for now. Let's just make sure this is fine -- I'll write down the
declaration of binary trees, to make sure we're not missing anything,
and typecheck it with Makam.

\begin{verbatim}
%type (datatype_declaration
  (fun tree \ensuremath{\Rightarrow} dbindnext (fun a \ensuremath{\Rightarrow} dbindbase
    [ (* leaf *) [],
      (* node *) [tconstr tree [a], a, tconstr tree [a]] ]))).
>> (...) : datatype_declaration (typ * unit) (constructor * constructor * unit)
\end{verbatim}

\heroSTUDENT{} Looks good. Should we proceed to the actual well-formedness for
datatype declarations? I think we will need a predicate to keep track of
information about a constructor -- which datatype it belongs to and what
arguments it expects. That way we can carry that information in the
assumptions context.

\begin{verbatim}
constructor_info :
  typeconstructor Arity \ensuremath{\to} constructor \ensuremath{\to} dbind typ Arity (list typ) \ensuremath{\to} prop.
\end{verbatim}

\heroADVISOR{} Yes, and we are mostly ready otherwise:

\begin{verbatim}
wfprogram (datatype (datatype_declaration ConstructorDecls) Program') :-
  (dt:(typeconstructor T) \ensuremath{\to} ([PolyTypes]
    openmany (ConstructorDecls dt) (pfun tvars constructor_decls \ensuremath{\Rightarrow} (
      constructor_polytypes constructor_decls tvars PolyTypes)),
    openmany (Program' dt) (pfun constructors program' \ensuremath{\Rightarrow}
      assumemany (constructor_info dt) constructors PolyTypes
      (wfprogram program')))).
\end{verbatim}

\heroSTUDENT{} This is a tricky piece of code. Let me stare at it for a while.
(\ldots{}) What is this predicate, \texttt{constructor\_polytypes}?

\heroADVISOR{} I'm using that in order to re-abstract over the type
variables\ldots{}. See, in the constructor declaration, we've introduced
a number of type variables. We need to abstract over them, in order to
get the polymorphic type of each constructor for the rest of the
program. Note that \texttt{PolyTypes} can't capture the type variables
\texttt{tvars} we introduce.

\heroSTUDENT{} I think I got it. Let me try to implement it.

\heroADVISOR{} Here's a hint.

\begin{verbatim}
(x:typ \ensuremath{\to} y:typ \ensuremath{\to} applymany PolyType [x, y] (arrow y x)) ?
>> Yes:
>> PolyType = dbindnext (fun x \ensuremath{\Rightarrow} dbindnext (fun y \ensuremath{\Rightarrow} dbindbase (arrow y x)))
\end{verbatim}

\begin{scenecomment}
(After a few attempts, Hagop comes up with the following definition.)
\end{scenecomment}

\begin{verbatim}
constructor_polytypes : [Arity Ctors PolyTypes]
  ctor_declaration Ctors \ensuremath{\to} subst typ Arity \ensuremath{\to}
  subst (dbind typ Arity (list typ)) PolyTypes \ensuremath{\to} prop.

constructor_polytypes [] _ [].
constructor_polytypes (CtorType :: CtorTypes) TypVars (PolyType :: PolyTypes) :-
  applymany PolyType TypVars CtorType,
  constructor_polytypes CtorTypes TypVars PolyTypes.
\end{verbatim}

\heroSTUDENT{} I see what you were getting at. I think this is an interesting
use of \texttt{applymany}: we are using it in the opposite direction
than what we have used it so far. We are giving it \texttt{TypVars} and
\texttt{CtorType} as inputs, and then we get \texttt{PolyType}, with all
the needed binders, as an output. And since the way we're using it,
\texttt{PolyType} cannot capture the \texttt{TypVars}, it all works out
correctly!

\heroADVISOR{} Excellent! Let's add the term-level former for constructors,
too.

\heroSTUDENT{} That is easy, compared to what we just did.

\begin{verbatim}
constr : constructor \ensuremath{\to} list term \ensuremath{\to} term.
typeof (constr Constructor Args) (tconstr TypConstr TypArgs) :-
  constructor_info TypConstr Constructor PolyType,
  applymany PolyType TypArgs Typs,
  map typeof Args Typs.
\end{verbatim}

\heroADVISOR{} You're getting the hang of this. Let's do something actually
difficult, then; type synonyms.

  
  \section{Adding Type Synonyms}
  Let us proceed to add type synonyms:

\begin{verbatim}
type_synonym : dbind typ T typ \ensuremath{\to} (typeconstructor T \ensuremath{\to} program) \ensuremath{\to} program.

type_synonym_info : typeconstructor T \ensuremath{\to} dbind typ T typ \ensuremath{\to} prop.

wfprogram (type_synonym Syn Program') :-
  (t:(typeconstructor T) \ensuremath{\to}
   type_synonym_info t Syn \ensuremath{\to}
   wfprogram (Program' t)).
\end{verbatim}

Simple enough. How to typecheck them, though? We need something like the
conversion rule:

\begin{displaymath}
\inferrule{
  \Gamma \vdash e : \tau \\ \tau =_{\delta} \tau'
}{
  \Gamma \vdash e : \tau'
}
\end{displaymath}

Here \(=_{\delta}\) means equality up to expanding type synonyms.

We will need a type-equality predicate:

\begin{verbatim}
teq : typ \ensuremath{\to} typ \ensuremath{\to} prop.
\end{verbatim}

A naive attempt at the conversion rule would be:

\begin{verbatim}
typeof E T :- typeof E T', teq T T'.
\end{verbatim}

However, it is easy to see that this rule leads to divergence: it does a
recursive call to itself.

We can do a bit better. We only need to use the conversion rule in cases
where we already know something about the type \texttt{T} of the
expression, but our typing rules do not match that type. In
bi-directional typing parlance, instead of analyzing the type \texttt{T}
of the expression \texttt{E}, we want to synthesize the type starting
from a new meta-variable \texttt{T\textquotesingle{}}, and then check
that the two types are equal using \texttt{teq}. So we need to change
our rule to apply only in the case where \texttt{T} starts with a
concrete type constructor, rather than when it is an uninstantiated
meta-variable.

It is typical for a logic-programming language to have a predicate that
only succeeds when a specific term is uninstantiated (usually called
\texttt{var}). In Makam this is called \texttt{refl.isunif} -- the
\texttt{refl} namespace prefix standing for the fact that we call these
kinds of predicates ``reflective,'' as they give us extra-logical
information about the form of a term (sometimes referred to as
``meta-predicates'' in Prolog parlance). Our second attempt thus looks
as follows:

\begin{verbatim}
typeof E T :- not(refl.isunif T), typeof E T', teq T T'.
\end{verbatim}

Upon further consideration, we see that this rule leads to an infinite
loop as well: since \texttt{teq} should be reflexive, for every proof of
\texttt{typeof\ E\ T\textquotesingle{}} through the other rules, a new
proof using this rule will be discovered, which will lead to another
proof for it, etc. One fix is to make sure that this rule is only used
once at the end, if typing using the other rules fails:

\begin{verbatim}
typeof, typeof_cases, typeof_conversion : term \ensuremath{\to} typ \ensuremath{\to} prop.
typeof E T :-
  if (typeof_cases E T)
  then success
  else (typeof_conversion E T).
typeof_cases (app E1 E2) T' :-
  typeof E1 (arrow T1 T2),
  typeof E2 T1.
...
typeof_conversion E T :-
  not(refl.isunif T), typeof_cases E T', teq T T'.
\end{verbatim}

However, this would require changing every typing rule we had. Instead,
we can do a trick, to force the rule to only fire once for each
expression \texttt{E}, remembering the fact that we have used the rule
already:

\begin{verbatim}
already_in : [A] A \ensuremath{\to} prop.
typeof E T :-
  not(refl.isunif T),
  not(already_in (typeof E)),
  (already_in (typeof E) \ensuremath{\to} typeof E T'),
  teq T T'.
\end{verbatim}

Also, we need to make sure that we also take the conversion rule into
account for patterns:

\begin{verbatim}
typeof (P : patt A B) S' S T :-
  not(refl.isunif T),
  not(already_in (typeof P)),
  (already_in (typeof P) \ensuremath{\to} typeof P S' S T'),
  teq T T'.
\end{verbatim}

Now let us go and define the actual \texttt{teq} predicate. A first
approach would be to just write out each case individually:

\begin{verbatim}
teq (arrow T1 T2) (arrow T1' T2') :- teq T1 T1', teq T2 T2'.
teq (product TS) (product TS') :- map teq TS TS'.
teq (arrowmany TS T) (arrowmany TS' T') :- teq T T', map teq TS TS'.
teq nat nat.
teq (forall T) (forall T') :- (x:typ \ensuremath{\to} teq x x \ensuremath{\to} teq (T x) (T' x)).
teq (tconstr TC Args) (tconstr TC Args') :- map teq Args Args'.
teq (tconstr TC Args) T' :-
  type_synonym_info TC Syn,
  applymany Syn Args T,
  teq T T'.
teq T' (tconstr TC Args) :-
  type_synonym_info TC Syn,
  applymany Syn Args T,
  teq T' T.
\end{verbatim}

Only the two last cases are important; the rest is boilerplate that
performs structural recursion through the type. Can we do better than
that?

Let us ruminate on a possible solution. We want to handle the case where
we have a constructor applied to a number of arguments generically, so
roughly something like:

\begin{verbatim}
teq (Constructor Arguments) (Constructor Arguments') :-
  map teq Arguments Arguments'.
\end{verbatim}

What we mean here, taking the \texttt{arrow} rule as an example, is that
\texttt{Constructor} would match with \texttt{arrow}, and
\texttt{Arguments} would get instantiated with the list of arguments of
the constructor. One thing to be careful about, though, is that the
types of arguments are not all the same. As a result, instead of a
homogeneous list, we need a heterogeneous list, which is simple to
represent using the existential type, \texttt{dyn}:

\begin{verbatim}
dyn : type.
dyn : A \ensuremath{\to} dyn.
\end{verbatim}

So the type of \texttt{Arguments} should be \texttt{list\ dyn} rather
than \texttt{list\ typ}. The type of \texttt{teq} will need to be
changed, so that we can apply it for any different type, rather than
just \texttt{typ}:

\begin{verbatim}
teq : [A] A \ensuremath{\to} A \ensuremath{\to} prop.
\end{verbatim}

Furthermore, since we have a heterogeneous list, we need a \texttt{map}
that uses polymorphic recursion: it needs take a polymorphic function as
an argument, so that it can be used at different types for different
elements of the list.

\begin{verbatim}
dyn.map : (forall A. [A] A \ensuremath{\to} A \ensuremath{\to} prop) \ensuremath{\to} list dyn \ensuremath{\to} list dyn \ensuremath{\to} prop.
\end{verbatim}

This type is in contrast to one like
\texttt{{[}A{]}\ (A\ \ensuremath{\to}\ A\ \ensuremath{\to}\ prop)\ \ensuremath{\to}\ list\ dyn\ \ensuremath{\to}\ list\ dyn\ \ensuremath{\to}\ prop},
which would instantiate the type \texttt{A} to the type of the first
element of the list, making further applications to different types
impossible.

Makam currently does not provide higher-rank types as the above type
suggests -- nor do any \foreignlanguage{greek}{λ}Prolog implementations that we are aware of.
Instead, it provides a way to side-step this issue, through a predicate
that replaces polymorphic type variables with fresh variables, allowing
it to be instantiated with new types. This is called \texttt{dyn.call},
and \texttt{dyn.map} can be defined in terms of it:

\begin{verbatim}
dyn.call : [B] (A \ensuremath{\to} A \ensuremath{\to} prop) \ensuremath{\to} B \ensuremath{\to} B \ensuremath{\to} prop.
dyn.map : (A \ensuremath{\to} A \ensuremath{\to} prop) \ensuremath{\to} list dyn \ensuremath{\to} list dyn \ensuremath{\to} prop.
dyn.map P [] [].
dyn.map P (HD :: TL) (HD' :: TL') :- dyn.call P HD HD', dyn.map P TL TL'.
\end{verbatim}

(\texttt{dyn.call} is itself defined in terms of a more fundamental
predicate \texttt{dyn.duphead} that creates a fresh version of a single
polymorphic constructor with fresh type variables.)

Based on these, the only thing missing is a way to actually check
whether a term is a ground term that can be decomposed into a
constructor and a list of arguments. Makam provides that functionality
in the form of the \texttt{refl.headargs} predicate:

\begin{verbatim}
refl.headargs : B \ensuremath{\to} A \ensuremath{\to} list dyn \ensuremath{\to} prop.
\end{verbatim}

(Other Prolog implementations also provide predicates towards the same
effect; for example, SWI-Prolog provides
\texttt{compound\_name\_arguments}, which is quite similar. Though such
predicates are not typical in other \foreignlanguage{greek}{λ}Prolog implementations, they should
not be viewed as a hack: we could always define these within the
language if we maintained a discipline, where we added a rule to
\texttt{refl.headargs} for every constructor that we introduce. For
example:

\begin{verbatim}
arrowmany : list typ \ensuremath{\to} typ \ensuremath{\to} typ.
refl.headargs (arrowmany TS T) [arrowmany, [dyn TS, dyn T]].
\end{verbatim}

The only other wrinkle would be to check via \texttt{refl.isunif} that
we are not instantiating a unification variable.)

We are now ready to proceed to defining the boilerplate generically. We
will do this as a reusable higher-order predicate for structural
recursion, which we will use to implement \texttt{teq}. We will define
it in open-recursion style, providing the predicate to use on recursive
calls as an argument:

\begin{verbatim}
structural_recursion : [B] (A \ensuremath{\to} A \ensuremath{\to} prop) \ensuremath{\to} B \ensuremath{\to} B \ensuremath{\to} prop.

structural_recursion Rec X Y :-
  refl.headargs X Constructor Arguments,
  dyn.map Rec Arguments Arguments',
  refl.headargs Y Constructor Arguments'.
\end{verbatim}

We also need to handle built-in types, such as the meta-level
\texttt{int} and \texttt{string} types, in case they are used as
arguments with other constructors:

\begin{verbatim}
structural_recursion Rec (X : string) (X : string).
structural_recursion Rec (X : int) (X : int).
\end{verbatim}

And last, we need to handle the case of the meta-level function type as
well:

\begin{verbatim}
structural_recursion Rec (X : A \ensuremath{\to} B) (Y : A \ensuremath{\to} B) :-
  (x:A \ensuremath{\to} structural_recursion Rec x x \ensuremath{\to} structural_recursion Rec (X x) (Y x)).
\end{verbatim}

We are done! Now we can define \texttt{teq} using
\texttt{structural\_recursion}, through an auxiliary predicate called
\texttt{teq\_aux}. We only need to define the non-trivial cases for it,
using \texttt{structural\_recursion} for the rest, while tying the open
recursion knot at the same time:

\begin{verbatim}
teq_aux : [A] A \ensuremath{\to} A \ensuremath{\to} prop.

teq T T' :- teq_aux T T'.

teq_aux T T' :-
  structural_recursion teq_aux T T'.

teq_aux (tconstr TC Args) T' :-
  type_synonym_info TC Synonym,
  applymany Synonym Args T,
  teq_aux T T'.

teq_aux T' (tconstr TC Args) :-
  type_synonym_info TC Synonym,
  applymany Synonym Args T,
  teq_aux T' T.
\end{verbatim}

Other than minimizing the boilerplate, the great thing about using
\texttt{structural\_recursion} is that no adaptation needs to be done
when we add any new constructor to our \texttt{typ} datatype -- even if
it uses new types that we have not defined before. For example, we did
not have to take any special provision to handle types we defined
earlier such as \texttt{dbind} -- everything works out thanks to the
reflective predicates we are using. (Mention something about the
expression problem?)

The one form of terms that \texttt{structural\_recursion} does not
handle are uninstantiated unification variables. We have found that it
works well to define special handling of unification variables for each
new generically recursive predicate. In this case, \texttt{teq} is only
supposed to be used with ground terms, so it is fine if we fail when we
encounter a unification variable.

Let us try out an example:

\begin{verbatim}
ascribe : term \ensuremath{\to} typ \ensuremath{\to} term.
typeof (ascribe E T) T :- typeof E T.

wfprogram (
  (type_synonym (dbindnext (fun a \ensuremath{\Rightarrow} dbindbase (product [a, a])))
  (fun bintuple \ensuremath{\Rightarrow} 
  
  main (lam (tconstr bintuple [product [nat, nat]])
            (fun x \ensuremath{\Rightarrow} 
    case_or_else x
    (patt_tuple [patt_tuple [patt_wild, patt_wild], patt_tuple [patt_wild, patt_wild]])
    (dbindbase (tuple []))
    (tuple [])
  ))
))) ?
>> Yes.
\end{verbatim}

Let us make sure we do not diverge on type error:

\begin{verbatim}
wfprogram (
  (type_synonym (dbindnext (fun a \ensuremath{\Rightarrow} dbindbase (product [a, a])))
  (fun bintuple \ensuremath{\Rightarrow} 
  
  main (lam (tconstr bintuple [product [nat, nat]])
            (fun x \ensuremath{\Rightarrow} 
    case_or_else x
    (patt_tuple [patt_tuple [patt_wild], patt_tuple [patt_wild, patt_wild]])
    (dbindbase (tuple []))
    (tuple [])
  ))
))) ?
>> Impossible.
\end{verbatim}

  
  \section{Contextual Types}
  \section{Where our heroes tackle dependencies, contexts, and a new level
of
meta}\label{where-our-heroes-tackle-dependencies-contexts-and-a-new-level-of-meta}

\heroSTUDENT{} I'm fairly confident by now that Makam should be able to handle
the research idea we want to try out. Shall we get to it?

\heroADVISOR{} Yes, it is time. So, what we are aiming to do, is add a
facility for type-safe, heterogeneous meta-programming to our object
language, similar to MetaHaskell \citep{mainland2012explicitly}. This
way we can manipulate the terms of a separate object language in a
type-safe manner.

\heroSTUDENT{} Exactly. We'd like our object language to be a formal logic, so
our language will be similar to Beluga \citep{pientka2010beluga} or
VeriML \citep{stampoulis2013veriml}. We'll have to be able to pattern
match over the terms of the object language, too, so they are runtime
entities\ldots{}. But we don't need to do all of that, let's just do a
basic version for now, and I can do the rest on my own.

\heroADVISOR{} Sounds good. So, I think the fragment we should do is this: we
will have dependent functions over a distinguished language of
\emph{dependent indices}. We need the dependency so that, for example,
we can take an object-level type as an argument, and return an
object-level term that uses that type.

\heroSTUDENT{} Exactly. Dependent products should be similar, but we can skip
them for now, and just add a way to return an object-level term from the
meta-level.

\heroADVISOR{} Good idea. We are getting into many levels of meta -- there's
the meta-language we're using, Makam; there's the object language we are
encoding, which is a meta-language in itself, let's call that
Heterogeneous Meta ML Light (HMML?); and there's the ``object-object''
language that HMML is manipulating. And let's keep that last one simple:
the simply typed lambda calculus (STLC).

\heroSTUDENT{} Great. So, our dependent indices will be the types and terms of
STLC -- actually, the open terms of STLC.

\heroADVISOR{} It's a plan. So, let's get to it. Let's first add distinguished
sorts for dependent indices, and dependent classifiers -- we'll use
those to type-check the indices, using an appropriate predicate. Let's
also have a distinguished type for \emph{dependent variables}, that is,
variables of dependent indices; and a way to substitute such a variable
for an object.

\begin{verbatim}
depindex, depclassifier, depvar : type.
depclassify : depindex \ensuremath{\to} depclassifier \ensuremath{\to} prop.
depclassify : depvar \ensuremath{\to} depclassifier \ensuremath{\to} prop.
depwf : depclassifier \ensuremath{\to} prop.
depsubst : [A] (depvar \ensuremath{\to} A) \ensuremath{\to} depindex \ensuremath{\to} A \ensuremath{\to} prop.
\end{verbatim}

\newcommand\dep[1]{\ensuremath{#1_{\text{d}}}}
\newcommand\lift[1]{\ensuremath{\langle#1\rangle}}

\heroSTUDENT{} Right, we might need to treat variables specially, so it's good
that they're a different type. And we might need to check that
classifiers are well-formed.

\heroADVISOR{} Now, we have a few typing rules to add. I'll use
``\(\dep{\cdot}\)'' to signify things that have to do with the dependent
indices.

\vspace{-1em}\begin{mathpar}
\small
\inferrule{\dep{\Psi} \dep{\vdash} \dep{i} : \dep{c}}
          {\Gamma; \dep{\Psi} \vdash \lift{\dep{i}} : \lift{\dep{c}}}

\inferrule{\Gamma; \dep{\Psi}, \; \dep{v} : \dep{c} \vdash e : \tau \\ \dep{\Psi} \dep{\vdash} \dep{c} \; \text{wf}}
          {\Gamma; \dep{\Psi} \vdash \Lambda \dep{v} : \dep{c}.e : \Pi \dep{v} : \dep{c}.\tau}

\inferrule{\Gamma; \dep{\Psi} \vdash e : \Pi \dep{v} : \dep{c}.\tau \\ \dep{\Psi} \dep{\vdash} \dep{i} : \dep{c}}
          {\Gamma; \dep{\Psi} \vdash e @ \dep{i} : \dep{\text{subst}}(\tau, [\dep{i}/\dep{v}])}
\end{mathpar}

\heroSTUDENT{} Those are very easy to transcribe to Makam.

\begin{verbatim}
lamdep : depclassifier \ensuremath{\to} (depvar \ensuremath{\to} term) \ensuremath{\to} term.
appdep : term \ensuremath{\to} depindex \ensuremath{\to} term.
liftdep : depindex \ensuremath{\to} term. liftdep : depclassifier \ensuremath{\to} typ.
pidep : depclassifier \ensuremath{\to} (depvar \ensuremath{\to} typ) \ensuremath{\to} typ.

typeof (lamdep C EF) (pidep C TF) :-
  (v:depvar \ensuremath{\to} depclassify v C \ensuremath{\to} typeof (EF v) (TF v)), depwf C.
typeof (appdep E I) T' :- typeof E (pidep C TF), depclassify I C, depsubst TF I T'.
typeof (liftdep I) (liftdep C) :- depclassify I C.
\end{verbatim}

\heroADVISOR{} Great. Just wanted to say, this framework is quite general. We
could instantiate dependent indices with a language of natural numbers,
equality predicates, and equality proofs; this would be quite similar to
the Dependent ML formulation of \citet{licata2005formulation}. But let's
go back to what we're trying to do. I'll add the object language in a
separate namespace prefix -- we can use `\texttt{\%extend}' for going
into a namespace -- and I'll just copy-paste our STLC code from earlier
on.

\begin{verbatim}
%extend object.
term : type. typ : type. typeof : term \ensuremath{\to} typ \ensuremath{\to} prop.
...
%end.
\end{verbatim}

\heroSTUDENT{} Great! I'll make these into dependent indices now, including
both types and terms.

\begin{verbatim}
iterm : object.term \ensuremath{\to} depindex.     ityp : object.typ \ensuremath{\to} depindex.
ctyp : object.typ \ensuremath{\to} depclassifier.  cext : depclassifier.

depclassify (iterm E) (ctyp T) :- object.typeof E T.
depclassify (ityp T) cext :- object.wftyp T.
depwf (ctyp T) :- object.wftyp T.
depwf cext.
\end{verbatim}

\heroADVISOR{} Right, we'll need to check that types are well-formed, too.
Right now, they are all well-formed by construction, but let's prepare
for any additions, by setting up a structurally recursive predicate. The
\texttt{wftyp\_cases} predicate will hold the important type-checking
cases, and it will have an extra argument to say whether it applies or
not.

\begin{verbatim}
%extend object.
wftyp : typ \ensuremath{\to} prop. wftyp_aux : [A] A \ensuremath{\to} A \ensuremath{\to} prop.
wftyp_cases : [A] A \ensuremath{\to} A \ensuremath{\to} bool \ensuremath{\to} prop.
wftyp T :- wftyp_aux T T.
wftyp_aux T T :- if (wftyp_cases T T Applies)
                 then (eq Applies true)
                 else (eq Applies false, structural_recursion wftyp_aux T T).
%end.
\end{verbatim}

\heroSTUDENT{} I see -- if a type-checking rule applies, but fails, we don't
want to proceed to also try structural recursion; it would defeat the
purpose. Neat trick. I also see that your structural recursion just
needs to do a simple visit, it does not need to produce an output; hence
the repeat of the same \texttt{typ} argument. Let's prepare for
substitutions too, in the same way.

\begin{verbatim}
depsubst_aux : [A] depvar \ensuremath{\to} depindex \ensuremath{\to} A \ensuremath{\to} A \ensuremath{\to} prop.
depsubst_cases : [A] depvar \ensuremath{\to} depindex \ensuremath{\to} A \ensuremath{\to} A \ensuremath{\to} bool \ensuremath{\to} prop.
depsubst F I Res :- (v:depvar \ensuremath{\to} depsubst_aux v I (F v) Res).
depsubst_aux Var Replace Where Result :-
  if (depsubst_cases Var Replace Where Result Applies)
  then (eq Applies true)
  else (eq Applies false,
        structural_recursion (depsubst_aux Var Replace) Where Result).
\end{verbatim}

\heroADVISOR{} Great! We only have one thing missing: we need to close the
loop, being able to refer to dependent variables from within
object-level terms and types. By the way, we are very much following the
construction of \textit{(reference elided for blind submission)}.

\heroSTUDENT{} I got this.

\begin{verbatim}
%extend object.
varterm : depvar \ensuremath{\to} term.  vartyp : depvar \ensuremath{\to} typ.
typeof (varterm V) T :- depclassify V (ctyp T).
wftyp_cases (vartyp V) (vartyp V) true :- depclassify V cext.
%end.

depsubst_cases Var (iterm Replace) (object.varterm Var) Replace true.
depsubst_cases Var (ityp Replace)  (object.vartyp Var)  Replace true.
\end{verbatim}

\heroADVISOR{} This is exciting, let me try this out! I'll do a function that
takes an object-level type and returns the object-level identity
function for it.

\begin{verbatim}
typeof (lamdep cext (fun t \ensuremath{\Rightarrow}
         (liftdep (iterm (object.lam (object.vartyp t) (fun x \ensuremath{\Rightarrow} x)))))) T ?
>> Yes!!!!!
>> T := pidep cext (fun t \ensuremath{\Rightarrow}
>>        liftdep (ctyp (object.arrow (object.vartyp t) (object.vartyp t))))
\end{verbatim}

\heroSTUDENT{} Look, even the Makam REPL is excited!

\heroADVISOR{} Wait until it sees what we have in store for it next: open STLC
terms in our indices!

\heroSTUDENT{} Good thing I've printed out the contextual types paper by
\citet{nanevski2008contextual}. (\ldots{}) OK, so it says here that we
can use contextual types to record at the type level, the context that
open terms depend on. So let's say, an open \texttt{object.term} of type
\(\tau\) that mentions variables of a \(\Phi\) context would have a
contextual type of the form \([\Phi] \tau\). This is some sort of modal
typing, with a precise context.

\heroADVISOR{} Right. So in our case, open STLC terms depend on a number of
variables, and we will need to keep track of the STLC types of those
variables, in order to maintain type safety. So, let's add a new
dependent index for open STLC terms; and, a dependent classifier for
their contextual type, which records the types of the variables that the
term depends on, as well as the actual type of the term itself.

\heroSTUDENT{} Let me see. I think something like this is what we want:

\begin{verbatim}
iopen_term : bindmany object.term object.term \ensuremath{\to} depindex.
cctx_typ : list object.typ \ensuremath{\to} object.typ \ensuremath{\to} depclassifier.
\end{verbatim}

\heroADVISOR{} That looks right to me. I can write the classification and
well-formedness rules for those.

\begin{verbatim}
depclassify (iopen_term XS_E) (cctx_typ TS T) :-
  openmany XS_E (pfun xs e \ensuremath{\Rightarrow}
    assumemany object.typeof xs TS (object.typeof e T),
    foreach object.wftyp TS).
depwf (cctx_typ TS T) :- foreach object.wftyp TS, object.wftyp T.
\end{verbatim}

\heroSTUDENT{} That makes a lot of sense. I see you are also checking
well-formedness for the types that the context introduces; and
\texttt{foreach} is exactly like \texttt{map}, but there's no output, so
it applies a single-argument predicate to each element of the list.

\heroADVISOR{} Right. We now get to the tricky part: referring to variables
that stand for open terms within other terms! You know what those are,
right? Those are Object-level Object-level Meta-variables.

\heroSTUDENT{} My head hurts, I'm getting
\href{https://en.wikipedia.org/wiki/Out_of_memory}{OOM} errors. Maybe
this is easier to implement in Makam than to talk about.

\heroADVISOR{} Might be so. Well, let me just say this: those variables will
stand for open terms that depend on a specific context \(\Phi\), but we
might use them at a different context \(\Phi'\). We need a
\emph{substitution} \(\sigma\) to go from the context they were defined,
to the current context.

\heroSTUDENT{} OK, and then we need to apply that substitution \(\sigma\) when
we substitute an actual open term for the metavariable. I know what to
do:

\begin{verbatim}
%extend object.
varmeta : depvar \ensuremath{\to} list term \ensuremath{\to} term.
typeof (varmeta V ES) T :- depclassify E (cctx_typ TS T), map object.typeof ES TS.
%end.
depsubst_cases Var (iopen_term XS_E) (object.varmeta Var ES) Result true :-
  applymany XS_E ES E', depsubst_aux Var (iopen_term XS_E) E' Result.
\end{verbatim}

\heroADVISOR{} That should be it, let's try this out! Let's do meta-level
application, maybe? So, take a ``function'' body that needs a single
argument, and an instantiation for that argument, and do the
substitution at the meta-level. This will be sort-of like inlining. And
let's use unification variables wherever it makes sense, to push our
rules to infer what they can for themselves!

\begin{verbatim}
typeof (lamdep _ (fun t1 \ensuremath{\Rightarrow} (lamdep _ (fun t2 \ensuremath{\Rightarrow}
       (lamdep (cctx_typ [object.vartyp t1] (object.vartyp t2)) (fun f \ensuremath{\Rightarrow}
       (lamdep _ (fun a \ensuremath{\Rightarrow} (liftdep (iopen_term (bindbase (
         (object.varmeta f [object.varterm a]))))))))))))) T ?
>> Yes:
>> T := (pidep (cext (fun t1 \ensuremath{\Rightarrow} pidep (cext (fun t2 \ensuremath{\Rightarrow}
>>      (pidep (cctx_typ [object.vartyp t1] (object.vartyp t2)) (fun f \ensuremath{\Rightarrow}
>>      (pidep (ctyp (object.vartyp t1)) (fun a \ensuremath{\Rightarrow}
>>      (liftdep (cctx_typ [] (object.vartyp t2))))))))))))
\end{verbatim}

\heroSTUDENT{} That's it! That's it! I cannot believe how easy this was!

\heroAUDIENCE{} Neither can we believe that you thought this was easy!

\heroAUTHOR{} Trust me, you should have seen how many weeks it took me to
implement something like this in OCaml\ldots{}. it was enough to make me
start working on Makam. That took two years, but now we can at least
show it in 24 pages of a single-column PDF!

\heroADVISOR{} Where are all these voices coming from?

\heroSTUDENT{}
\textit{(Joke elided to avoid issues with double-blind submission.)}

  
  \section{Hindley-Milner Polymorphism}
  \section{Where our heroes implement type generalization, tying loose
ends}\label{where-our-heroes-implement-type-generalization-tying-loose-ends}

\begin{verse}
``We promised we'll do Hindley-Milner, we don't want you to be sad. \\
This paper is coming to an end soon, and it wasn't all that bad. \\
\hspace{1em}\vspace{-0.5em} \\
We'll gather all unification variables, using structural recursion. \\
And if you haven't guessed it yet, we'll use some term reflection.''
\end{verse}

\heroSTUDENT{} I have an idea for implementing type generalization for
polymorphic \texttt{let} in the style of
\citet{damas1984type,hindley1969principal,milner1978theory}. I remember
the typing rule looks like this:

\vspace{-1.2em}\begin{mathpar}
\inferrule{\Gamma \vdash e : \tau \\ \vec{a} = \text{fv}(\tau) - \text{fv}(\Gamma) \\ \Gamma, x : \forall \vec{a}.\tau \vdash e' : \tau'}{\Gamma \vdash \text{let} \; x = e \; \text{in} \; e' : \tau'}
\end{mathpar}

\heroADVISOR{} Right, and we don't have any side-effectful operations, so, no
need for a value restriction. Let's assume a predicate for generalizing
the type, for now; the rest of the rule is easy:

\begin{verbatim}
generalize : typ \ensuremath{\to} typ \ensuremath{\to} prop.
let : term \ensuremath{\to} (term \ensuremath{\to} term) \ensuremath{\to} term.
typeof (let E F) T' :-
  typeof E T, generalize T Tgen, (x:term \ensuremath{\to} typeof x Tgen \ensuremath{\to} typeof (F x) T').
\end{verbatim}

\heroSTUDENT{} Right, so for generalization, based on the typing rule, we need
the following ingredients:

\begin{itemize}
\tightlist
\item
  something that picks out free variables from a type -- or, in our
  setting, uninstantiated unification variables
\item
  something that picks out free variables from the local context
\item
  a way to turn something that includes unification variables into a
  \texttt{forall} type
\end{itemize}

\noindent
Those look like things that we should be able to do with our generic
recursion and with the reflective predicates we've been using!

\heroADVISOR{} Indeed! So, I've done this before, and I need to leave for home
soon, so bear with me for a bit. There's this generic operation in the
Makam standard library, called \texttt{generic.fold}. It is quite
similar to \texttt{structural\_recursion}, but it does a fold through a
term, carrying an accumulator through. Pretty standard, really, and its
code is similar to what we did already. I'll use it to define a
predicate that returns \emph{one} unification variable of the right type
from a term, if at least one exists.

\begin{verbatim}
findunif : [A B] option B \ensuremath{\to} A \ensuremath{\to} option B \ensuremath{\to} prop.
findunif (some X) _ (some X).
findunif none (X : B) (some (X : B)) :- refl.isunif X.
findunif In X Out :- generic.fold findunif In X Out.
findunif : [A B] A \ensuremath{\to} B \ensuremath{\to} prop.  findunif T X :- findunif none T (some X).
\end{verbatim}

\heroSTUDENT{} Oh, the second rule is the important one -- it will only match
when we encounter a unification variable of the same type as the one we
require, thanks to type specialization.

\heroADVISOR{} Exactly. Now we add a predicate that, given a specific
unification variable and a specific term, replaces its occurrences with
the term. I'll show you later why this operation is necessary. Here I'll
need another reflective predicate, \texttt{refl.sameunif}, that succeeds
when its two arguments are the same exact unification variable;
\texttt{eq} would just unify them, which is not what we want.

\begin{verbatim}
replaceunif : [A B] A \ensuremath{\to} A \ensuremath{\to} B \ensuremath{\to} B \ensuremath{\to} prop.
replaceunif Which ToWhat Where Result :- refl.isunif Where,
  if (refl.sameunif Which Where) then (eq (dyn Result) (dyn ToWhat))
  else (eq Result Where).
replaceunif Which ToWhat Where Result :- not(refl.isunif Where),
  structural_recursion (replaceunif Which ToWhat) Where Result.
\end{verbatim}

\heroADVISOR{} And last, we'll need an auxiliary predicate that tells us
whether a unification variable exists within a term. You can do that
yourself; it's similar to the above.

\heroSTUDENT{} Yes, I think I know how to do that.

\begin{verbatim}
hasunif : [A B] B \ensuremath{\to} bool \ensuremath{\to} A \ensuremath{\to} bool \ensuremath{\to} prop.
hasunif _ true _ true.
hasunif X false Y true :- refl.sameunif X Y.
hasunif X In Y Out :- generic.fold (hasunif X) In Y Out.
hasunif : [A B] A \ensuremath{\to} B \ensuremath{\to} prop. hasunif Term Var :- hasunif Var false Term true.
\end{verbatim}

\heroADVISOR{} OK, we are now mostly ready to implement \texttt{generalize}.
We'll do this recursively. The base case is when there are no
unification variables within a type left:

\begin{verbatim}
generalize T T :- not(findunif T X).
\end{verbatim}

\heroSTUDENT{} Ah, I see what you are getting at. For the recursive case, we
will pick out the first unification variable that we come upon using
\texttt{findunif}. We will generalize over it using \texttt{replaceunif}
and then proceed to the rest. But don't we have to skip over the
unification variables that are in the \(\Gamma\) environment?

\heroADVISOR{} Well, that's the last hurdle. Let's assume a predicate that
gives us all the types in the environment, and write the recursive case
down:

\begin{verbatim}
get_types_in_environment : [A] A \ensuremath{\to} prop.
generalize T Res :- 
  findunif T Var, get_types_in_environment GammaTypes,
  (x:typ \ensuremath{\to} (replaceunif Var x T (T' x), generalize (T' x) (T'' x))),
  if (hasunif GammaTypes Var) then (eq Res (T'' Var)) else (eq Res (forall T'')).
\end{verbatim}

\heroSTUDENT{} Oh, clever. But what should
\texttt{get\_types\_in\_environment} be? Don't we have to go back and
thread a list of types through our \texttt{typeof} predicate, every time
we introduce a new \texttt{typeof\ x\ T\ \ensuremath{\to}} assumption?

\heroADVISOR{} Well, we came this far without rewriting our rules, so it's a
shame to do that now! Maybe we'll be excused to use yet another
reflective predicate that does what we want? There is a way to get a
list of all the local assumptions for the \texttt{typeof} predicate; it
turns out that all the rules and connectives are normal \foreignlanguage{greek}{λ}Prolog terms
like any other, so there's not really much magic to it. And those
assumptions will include just the types in \(\Gamma\)\ldots{}.

\begin{verbatim}
get_types_in_environment Assumptions :-
  refl.assume_get (typeof : term \ensuremath{\to} typ \ensuremath{\to} prop) Assumptions.
\end{verbatim}

\heroSTUDENT{} Wait. It can't be.

\begin{verbatim}
typeof (let (lam _ (fun x \ensuremath{\Rightarrow} let x (fun y \ensuremath{\Rightarrow} y))) (fun id \ensuremath{\Rightarrow} id)) T ?
>> Yes:
>> T := forall (fun a \ensuremath{\Rightarrow} arrow a a)
\end{verbatim}

\heroADVISOR{} And yet, it can.

}

\section{Conclusion}

\TODO{} We conclude the paper.

\bibliography{main}

\end{document}
