\section{Where our heroes tackle dependencies, contexts, and a new level
of
meta}\label{where-our-heroes-tackle-dependencies-contexts-and-a-new-level-of-meta}

\heroSTUDENT{} I'm fairly confident by now that Makam should be able to handle
the research idea we want to try out. Shall we get to it?

\heroADVISOR{} Yes, it is time. So, what we are aiming to do is add a facility
for type-safe, heterogeneous meta-programming to our object language,
similar to MetaHaskell \citep{mainland2012explicitly}. This way we can
manipulate the terms of a separate object language in a type-safe
manner.

\heroSTUDENT{} Exactly. We'd like our object language to be a formal logic, so
our language will be similar to Beluga \citep{beluga-main-reference} or
VeriML \citep{stampoulis2013veriml}. We'll have to be able to pattern
match over the terms of the object language, too, so they are runtime
entities\ldots{}. But we don't need to do all of that; let's just do a
basic version for now, and I can do the rest on my own.

\heroADVISOR{} Sounds good. So, I think the fragment we should do is this: we
will have dependent functions over a distinguished language of
\emph{dependent indices}. We need the dependency so that, for example,
we can take an object-level type as an argument and return an
object-level term that uses that type.

\heroSTUDENT{} Exactly. Dependent products should be similar, but we can skip
them for now and just add a way to return an object-level term from the
meta-level.

\heroADVISOR{} Good idea. We are getting into many levels of meta -- there's
the meta-language we're using, Makam; there's the object language we are
encoding, which is a meta-language in itself, let's call that
Heterogeneous Meta ML Light (HMML?); and there's the ``object-object''
language that HMML is manipulating. And let's keep that last one simple:
the simply typed lambda calculus (STLC).

\heroSTUDENT{} Great. So, our dependent indices will be the types and terms of
STLC -- actually, the open terms of STLC.

\heroADVISOR{} It's a plan. So, let's get to it. Let's first add distinguished
sorts for dependent indices and dependent classifiers -- we'll use those
to type-check the indices, with an appropriate predicate. Let's also
have a distinguished type for \emph{dependent variables}, that is,
variables of dependent indices; and a way to substitute such a variable
for an object.

\begin{verbatim}
depindex, depclassifier, depvar : type.
depclassify : depindex \ensuremath{\to} depclassifier \ensuremath{\to} prop.
depclassify : depvar \ensuremath{\to} depclassifier \ensuremath{\to} prop.
depwf : depclassifier \ensuremath{\to} prop.
depsubst : [A] (depvar \ensuremath{\to} A) \ensuremath{\to} depindex \ensuremath{\to} A \ensuremath{\to} prop.
\end{verbatim}

\newcommand\dep[1]{\ensuremath{#1_{\text{d}}}}
\newcommand\lift[1]{\ensuremath{\langle#1\rangle}}

\heroSTUDENT{} Right, we might need to check that classifiers are well-formed.
And we might need to treat variables specially, so it's good that
they're a different type. So, that's why you made substitution a
predicate, rather than using the normal HOAS function application
\texttt{F\ X} directly, as we have been doing so far. I know that when
we add variables that stand for open STLC terms, there will be some
extra computation involved to substitute them for an open term, so the
normal application won't work as is.

\heroADVISOR{} Exactly; and that extra computation will be necessary in order
to maintain type-safety. Hopefully, we won't have to write any
unnecessary cases, though! Now, we have a few typing rules to add. I'll
use ``\(\dep{\cdot}\)'' to signify things that have to do with the
dependent indices.

\vspace{-1.5em}\begin{mathpar}
\inferrule{\dep{\Psi} \dep{\vdash} \dep{i} : \dep{c}}
          {\Gamma; \dep{\Psi} \vdash \lift{\dep{i}} : \lift{\dep{c}}}

\inferrule{\Gamma; \dep{\Psi}, \; \dep{v} : \dep{c} \vdash e : \tau \\ \dep{\Psi} \dep{\vdash} \dep{c} \; \text{wf}}
          {\Gamma; \dep{\Psi} \vdash \Lambda \dep{v} : \dep{c}.e : \Pi \dep{v} : \dep{c}.\tau}

\inferrule{\Gamma; \dep{\Psi} \vdash e : \Pi \dep{v} : \dep{c}.\tau \\ \dep{\Psi} \dep{\vdash} \dep{i} : \dep{c}}
          {\Gamma; \dep{\Psi} \vdash e @ \dep{i} : \dep{\text{subst}}(\tau, [\dep{i}/\dep{v}])}
\end{mathpar}

\heroSTUDENT{} Those are very easy to transcribe to Makam.

\begin{verbatim}
lamdep : depclassifier \ensuremath{\to} (depvar \ensuremath{\to} term) \ensuremath{\to} term.
appdep : term \ensuremath{\to} depindex \ensuremath{\to} term.
liftdep : depindex \ensuremath{\to} term. liftdep : depclassifier \ensuremath{\to} typ.
pidep : depclassifier \ensuremath{\to} (depvar \ensuremath{\to} typ) \ensuremath{\to} typ.
typeof (lamdep C EF) (pidep C TF) :-
  (v:depvar \ensuremath{\to} depclassify v C \ensuremath{\to} typeof (EF v) (TF v)), depwf C.
typeof (appdep E I) T' :- typeof E (pidep C TF), depclassify I C, depsubst TF I T'.
typeof (liftdep I) (liftdep C) :- depclassify I C.
\end{verbatim}

\heroADVISOR{} Looks nice. Just wanted to say, this framework is quite
general. We could instantiate dependent indices with a language of
natural numbers, equality predicates, and equality proofs, which would
be quite similar to the Dependent ML formulation of
\citet{licata2005formulation}. But let's go back to what we're trying to
do. I'll add the object language in a separate namespace prefix -- we
can use `\texttt{\%extend}' for going into a namespace -- and I'll just
copy-paste our STLC code from earlier on.

\begin{verbatim}
%extend object.
term : type. typ : type. typeof : term \ensuremath{\to} typ \ensuremath{\to} prop.
...
%end.
\end{verbatim}

\heroSTUDENT{} Great! I'll make these into dependent indices now, including
both types and terms.

\begin{verbatim}
iterm : object.term \ensuremath{\to} depindex.     ityp : object.typ \ensuremath{\to} depindex.
ctyp : object.typ \ensuremath{\to} depclassifier.  cext : depclassifier.
depclassify (iterm E) (ctyp T) :- object.typeof E T.
depclassify (ityp T) cext :- object.wftyp T.
depwf (ctyp T) :- object.wftyp T.
depwf cext.
\end{verbatim}

\heroADVISOR{} Right, we'll need to check that types are well-formed, too.
Right now, they are all well-formed by construction, but let's prepare
for any additions, by setting up a structurally recursive predicate. The
\texttt{wftyp\_cases} predicate will hold the important type-checking
cases, and we will have an extra predicate to say whether those cases
apply or not for a specific \texttt{typ}.

\begin{verbatim}
%extend object.
wftyp : typ \ensuremath{\to} prop. wftyp_aux : [A] A \ensuremath{\to} A \ensuremath{\to} prop.
wftyp_cases, wftyp_applies : [A] A \ensuremath{\to} prop.
wftyp T :- wftyp_aux T T.
wftyp_aux T T :- if (wftyp_applies T)
                 then (wftyp_cases T)
                 else (structural_recursion wftyp_aux T T).
%end.
\end{verbatim}

\heroSTUDENT{} I see -- if a type-checking rule applies, but fails, we don't
want to proceed to also try structural recursion; it would defeat the
purpose. Neat trick. I also see that your structural recursion just
needs to do a simple visit and it does not need to produce an output;
hence the repeat of the same \texttt{typ} argument. Let's prepare for
substitutions, too, in the same way.

\begin{verbatim}
depsubst_aux, depsubst_cases : [A] depvar \ensuremath{\to} depindex \ensuremath{\to} A \ensuremath{\to} A \ensuremath{\to} prop.
depsubst_applies : [A] depvar \ensuremath{\to} A \ensuremath{\to} prop.
depsubst F I Res :- (v:depvar \ensuremath{\to} depsubst_aux v I (F v) Res).
depsubst_aux Var Replace Where Res :-
  if (depsubst_applies Var Where)
  then (depsubst_cases Var Replace Where Res)
  else (structural_recursion (depsubst_aux Var Replace) Where Res).
\end{verbatim}

\heroADVISOR{} Great! We only have one thing missing: we need to close the
loop, being able to refer to a dependent variable from within an
object-level term or type.

\heroSTUDENT{} I got this.

\begin{verbatim}
%extend object.
varterm : depvar \ensuremath{\to} term.  vartyp : depvar \ensuremath{\to} typ.
typeof (varterm V) T :- depclassify V (ctyp T).
wftyp_applies (vartyp V). wftyp_cases (vartyp V) :- depclassify V cext.
%end.
depsubst_applies Var (object.varterm Var).
depsubst_cases Var (iterm Replace) (object.varterm Var) Replace.
depsubst_applies Var (object.vartyp Var).
depsubst_cases Var (ityp Replace)  (object.vartyp Var)  Replace.
\end{verbatim}

\heroADVISOR{} This is exciting; let me try it out! I'll do a function that
takes an object-level type and returns the object-level identity
function for it.

\begin{verbatim}
typeof (lamdep cext (fun t \ensuremath{\Rightarrow}
         (liftdep (iterm (object.lam (object.vartyp t) (fun x \ensuremath{\Rightarrow} x)))))) T ?
>> Yes!!!!!
>> T := pidep cext (fun t \ensuremath{\Rightarrow}
>>        liftdep (ctyp (object.arrow (object.vartyp t) (object.vartyp t))))
\end{verbatim}

\heroSTUDENT{} Look, even the Makam REPL is excited!

\heroADVISOR{} Wait until it sees what we have in store for it next: open STLC
terms in our indices!

\heroSTUDENT{} Good thing I've printed out the contextual types paper by
\citet{nanevski2008contextual}. (\ldots{}) OK, so it says here that we
can use contextual types to record, at the type level, the context that
open terms depend on. So let's say, an open \texttt{object.term} of type
\(\tau\) that mentions variables of a \(\Phi\) context would have a
contextual type of the form \([\Phi] \tau\). This is some sort of modal
typing, with a precise context.

\heroADVISOR{} Right. So in our case, open STLC terms depend on a number of
variables, and we will need to keep track of the STLC types of those
variables, in order to maintain type safety. So, let's add a new
dependent index for open STLC terms, and a dependent classifier for
their contextual types, which record the types of the variables that the
term depends on, as well as the actual type of the term itself.

\heroSTUDENT{} Let me see. I think something like this is what we want:

\begin{verbatim}
iopen_term : bindmany object.term object.term \ensuremath{\to} depindex.
cctx_typ : list object.typ \ensuremath{\to} object.typ \ensuremath{\to} depclassifier.
\end{verbatim}

\heroADVISOR{} That looks right to me. I can write the classification and
well-formedness rules for those.

\begin{verbatim}
depclassify (iopen_term XS_E) (cctx_typ TS T) :-
  openmany XS_E (pfun xs e \ensuremath{\Rightarrow}
    assumemany object.typeof xs TS (object.typeof e T),
    foreach object.wftyp TS).
depwf (cctx_typ TS T) :- foreach object.wftyp TS, object.wftyp T.
\end{verbatim}

\heroSTUDENT{} That makes a lot of sense. I see you are also checking
well-formedness for the types that the context introduces; and
\texttt{foreach} is exactly like \texttt{map}, but there's no output, so
it applies a single-argument predicate to each element of the list.

\heroADVISOR{} Right. We now get to the tricky part: referring to variables
that stand for open terms within other terms! You know what those are,
right? Those are Object-level Object-level Meta-variables.

\heroSTUDENT{} My head hurts; I'm getting
\href{https://en.wikipedia.org/wiki/Out_of_memory}{OOM} errors. Maybe
this is easier to implement in Makam than to talk about.

\heroADVISOR{} Maybe so. Well, let me just say this: those variables will
stand for open terms that depend on a specific context \(\Phi\), but we
might use them at a different context \(\Phi'\). We need a
\emph{substitution} \(\sigma\) to go from the context they were defined
into the current context.

\heroSTUDENT{} OK, and then we need to apply that substitution \(\sigma\) when
we substitute an actual open term for the metavariable. I know what to
do:

\vspace{-0.5em}

\begin{verbatim}
%extend object.
varmeta : depvar \ensuremath{\to} list term \ensuremath{\to} term.
typeof (varmeta V ES) T :- depclassify V (cctx_typ TS T), map object.typeof ES TS.
%end.
depsubst_applies Var (object.varmeta Var _).
depsubst_cases Var (iopen_term XS_E) (object.varmeta Var ES) Result :-
  applymany XS_E ES E', depsubst_aux Var (iopen_term XS_E) E' Result.
\end{verbatim}

\heroADVISOR{} That should be it; let's try this out! Let's do meta-level
application, maybe? So, take a ``function'' body that needs a single
argument, and an instantiation for that argument, and do the
substitution at the meta-level. This will be sort of like inlining. And
let's use unification variables wherever it makes sense, to push our
rules to infer what they can for themselves!

\begin{verbatim}
typeof (lamdep _ (fun t1 \ensuremath{\Rightarrow} (lamdep _ (fun t2 \ensuremath{\Rightarrow}
       (lamdep (cctx_typ [object.vartyp t1] (object.vartyp t2)) (fun f \ensuremath{\Rightarrow}
       (lamdep _ (fun a \ensuremath{\Rightarrow} (liftdep (iopen_term (bindbase (
         (object.varmeta f [object.varterm a]))))))))))))) T ?
>> Yes:
>> T := (pidep cext (fun t1 \ensuremath{\Rightarrow} pidep cext (fun t2 \ensuremath{\Rightarrow}
>>      (pidep (cctx_typ [object.vartyp t1] (object.vartyp t2)) (fun f \ensuremath{\Rightarrow}
>>      (pidep (ctyp (object.vartyp t1)) (fun a \ensuremath{\Rightarrow}
>>      (liftdep (cctx_typ [] (object.vartyp t2)))))))))).
\end{verbatim}

\begin{scenecomment}
(Our heroes try out a few more examples to convince themselves that this works.)
\end{scenecomment}

\heroSTUDENT{} That's it! That's it! I cannot believe how easy this was!

\heroAUDIENCE{} Neither can we believe that you thought this was easy!

\heroAUTHOR{} Trust me, you should have seen how many weeks it took me to
implement something like this in OCaml\ldots{}. it was enough to make me
start working on Makam. That took two years, but now we can at least
show it in 24 pages of a single-column PDF!

\heroADVISOR{} Where are all these voices coming from?

\heroSTUDENT{}
\textit{(Joke elided to avoid issues with double-blind submission.)}
