\section{Where our heroes set out on a road to prototype a type
system}\label{where-our-heroes-set-out-on-a-road-to-prototype-a-type-system}

\hero{HAGOP (Student)} (\ldots{}) So yes, I think my next step should be
writing a toy implementation of the type system we have in mind, so that
we can try out some examples and see what works and what does not.

\hero{ROZA (Advisor)} Definitely -- trying out examples will help you
refine your ideas, too.

\heroSTUDENT{} Let's see, though; we have an ML core, dependently typed
constructs, and contextual types like in
\citet{nanevski2008contextual}\ldots{} I guess I will need a few months?

\heroADVISOR{} That sounds like a lot. Why don't you use some kind of
metalanguage to implement the prototype?

\heroSTUDENT{} You mean a tool like Racket/PLT Redex
\citep{tobin2011languages,felleisen2009semantics}, the K Framework
\citep{rosu2010overview,ellison2009rewriting}, Spoofax
\citep{kats2010spoofax}, or CRSX \citep{rose2011crsx}?

\heroADVISOR{} Yes, all of those should be good choices. I was thinking though
that we could use higher-order logic programming\ldots{} it's a
formalism that is well-suited to what we want to do, since we will need
all sorts of different binding constructs, and the type system we are
thinking about is quite advanced.

\heroSTUDENT{} Oh, so you mean \foreignlanguage{greek}{λ}Prolog \citep{miller1988overview} or LF
\citep{pfenning1999system}.

\heroADVISOR{} Yes. Actually, a few years back a friend of mine worked on this
new implementation of \foreignlanguage{greek}{λ}Prolog just for this purpose -- rapid prototyping
of languages. It's called Makam. It should be able to handle what we
have in mind nicely, and we should not need to spend more than a few
hours on it!

\heroSTUDENT{} Sounds great! Anything I can read up on Makam then?

\heroADVISOR{} Not much, unfortunately\ldots{} But I know the language and its
standard library quite well, so let's work on this together; it'll be
fun. I'll show you how things work along the way!

\begin{scenecomment}
(Our heroes install Makam from --elided for blind reviewing-- and figure out how to run the REPL.)
\end{scenecomment}
