\heroADVISOR{} That's great! I understand that this was a little tricky, but
still, it was not too bad, right? Actually, I know of one thing that is
quite simple to do: the evaluation rule. On paper we typically write
something roughly like:

\vspace{-1.5em}\begin{mathpar}
\inferrule{e_1 \Downarrow v_1 \\ \texttt{match}(p, v_1) \leadsto \sigma \\ e_2[\sigma/xs] \Downarrow v_2}
          {\texttt{case\_or\_else}(e_1, p \mapsto xs.e_2, e_3) \Downarrow v_2}

\inferrule{e_1 \Downarrow v_1 \\ \texttt{match}(p, v_1) \not\leadsto \\ e_3 \Downarrow v_3}
          {\texttt{case\_or\_else}(e_1, p \mapsto xs.e_2, e_3) \Downarrow v_3}
\end{mathpar}

\noindent
So \texttt{match} tries to unify a pattern with a term and yields a
substitution \(\sigma\) for the pattern variables if successful, which
is then applied to the body of the branch. If there is no \texttt{match}
to be found, then we use the \texttt{else} branch.

\heroSTUDENT{} Hmm\ldots{} so do we need two predicates, one for the case
where the \texttt{match} is successful and one to check that a pattern
\emph{does not} match a scrutinee?

\heroADVISOR{} Actually we could have a single \texttt{match} predicate. And
we can use the logical \texttt{if-then-else} construct for the two
cases, which we have not seen so far. Let me write down the evaluation
rule, and I'll explain:

\begin{verbatim}
\tokpropconst{match} : [\tokmetavariable{NBefore} \tokmetavariable{NAfter}]
  (\tokmetavariable{Pattern}: \toktypeid{patt} \tokmetavariable{NBefore} \tokmetavariable{NAfter}) (\tokmetavariable{Scrutinee}: \toktypeid{term})
  (\tokmetavariable{SubstBefore}: \tokstdtypeid{vector} \toktypeid{term} \tokmetavariable{NBefore}) (\tokmetavariable{SubstAfter}: \tokstdtypeid{vector} \toktypeid{term} \tokmetavariable{NAfter}) \tokarrowtype{\ensuremath{\to}}
  \tokbuiltintype{prop}.
\tokpropconst{eval} (\tokobjconst{case_or_else} \tokmetavariable{Scrutinee} \tokmetavariable{Pattern} \tokmetavariable{Body} \tokmetavariable{Else}) \tokmetavariable{V'} \toksymbol{\ensuremath{:\!-}}
  \tokpropconst{eval} \tokmetavariable{Scrutinee} \tokmetavariable{V},
  \tokkeyword{if} (\tokpropconst{match} \tokmetavariable{Pattern} \tokmetavariable{V} \tokstdconst{vnil} \tokmetavariable{Subst})
  \tokkeyword{then} (\tokstdconst{vapplymany} \tokmetavariable{Body} \tokmetavariable{Subst} \tokmetavariable{Body'}, \tokpropconst{eval} \tokmetavariable{Body'} \tokmetavariable{V'})
  \tokkeyword{else} (\tokpropconst{eval} \tokmetavariable{Else} \tokmetavariable{V'}).
\end{verbatim}

\noindent
The \texttt{if-then-else} construct behaves as follows: when there is at
least one way to prove the condition, it proceeds to the \texttt{then}
branch, otherwise it goes to the \texttt{else} branch. Pretty standard,
really. It is one thing that the Prolog cut statement, \texttt{!}, is
useful for, but I find that using cut makes for less readable code.
\citet{kiselyov05backtracking} is worth reading for alternatives to the
cut statement and the semantics of
\texttt{if}-\texttt{then}-\texttt{else} and \texttt{not} in logic
programming, and Makam follows that paper closely.

\heroSTUDENT{} I see. Now, I noticed a \texttt{vapplymany} predicate -- what
is that?

\heroADVISOR{} That is a standard-library predicate. It is used to perform
simultaneous substitution for all the variables in our multiple binding
type, \texttt{vbindmany} (also available as \texttt{appmany} for
\texttt{bindmany}). Or another way to say it, it's the equivalent of
HOAS function application for \texttt{vbindmany}:

\begin{verbatim}
\tokstdconst{vapplymany} : [\tokmetavariable{N}] \tokstdtypeid{vbindmany} \tokmetavariable{Var} \tokmetavariable{N} \tokmetavariable{Body} \tokarrowtype{\ensuremath{\to}} \tokstdtypeid{vector} \tokmetavariable{Var} \tokmetavariable{N} \tokarrowtype{\ensuremath{\to}} \tokmetavariable{Body} \tokarrowtype{\ensuremath{\to}} \tokbuiltintype{prop}.
\tokstdconst{vapplymany} (\tokstdconst{vbody} \tokmetavariable{Body}) \tokstdconst{vnil} \tokmetavariable{Body}.
\tokstdconst{vapplymany} (\tokstdconst{vbind} \tokmetavariable{F}) (\tokstdconst{vcons} \tokmetavariable{E} \tokmetavariable{ES}) \tokmetavariable{Body} \toksymbol{\ensuremath{:\!-}} \tokstdconst{vapplymany} (\tokmetavariable{F} \tokmetavariable{E}) \tokmetavariable{ES} \tokmetavariable{Body}.
\end{verbatim}

\heroSTUDENT{} I see\ldots{} OK, I think I know how to continue. I will write
a few of the \texttt{match} rules down.

\begin{verbatim}
\tokpropconst{match} \tokobjconst{patt_var} \tokmetavariable{X} \tokmetavariable{Subst} \tokmetavariable{Subst'} \toksymbol{\ensuremath{:\!-}} \tokstdconst{vsnoc} \tokmetavariable{Subst} \tokmetavariable{X} \tokmetavariable{Subst'}.
\tokpropconst{match} \tokobjconst{patt_wild} \tokmetavariable{X} \tokmetavariable{Subst} \tokmetavariable{Subst}.
\tokpropconst{match} \tokobjconst{patt_ozero} \tokobjconst{ozero} \tokmetavariable{Subst} \tokmetavariable{Subst}.
\tokpropconst{match} (\tokobjconst{patt_osucc} \tokmetavariable{P}) (\tokobjconst{osucc} \tokmetavariable{V}) \tokmetavariable{Subst} \tokmetavariable{Subst'} \toksymbol{\ensuremath{:\!-}}
  \tokpropconst{match} \tokmetavariable{P} \tokmetavariable{V} \tokmetavariable{Subst} \tokmetavariable{Subst'}.
\end{verbatim}

\begin{scenecomment}
(Our heroes also write down the rules for multiple patterns and tuples, which are
available in the unabridged version of this story.)
\end{scenecomment}

\heroSTUDENT{} Let's try this out with a simple example -- how about
predecessor for natural numbers?

\begin{verbatim}
\tokpropconst{eval} (\tokobjconst{case_or_else} (\tokobjconst{osucc} (\tokobjconst{osucc} \tokobjconst{ozero}))
     (\tokobjconst{patt_osucc} \tokobjconst{patt_var}) (\tokstdconst{vbind} (\tokkeyword{fun} \tokconst{pred} \tokkeyword{\ensuremath{\Rightarrow}} \tokstdconst{vbody} \tokconst{pred}))
     \tokobjconst{ozero})
  \tokmetavariable{V} ?
\tokquery{>>} \tokquery{Yes:}
\tokquery{>>} \tokmetavariable{V} \toksymbol{:=} \tokobjconst{osucc} \tokobjconst{ozero}.
\end{verbatim}
