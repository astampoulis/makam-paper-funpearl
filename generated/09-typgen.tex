\section{Where our hero Roza implements type generalization, tying loose
ends}\label{where-our-hero-roza-implements-type-generalization-tying-loose-ends}

\begin{verse}
``We mentioned Hindley-Milner / we don't want you to be sad. \\
This paper's going to end soon / and it wasn't all that bad. \\
\hspace{1em}\vspace{-0.5em} \\
(Before we get to that, though / it's time to take a break. \\
If taksims seem monotonous / then put on some Nick Drake.) \\
\hspace{1em}\vspace{-0.5em} \\
We'll gather unif-variables / with structural recursion \\
and if you haven't guessed it yet / we'll get to use reflection.''
\end{verse}

\heroADVISOR{} Let me now show you how to implement type generalization for
polymorphic \texttt{let} in the style of
\citet{damas1984type,hindley1969principal,milner1978theory}. I've done
this
before\footnote{Existing work that has considered the problem of ML type generalization
in the \lamprolog setting can be found in \citet{typgen-lamprolog-1} and \citet{typgen-lamprolog-2}.},
and I need to leave for home soon, so bear with me for a bit. The gist
will be to reuse the unification support of our metalanguage, capturing
the \emph{metalevel unification variables} and generalizing over them.
That way we will have a very short implementation, and we won't have to
do a deep embedding of unification!

\heroSTUDENT{} So -- you're saying that in \lamprolog, other than reusing the
metalevel function type for implementing object-level substitution, we
can also reuse metalevel unification for the object level as well.

\identNormal

\heroADVISOR{} Exactly! First of all, the typing rule for a generalizing let
looks like this:

\vspace{-1.2em}\begin{mathpar}
\inferrule{\Gamma \vdash e : \tau \\ \vec{a} = \text{fv}(\tau) - \text{fv}(\Gamma) \\ \Gamma, x : \forall \vec{a}.\tau \vdash e' : \tau'}{\Gamma \vdash \text{let} \; x = e \; \text{in} \; e' : \tau'}
\end{mathpar}

We don't have any side-effectful operations, so there is no need for a
value restriction. Transcribing this to Makam is easy, if we assume a
predicate for generalizing the type, for now:

\begin{verbatim}
generalize : (Type: typ) (GeneralizedType: typ) \ensuremath{\to} prop.
let : term \ensuremath{\to} (term \ensuremath{\to} term) \ensuremath{\to} term.
typeof (let E F) T' :-
  typeof E T, generalize T Tgen, (x:term \ensuremath{\to} typeof x Tgen \ensuremath{\to} typeof (F x) T').
\end{verbatim}

Now, for generalization itself, we need the following ingredients based
on the typing rule:

\begin{itemize}
\tightlist
\item
  something that picks out free variables from a type, standing for the
  \(\text{fv}(\tau)\) part -- or, in our setting, this should really be
  read as uninstantiated unification variables. Those are the
  Makam-level unification variables that have not been forced to unify
  with concrete types because of the rest of the typing rules.
\item
  something that picks out free variables from the local context: the
  \(\text{fv}(\Gamma)\) part. Again, these are the uninstantiated
  unification variables rather than the free variables. In our case, the
  context \(\Gamma\) is represented by the local \texttt{typeof}
  assumptions that our typing rules add, so we'll have to look at those
  somehow.
\item
  a way to turn something that includes unification variables into a
  \(\forall\) type, corresponding to the \(\forall \vec{a}.\tau\) part.
  This essentially abstracts over a number of variables and uses them as
  the replacement for the ones inside \(\tau\).
\end{itemize}

All of those look like things that we should be able to do with our
generic recursion and with the reflective predicates we've been using!
However, to make the implementation simpler, we will generalize over one
variable at a time, instead of all at once -- but that should be
entirely equivalent to what's described in the typing rule.

First, we will define a \texttt{findunif} predicate that returns
\emph{one} unification variable \emph{of the right type} from a term, if
at least one such variable exists. To implement it, we will make use of
a generic operation in the Makam standard library, called
\texttt{generic\_fold}. It is quite similar to
\texttt{structural\_recursion}, but it does a fold through a term,
carrying an accumulator through. Pretty standard, really, and its code
is similar to what we did already for \texttt{structural\_recursion},
with no new surprises.

\begin{verbatim}
findunif_aux : [Any VarType]
  (Var: option VarType) (Current: Any) (Var': option VarType) \ensuremath{\to} prop.
findunif_aux (some Var) _ (some Var).
findunif_aux none (Current : CurrentType) (Result: option VarType) :-
  refl.isunif Current,
  if (dyn.eq Result (some Current)) then success
  else (eq Result none).
findunif_aux (In: option B) Current Out :-
  generic_fold @findunif_aux In Current Out.
findunif : [Any VarType] (Search: Any) (Found: VarType) \ensuremath{\to} prop.
findunif Search Found :- findunif_aux none Search (some Found).
\end{verbatim}

Here, the second rule of \texttt{findunif\_aux} is the important one --
it will only succeed when we encounter a unification variable of the
\emph{same type} \texttt{VarType} as the one we require. This rule
relies on the dynamic \texttt{typecase} aspect of the ad-hoc
polymorphism in \lamprolog, making use of the \texttt{dyn.eq}
standard-library predicate, which has a lax typing:

\begin{verbatim}
dyn.eq : [A B] A \ensuremath{\to} B \ensuremath{\to} prop.
dyn.eq X X.
\end{verbatim}

The difference between this and \texttt{eq} is that the types \texttt{A}
and \texttt{B} are only unified at runtime, rather than statically too.
Otherwise, our rule would only apply when the type \texttt{CurrentType}
of the current unification variable we are visiting already matches the
type that we are searching for, \texttt{VarType}.

With this predicate, we should already be able to find \emph{one} (as
opposed to all, as described above) uninstantiated unification variable
from a type. Here is an example of its use:

\begin{verbatim}
findunif (arrowmany TS T) (X: typ) ?
>> Yes:
>> X := T.
\end{verbatim}

Now we add a predicate \texttt{replaceunif} that, given a specific
unification variable and a specific term, replaces its occurrences with
the term. This will be needed as part of the \(\forall \vec{a}.\tau\)
operation of the rule. Here I'll need another reflective predicate,
\texttt{refl.sameunif}, that succeeds when its two arguments are the
same exact unification variable; \texttt{eq} would just unify them,
which is not what we want.

\begin{verbatim}
replaceunif : [VarType Any]
  (Which: VarType) (ToWhat: VarType) (Where: Any) (Result: Any) \ensuremath{\to} prop.
replaceunif Which ToWhat Where ToWhat :-
  refl.isunif Where, refl.sameunif Which Where.
replaceunif Which ToWhat Where Where :-
  refl.isunif Where, not(refl.sameunif Which Where).
replaceunif Which ToWhat Where Result :- not(refl.isunif Where),
  structural_recursion @(replaceunif Which ToWhat) Where Result.
\end{verbatim}

Last, we'll need an auxiliary predicate that tells us whether a
unification variable exists within a term. This is easy; it's similar to
the above.

\begin{verbatim}
hasunif : [VarType Any] VarType \ensuremath{\to} bool \ensuremath{\to} Any \ensuremath{\to} bool \ensuremath{\to} prop.
hasunif _ true _ true.
hasunif X false Y true :- refl.sameunif X Y.
hasunif X In Y Out :- generic_fold @(hasunif X) In Y Out.

hasunif : [VarType Any] VarType \ensuremath{\to} Any \ensuremath{\to} prop.
hasunif Var Term :- hasunif Var false Term true.
\end{verbatim}

We are now mostly ready to implement \texttt{generalize}. We'll do this
recursively. The base case is when there are no unification variables
within a type left:

\begin{verbatim}
generalize T T :- not(findunif T X).
\end{verbatim}

For the recursive case, we will pick out the first unification variable
that we come upon using \texttt{findunif}. We will generalize over it
using \texttt{replaceunif} and then proceed to the rest. Still, there is
a last hurdle: we have to skip over the unification variables that are
in the \(\Gamma\) environment. For the time being, let's assume a
predicate that gives us all the types in the environment, so we can
write the recursive case down:

\begin{verbatim}
get_types_in_environment : [A] A \ensuremath{\to} prop.
generalize T Res :-
  findunif T Var, get_types_in_environment GammaTypes,
  (x:typ \ensuremath{\to} (replaceunif Var x T (T' x), generalize (T' x) (T'' x))),
  if (hasunif Var GammaTypes)
  then (eq Res (T'' Var))
  else (eq Res (tforall T'')).
\end{verbatim}

\identDialog

\heroSTUDENT{} Oh, clever. But what should
\texttt{get\_types\_in\_environment} be? Don't we have to go back and
thread a list of types through our \texttt{typeof} predicate, every time
we introduce a new \texttt{typeof\ x\ T\ \ensuremath{\to}} assumption?

\heroADVISOR{} Well, we came this far without significantly rewriting our
rules, so it's a shame to do that now! Maybe we'll be excused to use yet
another reflective predicate that does what we want? There is a way to
get a list of all the local assumptions for the \texttt{typeof}
predicate; it turns out that all the rules and connectives are normal
\lamprolog terms like any other, so there's not really much magic to it.
And those assumptions will include just the types in \(\Gamma\)\ldots{}.

\begin{verbatim}
get_types_in_environment Assumptions :-
  refl.assume_get typeof Assumptions.
\end{verbatim}

\heroSTUDENT{} Wait. It can't be.

\begin{verbatim}
typeof (let (lam _ (fun x \ensuremath{\Rightarrow} let x (fun y \ensuremath{\Rightarrow} y))) (fun id \ensuremath{\Rightarrow} id)) T ?
>> Yes:
>> T := tforall (fun a \ensuremath{\Rightarrow} arrow a a).
\end{verbatim}

\heroADVISOR{} And yet, it can.
