\section{In which our heroes get the easy stuff out of the
way}\label{in-which-our-heroes-get-the-easy-stuff-out-of-the-way}

\heroSTUDENT{} OK, let's just start with the simply typed lambda calculus to
see how this works. Let's define just the basics: application, lambda
abstraction, and the arrow type.

\heroADVISOR{} Right. We will first need to define the two meta-types for
these two sorts:

\importantCodeblock{}

\begin{verbatim}
\toktypeid{term} : \tokbuiltintype{type}.
\toktypeid{typ} : \tokbuiltintype{type}.
\end{verbatim}

\importantCodeblockEnd{}

\heroSTUDENT{} Oh, so \texttt{type} is the reserved keyword for the meta-level
kind of types, and we'll use \texttt{typ} for our object-level types?

\heroADVISOR{} Exactly. And let's do the easy constructors first:

\importantCodeblock{}

\begin{verbatim}
\tokobjconst{app} : \toktypeid{term} \tokarrowtype{\ensuremath{\to}} \toktypeid{term} \tokarrowtype{\ensuremath{\to}} \toktypeid{term}.
\tokobjconst{arrow} : \toktypeid{typ} \tokarrowtype{\ensuremath{\to}} \toktypeid{typ} \tokarrowtype{\ensuremath{\to}} \toktypeid{typ}.
\end{verbatim}

\importantCodeblockEnd{}

\heroSTUDENT{} So we add constructors to a type at any point; we do not list
them out when we define it like in Haskell. But how about lambdas? I
have heard that \foreignlanguage{greek}{λ}Prolog supports higher-order abstract syntax
\citep{hoas-standard-reference}, which should make those really easy to
add, too, right?

\heroADVISOR{} Yes, functions at the meta level are parametric, so they
correspond exactly to single-variable binding -- they cannot perform any
computation such as pattern matching on their arguments and thus we do
not have to worry about exotic terms. So this works fine for
Church-style lambdas:

\importantCodeblock{}

\begin{verbatim}
\tokobjconst{lam} : \toktypeid{typ} \tokarrowtype{\ensuremath{\to}} (\toktypeid{term} \tokarrowtype{\ensuremath{\to}} \toktypeid{term}) \tokarrowtype{\ensuremath{\to}} \toktypeid{term}.
\end{verbatim}

\importantCodeblockEnd{}

\heroSTUDENT{} I see. And how about the typing judgment,
\(\Gamma \vdash e : \tau\) ?

\heroADVISOR{} Let's add a \emph{predicate} for that: a new constructor for
the type of \texttt{prop}ositions. It will relate a \texttt{term} \(e\)
to its \texttt{typ}, \(\tau\). Only, no \(\Gamma\), there is an implicit
context of assumptions that will serve the same purpose.

\importantCodeblock{}

\begin{verbatim}
\tokpropconst{typeof} : \toktypeid{term} \tokarrowtype{\ensuremath{\to}} \toktypeid{typ} \tokarrowtype{\ensuremath{\to}} \tokbuiltintype{prop}.
\end{verbatim}

\importantCodeblockEnd{}

\heroSTUDENT{} I see. We now need to give the rules that make up the
predicate, right? Let me see if I can get the typing rule for
application. I know that in Prolog we write the conclusion of a rule
first, and the premises follow the \texttt{\ensuremath{:\!-}} sign. Does something like
this work?

\importantCodeblock{}

\begin{verbatim}
\tokpropconst{typeof} (\tokobjconst{app} \tokmetavariable{E1} \tokmetavariable{E2}) \tokmetavariable{T'} \toksymbol{\ensuremath{:\!-}} \tokpropconst{typeof} \tokmetavariable{E1} (\tokobjconst{arrow} \tokmetavariable{T} \tokmetavariable{T'}), \tokpropconst{typeof} \tokmetavariable{E2} \tokmetavariable{T}.
\end{verbatim}

\importantCodeblockEnd{}

\heroADVISOR{} Yes! That's exactly right. Makam uses capital letters for
unification variables.

\heroSTUDENT{} I will need help with the lambda typing rule, though. What's
the equivalent of extending the context as in \(\Gamma, \; x : \tau\)?

\heroADVISOR{} Simple: we introduce a fresh constructor for terms and a new
typing rule for it:

\importantCodeblock{}

\begin{verbatim}
\tokpropconst{typeof} (\tokobjconst{lam} \tokmetavariable{T1} \tokmetavariable{E}) (\tokobjconst{arrow} \tokmetavariable{T1} \tokmetavariable{T2}) \toksymbol{\ensuremath{:\!-}}
  (\tokconst{x}:\toktypeid{term} \toksymbol{\ensuremath{\to}} \tokpropconst{typeof} \tokconst{x} \tokmetavariable{T1} \toksymbol{\ensuremath{\to}} \tokpropconst{typeof} (\tokmetavariable{E} \tokconst{x}) \tokmetavariable{T2}).
\end{verbatim}

\importantCodeblockEnd{}

\heroSTUDENT{} Hmm, so `\texttt{x:term\ \ensuremath{\to}}' introduces the fresh
constructor standing for the new variable, and
`\texttt{typeof\ x\ T1\ \ensuremath{\to}}' introduces the new assumption?
Oh, and we need to get to the body of the lambda function in order to
type-check it, so that's why you do \texttt{E\ x}.

\heroADVISOR{} Yes. Note that the introductions are locally scoped, so they
are only in effect for the recursive call `\texttt{typeof\ (E\ x)\ T2}'.

\heroSTUDENT{} Makes sense. So do we have a type checker already?

\heroADVISOR{} We do! We can issue \emph{queries} of the \texttt{typeof}
predicate to type-check terms. Observe:

\importantCodeblock{}

\begin{verbatim}
\tokpropconst{typeof} (\tokobjconst{lam} \tokmetavariable{_} (\tokkeyword{fun} \tokconst{x} \tokkeyword{\ensuremath{\Rightarrow}} \tokconst{x})) \tokmetavariable{T} ?
\tokquery{>>} \tokquery{Yes:}
\tokquery{>>} \tokmetavariable{T} \toksymbol{:=} \tokobjconst{arrow} \tokmetavariable{T1} \tokmetavariable{T1}.
\end{verbatim}

\importantCodeblockEnd{}

\heroSTUDENT{} Cool! So \texttt{fun} for metalevel functions, underscores for
unification variables we don't care about, and \texttt{?} for queries.
But wait, last time I implemented unification in my toy STLC
implementation it was easy to make it go into an infinite loop with
\(\lambda x. x x\). Does that work?

\heroADVISOR{} Well, you were missing the occurs-check. \foreignlanguage{greek}{λ}Prolog unification
includes it:

\begin{verbatim}
\tokpropconst{typeof} (\tokobjconst{lam} \tokmetavariable{_} (\tokkeyword{fun} \tokconst{x} \tokkeyword{\ensuremath{\Rightarrow}} \tokobjconst{app} \tokconst{x} \tokconst{x})) \tokmetavariable{T'} ?
\tokquery{>>} \tokquery{Impossible.}
\end{verbatim}

\heroSTUDENT{} Right. So let's see, what else can we do? How about adding
tuples to our language? Can we use something like a polymorphic list?

\heroADVISOR{} Sure, \foreignlanguage{greek}{λ}Prolog has polymorphic types and higher-order
predicates. Here's how lists are defined in the standard library:

\begin{verbatim}
\tokstdtypeid{list} : \tokbuiltintype{type} \tokarrowtype{\ensuremath{\to}} \tokbuiltintype{type}.
\tokstdconst{nil} : \tokstdtypeid{list} \tokmetavariable{A}.
\tokstdconst{cons} : \tokmetavariable{A} \tokarrowtype{\ensuremath{\to}} \tokstdtypeid{list} \tokmetavariable{A} \tokarrowtype{\ensuremath{\to}} \tokstdtypeid{list} \tokmetavariable{A}.

\tokstdconst{map} : (\tokmetavariable{A} \tokarrowtype{\ensuremath{\to}} \tokmetavariable{B} \tokarrowtype{\ensuremath{\to}} \tokbuiltintype{prop}) \tokarrowtype{\ensuremath{\to}} \tokstdtypeid{list} \tokmetavariable{A} \tokarrowtype{\ensuremath{\to}} \tokstdtypeid{list} \tokmetavariable{B} \tokarrowtype{\ensuremath{\to}} \tokbuiltintype{prop}.
\tokstdconst{map} \tokmetavariable{P} \tokstdconst{nil} \tokstdconst{nil}.
\tokstdconst{map} \tokmetavariable{P} (\tokstdconst{cons} \tokmetavariable{X} \tokmetavariable{XS}) (\tokstdconst{cons} \tokmetavariable{Y} \tokmetavariable{YS}) \toksymbol{\ensuremath{:\!-}} \tokmetavariable{P} \tokmetavariable{X} \tokmetavariable{Y}, \tokstdconst{map} \tokmetavariable{P} \tokmetavariable{XS} \tokmetavariable{YS}.
\end{verbatim}

\heroSTUDENT{} Nice! I guess that's why you wanted to go with \foreignlanguage{greek}{λ}Prolog for
doing this instead of LF, since you cannot use polymorphism there?

\heroADVISOR{} Indeed. We will see, once we figure out what our language
should be, one thing we could do is monomorphize our definitions to LF,
and then we could even use Beluga \citep{beluga-main-reference} to do
all of our metatheoretic proofs. Or maybe we could use Abella
\citep{abella-main-reference} directly.

\heroSTUDENT{} Sounds good. So, for tuples, this should work:

\importantCodeblock{}

\begin{verbatim}
\tokobjconst{tuple} : \tokstdtypeid{list} \toktypeid{term} \tokarrowtype{\ensuremath{\to}} \toktypeid{term}.
\tokobjconst{product} : \tokstdtypeid{list} \toktypeid{typ} \tokarrowtype{\ensuremath{\to}} \toktypeid{typ}.
\tokpropconst{typeof} (\tokobjconst{tuple} \tokmetavariable{ES}) (\tokobjconst{product} \tokmetavariable{TS}) \toksymbol{\ensuremath{:\!-}} \tokstdconst{map} \tokpropconst{typeof} \tokmetavariable{ES} \tokmetavariable{TS}.
\end{verbatim}

\importantCodeblockEnd{}

\heroADVISOR{} Yes, and there is syntactic sugar for \texttt{cons} and
\texttt{nil} too:

\begin{verbatim}
\tokpropconst{typeof} (\tokobjconst{lam} \tokmetavariable{_} (\tokkeyword{fun} \tokconst{x} \tokkeyword{\ensuremath{\Rightarrow}} \tokobjconst{lam} \tokmetavariable{_} (\tokkeyword{fun} \tokconst{y} \tokkeyword{\ensuremath{\Rightarrow}} \tokobjconst{tuple} [\tokconst{x}, \tokconst{y}]))) \tokmetavariable{T} ?
\tokquery{>>} \tokquery{Yes:}
\tokquery{>>} \tokmetavariable{T} \toksymbol{:=} \tokobjconst{arrow} \tokmetavariable{T1} (\tokobjconst{arrow} \tokmetavariable{T2} (\tokobjconst{product} [\tokmetavariable{T1}, \tokmetavariable{T2}])).
\end{verbatim}

\heroSTUDENT{} So how about evaluation? Can we write the big-step semantics
too?

\heroADVISOR{} Why not? Let's add a predicate and do the two easy rules:

\importantCodeblock{}

\begin{verbatim}
\tokpropconst{eval} : \toktypeid{term} \tokarrowtype{\ensuremath{\to}} \toktypeid{term} \tokarrowtype{\ensuremath{\to}} \tokbuiltintype{prop}.
\tokpropconst{eval} (\tokobjconst{lam} \tokmetavariable{T} \tokmetavariable{F}) (\tokobjconst{lam} \tokmetavariable{T} \tokmetavariable{F}).
\tokpropconst{eval} (\tokobjconst{tuple} \tokmetavariable{ES}) (\tokobjconst{tuple} \tokmetavariable{VS}) \toksymbol{\ensuremath{:\!-}} \tokstdconst{map} \tokpropconst{eval} \tokmetavariable{ES} \tokmetavariable{VS}.
\end{verbatim}

\importantCodeblockEnd{}

\heroSTUDENT{} OK, let me try my hand at the beta-redex case. I'll just do
call-by-value. I know that when using HOAS, function application is
exactly capture-avoiding substitution, so this should be fine:

\importantCodeblock{}

\begin{verbatim}
\tokpropconst{eval} (\tokobjconst{app} \tokmetavariable{E} \tokmetavariable{E'}) \tokmetavariable{V''} \toksymbol{\ensuremath{:\!-}} \tokpropconst{eval} \tokmetavariable{E} (\tokobjconst{lam} \tokmetavariable{_} \tokmetavariable{F}), \tokpropconst{eval} \tokmetavariable{E'} \tokmetavariable{V'}, \tokpropconst{eval} (\tokmetavariable{F} \tokmetavariable{V'}) \tokmetavariable{V''}.
\end{verbatim}

\importantCodeblockEnd{}

\heroADVISOR{} Exactly! See, I told you this would be easy!
