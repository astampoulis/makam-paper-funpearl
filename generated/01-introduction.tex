\section{In which our heroes set out on a road to prototype a type
system}\label{in-which-our-heroes-set-out-on-a-road-to-prototype-a-type-system}

\hero{HAGOP (Student)} (\ldots{}) So yes, I think my next step should be
writing a toy implementation of the type system we have in mind, so that
we can try out some examples and see what works and what does not.

\hero{ROZA (Advisor)} Definitely -- trying out examples will help you
refine your ideas, too.

\heroSTUDENT{} Let's see, though; we have the simply typed \foreignlanguage{greek}{λ}-calculus, some ML
core features, a staging construct, and contextual types like in
\citet{nanevski2008contextual}\ldots{} I guess I will need a few weeks?

\heroADVISOR{} That sounds like a lot. Why don't you use some kind of
metalanguage to implement the prototype?

\heroSTUDENT{} You mean a tool like Racket \citep{racket-manifesto}, PLT Redex
\citep{felleisen2009semantics}, the K Framework
\citep{k-framework-main-reference} or Spoofax
\citep{spoofax-main-reference}?

\heroADVISOR{} Yes, all of those should be good choices. I was thinking though
that we could use higher-order logic programming\ldots{} it's a
formalism that is well-suited to what we want to do, since we will need
all sorts of different binding constructs, and the type system we are
thinking about is quite advanced.

\heroSTUDENT{} Oh, so you mean \foreignlanguage{greek}{λ}Prolog \citep{miller1988overview} or LF
\citep{lf-main-reference}.

\heroADVISOR{} Yes. Actually, a few years back a friend of mine worked on this
new implementation of \foreignlanguage{greek}{λ}Prolog just for this purpose -- rapid prototyping
of languages. It's called Makam. It should be able to handle what we
have in mind nicely, and we should not need to spend more than a few
hours on it!

\heroSTUDENT{} Sounds great! Anything I can read up on Makam then?

\heroADVISOR{} Not much, unfortunately\ldots{} But I know the language and its
standard library quite well, so let's work on this together; it'll be
fun. I'll show you how things work along the way!

\begin{scenecomment}
(Our heroes install Makam from --elided for blind reviewing-- and figure out how to run the REPL.)
\end{scenecomment}

\section{In which we get a premonition of things to
come}\label{in-which-we-get-a-premonition-of-things-to-come}

\identNormal

\emph{Chapter 3} serves as a tutorial to \lamprolog/Makam, showing the
basic usage of the language to encode the static and dynamic semantics
of the simply typed lambda calculus. \emph{Chapter 4} explores the
question of how to implement multiple-variable binding, culminating in a
reusable polymorphic datatype. \emph{Chapters 5 and 6} present a novel
account of how GADTs are directly supported in \lamprolog thanks to the
presence of ad-hoc polymorphism and showcase their use for accurate
encodings of mutually recursive definitions and pattern matching.
\emph{Chapter 7} describes a novel way to define operations by
structural recursion in \lamprolog/Makam while only giving the essential
cases, motivating them through the example of encoding a simple
conversion rule. The following chapters make use of the presented
features to implement polymorphism and algebraic datatypes
(\emph{Chapter 8}), heterogeneous staging constructs with contextual
typing (\emph{Chapter 9}) and Hindley-Milner type generalization
(\emph{Chapter 10}). We then summarize and compare to related work.

\identDialog
