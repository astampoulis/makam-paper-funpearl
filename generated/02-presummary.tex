\section{In which our readers get a premonition of things to
come}\label{in-which-our-readers-get-a-premonition-of-things-to-come}

\identNormal

\emph{Section 3} serves as a tutorial to \lamprolog/Makam, showing the
basic usage of the language to encode the static and dynamic semantics
of the simply typed lambda calculus. \emph{Section 4} explores the
question of how to implement multiple-variable binding, culminating in a
reusable polymorphic datatype. \emph{Sections 5 and 6} present a novel
account of how GADTs are directly supported in \lamprolog thanks to the
presence of ad-hoc polymorphism and showcase their use for accurate
encodings of mutually recursive definitions and pattern matching.
\emph{Section 7} describes a novel way to define operations by
structural recursion in \lamprolog/Makam while only giving the essential
cases, motivating them through the example of encoding a simple
conversion rule. The following sections make use of the presented
features to implement polymorphism and algebraic datatypes
(\emph{Section 8}), heterogeneous staging constructs with contextual
typing (\emph{Section 9}) and Hindley-Milner type generalization
(\emph{Section 10}). We then summarize and compare to related work.

We encourage readers to skim through the paper as a first pass, focusing
on the \highlightedtext{highlighted code}. These highlights provide a
rough picture of the Makam code needed to implement a typechecker for a
small ML-like language, along with a few key definitions from the
standard library.

\identDialog
