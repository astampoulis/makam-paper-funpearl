\heroADVISOR{} That's great! I understand that this was a little tricky, but
still, it was not too bad, right? Actually, I know of one thing that is
quite simple to do: the evaluation rule. On paper we typically write
something roughly like:

\vspace{-1.5em}\begin{mathpar}
\inferrule{e_1 \Downarrow v_1 \\ \texttt{match}(p, v_1) \leadsto \sigma \\ e_2[\sigma/xs] \Downarrow v_2}
          {\texttt{case\_or\_else}(e_1, p \mapsto xs.e_2, e_3) \Downarrow v_2}

\inferrule{e_1 \Downarrow v_1 \\ \texttt{match}(p, v_1) \not\leadsto \\ e_3 \Downarrow v_3}
          {\texttt{case\_or\_else}(e_1, p \mapsto xs.e_2, e_3) \Downarrow v_3}
\end{mathpar}

\noindent
So \texttt{match} tries to unify a pattern with a term and yields a
substitution \(\sigma\) for the pattern variables if successful, which
is then applied to the body of the branch. If there is no \texttt{match}
to be found, then we use the \texttt{else} branch.

\heroSTUDENT{} Hmm\ldots{} so do we need two predicates, one for the case
where the \texttt{match} is successful and one to check that a pattern
\emph{does not} match a scrutinee?

\heroADVISOR{} Actually we could have a single \texttt{match} predicate. And
we can use the logical \texttt{if-then-else} construct for the two
cases, which we have not seen so far. Let me write down the evaluation
rule, and I'll explain:

\begin{verbatim}
match : [NBefore NAfter]
  (Pattern: patt NBefore NAfter) (Scrutinee: term)
  (SubstBefore: vector term NBefore) (SubstAfter: vector term NAfter) \ensuremath{\to}
  prop.
eval (case_or_else Scrutinee Pattern Body Else) V' \ensuremath{:\!-}
  eval Scrutinee V,
  if (match Pattern V vnil Subst)
  then (vapplymany Body Subst Body', eval Body' V')
  else (eval Else V').
\end{verbatim}

\noindent
The \texttt{if-then-else} construct behaves as follows: when there is at
least one way to prove the condition, it proceeds to the \texttt{then}
branch, otherwise it goes to the \texttt{else} branch. Pretty standard,
really. It is one thing that the Prolog cut statement, \texttt{!}, is
useful for, but I find that using cut makes for less readable code.
\citet{kiselyov05backtracking} is worth reading for alternatives to the
cut statement and the semantics of
\texttt{if}-\texttt{then}-\texttt{else} and \texttt{not} in logic
programming, and Makam follows that paper closely.

\heroSTUDENT{} I see. Now, I noticed a \texttt{vapplymany} predicate -- what
is that?

\heroADVISOR{} That is a standard-library predicate. It is used to perform
simultaneous substitution for all the variables in our multiple binding
type, \texttt{vbindmany} (also available as \texttt{appmany} for
\texttt{bindmany}). Or another way to say it, it's the equivalent of
HOAS function application for \texttt{vbindmany}:

\begin{verbatim}
vapplymany : [N] vbindmany Var N Body \ensuremath{\to} vector Var N \ensuremath{\to} Body \ensuremath{\to} prop.
vapplymany (vbody Body) vnil Body.
vapplymany (vbind F) (vcons E ES) Body \ensuremath{:\!-} vapplymany (F E) ES Body.
\end{verbatim}

\heroSTUDENT{} I see\ldots{} OK, I think I know how to continue. I will write
a few of the \texttt{match} rules down.

\begin{verbatim}
match patt_var X Subst Subst' \ensuremath{:\!-} vsnoc Subst X Subst'.
match patt_wild X Subst Subst.
match patt_ozero ozero Subst Subst.
match (patt_osucc P) (osucc V) Subst Subst' \ensuremath{:\!-}
  match P V Subst Subst'.
\end{verbatim}

\begin{scenecomment}
(Our heroes also write down the rules for multiple patterns and tuples, which are
available in the unabridged version of this story.)
\end{scenecomment}

\heroSTUDENT{} Let's try this out with a simple example -- how about
predecessor for natural numbers?

\begin{verbatim}
eval (case_or_else (osucc (osucc ozero))
     (patt_osucc patt_var) (vbind (fun pred \ensuremath{\Rightarrow} vbody pred))
     ozero)
  V ?
>> Yes:
>> V := osucc ozero.
\end{verbatim}
