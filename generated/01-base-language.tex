\section{Where our heroes get the easy stuff out of the
way}\label{where-our-heroes-get-the-easy-stuff-out-of-the-way}

STUDENT. OK, let's just start with the simply typed lambda calculus to
see how this works. Let's define just the basics, application, lambda
abstraction and the arrow type.

ADVISOR. Right. We will first need to define the two meta-types for
these two sorts:

\begin{verbatim}
term : type.
typ : type.
\end{verbatim}

STUDENT. Oh, so \texttt{type} is the reserved keyword for the meta-level
kind of types, and we'll use \texttt{typ} for our object-level types?

ADVISOR. Exactly. And let's do the easy constructors first:

\begin{verbatim}
app : term \ensuremath{\to} term \ensuremath{\to} term.
arrow : typ \ensuremath{\to} typ \ensuremath{\to} typ.
\end{verbatim}

STUDENT. So we add constructors to a type at any point, we do not list
them out when we define it like in Haskell. But how about lambdas? I
have heard that \lamprolog supports higher-order abstract syntax, which
should make those really easy to add too, right?

ADVISOR. Yes, functions at the meta-level are parametric, so they
correspond exactly to single variable binding -- they cannot perform any
computation, and thus we do not have to worry about exotic terms. So
this works fine for Church-style lambdas:

\begin{verbatim}
lam : typ \ensuremath{\to} (term \ensuremath{\to} term) \ensuremath{\to} term.
\end{verbatim}

STUDENT. I see. And how about the typing judgement,
\(\Gamma \vdash e : \tau\) ?

ADVISOR. Let's add a predicate for that. Only, no \(\Gamma\), there is
an implicit context of assumptions:

\begin{verbatim}
typeof : term \ensuremath{\to} typ \ensuremath{\to} prop.
\end{verbatim}

STUDENT. Let me see if I can get the typing rule for application. I know
that in Prolog we write the conclusion of a rule first, and the premises
follow the \texttt{:-} sign. Does something like this work?

\begin{verbatim}
typeof (app E1 E2) T' :-
  typeof E1 (arrow T T'), typeof E2 T.
\end{verbatim}

ADVISOR. Yes! That's exactly right. Makam uses capital letters for
unification variables.

STUDENT. I will need help with the lambda typing rule though. What's the
equivalent of extending the context as in \(\Gamma, x : \tau\) ?

ADVISOR. Simple, we introduce a fresh constructor for terms, and a new
typing rule for it:

\begin{verbatim}
typeof (lam T1 E) (arrow T1 T2) :-
  (x:term \ensuremath{\to} typeof x T1 \ensuremath{\to} typeof (E x) T2).
\end{verbatim}

STUDENT. Hmm, so \texttt{x:term\ \ensuremath{\to}} introduces the fresh
constructor standing for the new variable, and
\texttt{typeof\ x\ T1\ \ensuremath{\to}} introduces the new assumption?
Oh, and we need to get to the body of the lambda function in order to
type-check it, that's why you do \texttt{E\ x}.

ADVISOR. Yes. Note that the introductions are locally scoped, so they
are only into effect for the recursive call to \texttt{typeof}.

STUDENT. Makes sense. So do we have a type checker already? Can we run
queries?

ADVISOR. We do! Observe:

\begin{verbatim}
typeof (lam _ (fun x \ensuremath{\Rightarrow} x)) T ?
>> Yes:
>> T := arrow T1 T1
\end{verbatim}

STUDENT. Cool! But wait, last time I implemented unification in my toy
STLC implementation it was easy to make it go into an infinite loop with
\(\lambda x. x x\). How does that work here?

ADVISOR. Well you were missing the occurs-check. \lamprolog unification
includes it:

\begin{verbatim}
typeof (lam _ (fun x \ensuremath{\Rightarrow} app x x)) T' ?
>> Impossible.
\end{verbatim}

STUDENT. Right. Cool, so what else can we do? How about adding tuples to
our language? Can we use something like a polymorphic list?

ADVISOR. Sure! \lamprolog has polymorphic types and higher-order
predicates:

\begin{verbatim}
list : type \ensuremath{\to} type.
nil : list A.
cons : A \ensuremath{\to} list A \ensuremath{\to} list A.

map : (A \ensuremath{\to} B \ensuremath{\to} prop) \ensuremath{\to} list A \ensuremath{\to} list B \ensuremath{\to} prop.
map P nil nil.
map P (cons X XS) (cons Y YS) :- P X Y, map P XS YS.
\end{verbatim}

STUDENT. Nice! So this should work:

\begin{verbatim}
tuple : list term \ensuremath{\to} term.
product : list typ \ensuremath{\to} typ.
typeof (tuple ES) (product TS) :-
  map typeof ES TS.
\end{verbatim}

ADVISOR. It does! And we can use syntactic sugar for \texttt{cons} and
\texttt{nil} too:

\begin{verbatim}
typeof (lam _ (fun x \ensuremath{\Rightarrow} lam _ (fun y \ensuremath{\Rightarrow} tuple [x, y]))) T ?
>> Yes:
>> T := arrow T1 (arrow T2 (product [T1, T2]))
\end{verbatim}

STUDENT. So how about evaluation? Can we write the big-step semantics
too?

ADVISOR. Why not? Let's add a predicate and do the two easy rules:

\begin{verbatim}
eval : term \ensuremath{\to} term \ensuremath{\to} prop.
eval (lam T F) (lam T F).
eval (tuple ES) (tuple VS) :- map eval ES VS.
\end{verbatim}

STUDENT. OK, let me try my hand at the beta-redex case. I'll just do
call-by-value. And I think in \(\lamprolog\) function application is
exactly capture-avoiding substitution, so this should be fine:

\begin{verbatim}
eval (app E E') V'' :-
  eval E (lam _ F), eval E' V', eval (F V') V''.
\end{verbatim}

ADVISOR. Exactly! See, I told you this would be easy!
