\section{Where the legend of the GADTs and the Ad-Hoc Polymorphism is
recounted}\label{where-the-legend-of-the-gadts-and-the-ad-hoc-polymorphism-is-recounted}

\identNormal\it

Once upon a time, our republic lacked one of the natural wonders that it
is now well-known for, and which is now regularly enjoyed by tourists
and inhabitants alike. I am talking of course about the Great Arboretum
of Dangling Trees, known as GADTs for short. Then settlers from the
far-away land of the Dependency started coming to the republic, and
started speaking of Lists that Knew Their Length, of Terms that Knew
Their Types, of Collections of Elements that were Heterogeneous, and
about the other magical beings of their home. And they set out to build
a natural environment for these beings on the republic, namely the GADTs
that we know and love, to remind them of home a little. And their work
was good and was admired by many.

A long time passed, and dispatches from another far-away land came to
the republic, written by authors whose names are now lost in the sea of
anonymity, and I fear might forever remain so. And the dispatches went
something like this.

\rm

\heroAUTHOR{} \ldots{} In my land of \lamprolog that I speak of, the type
system is a subset of System F\(_\omega\) that should be familiar to you
-- the simply typed lambda calculus, plus prenex polymorphism, plus
simple type constructors of the form
\texttt{type\ *\ ...\ *\ type\ \ensuremath{\to}\ type}. There is also a
limited form of rank-2 polymorphism, allowing types of the form
\texttt{forall\ A\ T}, which are inhabited by unapplied polymorphic
constants through the notation \texttt{@foo}. There is a \texttt{prop}
sort for propositions, which is a normal type, but also a bit special:
its terms are not just values but are also computations, activated when
queried upon.

However, the language of this land has a distinguishing feature, called
Ad-Hoc Polymorphism. You see, the different rules that define a
predicate in our language can \emph{specialize} their type arguments.
This can be used to define polymorphic predicates that behave
differently for different types, like this, where we are essentially
doing a \texttt{typecase} and we choose a rule depending on the
\emph{type} of the argument (as opposed to its value):

\begin{verbatim}
print : [A] A \ensuremath{\to} prop.
print (I: int) :- (... code for printing integers ...)
print (S: string) :- (... code for printing strings ...)
\end{verbatim}

The local dialects Teyjus
\citep{teyjus-main-reference,teyjus-2-implementation} and Makam include
this feature, while it is not encountered in other dialects like ELPI
\citep{elpi-main-reference}. In the Makam dialect in particular, type
variables are understood to be parametric by default. In order to make
them ad-hoc and allow specializing them in rules, we need to denote them
using the \texttt{{[}A{]}} notation.

Of course, this feature has both to do with the statics as well as the
dynamics of our language: and while dynamically it means something akin
to a \texttt{typecase}, statically, it means that rules might specialize
their type variables, and this remains so for type-checking the whole
rule.

But alas! Is it not type specialization during pattern matching that is
an essential feature of the GADTs of your land? Maybe that means that we
can use Ad-Hoc Polymorphism not just to do \texttt{typecase} but also to
work with GADTs in our land? Consider this! The venerable List that
Knows Its Length:

\begin{verbatim}
zero : type. succ : type \ensuremath{\to} type.
vector : type \ensuremath{\to} type \ensuremath{\to} type.
vnil : vector A zero.
vcons : A \ensuremath{\to} vector A N \ensuremath{\to} vector A (succ N).
\end{verbatim}

And now for the essential \texttt{vmap}:

\begin{verbatim}
vmap : [N] (A \ensuremath{\to} B \ensuremath{\to} prop) \ensuremath{\to} vector A N \ensuremath{\to} vector B N \ensuremath{\to} prop.
vmap P vnil vnil.
vmap P (vcons X XS) (vcons Y YS) :- P X Y, vmap P XS YS.
\end{verbatim}

In each rule, the first argument already specializes the \texttt{N} type
-- in the first rule to \texttt{zero}, in the second, to
\texttt{succ\ N}. And so erroneous rules that do not respect this
specialization would not be accepted as well-typed sayings in our
language:

\begin{verbatim}
vmap P vnil (vcons X XS) :- ...
\end{verbatim}

And we should note that in this usage of Ad-Hoc Polymorphism for GADTs,
it is only the increased precision of the statics that we care about.
Dynamically, the rules for \texttt{vmap} can perform normal term-level
unification and only look at the constructors \texttt{vnil} and
\texttt{vcons} to see whether each rule applies, rather than relying on
the \texttt{typecase} aspects we spoke of before.

Coupling this with the binding constructs that I talked to you earlier
about, we can build new magical beings, like the \emph{Bind that Knows
Its Length}:

\begin{verbatim}
vbindmany : (Var: type) (N: type) (Body: type) \ensuremath{\to} type.
vbody : Body \ensuremath{\to} vbindmany Var zero Body.
vbind : (Var \ensuremath{\to} vbindmany Var N Body) \ensuremath{\to} vbindmany Var (succ N) Body.
\end{verbatim}

(Whereby I am using notation of the Makam dialect in my definition of
\texttt{vbindmany} that allows me to name parameters, purely for the
purposes of increased clarity.)

In the \texttt{openmany} version for \texttt{vbindmany}, the rules are
exactly as before, though the static type is more precise:

\begin{verbatim}
vopenmany : [N] vbindmany Var N Body \ensuremath{\to} (vector Var N \ensuremath{\to} Body \ensuremath{\to} prop) \ensuremath{\to} prop.
vopenmany (vbody Body) Q :- Q vnil Body.
vopenmany (vbind F) Q :-
  (x:A \ensuremath{\to} vopenmany (F x) (fun xs \ensuremath{\Rightarrow} Q (vcons x xs))).
\end{verbatim}

We can also showcase the \emph{Accurate Encoding of the Letrec}:

\begin{verbatim}
vletrec : vbindmany term N (vector term N * term) \ensuremath{\to} term.
\end{verbatim}

And that is the way that the land of \lamprolog supports GADTs, without
needing the addition of any feature, all thanks to the existing support
for Ad-Hoc Polymorphism.

\identDialog
