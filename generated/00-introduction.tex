\section{Where our heroes set on a road to prototype a type
system}\label{where-our-heroes-set-on-a-road-to-prototype-a-type-system}

\heroSTUDENT{} (\ldots{}) So yes, I think my next step should be writing a toy
implementation of the type system we have in mind, so that we can try
out some examples and see what works and what does not.

\heroADVISOR{} Definitely -- trying out examples will help us refine our
ideas, too.

\heroSTUDENT{} Let's see, though; we have an ML core, dependently typed
constructs, and contextual types like in
\citet{nanevski2008contextual}\ldots{} I guess I will need a few months?

\heroADVISOR{} That sounds like a lot. Why don't you use some kind of
metalanguage to implement the prototype?

\heroSTUDENT{} You mean a tool like PLT Redex \citep{felleisen2009semantics},
the K Framework \citep{rosu2010overview,ellison2009rewriting}, Spoofax
\citep{kats2010spoofax}, Needle \& Knot \citep{keuchel2016needle}, or
CRSX \citep{rose2011crsx}?

\heroADVISOR{} Sure, all fine choices. Though I do not think these frameworks
have been used to implement a type system as advanced as the one we have
in mind, or can handle all the binding constructs we will need\ldots{} I
was actually thinking we should use higher-order logic programming.

\heroSTUDENT{} Oh, so \foreignlanguage{greek}{λ}Prolog \citep{miller1988overview} or LF
\citep{pfenning1999system}.

\heroADVISOR{} Yes. Actually, a few years back I worked on this new
implementation of \foreignlanguage{greek}{λ}Prolog just for this purpose -- rapid prototyping of
languages. It's called Makam. It should be able to handle what we have
in mind, and we won't need more than a few hours.

\heroSTUDENT{} Sounds great! Anything I can read up on it?

\heroADVISOR{} Not much, unfortunately\ldots{} Let's just work on this
together, it'll be fun.

\begin{scenecomment}
(Our heroes install Makam from
\if@ACM@anonymous{\url{http://github.com/astampoulis/makam}}\else{--elided for blind
reviewing--}\fi\xspace
and figure out how to run the REPL.)
\end{scenecomment}
