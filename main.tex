%% For double-blind review submission, w/o CCS and ACM Reference (max submission space)
%\documentclass[acmsmall,review,anonymous]{acmart}\settopmatter{printfolios=true,printccs=false,printacmref=false}
%% For double-blind review submission, w/ CCS and ACM Reference
%\documentclass[acmsmall,review,anonymous]{acmart}\settopmatter{printfolios=true}
%% For single-blind review submission, w/o CCS and ACM Reference (max submission space)
%\documentclass[acmsmall,review]{acmart}\settopmatter{printfolios=true,printccs=false,printacmref=false}
%% For single-blind review submission, w/ CCS and ACM Reference
%\documentclass[acmsmall,review]{acmart}\settopmatter{printfolios=true}
%% For final camera-ready submission, w/ required CCS and ACM Reference
\documentclass[acmsmall,screen]{acmart}

\usepackage[utf8]{inputenc}
\usepackage[T1]{fontenc}
\usepackage{microtype}
\usepackage[greek,english]{babel}
\usepackage{booktabs}
\usepackage{subcaption}
\usepackage{alltt}
\usepackage{xspace}
\usepackage{mathpartir}
\usepackage{indentfirst}
\usepackage{hyperref}
\usepackage{color}
\usepackage{fancyvrb}

\bibliographystyle{shared/ACM-Reference-Format}
\citestyle{acmauthoryear}   %% For author/year citations

%%% The following is specific to ICFP'18 and the paper
%%% 'Prototyping a Functional Language using Higher-Order Logic Programming: A Functional Pearl on Learning the Ways of λProlog/Makam'
%%% by Antonis Stampoulis and Adam Chlipala.
%%%
\setcopyright{rightsretained}
\acmPrice{}
\acmDOI{10.1145/3236788}
\acmYear{2018}
\copyrightyear{2018}
\acmJournal{PACMPL}
\acmVolume{2}
\acmNumber{ICFP}
\acmArticle{93}
\acmMonth{9}

\begin{document}

\title{Prototyping a Functional Language using Higher-Order Logic Programming}
\subtitle{A Functional Pearl on learning the ways of \lamprolog/Makam}

\author{Antonis Stampoulis}
\affiliation{
  \institution{Originate Inc.}
  \city{New York}
  \state{New York}
  \country{USA}
}
\email{antonis.stampoulis@gmail.com}

\author{Adam Chlipala}
\affiliation{
  \institution{MIT CSAIL}
  \city{Cambridge}
  \state{Massachusetts}
  \country{USA}
}
\email{adamc@csail.mit.edu}

%% -- Macro definitions
\newcommand\TODO[0]{\textbf{TODO}}
\newcommand\lamprolog[0]{\foreignlanguage{greek}{λ}Prolog\xspace}
\newcommand\fomega[0]{F$\omega$\xspace}

\newsavebox{\selvestebox}
\newtoggle{important}
\togglefalse{important}
\definecolor{highlightcolor}{rgb}{0.81,0.83,0.95}
\newcommand\importantCodeblock[0]{\toggletrue{important}}
\newcommand\importantCodeblockEnd[0]{\togglefalse{important}}
\renewenvironment{verbatim}{\iftoggle{important}{\newenvironment{colbox}{\begin{lrbox}{\selvestebox}%
   \begin{minipage}{\dimexpr\columnwidth-2\fboxsep\relax}}{\end{minipage}\end{lrbox}%
   \begin{center}\hspace{-6pt}%
   \colorbox{highlightcolor}{\usebox{\selvestebox}}%
   \end{center}}}{\newenvironment{colbox}{}{}}%
\begin{colbox}\begin{quote}\begin{alltt}}{\end{alltt}\end{quote}\end{colbox}}
\newcommand\highlightedtext[1]{{\setlength{\fboxsep}{2pt}\colorbox{highlightcolor}{#1}}}

\newcommand\hide[1]{}
\newcommand\tightlist[0]{\itemsep1pt\parskip0pt\parsep0pt}
\renewcommand\thesection{\textbf{CHAPTER \arabic{section}}}
\renewcommand\thesubsection{\textbf{SECTION \arabic{section}.\arabic{subsection}}}
\newcommand\hero[1]{\textit{#1}.}
\newcommand\heroSTUDENT[0]{\hero{HAGOP}}
\newcommand\heroADVISOR[0]{\hero{ROZA}}
\newcommand\heroAUDIENCE[0]{\hero{AUDIENCE}}
\newcommand\heroAUTHOR[0]{\hero{ANONYMOUS AUTHOR}}
\newenvironment{scenecomment}{\em\noindent}{}
\newcommand\identDialog[0]{\setlength{\leftskip}{1em}\setlength{\parindent}{-1em}}
\newcommand\identNormal[0]{\setlength{\leftskip}{0em}\setlength{\parindent}{0em}}
\newenvironment{normalident}{\identNormal}{\identDialog}
\newcommand\rulename[1]{\textsc{#1}}
\newenvironment{versy}{\begin{center}\begin{minipage}{0.85\textwidth}\begin{verse}}{\end{verse}\end{minipage}\end{center}}

\definecolor{todo}{rgb}{0.75,0.00,0.00}
\newcommand\todo[1]{\textcolor{todo}{#1}}
\newcommand\heroTODO[0]{\textbf{\todo{TODO}}.}
\newcommand\heroNEEDFEEDBACK[0]{\textbf{\todo{Need Feedback}}:}

% =====
% used for highlighting from pandoc
\DefineVerbatimEnvironment{Highlighting}{Verbatim}{commandchars=\\\{\}}
\newenvironment{Shaded}{\hspace{-0.5em}\begin{quote}}{\end{quote}}
\newcommand{\KeywordTok}[1]{\textcolor[rgb]{0.00,0.44,0.13}{\textbf{{#1}}}}
\newcommand{\DataTypeTok}[1]{\textcolor[rgb]{0.56,0.13,0.00}{{#1}}}
\newcommand{\DecValTok}[1]{\textcolor[rgb]{0.25,0.63,0.44}{{#1}}}
\newcommand{\BaseNTok}[1]{\textcolor[rgb]{0.25,0.63,0.44}{{#1}}}
\newcommand{\FloatTok}[1]{\textcolor[rgb]{0.25,0.63,0.44}{{#1}}}
\newcommand{\CharTok}[1]{\textcolor[rgb]{0.25,0.44,0.63}{{#1}}}
\newcommand{\StringTok}[1]{\textcolor[rgb]{0.25,0.44,0.63}{{#1}}}
\newcommand{\CommentTok}[1]{\textcolor[rgb]{0.38,0.63,0.69}{\textit{{#1}}}}
\newcommand{\OtherTok}[1]{\textcolor[rgb]{0.00,0.44,0.13}{{#1}}}
\newcommand{\AlertTok}[1]{\textcolor[rgb]{1.00,0.00,0.00}{\textbf{{#1}}}}
\newcommand{\FunctionTok}[1]{\textcolor[rgb]{0.02,0.16,0.49}{{#1}}}
\newcommand{\RegionMarkerTok}[1]{{#1}}
\newcommand{\ErrorTok}[1]{\textcolor[rgb]{1.00,0.00,0.00}{\textbf{{#1}}}}
\newcommand{\NormalTok}[1]{{#1}}
% =====

% ====
% highlighting makam

\newcommand\colorgreen[1]{\textcolor[rgb]{0.00,0.44,0.13}{#1}}
\newcommand\colorred[1]{\textcolor[rgb]{0.56,0.13,0.00}{#1}}
\newcommand\colorblue[1]{\textcolor[rgb]{0.02,0.16,0.49}{#1}}

\newcommand\tokkeyword[1]{#1}
\newcommand\toksymbol[1]{#1}
\newcommand\tokarrowtype[1]{#1}

\newcommand\toknumber[1]{#1}
\newcommand\tokstring[1]{#1}

\newcommand\tokmetavariable[1]{\colorblue{#1}}

\newcommand\tokbuiltintype[1]{\textbf{#1}}
\newcommand\tokstdtypeid[1]{#1}
\newcommand\toktypeid[1]{\colorred{#1}}

\newcommand\tokstdconst[1]{#1}
\newcommand\tokpropconst[1]{\colorgreen{#1}}
\newcommand\tokobjconst[1]{\colorred{\textbf{#1}}}
\newcommand\tokconst[1]{\colorred{\textbf{#1}}}

\newcommand\tokquery[1]{#1}

\newcommand\tokcomment[1]{#1}
\newcommand\tokdirective[1]{#1}

% ====

\begin{abstract}
We demonstrate how the framework of \emph{higher-order logic programming}, as exemplified
in the \lamprolog language design, is a prime vehicle for rapid prototyping of
implementations for programming languages with sophisticated type systems. We present the
literate development of a type checker for a language with a number of complicated
features, culminating in a standard ML-style core with algebraic datatypes and
type generalization, extended with staging constructs that are generic over a
separately defined language of terms. We add each new feature in sequence,
with little to no changes to existing code. Scaling the
higher-order logic programming approach to this setting required us to develop
approaches to challenges like complex variable binding patterns in object languages
and performing generic structural traversals of code, making use of novel constructions
in the setting of \lamprolog, such as GADTs and generic programming. For our development,
we make use of Makam, a new implementation of \lamprolog, which we introduce in tutorial
style as part of our (quasi-)literate development.
\end{abstract}

 \begin{CCSXML}
<ccs2012>
<concept>
<concept_id>10011007.10011006.10011008.10011009.10011015</concept_id>
<concept_desc>Software and its engineering~Constraint and logic languages</concept_desc>
<concept_significance>500</concept_significance>
</concept>
<concept>
<concept_id>10011007.10011006.10011039</concept_id>
<concept_desc>Software and its engineering~Formal language definitions</concept_desc>
<concept_significance>300</concept_significance>
</concept>
<concept>
<concept_id>10011007.10011074.10011092.10010876</concept_id>
<concept_desc>Software and its engineering~Software prototyping</concept_desc>
<concept_significance>100</concept_significance>
</concept>
</ccs2012>
\end{CCSXML}

\ccsdesc[500]{Software and its engineering~Constraint and logic languages}
\ccsdesc[300]{Software and its engineering~Formal language definitions}
\ccsdesc[100]{Software and its engineering~Software prototyping}

\keywords{higher-order logic programming, programming language prototyping, metaprogramming}

\maketitle

{
  \setlength{\parskip}{3pt}
  \renewcommand{\labelitemi}{\textendash}

  % Hanging Indent
  \identDialog{}

  % No Indent
  % \identNormal{}

  \section{Where our heroes set out on a road to prototype a type
system}\label{where-our-heroes-set-out-on-a-road-to-prototype-a-type-system}

\hero{HAGOP (Student)} (\ldots{}) So yes, I think my next step should be
writing a toy implementation of the type system we have in mind, so that
we can try out some examples and see what works and what does not.

\hero{ROZA (Advisor)} Definitely -- trying out examples will help you
refine your ideas, too.

\heroSTUDENT{} Let's see, though; we have the simply typed \foreignlanguage{greek}{λ}-calculus, some ML
core features, a staging construct, and contextual types like in
\citet{nanevski2008contextual}\ldots{} I guess I will need a few weeks?

\heroADVISOR{} That sounds like a lot. Why don't you use some kind of
metalanguage to implement the prototype?

\heroSTUDENT{} You mean a tool like Racket \citep{racket-manifesto}, PLT Redex
\citep{felleisen2009semantics}, the K Framework
\citep{k-framework-main-reference} or Spoofax
\citep{spoofax-main-reference}?

\heroADVISOR{} Yes, all of those should be good choices. I was thinking though
that we could use higher-order logic programming\ldots{} it's a
formalism that is well-suited to what we want to do, since we will need
all sorts of different binding constructs, and the type system we are
thinking about is quite advanced.

\heroSTUDENT{} Oh, so you mean \foreignlanguage{greek}{λ}Prolog \citep{miller1988overview} or LF
\citep{lf-main-reference}.

\heroADVISOR{} Yes. Actually, a few years back a friend of mine worked on this
new implementation of \foreignlanguage{greek}{λ}Prolog just for this purpose -- rapid prototyping
of languages. It's called Makam. It should be able to handle what we
have in mind nicely, and we should not need to spend more than a few
hours on it!

\heroSTUDENT{} Sounds great! Anything I can read up on Makam then?

\heroADVISOR{} Not much, unfortunately\ldots{} But I know the language and its
standard library quite well, so let's work on this together; it'll be
fun. I'll show you how things work along the way!

\begin{scenecomment}
(Our heroes install Makam from --elided for blind reviewing-- and figure out how to run the REPL.)
\end{scenecomment}


  \section{In which our readers get a premonition of things to
come}\label{in-which-our-readers-get-a-premonition-of-things-to-come}

\identNormal

\emph{Section 3} serves as a tutorial to \lamprolog/Makam, showing the
basic usage of the language to encode the static and dynamic semantics
of the simply typed lambda calculus. \emph{Section 4} explores the
question of how to implement multiple-variable binding, culminating in a
reusable polymorphic datatype. \emph{Sections 5 and 6} present a novel
account of how GADTs are directly supported in \lamprolog thanks to the
presence of ad-hoc polymorphism and showcase their use for accurate
encodings of mutually recursive definitions and pattern matching.
\emph{Section 7} describes a novel way to define operations by
structural recursion in \lamprolog/Makam while only giving the essential
cases, motivating them through the example of encoding a simple
conversion rule. The following sections make use of the presented
features to implement polymorphism and algebraic datatypes
(\emph{Section 8}), heterogeneous staging constructs with contextual
typing (\emph{Section 9}) and Hindley-Milner type generalization
(\emph{Section 10}). We then summarize and compare to related work.

We encourage readers to skim through the paper as a first pass, focusing
on the \highlightedtext{highlighted code}. These highlights provide a
rough picture of the Makam code needed to implement a typechecker for a
small ML-like language, along with a few key definitions from the
standard library.

\identDialog

  
  \section{In which our heroes get the easy stuff out of the
way}\label{in-which-our-heroes-get-the-easy-stuff-out-of-the-way}

\heroSTUDENT{} OK, let's just start with the simply typed lambda calculus to
see how this works. Let's define just the basics: application, lambda
abstraction, and the arrow type.

\heroADVISOR{} Right. We will first need to define the two meta-types for
these two sorts:

\importantCodeblock{}

\begin{verbatim}
term : type.
typ : type.
\end{verbatim}

\importantCodeblockEnd{}

\heroSTUDENT{} Oh, so \texttt{type} is the reserved keyword for the meta-level
kind of types, and we'll use \texttt{typ} for our object-level types?

\heroADVISOR{} Exactly. And let's do the easy constructors first:

\importantCodeblock{}

\begin{verbatim}
app : term \ensuremath{\to} term \ensuremath{\to} term.
arrow : typ \ensuremath{\to} typ \ensuremath{\to} typ.
\end{verbatim}

\importantCodeblockEnd{}

\heroSTUDENT{} So we add constructors to a type at any point; we do not list
them out when we define it like in Haskell. But how about lambdas? I
have heard that \foreignlanguage{greek}{λ}Prolog supports higher-order abstract syntax
\citep{hoas-standard-reference}, which should make those really easy to
add, too, right?

\heroADVISOR{} Yes, functions at the meta level are parametric, so they
correspond exactly to single-variable binding -- they cannot perform any
computation such as pattern matching on their argument and thus we do
not have to worry about exotic terms. So this works fine for
Church-style lambdas:

\importantCodeblock{}

\begin{verbatim}
lam : typ \ensuremath{\to} (term \ensuremath{\to} term) \ensuremath{\to} term.
\end{verbatim}

\importantCodeblockEnd{}

\heroSTUDENT{} I see. And how about the typing judgment,
\(\Gamma \vdash e : \tau\) ?

\heroADVISOR{} Let's add a \emph{predicate} for that: a new constructor for
the type of \texttt{prop}ositions. It will relate a \texttt{term} \(e\)
to its \texttt{typ}, \(\tau\). Only, no \(\Gamma\), there is an implicit
context of assumptions that will serve the same purpose.

\importantCodeblock{}

\begin{verbatim}
typeof : term \ensuremath{\to} typ \ensuremath{\to} prop.
\end{verbatim}

\importantCodeblockEnd{}

\heroSTUDENT{} I see. We now need to give the rules that make up the
predicate, right? Let me see if I can get the typing rule for
application. I know that in Prolog we write the conclusion of a rule
first, and the premises follow the \texttt{\ensuremath{:\!-}} sign. Does something like
this work?

\importantCodeblock{}

\begin{verbatim}
typeof (app E1 E2) T' \ensuremath{:\!-} typeof E1 (arrow T T'), typeof E2 T.
\end{verbatim}

\importantCodeblockEnd{}

\heroADVISOR{} Yes! That's exactly right. Makam uses capital letters for
unification variables.

\heroSTUDENT{} I will need help with the lambda typing rule, though. What's
the equivalent of extending the context as in \(\Gamma, \; x : \tau\) ?

\heroADVISOR{} Simple: we introduce a fresh constructor for terms and a new
typing rule for it:

\importantCodeblock{}

\begin{verbatim}
typeof (lam T1 E) (arrow T1 T2) \ensuremath{:\!-}
  (x:term \ensuremath{\to} typeof x T1 \ensuremath{\to} typeof (E x) T2).
\end{verbatim}

\importantCodeblockEnd{}

\heroSTUDENT{} Hmm, so `\texttt{x:term\ \ensuremath{\to}}' introduces the fresh
constructor standing for the new variable, and
`\texttt{typeof\ x\ T1\ \ensuremath{\to}}' introduces the new assumption?
Oh, and we need to get to the body of the lambda function in order to
type-check it, so that's why you do \texttt{E\ x}.

\heroADVISOR{} Yes. Note that the introductions are locally scoped, so they
are only in effect for the recursive call `\texttt{typeof\ (E\ x)\ T2}'.

\heroSTUDENT{} Makes sense. So do we have a type checker already?

\heroADVISOR{} We do! We can issue \emph{queries} of the \texttt{typeof}
predicate to type check terms. Observe:

\importantCodeblock{}

\begin{verbatim}
typeof (lam _ (fun x \ensuremath{\Rightarrow} x)) T ?
>> Yes:
>> T := arrow T1 T1.
\end{verbatim}

\importantCodeblockEnd{}

\heroSTUDENT{} Cool! So \texttt{fun} for metalevel functions, underscores for
unification variables we don't care about, and \texttt{?} for queries.
But wait, last time I implemented unification in my toy STLC
implementation it was easy to make it go into an infinite loop with
\(\lambda x. x x\). How does that work here?

\heroADVISOR{} Well, you were missing the occurs-check. \foreignlanguage{greek}{λ}Prolog unification
includes it:

\begin{verbatim}
typeof (lam _ (fun x \ensuremath{\Rightarrow} app x x)) T' ?
>> Impossible.
\end{verbatim}

\heroSTUDENT{} Right. So let's see, what else can we do? How about adding
tuples to our language? Can we use something like a polymorphic list?

\heroADVISOR{} Sure, \foreignlanguage{greek}{λ}Prolog has polymorphic types and higher-order
predicates. Here's how lists are defined in the standard library:

\begin{verbatim}
list : type \ensuremath{\to} type.
nil : list A.
cons : A \ensuremath{\to} list A \ensuremath{\to} list A.

map : (A \ensuremath{\to} B \ensuremath{\to} prop) \ensuremath{\to} list A \ensuremath{\to} list B \ensuremath{\to} prop.
map P nil nil.
map P (cons X XS) (cons Y YS) \ensuremath{:\!-} P X Y, map P XS YS.
\end{verbatim}

\heroSTUDENT{} Nice! I guess that's why you wanted to go with \foreignlanguage{greek}{λ}Prolog for
doing this instead of LF, since you cannot use polymorphism there?

\heroADVISOR{} Indeed. We will see, once we figure out what our language
should be, one thing we could do is monomorphize our definitions to LF,
and then we could even use Beluga \citep{beluga-main-reference} to do
all of our metatheoretic proofs. Or maybe we could use Abella
\citep{abella-main-reference} directly.

\heroSTUDENT{} Sounds good. So, for tuples, this should work:

\importantCodeblock{}

\begin{verbatim}
tuple : list term \ensuremath{\to} term.
product : list typ \ensuremath{\to} typ.
typeof (tuple ES) (product TS) \ensuremath{:\!-} map typeof ES TS.
\end{verbatim}

\importantCodeblockEnd{}

\heroADVISOR{} Yes, and there is syntactic sugar for \texttt{cons} and
\texttt{nil} too:

\begin{verbatim}
typeof (lam _ (fun x \ensuremath{\Rightarrow} lam _ (fun y \ensuremath{\Rightarrow} tuple [x, y]))) T ?
>> Yes:
>> T := arrow T1 (arrow T2 (product [T1, T2])).
\end{verbatim}

\heroSTUDENT{} So how about evaluation? Can we write the big-step semantics
too?

\heroADVISOR{} Why not? Let's add a predicate and do the two easy rules:

\importantCodeblock{}

\begin{verbatim}
eval : term \ensuremath{\to} term \ensuremath{\to} prop.
eval (lam T F) (lam T F).
eval (tuple ES) (tuple VS) \ensuremath{:\!-} map eval ES VS.
\end{verbatim}

\importantCodeblockEnd{}

\heroSTUDENT{} OK, let me try my hand at the beta-redex case. I'll just do
call-by-value. I know that when using HOAS, function application is
exactly capture-avoiding substitution, so this should be fine:

\importantCodeblock{}

\begin{verbatim}
eval (app E E') V'' \ensuremath{:\!-} eval E (lam _ F), eval E' V', eval (F V') V''.
\end{verbatim}

\importantCodeblockEnd{}

\heroADVISOR{} Exactly! See, I told you this would be easy!

  
  \section{In which our heroes add parentheses and discover how to do
multiple
binding}\label{in-which-our-heroes-add-parentheses-and-discover-how-to-do-multiple-binding}

\heroSTUDENT{} Still, I feel like we've been playing to the strengths of
\foreignlanguage{greek}{λ}Prolog\ldots{}. Yes, single-variable binding, substitutions, and so on
work nicely, but how about any other form of binding? Say, binding
multiple variables at the same time? We are definitely going to need
that for the language we have in mind. I was under the impression that
HOAS encodings do not work for that -- for example, I was reading
\citet{keuchel2016needle} recently and I remember something to that end.

\heroADVISOR{} That's not really true; having first-class support for
single-variable binders should be enough. But let's try it out, maybe
adding multiple-argument functions for example -- I mean uncurried ones.
Want to give it a try?

\heroSTUDENT{} Let me see. We want the terms to look roughly like this:

\begin{verbatim}
lammany (fun x \ensuremath{\Rightarrow} fun y \ensuremath{\Rightarrow} tuple [y, x])
\end{verbatim}

For the type of \texttt{lammany}, I want to write something like this,
but I know this is wrong.

\begin{verbatim}
lammany : (list term \ensuremath{\to} term) \ensuremath{\to} term.
\end{verbatim}

\heroADVISOR{} Yes, that doesn't quite work. It would introduce a fresh
variable for \texttt{list}s, not a number of fresh variables for
\texttt{term}s. HOAS functions are parametric, too, which means we
cannot even get to the potential elements of the fresh \texttt{list}
inside the \texttt{term}.

\heroSTUDENT{} Right. So I don't know, instead we want to use a type that
stands for \texttt{term\ \ensuremath{\to}\ term},
\texttt{term\ \ensuremath{\to}\ term\ \ensuremath{\to}\ term}, and so on.
Can we write \texttt{term\ \ensuremath{\to}\ ...\ \ensuremath{\to}\ term}?

\heroADVISOR{} Well, not quite, but we have already seen something similar, a
type that roughly stands for \texttt{term\ *\ ...\ *\ term}, and we did
not need anything special for that\ldots{}.

\heroSTUDENT{} You mean the \texttt{list} type?

\heroADVISOR{} Exactly. What do you think about this definition?

\begin{verbatim}
bindmanyterms : type.
bindnil : term \ensuremath{\to} bindmanyterms.
bindcons : (term \ensuremath{\to} bindmanyterms) \ensuremath{\to} bindmanyterms.
\end{verbatim}

\heroSTUDENT{} Hmm. That looks quite similar to lists; the parentheses in
\texttt{cons} are different. \texttt{nil} gets an extra \texttt{term}
argument, too\ldots{}.

\heroADVISOR{} Yes\ldots{} So what is happening here is that \texttt{bindcons}
takes a single argument, introducing a binder; and \texttt{bindnil} is
when we get to the body and don't need any more binders. Maybe we should
name them accordingly.

\heroSTUDENT{} Right, and could we generalize their types? Maybe that will
help me get a better grasp of it. How is this?

\begin{verbatim}
bindmany : type \ensuremath{\to} type \ensuremath{\to} type.
body : Body \ensuremath{\to} bindmany Variable Body.
bind : (Variable \ensuremath{\to} bindmany Variable Body) \ensuremath{\to} bindmany Variable Body.
\end{verbatim}

\heroADVISOR{} This looks great! That is exactly what's in the Makam standard
library, actually. And we can now define \texttt{lammany} using it --
and our example term from before.

\begin{verbatim}
lammany : bindmany term term \ensuremath{\to} term.
lammany (bind (fun x \ensuremath{\Rightarrow} bind (fun y \ensuremath{\Rightarrow} body (tuple [y,x]))))
\end{verbatim}

\heroSTUDENT{} I see. That is an interesting datatype. Is there some reference
about it?

\heroADVISOR{} Not that I know of, at least where it is called out as a
reusable datatype -- though the construction is definitely part of PL
folklore. After I started using this in Makam, I noticed similar
constructions in the wild, for example in MTac \citep{ziliani2013mtac},
for parametric HOAS implementation of telescopes in Coq.

\heroSTUDENT{} Interesting. So how do we work with \texttt{bindmany}? What's
the typing rule like?

\heroADVISOR{} The rule is written like this, and I'll explain what goes into
it:

\begin{verbatim}
arrowmany : list typ \ensuremath{\to} typ \ensuremath{\to} typ.
typeof (lammany F) (arrowmany TS T) \ensuremath{:\!-}
  openmany F (fun xs body \ensuremath{\Rightarrow} assumemany typeof xs TS (typeof body T)).
\end{verbatim}

\heroSTUDENT{} Let me see if I can read this\ldots{} \texttt{openmany} somehow
gives you fresh variables \texttt{xs} for the binders, plus the
\texttt{body} of the \texttt{lammany}; and then the
\texttt{assumemany\ typeof} part is what corresponds to extending the
\(\Gamma\) context with multiple typing assumptions?

\heroADVISOR{} Yes, and then we typecheck the \texttt{body} in that local
context that includes the fresh variables and the typing assumptions.
But let's do one step at a time. \texttt{openmany} is simple; we iterate
through the nested binders, introducing one fresh variable at a time. We
also substitute each bound variable for the current fresh variable, so
that when we get to the body, it only uses the fresh variables we
introduced.

\begin{verbatim}
openmany : bindmany A B \ensuremath{\to} (list A \ensuremath{\to} B \ensuremath{\to} prop) \ensuremath{\to} prop.
openmany (body Body) Q \ensuremath{:\!-} Q [] Body.
openmany (bind F) Q \ensuremath{:\!-} (x:A \ensuremath{\to} openmany (F x) (fun xs \ensuremath{\Rightarrow} Q (x :: xs))).
\end{verbatim}

\heroSTUDENT{} I see. I guess \texttt{assumemany} is similar, introducing one
assumption at a time?

\begin{verbatim}
assumemany : (A \ensuremath{\to} B \ensuremath{\to} prop) \ensuremath{\to} list A \ensuremath{\to} list B \ensuremath{\to} prop \ensuremath{\to} prop.
assumemany P [] [] Q \ensuremath{:\!-} Q.
assumemany P (X :: XS) (T :: TS) Q \ensuremath{:\!-} (P X T \ensuremath{\to} assumemany P XS TS Q).
\end{verbatim}

\heroADVISOR{} Yes, exactly! Just a note, though -- \lamprolog typically does
not allow the definition of \texttt{assumemany}, where a non-concrete
predicate like \texttt{P\ X\ Y} is used as an assumption, because of
logical reasons. Makam allows this form statically and so does ELPI
\citep{elpi-main-reference}, another \lamprolog implementation, though
there are instantiations of \texttt{P} that will fail at
run-time\footnote{The logical reason why this is not allowed in \lamprolog is that it violates the property of existence of uniform proofs; see \citet{assumemany-issue} for more information. An example of a goal that will fail at runtime is anything that includes a logical-or (denoted as ``\texttt{;}'') as an assumption, like ``\texttt{(typeof X T1; typeof X T2) \ensuremath{\to} ...}''.}.

\heroSTUDENT{} I see. But in that case we could just manually inline
\texttt{assumemany\ typeof} instead, so that's not a big problem, just
more verbose. But can I try our typing rule out?

\begin{verbatim}
typeof (lammany (bind (fun x \ensuremath{\Rightarrow} bind (fun y \ensuremath{\Rightarrow} body (tuple [y, x]))))) T ?
>> Yes:
>> T := arrowmany [T1, T2] (product [T2, T1]).
\end{verbatim}

\noindent
Great, I think I got the hang of this. Let me try to see if I can add a
multiple-argument application construct \texttt{appmany} and its
evaluation rules.

\begin{verbatim}
appmany : term \ensuremath{\to} list term \ensuremath{\to} term.
typeof (appmany E ES) T \ensuremath{:\!-}
  typeof E (arrowmany TS T), map typeof ES TS.
eval (appmany E ES) V \ensuremath{:\!-}
  eval E (lammany XS_E'), map eval ES VS, (...).
\end{verbatim}

\noindent
How can I do simultaneous substitution of all of the \texttt{XS} for
\texttt{VS}?

\heroADVISOR{} You'll need another standard-library predicate for
\texttt{bindmany}, which iteratively uses HOAS function application to
perform a number of substitutions:

\begin{verbatim}
applymany : bindmany A B \ensuremath{\to} list A \ensuremath{\to} B \ensuremath{\to} prop.
applymany (body B) [] B.
applymany (bind F) (X :: XS) B \ensuremath{:\!-} applymany (F X) XS B.
\end{verbatim}

\begin{verbatim}
eval (appmany E ES) V \ensuremath{:\!-}
  eval E (lammany XS_E'), map eval ES VS,
  applymany XS_E' VS E'', eval E'' V.
\end{verbatim}

\heroSTUDENT{} I see, that makes sense. Can I ask you something that worries
me, though -- all these fancy higher-order abstract binders, how do
we\ldots{} make them concrete? Say, how do we print them?

\heroADVISOR{} That's actually quite easy. We just add a concrete name to
them. A plain old \texttt{string}. Our typing rules etc. do not care
about it, but we could use it for parsing concrete syntax into our
abstract binding syntax, or for pretty-printing\ldots{}. All those are
stories for another time, though; let's just say that we could have
defined \texttt{bind} with an extra \texttt{string} argument,
representing the concrete name; and then \texttt{openmany} would just
ignore it.

\begin{verbatim}
bind : string \ensuremath{\to} (Var \ensuremath{\to} bindmany Var Body) \ensuremath{\to} bindmany Var Body.
\end{verbatim}

\heroSTUDENT{} Interesting. I would like to see more about this, but maybe
some other time. I thought of another thing that could be challenging:
mutually recursive \texttt{let\ rec}s?

\heroADVISOR{} Sure. Let's take this term for example:

\begin{verbatim}
let rec f = f_def and g = g_def in body
\end{verbatim}

\noindent
If we write this in a way where the binding structure is explicit, we
would bind \texttt{f} and \texttt{g} simultaneously and then write the
definitions and the body in that scope:

\begin{verbatim}
letrec (fun f \ensuremath{\Rightarrow} fun g \ensuremath{\Rightarrow} ([f_def, g_def], body))
\end{verbatim}

\heroSTUDENT{} I think I know how to do this then! How does this look?

\begin{verbatim}
letrec : bindmany term (list term * term) \ensuremath{\to} term.
\end{verbatim}

\heroADVISOR{} Exactly! Want to try writing the typing rules?

\heroSTUDENT{} Maybe something like this?

\begin{verbatim}
typeof (letrec XS_DefsBody) T' \ensuremath{:\!-}
  openmany XS_DefsBody (fun xs (defs, body) \ensuremath{\Rightarrow}
    assumemany typeof xs TS (map typeof defs TS),
    assumemany typeof xs TS (typeof body T')).
\end{verbatim}

\heroADVISOR{} Almost! You have used the syntax we use for writing rule
premises in the \texttt{fun} argument of \texttt{openmany}; the Makam
grammar only allows that with the syntactic form \texttt{pfun} instead,
which is used to write anonymous predicates. Since this \texttt{pfun}
argument will be a predicate and can thus perform computation, you are
also able to destructure parameters like you did here on
\texttt{(defs,\ body)} -- that doesn't work for normal functions in the
general case, since they need to treat arguments parametrically. This
works by performing unification of the parameter with the given term --
so \texttt{defs} and \texttt{body} need to be capitalized so that they
are understood to be unification variables.

\begin{verbatim}
typeof (letrec XS_DefsBody) T' \ensuremath{:\!-}
  openmany XS_DefsBody (pfun XS (Defs, Body) \ensuremath{\Rightarrow}
    assumemany typeof XS TS (map typeof Defs TS),
    assumemany typeof XS TS (typeof Body T')).
\end{verbatim}

\heroSTUDENT{} Ah, I
see\footnote{There is a subtlety here, having to do with the free variables that a unification variable
can capture. In \lamprolog, a unification variable is allowed to capture all the free variables in scope at the
point where it is introduced, as well as any variables it is explicitly applied to. \texttt{Defs} and \texttt{Body} are introduced as unification variables when we get to execute the \texttt{pfun}; otherwise, all unification variables used in a rule get introduced when we check whether the rule fires. As a result, \texttt{Defs} and \texttt{Body} can capture the \texttt{xs} variables that \texttt{openmany} introduces, whereas \texttt{T'} cannot. In \lamprolog terms, the \texttt{pfun} notation of Makam desugars to existential quantification of any (capitalized) unification variables that are mentioned while destructuring an argument, like the variables \texttt{Defs} and \texttt{Body}.}.
Let me ask you something though: one thing I noticed with our
representation of \texttt{letrec} is that we have to be careful so that
the number of binders matches the number of definitions we give. Our
typing rules disallow that, but I wonder if there's a way to have a more
accurate representation for \texttt{letrec} which includes that
requirement?

\heroADVISOR{} Funny you should ask that\ldots{} Let me tell you a story.

  
  \section{In which the legend of the GADTs and the Ad-Hoc Polymorphism is
recounted}\label{in-which-the-legend-of-the-gadts-and-the-ad-hoc-polymorphism-is-recounted}

\identNormal\it

Once upon a time, our republic lacked one of the natural wonders that it
is now well-known for, and which is now regularly enjoyed by tourists
and inhabitants alike. I am talking of course about the Great Arboretum
of Dangling Trees, known as GADTs for short. Then settlers from the
far-away land of the Dependency started coming to the republic, and
started speaking of Lists that Knew Their Length, of Terms that Knew
Their Types, of Collections of Elements that were Heterogeneous, and
about the other magical beings of their home. And they set out to build
a natural environment for these beings on the republic, namely the GADTs
that we know and love, to remind them of home a little. And their work
was good and was admired by many.

A long time passed, and dispatches from another far-away land came to
the republic, written by authors whose names are now lost in the sea of
anonymity, and I fear might forever remain so. And the dispatches went
something like this.

\rm

\heroAUTHOR{} \ldots{} In my land of \lamprolog that I speak of, the type
system is a subset of System F\(_\omega\) that should be familiar to you
-- the simply typed lambda calculus, plus prenex polymorphism, plus
simple type constructors of the form
\texttt{type\ *\ ...\ *\ type\ \ensuremath{\to}\ type}. There is also a
limited form of rank-2 polymorphism, allowing types of the form
\texttt{forall\ A\ T}, which are inhabited by unapplied polymorphic
constants through the notation \texttt{@foo}. There is a \texttt{prop}
sort for propositions, which is a normal type, but also a bit special:
its terms are not just values but are also computations, activated when
queried upon.

However, the language of this land has a distinguishing feature, called
Ad-Hoc Polymorphism. You see, the different rules that define a
predicate in our language can \emph{specialize} their type arguments.
This can be used to define polymorphic predicates that behave
differently for different types, like this, where we are essentially
doing a \texttt{typecase} and we choose a rule depending on the
\emph{type} of the argument (as opposed to its value):

\begin{verbatim}
print : [A] A \ensuremath{\to} prop.
print (I: int) \ensuremath{:\!-} (... code for printing integers ...)
print (S: string) \ensuremath{:\!-} (... code for printing strings ...)
\end{verbatim}

The local dialects Teyjus
\citep{teyjus-main-reference,teyjus-2-implementation} and Makam include
this feature, while it is not encountered in other dialects like ELPI
\citep{elpi-main-reference}. In the Makam dialect in particular, type
variables are understood to be parametric by default. In order to make
them ad-hoc and allow specializing them in rules, we need to denote them
using the \texttt{{[}A{]}} notation.

Of course, this feature has both to do with the statics as well as the
dynamics of our language: and while dynamically it means something akin
to a \texttt{typecase}, statically, it means that rules might specialize
their type variables, and this remains so for type-checking the whole
rule.

But alas! Is it not type specialization during pattern matching that is
an essential feature of the GADTs of your land? Maybe that means that we
can use Ad-Hoc Polymorphism not just to do \texttt{typecase} but also to
work with GADTs in our land? Consider this! The venerable List that
Knows Its Length:

\begin{verbatim}
zero : type. succ : type \ensuremath{\to} type.
vector : type \ensuremath{\to} type \ensuremath{\to} type.
vnil : vector A zero.
vcons : A \ensuremath{\to} vector A N \ensuremath{\to} vector A (succ N).
\end{verbatim}

And now for the essential \texttt{vmap}:

\begin{verbatim}
vmap : [N] (A \ensuremath{\to} B \ensuremath{\to} prop) \ensuremath{\to} vector A N \ensuremath{\to} vector B N \ensuremath{\to} prop.
vmap P vnil vnil.
vmap P (vcons X XS) (vcons Y YS) \ensuremath{:\!-} P X Y, vmap P XS YS.
\end{verbatim}

In each rule, the first argument already specializes the \texttt{N} type
-- in the first rule to \texttt{zero}, in the second, to
\texttt{succ\ N}. And so erroneous rules that do not respect this
specialization would not be accepted as well-typed sayings in our
language:

\begin{verbatim}
vmap P vnil (vcons X XS) \ensuremath{:\!-} ...
\end{verbatim}

And we should note that in this usage of Ad-Hoc Polymorphism for GADTs,
it is only the increased precision of the statics that we care about.
Dynamically, the rules for \texttt{vmap} can perform normal term-level
unification and only look at the constructors \texttt{vnil} and
\texttt{vcons} to see whether each rule applies, rather than relying on
the \texttt{typecase} aspects we spoke of before.

Coupling this with the binding constructs that I talked to you earlier
about, we can build new magical beings, like the \emph{Bind that Knows
Its Length}:

\begin{verbatim}
vbindmany : (Var: type) (N: type) (Body: type) \ensuremath{\to} type.
vbody : Body \ensuremath{\to} vbindmany Var zero Body.
vbind : (Var \ensuremath{\to} vbindmany Var N Body) \ensuremath{\to} vbindmany Var (succ N) Body.
\end{verbatim}

(Whereby I am using notation of the Makam dialect in my definition of
\texttt{vbindmany} that allows me to name parameters, purely for the
purposes of increased clarity.)

In the \texttt{openmany} version for \texttt{vbindmany}, the rules are
exactly as before, though the static type is more precise:

\begin{verbatim}
vopenmany : [N] vbindmany Var N Body \ensuremath{\to} (vector Var N \ensuremath{\to} Body \ensuremath{\to} prop) \ensuremath{\to} prop.
vopenmany (vbody Body) Q \ensuremath{:\!-} Q vnil Body.
vopenmany (vbind F) Q \ensuremath{:\!-}
  (x:A \ensuremath{\to} vopenmany (F x) (fun xs \ensuremath{\Rightarrow} Q (vcons x xs))).
\end{verbatim}

We can also showcase the \emph{Accurate Encoding of the Letrec}:

\begin{verbatim}
vletrec : vbindmany term N (vector term N * term) \ensuremath{\to} term.
\end{verbatim}

And that is the way that the land of \lamprolog supports GADTs, without
needing the addition of any feature, all thanks to the existing support
for Ad-Hoc Polymorphism.

\identDialog

  
  \section{In which our hero Hagop adds pattern matching on his
own}\label{in-which-our-hero-hagop-adds-pattern-matching-on-his-own}

\begin{scenecomment}
(Roza had a meeting with another student, so Hagop took a small break, and is now back at his
office. He is trying to work out on his own how to encode patterns. He is fairly
confident at this point that having explicit support for single-variable
binding is enough to model most complicated forms of binding, especially when making use of
polymorphism and GADTs.)
\end{scenecomment}

\identNormal
\heroSTUDENT{} OK, so let's implement simple patterns and pattern-matching
like in ML\ldots{} First let's determine the right binding structure.
For a branch like:

\begin{verbatim}
| cons(hd, tl) \ensuremath{\to} ... hd .. tl ...
\end{verbatim}

the pattern introduces 2 variables, \texttt{hd} and \texttt{tl}, which
the body of the branch can refer to. But we can't really refer to those
variables in the pattern itself, at least for simple
patterns\footnote{There are counterexamples, like for or-patterns in some ML dialects, or for dependent pattern matching, where consequent uses of the same variable perform exact matches rather than unification. We choose to omit the handling of cases like those in the present work for presentation purposes.}\ldots{}.
So there's no binding going on really within the pattern; instead, once
we figure out how many variables a pattern introduces, we can do the
actual binding all at once, when we get to the body of the branch:

\begin{verbatim}
branch(pattern, bind [# of variables in pattern].body)
\end{verbatim}

So we could write the above branch in Makam like this:

\begin{verbatim}
branch(patt_cons patt_var patt_var,
       bind (fun hd \ensuremath{\Rightarrow} bind (fun tl \ensuremath{\Rightarrow} body (.. hd .. tl ..))))
\end{verbatim}

We do have to keep the order of variables consistent somehow, so
\texttt{hd} here should refer to the first occurrence of
\texttt{patt\_var}, and \texttt{tl} to the second. Based on these, I am
thinking that the type of \texttt{branch} should be something like:

\begin{verbatim}
branch : (Pattern: patt N) (Vars_Body: vbindmany term N term) \ensuremath{\to} ...
\end{verbatim}

Wait, before I get into the weeds let me just set up some things. First,
let's add a simple base type, say \texttt{nat}s, to have something to
work with as an example. I'll prefix their names with \texttt{o} for
``object language,'' so as to avoid ambiguity. And I will also add a
\texttt{case\_or\_else} construct, standing for a single-branch
pattern-match construct. It should be easy to extend to a
multiple-branch construct, but I want to keep things as simple as
possible. I'll inline what I had written for \texttt{branch} above into
the definition of \texttt{case\_or\_else}.

\begin{verbatim}
onat : typ. ozero : term. osucc : term \ensuremath{\to} term.
typeof ozero onat. typeof (osucc N) onat \ensuremath{:\!-} typeof N onat.
eval ozero ozero. eval (osucc E) (osucc V) \ensuremath{:\!-} eval E V.
\end{verbatim}

\begin{verbatim}
case_or_else : (Scrutinee: term)
  (Patt: patt N) (Vars_Body: vbindmany term N term)
  (Else: term) \ensuremath{\to} term.
\end{verbatim}

Now for the typing rule -- it will be something like this:

\begin{verbatim}
typeof (case_or_else Scrutinee Pattern Vars_Body Else) BodyT \ensuremath{:\!-}
  typeof Scrutinee T, typeof_patt Pattern T VarTypes,
  vopenmany Vars_Body (pfun Vars Body \ensuremath{\Rightarrow}
    vassumemany typeof Vars VarTypes (typeof Body BodyT)),
  typeof Else BodyT.
\end{verbatim}

Right, so when checking a pattern, we'll have to determine both what
type of scrutinee it matches, as well as the types of the variables that
it contains. We will also need \texttt{vassumemany} that is just like
\texttt{assumemany} from before but which takes \texttt{vector}
arguments instead of \texttt{list}.

\begin{verbatim}
typeof_patt : [N] patt N \ensuremath{\to} typ \ensuremath{\to} vector typ N \ensuremath{\to} prop.
vassumemany : [N] (A \ensuremath{\to} B \ensuremath{\to} prop) \ensuremath{\to} vector A N \ensuremath{\to} vector B N \ensuremath{\to} prop \ensuremath{\to} prop.
(...)
\end{verbatim}

Now, I can just go ahead and define the patterns, together with their
typing relation, \texttt{typeof\_patt}.

Let me just work one by one for each pattern.

\begin{verbatim}
patt_var : patt (succ zero).
typeof_patt patt_var T (vcons T vnil).
\end{verbatim}

OK, that's how we'll write pattern variables, introducing a single
variable of a specific \texttt{typ} into the body of the branch. And the
following should be good for the \texttt{onat}s I defined earlier.

\begin{verbatim}
patt_ozero : patt zero.
typeof_patt patt_ozero onat vnil.

patt_osucc : patt N \ensuremath{\to} patt N.
typeof_patt (patt_osucc P) onat VarTypes \ensuremath{:\!-} typeof_patt P onat VarTypes.
\end{verbatim}

A wildcard pattern will match any value and should not introduce a
variable into the body of the branch.

\begin{verbatim}
patt_wild : patt zero.
typeof_patt patt_wild T vnil.
\end{verbatim}

OK, and let's do patterns for our n-tuples\ldots{}. I guess I'll need a
type for lists of patterns too.

\begin{verbatim}
patt_tuple : pattlist N \ensuremath{\to} patt N.
typeof_patt (patt_tuple PS) (product TS) VarTypes \ensuremath{:\!-}
  typeof_pattlist PS TS VarTypes.
pattlist : (N: type) \ensuremath{\to} type.
pnil : patt zero.
pcons : patt N \ensuremath{\to} pattlist N' \ensuremath{\to} pattlist (N + N').
\end{verbatim}

Uh-oh\ldots{} don't think I can do that
\texttt{N\ +\ N\textquotesingle{}} really. In this \texttt{pcons} case,
my pattern basically looks like \texttt{(P,\ ...PS)}; and I want the
overall pattern to have as many variables as \texttt{P} and \texttt{PS}
combined. But the GADTs support in \lamprolog seems to be quite basic. I
do not think there's any notion of type-level functions like
plus\ldots{}.

However\ldots{} maybe I can work around that, if I change \texttt{patt}
to include an ``accumulator'' argument, say \texttt{NBefore}. Each
constructor for patterns will now define how many pattern variables it
adds to that accumulator, yielding \texttt{NAfter}, rather than defining
how many pattern variables it includes\ldots{} like this:

\begin{verbatim}
patt, pattlist : (NBefore: type) (NAfter: type) \ensuremath{\to} type.
patt_var : patt N (succ N).
patt_ozero : patt N N.
patt_osucc : patt N N' \ensuremath{\to} patt N N'.
patt_wild : patt N N.
patt_tuple : pattlist N N' \ensuremath{\to} patt N N'.

pnil : pattlist N N.
pcons : patt N N' \ensuremath{\to} pattlist N' N'' \ensuremath{\to} pattlist N N''.
\end{verbatim}

Yes, I think that should work. I have a little editing to do in my
existing predicates to use this representation instead. For top-level
patterns, we should always start with the accumulator being
\texttt{zero}\ldots{}

\begin{verbatim}
case_or_else : (Scrutinee: term)
  (Patt: patt zero N) (Vars_Body: vbindmany term N term)
  (Else: term) \ensuremath{\to} term.
\end{verbatim}

I also have to change \texttt{typeof\_patt}, so that it includes an
accumulator argument of its own:

\begin{verbatim}
typeof_patt : [NBefore NAfter] patt NBefore NAfter \ensuremath{\to} typ \ensuremath{\to}
  vector typ NBefore \ensuremath{\to} vector typ NAfter \ensuremath{\to} prop.

typeof (case_or_else Scrutinee Pattern Vars_Body Else) BodyT \ensuremath{:\!-}
  typeof Scrutinee T, typeof_patt Pattern T vnil VarTypes,
  vopenmany Vars_Body (pfun Vars Body \ensuremath{\Rightarrow}
    vassumemany typeof Vars VarTypes (typeof Body BodyT)),
  typeof Else BodyT.
\end{verbatim}

All right, let's proceed to the typing rules for patterns themselves:

\begin{verbatim}
typeof_patt patt_var T VarTypes VarTypes' \ensuremath{:\!-}
  vsnoc VarTypes T VarTypes'.
\end{verbatim}

OK, here I need \texttt{vsnoc} to add an element to the end of a vector.
That should yield the correct order for the types of pattern variables;
I am visiting the pattern left-to-right after all.

\begin{verbatim}
vsnoc : [N] vector A N \ensuremath{\to} A \ensuremath{\to} vector A (succ N) \ensuremath{\to} prop.
vsnoc vnil Y (vcons Y vnil).
vsnoc (vcons X XS) Y (vcons X XS_Y) \ensuremath{:\!-} vsnoc XS Y XS_Y.
\end{verbatim}

The rest is easy to adapt\ldots{}.

\begin{scenecomment}
(Our hero finishes adapting the rest of the rules for \texttt{typeof\_patt},
which are available in the unabridged version of this story.)
\end{scenecomment}

Let me see if this works! I'll try out the predecessor function:

\begin{verbatim}
typeof (lam _ (fun n \ensuremath{\Rightarrow} case_or_else n
  (patt_osucc patt_var) (vbind (fun pred \ensuremath{\Rightarrow} vbody pred))
  ozero)) T ?
>> Yes:
>> T := arrow onat onat.
\end{verbatim}

Great! Time to show this to Roza.

\identDialog


  \section{In which our heroes reflect on structural
recursion}\label{in-which-our-heroes-reflect-on-structural-recursion}

\heroADVISOR{} Your pattern matching encoding looks good! You seem to be
getting the hang of this. How about we do something challenging then?
Say, type synonyms?

\heroSTUDENT{} Type synonyms? You mean, introducing type definitions like
\texttt{type\ natpair\ =\ nat\ *\ nat}? That does not seem particularly
tricky.

\heroADVISOR{} I think we will face a couple of interesting issues with it,
the main one being how to do \emph{structural recursion} in a nice way.
But first, let me write out the necessary pen-and-paper rules, so that
we are on the same page. We'll do top-level type definitions, so let's
add a top-level notion of programs \(c\) and a well-formedness judgment
`\(\vdash c \; \text{wf}\)' for them. (We could do modules instead of
just programs, but I feel that would derail us a little.) We also need
an additional environment \(\Delta\) to store type definitions:

\vspace{-1.2em}\begin{mathpar}
\inferrule[WfProgram-Main]{\emptyset; \; \Delta \vdash e : \tau}
          {\Delta \vdash (\text{main} \; e) \; \text{wf}}

\inferrule[WfProgram-Type]{\Delta, \; \alpha = \tau \vdash c \; \text{wf}}
          {\Delta \vdash (\texttt{type} \; \alpha = \tau \; ; \; c) \; \text{wf}}

\inferrule[Typeof-Conv]
          {\Gamma; \Delta \vdash e : \tau \\\\ \Delta \vdash \tau =_\delta \tau'}
          {\Gamma; \Delta \vdash e : \tau'}

\inferrule[TypEq-Def]
          {\alpha = \tau \in \Delta}
          {\Delta \vdash \alpha =_\delta \tau}
\cdots
\end{mathpar}

\heroSTUDENT{} Right, we will need the conversion rule, so that we identify
types up to expanding their definitions; that's
\(\delta\)-equality\ldots{} And I see you haven't listed out all the
rules for that, but those are mostly standard.

\heroADVISOR{} Still, there are quite a few of those rules. Want to give
transcribing this to Makam a try?

\heroSTUDENT{} Yes, I got this. I'll add a new \texttt{typedef} predicate; I
will only use it for local assumptions, to correspond to the \(\Delta\)
context of type definitions. I will also do the well-formed program
rules:

\begin{verbatim}
typedef : (NewType: typ) (Definition: typ) \ensuremath{\to} prop.
program : type. 
main : term \ensuremath{\to} program. 
lettype : (Definition: typ) (A_Program: typ \ensuremath{\to} program) \ensuremath{\to} program.
\end{verbatim}

\importantCodeblock{}

\begin{verbatim}
wfprogram : program \ensuremath{\to} prop.
wfprogram (main E) \ensuremath{:\!-} typeof E T.
wfprogram (lettype T A_Program) \ensuremath{:\!-}
  (a:typ \ensuremath{\to} typedef a T \ensuremath{\to} wfprogram (A_Program a)).
\end{verbatim}

\importantCodeblockEnd{}

\noindent
Well, I can do the conversion rule and the type-equality judgment
too\ldots{}. I will name that \texttt{typeq}. I'll just write the one
rule for now, which should be sufficient for a small example:

\begin{verbatim}
typeq : (T1: typ) (T2: typ) \ensuremath{\to} prop.
typeof E T \ensuremath{:\!-} typeof E T', typeq T T'.
typeq A T \ensuremath{:\!-} typedef A T.

wfprogram (lettype (arrow onat onat) (fun a \ensuremath{\Rightarrow}
          (main (lam a (fun f \ensuremath{\Rightarrow} (app f ozero)))))) ?
>> (Complete silence)
\end{verbatim}

\heroADVISOR{} Time to \texttt{Ctrl-C} out of the infinite loop?

\heroSTUDENT{} Oh. Oh, right. I guess we hit a case where the proof-search
strategy of Makam fails to make progress?

\heroADVISOR{} Correct. The loop happens when the new \texttt{typeof} rule
gets triggered: it has \texttt{typeof\ E\ T\textquotesingle{}} as a
premise, but the same rule still applies to solve that goal, so the rule
will fire again, and so on. Makam just does depth-first search right
now; until my friend implements a more sophisticated search strategy, we
need to find another way to do this.

\heroSTUDENT{} I see. I guess we should switch to an algorithmic type system
then.

\heroADVISOR{} Yes. Fortunately we can do that with relatively painless edits
and additions. Consider this: we only need to use the conversion rule in
cases where we already know something about the type \texttt{T} of an
expression \texttt{E}, but the typing rules require that \texttt{E} has
a type \texttt{T\textquotesingle{}} of some other form. That was the
case above -- for \texttt{E\ =\ f}, we knew that \texttt{T\ =\ a}, but
the typing rule for \texttt{app} required that
\texttt{T\textquotesingle{}\ =\ arrow\ T1\ T2} for some \texttt{T1},
\texttt{T2}.

\heroSTUDENT{} Oh. In that case we could try \emph{not} propagating the
concrete type information we have? We could then use the conversion rule
to check that the type we end up with matches what we expect.

\heroADVISOR{} Exactly. So we need to change the rule you wrote to apply only
in the case where \texttt{T} starts with a concrete constructor, rather
than when it is an uninstantiated unification variable. We will then
check whether the resulting type \texttt{T\textquotesingle{}} is equal
to \texttt{T}, using our \texttt{typeq} predicate.

\heroSTUDENT{} Is that even possible? Is there a way in \foreignlanguage{greek}{λ}Prolog to tell
whether something is a unification variable?

\heroADVISOR{} There is! Most Prolog dialects have a predicate that does that
-- it's usually called \texttt{var}. In Makam it is called
\texttt{refl.isunif}, the \texttt{refl} namespace prefix standing for
\emph{reflective} predicates. So, here's how we can write it instead,
where I'll also use logical
negation\footnote{Makam follows \citet{kiselyov05backtracking} closely in terms of the semantics for logical if-then-else and logical negation.}:

\importantCodeblock{}

\begin{verbatim}
typeof E T \ensuremath{:\!-} not(refl.isunif T), typeof E T', typeq T T'.
\end{verbatim}

\importantCodeblockEnd{}

\heroSTUDENT{} Interesting. If we ever made a paper submission out of this,
some reviewers would not be happy about this \texttt{typeof} rule. But
sure. Oh, and we should add the conversion rule for
\texttt{typeof\_patt}, but that's almost the same as for terms.
(\ldots{}) I'll do \texttt{typeq} next.

\begin{verbatim}
typeq A T' \ensuremath{:\!-} typedef A T, typeq T T'.
typeq T' A \ensuremath{:\!-} typedef A T, typeq T T'.
\end{verbatim}

\heroADVISOR{} I like how you added the symmetric rule, but\ldots{} this is
subtle, but if \texttt{A} is a unification variable, we don't want to
unify it with an arbitrary synonym. So we need to check that \texttt{A}
is concrete
somehow\footnote{Though not supported in Makam, an alternative to this in other languages based on higher-order logic programming would be to add a \texttt{mode (i o)} declaration for \texttt{typedef}, so that \texttt{typedef A T} would fail if \texttt{A} is not concrete.}:

\begin{verbatim}
typeq A T' \ensuremath{:\!-} not(refl.isunif A), typedef A T, typeq T T'.
typeq T' A \ensuremath{:\!-} not(refl.isunif A), typedef A T, typeq T T'.
\end{verbatim}

\heroSTUDENT{} I see what you mean. OK, I'll continue on to the rest of the
cases\ldots{}.

\begin{verbatim}
typeq (arrow T1 T2) (arrow T1' T2') \ensuremath{:\!-}
  typeq T1 T1', typeq T2 T2'.
typeq (arrowmany TS T) (arrowmany TS' T') \ensuremath{:\!-}
  map typeq TS TS', typeq T T'.
\end{verbatim}

\heroADVISOR{} Writing boilerplate is not fun, is it?

\heroSTUDENT{} It is not. I wish we could just write the first two rules that
you wrote; they're the important ones, after all. All the others just
propagate the structural recursion through. Also, whenever we add a new
constructor for types, we'll have to remember to add a \texttt{typeq}
rule for it\ldots{}.

\heroADVISOR{} Right. Let's just use some magic instead.

\begin{scenecomment}
(Roza changes the type definition of \texttt{typeq} to \texttt{typeq : [Any] (T1: Any) (T2: Any) \ensuremath{\to} prop},
and adds a few lines:)
\end{scenecomment}

\importantCodeblock{}

\begin{verbatim}
typeq A T' \ensuremath{:\!-} not(refl.isunif A), typedef A T, typeq T T'.
typeq T' A \ensuremath{:\!-} not(refl.isunif A), typedef A T, typeq T T'.
typeq T T' \ensuremath{:\!-} structural_recursion @typeq T T'.
\end{verbatim}

\importantCodeblockEnd{}

\heroSTUDENT{} \ldots{} What just happened. Is \texttt{structural\_recursion}
some special Makam trick I don't know about yet?

\heroADVISOR{} Indeed. There is a little bit of trickery involved here, but
you will see that there is much less of it than you would expect, upon
close reflection. \texttt{structural\_recursion} is just a normal
standard-library predicate like any other; it essentially applies a
polymorphic predicate ``structurally'' to a term. Its implementation
will be a little special of course. But let's just think about how you
would write the rest of the rules of \texttt{typeq} generically, to
perform structural recursion.

\heroSTUDENT{} OK. Well, when looking at two \texttt{typ}s together, we have
to make sure that their constructors are the same and also that any
\texttt{typ}s they contain as arguments are recursively
\texttt{typeq}ual. So something like this:

\begin{verbatim}
typeq (Constructor Arguments) (Constructor Arguments') \ensuremath{:\!-}
  map typeq Arguments Arguments'.
\end{verbatim}

\heroADVISOR{} Right. Note, though, that the types of arguments might be
different than \texttt{typ}. So even if we start comparing two types at
the top level, we might end up having to compare two lists of types that
they contain -- imagine the case for \texttt{arrowmany} for example.

\heroSTUDENT{} I see! That's why you edited \texttt{typeq} to be polymorphic
above; you have extended it to work on \emph{any type} (of the
metalanguage) that might contain a \texttt{typ}.

\heroADVISOR{} Exactly. Now, the list of \texttt{Arguments} -- can you come up
with a type for them?

\heroSTUDENT{} We can use the GADT of heterogeneous lists for them; not all
the arguments of each constructor need to be of the same type!

\begin{verbatim}
typenil : type. typecons : (T: type) (TS: type) \ensuremath{\to} type.
hlist : (TypeList: type) \ensuremath{\to} type.
hnil : hlist typenil. hcons : T \ensuremath{\to} hlist TS \ensuremath{\to} hlist (typecons T TS).
\end{verbatim}

\heroADVISOR{} Great! We will need a heterogeneous \texttt{map} for these
lists too. We'll need a polymorphic predicate as an argument, since
we'll have to use it for \texttt{Arguments} of different types:

\begin{verbatim}
hmap : [TS] (P: forall A (A \ensuremath{\to} A \ensuremath{\to} prop)) (XS: hlist TS) (YS: hlist TS) \ensuremath{\to} prop.
hmap P hnil hnil.
hmap P (hcons X XS) (hcons Y YS) \ensuremath{:\!-} forall.call P X Y, hmap P XS YS.
\end{verbatim}

\noindent
As I mentioned before, the rank-2 polymorphism support in Makam is quite
limited, so you have to use \texttt{forall.call} explicitly to
instantiate the polymorphic \texttt{P} predicate accordingly and call
it.

\heroSTUDENT{} Let me try out an example of that:

\begin{verbatim}
change : [A]A \ensuremath{\to} A \ensuremath{\to} prop. change 1 2. change "foo" "bar".
hmap @change (hcons 1 (hcons "foo" hnil)) YS ?
>> Yes:
>> YS := hcons 2 (hcons "bar" hnil).
\end{verbatim}

\noindent
Looks good enough. So, going back to our generic rule -- is there a way
to actually write it? Maybe there's a reflective predicate we can use,
similar to how we used \texttt{refl.isunif} before to tell if a term is
an uninstantiated unification variable?

\heroADVISOR{} Exactly -- there is \texttt{refl.headargs}. It relates a
concrete term to its decomposition into a constructor and a list of
arguments\footnote{Other versions of Prolog have predicates toward the same effect; for example, SWI-Prolog \citep{wielemaker2012swi} provides `\texttt{compound\_{}name\_{}arguments}', which is quite similar.}.
This does not need an extra-logical feature save for
\texttt{refl.isunif}, though: we could define \texttt{refl.headargs}
without any special support, if we maintained a discipline whenever we
add a new constructor, roughly like this:

\begin{verbatim}
refl.headargs : (Term: TermT) (Constr: ConstrT) (Args: hlist ArgsTS) \ensuremath{\to} prop.

arrowmany : (TS: list typ) (T: typ) \ensuremath{\to} typ.
refl.headargs Term Constructor Args \ensuremath{:\!-}
  not(refl.isunif Term), eq Term (arrowmany TS T),
  eq Constructor arrowmany, eq Args (hcons TS (hcons T hnil)).
\end{verbatim}

By the way, \texttt{eq} is a standard-library predicate that simply
attempts unification of its two arguments:

\begin{verbatim}
eq : A \ensuremath{\to} A \ensuremath{\to} prop. eq X X.
\end{verbatim}

\heroSTUDENT{} I see. I think I can write the generic rule for \texttt{typeq}
now then!

\begin{verbatim}
typeq T T' \ensuremath{:\!-}
  refl.headargs T Constr Args,
  refl.headargs T' Constr Args',
  hmap @typeq Args Args'.
\end{verbatim}

\heroADVISOR{} That looks great! Simple, isn't it? You'll see that there are a
few more generic cases that are needed, though. Should we do that? We
can roll our own reusable \texttt{structural\_recursion} implementation
-- that way we will dispel all magic from its use that I showed you
earlier! I'll give you the type; you fill in the first case:

\importantCodeblock{}

\begin{verbatim}
structural_recursion : [Any] 
  (RecursivePred: forall A (A \ensuremath{\to} A \ensuremath{\to} prop))
  (X: Any) (Y: Any) \ensuremath{\to} prop.
\end{verbatim}

\importantCodeblockEnd{}

\heroSTUDENT{} Let me see. Oh, so, the first argument is a predicate -- are we
doing this in open-recursion style? I see. Well, I can adapt the case I
just wrote above.

\begin{verbatim}
structural_recursion Rec X Y \ensuremath{:\!-}
  refl.headargs X Constructor Arguments,
  refl.headargs Y Constructor Arguments',
  hmap Rec Arguments Arguments'.
\end{verbatim}

\heroADVISOR{} Nice. Now, here you assume that \texttt{X} and \texttt{Y} are
both concrete terms. What happens when \texttt{X} is concrete and
\texttt{Y} isn't, or the other way around? Hint: you can use this
\texttt{happly} predicate, to apply a list of arguments to a
constructor, and thus reconstruct a term:

\begin{verbatim}
happly : [Constr Args Terms] Constr \ensuremath{\to} hlist Args \ensuremath{\to} Term \ensuremath{\to} prop.
happly Constr hnil Constr.
happly Constr (hcons A AS) Term \ensuremath{:\!-} happly (Constr A) AS Term.
\end{verbatim}

\heroSTUDENT{} How about this? This way, we will decocompose the concrete
\texttt{X}, perform the transformation on the \texttt{Arguments}, and
then reapply the \texttt{Constructor} to get the result for \texttt{Y}.

\importantCodeblock{}

\begin{verbatim}
structural_recursion Rec X Y \ensuremath{:\!-}
  refl.headargs X Constructor Arguments,
  hmap Rec Arguments Arguments',
  happly Constructor Arguments' Y.
\end{verbatim}

\importantCodeblockEnd{}

\heroADVISOR{} That is exactly right. You need the symmetric case too but
that's entirely similar. Also, there is another type of concrete terms
in Makam: meta-level functions! It does not make sense to destructure
functions using \texttt{refl.headargs}, so it fails in that case, and we
have to treat them specially:

\importantCodeblock{}

\begin{verbatim}
structural_recursion Rec (X : A \ensuremath{\to} B) (Y : A \ensuremath{\to} B) \ensuremath{:\!-}
  (x:A \ensuremath{\to} structural_recursion Rec x x \ensuremath{\to}
    structural_recursion Rec (X x) (Y x)).
\end{verbatim}

\importantCodeblockEnd{}

\heroSTUDENT{} Ah, I see! Here you \emph{are} actually relying on the
\texttt{typecase} aspect of ad-hoc polymorphism, right? To check if
\texttt{X} and \texttt{Y} are of the meta-level function type.

\heroADVISOR{} Exactly. And you know what, that's all there is to it! So,
we've minimized the boilerplate, and we won't need any adaptation when
we add a new constructor -- even if we make use of all sorts of new and
complicated types.

\heroSTUDENT{} That's right: we do not need to do anything special for the
binding forms we defined, like \texttt{bindmany}\ldots{}. quite a payoff
for a small amount of code! But, wait, isn't
\texttt{structural\_recursion} missing a case: what happens if both
\texttt{X} and \texttt{Y} are uninstantiated unification variables?

\heroADVISOR{} You are correct, it would fail in that case. But in my
experience, it's better to define how to handle unification variables as
needed, in each new structurally recursive predicate. In this case, we
should never get into that situation based on how we have defined
\texttt{typeq}.

\begin{scenecomment}
(Roza and Hagop try out a few examples and convince themselves that this works OK and no endless loops happen when things don't typecheck correctly.)
\end{scenecomment}


  \section{In which our heroes break into song and add more ML
features}\label{in-which-our-heroes-break-into-song-and-add-more-ml-features}

\begin{scenecomment}
(Our heroes need a small break, so they work on a couple of features while improvising on a makam\footnote{Makam is the system of melodic modes of traditional Arabic and Turkish music that is also used in the Greek rembetiko. It is comprised of a set of scales, patterns of melodic development, and rules for improvisation.}. Roza is singing lyrics from the folk songs of her land, and Hagop is playing the oud. Their friend Lambros from the next office over joins them on the kemen\c{c}e.)
\end{scenecomment}

\begin{versy}
``You can skim this chapter / or skip it all the same. \\
It's mostly for completeness / since ML as a name \\
requires some poly-lambdas / as well as ADTs \\
so here we are dotting our i's / and crossing all our t's. \\
\hspace{1em} \vspace{-0.5em} \\
System F is easy / but later we might do \\
some type generalizing / like Hindley-Milner too \\
but if you're feeling tired / I told you just before \\
you can take a mini-break / like Lambros from next door.''
\end{versy}

\begin{verbatim}
tforall : (typ \ensuremath{\to} typ) \ensuremath{\to} typ.
polylam : (typ \ensuremath{\to} term) \ensuremath{\to} term.
polyapp : term \ensuremath{\to} typ \ensuremath{\to} term.
\end{verbatim}

\importantCodeblock{}

\begin{verbatim}
typeof (polylam E) (tforall T) \ensuremath{:\!-} (a:typ \ensuremath{\to} typeof (E a) (T a)).
typeof (polyapp E T) T' \ensuremath{:\!-} typeof E (tforall TF), eq T' (TF T).
\end{verbatim}

\importantCodeblockEnd{}

\begin{versy}
``The algebraic datatypes / caused all sorts of trouble \\
in the previous version / and since it was a double- \\
blind submission process / reviewers quite diverse \\
wonder who's the lunatic / who writes papers in verse.''
\end{versy}

\begin{verbatim}
datadef : type. datatype_bind : (Into: type) \ensuremath{\to} type.
datatype : (Def: datadef) (Rest: datatype_bind program) \ensuremath{\to} program.
\end{verbatim}

\begin{versy}
``The types seem fairly easy / the constructors might be hard. \\
So let's go step-by-step for now / or we'll be here next March. \\
We won't support the poly-types / to keep the system simple, \\
and arguments to constructors? / They'll all take just a single.''
\end{versy}

\begin{verbatim}
\textcolor{white}{\ensuremath{\rightsquigarrow}} \textsf{data nattree = Leaf of nat | Node of (nattree * nattree) ; rest}
\ensuremath{\rightsquigarrow} \textsf{data nattree = [ ("Leaf", nat), ("Node", nattree * nattree) ] ; rest}
\ensuremath{\rightsquigarrow} \textsf{data nattree = [ nat, nattree * nattree ] ; \foreignlanguage{greek}{λ}Leaf. \foreignlanguage{greek}{λ}Node. rest}
\ensuremath{\rightsquigarrow} \textsf{data (\foreignlanguage{greek}{λ}nattree. [nat, nattree * nattree]) ; \foreignlanguage{greek}{λ}nattree. \foreignlanguage{greek}{λ}Leaf. \foreignlanguage{greek}{λ}Node. rest}
\ensuremath{\rightsquigarrow} \textsf{data (\foreignlanguage{greek}{λ}nattree. [nat, nattree * nattree]) ;}
     \textsf{\foreignlanguage{greek}{λ}nattree. bind (\foreignlanguage{greek}{λ}Leaf. bind (\foreignlanguage{greek}{λ}Node. body rest))}
\ensuremath{\rightsquigarrow} datatype (mkdatadef (fun nattree \ensuremath{\Rightarrow} [nat, nattree * nattree]))
     (bind_datatype (fun nattree \ensuremath{\Rightarrow}
       bind (fun leaf \ensuremath{\Rightarrow} bind (fun node \ensuremath{\Rightarrow} body rest))))
\end{verbatim}

\begin{versy}
``Sometimes it just is better / to avoid all those words. \\
Just squint your eyes a little bit; / Hagop will strum some chords.''
\end{versy}

\begin{verbatim}
mkdatadef : (typ \ensuremath{\to} list typ) \ensuremath{\to} datadef.
constructor : type.
bind_datatype : (typ \ensuremath{\to} bindmany constructor A) \ensuremath{\to} datatype_bind A.
\end{verbatim}

\begin{versy}
``We are avoiding GADTs / they're good but up the ante. \\
And we have MetaML to do / (in prose 'cause I'm no Dante.) \\
We're almost there, we need to add / the \texttt{wfprogram} clause. \\
But first we'll need an env. with types that / \texttt{constructor}s expose.''
\end{versy}

\begin{verbatim}
constructor_typ : (DataType: typ) (C: constructor) (ArgType: typ) \ensuremath{\to} prop.
\end{verbatim}

\begin{versy}
``We go through the constructors / populating our new \texttt{prop}. \\
\texttt{DT} stands for datatype / -- the page is just too cropped.''
\end{versy}

\importantCodeblock{}

\begin{verbatim}
wfprogram (datatype (mkdatadef DT_ConstrArgTypes)
                    (bind_datatype DT_Constrs_Rest)) \ensuremath{:\!-}
  (dt:typ \ensuremath{\to} openmany (DT_Constrs_Rest dt) (pfun Constrs Rest \ensuremath{\Rightarrow}
    assumemany (constructor_typ dt) Constrs (DT_ConstrArgTypes dt)
    (wfprogram Rest))).
\end{verbatim}

\importantCodeblockEnd{}

\begin{versy}
``That's it, it's almost over / there's our wf-programs. \\
We can't use the constructors, though / remember \texttt{term}s and \texttt{patt}s ?''
\end{versy}

\begin{verbatim}
constr : (C: constructor) (Arg: term) \ensuremath{\to} term.
patt_constr : (C: constructor) (Arg: patt N N') \ensuremath{\to} patt N N'.
typeof (constr C Arg) Datatype \ensuremath{:\!-}
  constructor_typ Datatype C ArgType, typeof Arg ArgType.
typeof_patt (patt_constr C Arg) Datatype S S' \ensuremath{:\!-}
  constructor_typ Datatype C ArgType, typeof_patt Arg ArgType S S'.
\end{verbatim}

\begin{versy}
``That's all, there's no example / please, download Makam. \\
Trust me: you can run this code / and check that all tests pass.''
\end{versy}

  
  \section{In which our heroes tackle a new level of meta, contexts and
substitutions}\label{in-which-our-heroes-tackle-a-new-level-of-meta-contexts-and-substitutions}

\heroSTUDENT{} I'm fairly confident by now that Makam should be able to handle
the research idea we want to try out. Shall we get to it?

\heroADVISOR{} Yes, it is time. So, what we are aiming to do is add a facility
for type-safe, heterogeneous meta-programming to our object language,
similar to MetaHaskell \citep{mainland2012explicitly}. This way we can
manipulate the terms of a \emph{separate} object language in a type-safe
manner.

\heroSTUDENT{} Exactly. For the research language we have in mind, we aim for
our object language to be a formal logic, so our language will be
similar to Beluga \citep{beluga-main-reference} or VeriML
\citep{veriml-main-reference}. We will also need dependent functions and
pattern-matching over the object language\ldots{} But we don't need to
do all of that; let's just do a basic version for now, and I can do the
rest on my own.

\newcommand\dep[1]{\ensuremath{#1}}
\newcommand\lift[1]{\ensuremath{\langle#1\rangle}}
\newcommand\odash[0]{\ensuremath{\vdash_{\text{o}}}}
\newcommand\wf[0]{\ensuremath{\; \text{wf}}}
\newcommand\aq[1]{\ensuremath{\texttt{aq}(#1)}}
\newcommand\aqopen[1]{\ensuremath{\texttt{aqopen}(#1)}}

\heroADVISOR{} Sounds good. First, let's agree on some terminology, because a
lot of words are getting overloaded a lot. Let us call \emph{objects}
\(o\) any sorts of terms of the object language that we will be
manipulating. And, for lack of a better word, let us call \emph{classes}
\(c\) the ``types'' that characterize those objects through a typing
relation of the form \(\Psi \odash o : c\). It is unfortunate that these
names suggest object-orientation, but this is not the intent.

\heroSTUDENT{} I see what you are saying. Let's keep the objects simple -- to
start, let's just do the terms of the simply typed lambda calculus
(STLC). In that case our classes will just be the types of STLC. The
objects are run-time entities: essentially, our programs will be able to
``compute'' objects. So we need a way to return (or ``lift'') an object
\(o\) as a meta-level value \(\lift{o}\).

\heroADVISOR{} We are getting into many levels of meta -- there's the
metalanguage we're using, Makam; there's the object language we are
encoding, which is now becoming a metalanguage in itself, let's call
that Heterogeneous Meta ML Light (HMML?); and there's the
``object-object'' language that HMML is manipulating. One option would
be to have the object-object language be the full HMML metalanguage
itself, which would lead us to a homogeneous, multi-stage language like
MetaML \citep{metaml-main-reference}. But, I agree, we should keep the
object-object language simple: the STLC will suffice.

\heroSTUDENT{} Great. How about we try to do the standard example of a staged
\texttt{power} function? Here's a rough sketch, where I'm using
\texttt{\textasciitilde{}I} for antiquotation:

\begin{verbatim}
let power (n: onat): < stlc.arrow stlc.onat stlc.onat > =
  match n with
    0 \ensuremath{\Rightarrow} < stlc.lam (fun x \ensuremath{\Rightarrow} 1) >
  | S n' \ensuremath{\Rightarrow} letobj I = power n' in
      < stlc.lam (fun x \ensuremath{\Rightarrow} stlc.mult (stlc.app ~I x) x) >
\end{verbatim}

\heroADVISOR{} It's a plan. So, let's get to it. Should we write some of the
system down on paper first?

\heroSTUDENT{} Yes, that would be very useful. For this example, we will need
the lifting construct \(\lift{\cdot}\) and the \texttt{letobj} form.
Here are their typing rules, which depend on an appropriately defined
typing judgment \(\Psi \odash o : c\) for objects. In our case, this
will initially match the \(\Delta \vdash \hat{e} : \hat{t}\) typing
judgment for STLC (I'll use hats for terms of STLC, to disambiguate them
from terms of HMML). We use \(\dep{i}\) for variables standing for
objects, which we will call \emph{indices}. And we will need a way to
antiquote indices inside STLC terms, which means that we will have to
\emph{extend} the STLC terms as well as their typing judgment
accordingly. We store indices in the \(\Psi\) context, so the STLC
typing judgment will end up having the form
\(\Psi; \Delta \vdash \hat{e} : \hat{t}\). Last, I'll also write down
the evaluation rules of the new constructs, as they are quite simple.

\newcommand\stlce[0]{\hat{e}}
\newcommand\stlct[0]{\hat{t}}
\newcommand\stlc[1]{\hat{#1}}

\vspace{-1.5em}\begin{mathpar}
\begin{array}{ll}
\rulename{Ob-Ob-Syntax}                                                   & \rulename{HMML-Syntax} \\
\stlce  ::= \lambda x:\stlct.\stlce \; | \; \stlce_1 \; \stlce_2 \; | \; x \; | \; n \; | \; \stlce_1 * \stlce_2 \; | \; \textbf{\aq{i}} & e ::= \text{...} \; | \; \lift{\dep{o}} \; | \; \texttt{letobj} \; \dep{i} = \dep{e} \; \texttt{in} \; e' \\
\stlct  ::= \stlct_1 \to \stlct_2 \; | \; \stlc{\text{nat}} & \tau ::= \text{...} \; | \; \lift{\dep{c}} \\
\dep{o} ::= \stlce \hspace{1.5em} \dep{c} ::= \stlct &
\end{array}
\end{mathpar}\begin{mathpar}
\inferrule[Typeof-LiftObj]
          {\dep{\Psi} \odash \dep{o} : \dep{c}}
          {\Gamma; \dep{\Psi} \vdash \lift{\dep{o}} : \lift{\dep{c}}}

\inferrule[Typeof-LetObj]
          {\Gamma; \dep{\Psi} \vdash e : \lift{\dep{c}} \\ \Gamma; \dep{\Psi}, \; \dep{i} : \dep{c} \vdash e' : \tau \\ i \not\in \text{fv}(\tau)}
          {\Gamma; \dep{\Psi} \vdash \texttt{letobj} \; \dep{i} = e \; \texttt{in} \; e' : \tau}

\inferrule[STLC-Typeof-Antiquote]
          {\dep{i} : \stlct \in \Psi}
          {\Psi; \Delta \vdash \aq{\dep{i}} : \stlct}
\end{mathpar}\begin{mathpar}
\inferrule[Eval-LiftObj]
          {\hspace{1em}}{\lift{\dep{o}} \Downarrow \lift{\dep{o}}}

\inferrule[Eval-LetObj]
          {e \Downarrow \lift{\dep{o}} \\ e'[\dep{o}/\dep{i}] \Downarrow v}
          {\texttt{letobj} \; \dep{i} = e \; \texttt{in} \; e' \Downarrow v}

\begin{array}{l}
\rulename{SubstObj} \\
  e[\dep{o}/\dep{i}] = e' \; \text{is defined by} \\ \text{structural recursion, save for:} \\ \hspace{2em} {\aq{\dep{i}}[\stlce/\dep{i}] = \stlce}
\end{array}
\end{mathpar}

The typing rules should be quite simple to transcribe to Makam:

\begin{verbatim}
object, class, index : type.
classof : object \ensuremath{\to} class \ensuremath{\to} prop.
classof_index : index \ensuremath{\to} class \ensuremath{\to} prop.
subst_obj : (I_E: index \ensuremath{\to} term) (O: object) (E_OforI: term) \ensuremath{\to} prop.

liftobj : object \ensuremath{\to} term. liftclass : class \ensuremath{\to} typ.
typeof (liftobj O) (liftclass C) \ensuremath{:\!-} classof O C.

letobj : term \ensuremath{\to} (index \ensuremath{\to} term) \ensuremath{\to} term.
typeof (letobj E EF') T \ensuremath{:\!-}
  typeof E (liftclass C), (i:index \ensuremath{\to} classof_index i C \ensuremath{\to} typeof (EF' i) T).

eval (liftobj O) (liftobj O).
eval (letobj E I_E') V \ensuremath{:\!-}
  eval E (liftobj O), subst_obj I_E' O E', eval E' V.
\end{verbatim}

\heroADVISOR{} Great. I'll add the object language in a separate namespace
prefix -- we can use `\texttt{\%extend}' for going into a namespace --
and I'll just copy-paste our STLC code from earlier on, plus natural
numbers. Let me also add our new antiquote as a new STLC term
constructor!

\begin{verbatim}
%extend stlc.
term : type. typ : type. typeof : term \ensuremath{\to} typ \ensuremath{\to} prop.
...
aq : index \ensuremath{\to} term.
%end.
\end{verbatim}

\heroSTUDENT{} Time to add STLC terms as \texttt{object}s and their types as
\texttt{class}es. We can then give the corresponding rule for
\texttt{classof}. And I think that's it for the typing rules!

\begin{verbatim}
obj_term : stlc.term \ensuremath{\to} object. cls_typ : stlc.typ \ensuremath{\to} class.
classof (obj_term E) (cls_typ T) \ensuremath{:\!-} stlc.typeof E T.
stlc.typeof (stlc.aq I) T \ensuremath{:\!-} classof_index I (cls_typ T).
\end{verbatim}

\begin{scenecomment}
(Hagop transcribes the example from before. Writing out the term takes several lines, so he finds himself
wishing that Makam supported some way to write terms of object languages in their native syntax.)
\end{scenecomment}

\begin{verbatim}
typeof (letrec (bind (fun power \ensuremath{\Rightarrow} body ([ ..long term.. ], power)))) T ?
>> Yes:
>> T := arrow onat (liftclass (cls_typ (stlc.arrow stlc.onat stlc.onat))).
\end{verbatim}

\heroADVISOR{} That's great! The only thing missing to try out an evaluation
example too is implementing \texttt{subst\_obj}. Thanks to
\texttt{structural\_recursion} though, that is very easy:

\begin{verbatim}
subst_obj_aux, subst_obj_cases : [Any]
  (Var: index) (Replacement: object) (Where: Any) (Result: Any) \ensuremath{\to} prop.
subst_obj I_Term O Term_OforI \ensuremath{:\!-}
  (i:index \ensuremath{\to} subst_obj_aux i O (I_Term i) Term_OforI).

subst_obj_aux Var Replacement Where Result \ensuremath{:\!-}
  if (subst_obj_cases Var Replacement Where Result)
  then success
  else (structural_recursion @(subst_obj_aux Var Replacement) Where Result).
subst_obj_cases Var (obj_term Replacement) (stlc.aq Var) Replacement.
\end{verbatim}

\noindent
My definition here is quite subtle, so let me walk you through it.
First, we extend the \texttt{subst\_obj} predicate to work on any type
-- that's what \texttt{subst\_obj\_aux} is for. We set up the structural
recursion, by attempting to see whether the ``essential'' cases actually
apply -- those are captured in the \texttt{subst\_obj\_cases} predicate.
If they don't, that means we should proceed by structural recursion. I
did not mention it before, but the \texttt{@} notation that we used to
treat a polymorphic constant as a term of type \texttt{forall\ A\ T} can
be used with an arbitrary term as well, to assign it such a type if
possible. Finally, the essential case itself is a direct transcription
of the pen-and-paper version.

\heroSTUDENT{} Let me go and reread that a little. (\ldots{}) I think it makes
sense now. Well, is that all? Are we done?

\begin{verbatim}
eval (letrec (bind (fun power \ensuremath{\Rightarrow} body ([ (* .. definition of power *) ],
        (* body of letrec: *) app power (osucc (osucc ozero)))))) V ?
>> Yes!!!
>> V := < obj_term (\foreignlanguage{greek}{λ}x.x * ((\foreignlanguage{greek}{λ}a.a * (\foreignlanguage{greek}{λ}b.1) a) x)) >.
\end{verbatim}

\heroADVISOR{} See, even the Makam REPL is
excited\footnote{We have taken the liberty here to transcribe the result to more meaningful syntax to make it easier to verify.}!
That looks correct, even though there are a lot of administrative
redices. We should be able to fix that with the next kind of object in
our check-list, though: open STLC terms! That way, instead of having
\texttt{power} return an object containing a lambda function, it can
return an open term. Here's how I would write the same example from
before:

\begin{verbatim}
let power_aux (n: onat): < [ stlc.onat ] stlc.onat > =
  match n with
    0 \ensuremath{\Rightarrow} < [x]. 1 >
  | S n' \ensuremath{\Rightarrow} letobj I = power_aux n' in < [x]. stlc.mult ~(I/[x]) x >
\end{verbatim}

\noindent
We have to list out explicitly the variables that an open term depends
on, so that's the \texttt{{[}x{]}.} notation I use. Then, we can use
contextual types \citep{nanevski2008contextual} for the type of those
open terms.

\heroSTUDENT{} Good thing I've already printed the paper out. (\ldots{}) OK,
so it says here that we can use contextual types to record, at the type
level, the context that open terms depend on. So let's say an open
\texttt{stlc.term} of type \(t\) that mentions variables of a \(\Delta\)
context would have a contextual type of the form \([\Delta] t\). This is
some sort of modal typing, with a precise context.

\heroADVISOR{} Right. We now get to the tricky part: referring to variables
that stand for open terms within other terms! You know what those are,
right? Those are Object-level Object-level Meta-variables.

\heroSTUDENT{} My head hurts; I'm getting
\href{https://en.wikipedia.org/wiki/Out_of_memory}{OOM} errors. Maybe
this is easier to implement in Makam than to talk about.

\heroADVISOR{} Maybe so. Well, let me just say this: those variables will
stand for open terms that depend on a specific context \(\Delta\), but
we might use them at a different context \(\Delta'\). We need a
\emph{substitution} \(\sigma\) to go from the context of definition into
the current context. I think writing down the rules on paper will help:

\vspace{-2em}\begin{mathpar}
\begin{array}{ll}
\rulename{Ob-Ob-Syntax} & \\
\dep{o} ::= \text{...} \; | \; [x_1, \text{...}, x_n]. \stlce & \dep{c} ::= \text{...} \; | \; [\stlct_1, \text{...}, \stlct_n] \stlct \\
\stlce ::= \text{...} \; | \; \aqopen{i}/\sigma & 
\sigma ::= [\stlce_1, \text{...}, \stlce_n] \\
\end{array}
\end{mathpar}\begin{mathpar}
\inferrule[Classof-OpenTerm]
          {\Psi; x_1 : \stlct_1, \text{...}, x_n : \stlct_n \vdash \stlce : \stlct}
          {\Psi \odash [x_1, \text{...}, x_n]. \stlce : [\stlct_1, \text{...}, \stlct_n] \stlct}

\inferrule[STLC-TypeOf-AntiquoteOpen]
          {\dep{i} : [\stlct_1, \text{...}, \stlct_n] \stlct \in \Psi \\
           \forall i.\Psi; \Delta \vdash \stlce_i : \stlct_i}
          {\Psi; \Delta \vdash \aqopen{\dep{i}}/[\stlce_1, \text{...}, \stlce_n] : \stlct}
\end{mathpar}\begin{mathpar}
\begin{array}{l}
\rulename{SubstObj} \\
(\aqopen{\dep{i}}/\sigma)[[x_1, \text{...}, x_n]. \stlce / i] =
    \stlce[\stlce_1/x_1, \text{...}, \stlce_n/x_n] 
    \text{ if } \sigma[[x_1, \text{...}, x_n]. \stlce / i] = [\stlce_1, \text{...}, \stlce_n]
\end{array}
\end{mathpar}

\heroSTUDENT{} I've seen that rule for \rulename{SubstObj} before, and it is
still tricky\ldots{} We need to replace the open variables in \(e\)
through the substitution
\(\sigma = [\stlce^*_1, \text{...}, \stlce^*_n]\). However, the terms
\(\stlce^*_1\) through \(\stlce^*_n\) might mention the \(i\) index
themselves, so we first need to apply the top-level substitution for
\(i\) to \(\sigma\) itself! After that, we do replace the open variables
in \(\stlce\).

\heroADVISOR{} I feel that we are getting to the point where it's easier to
write things down in Makam rather than on paper:

\begin{verbatim}
obj_openterm : bindmany stlc.term stlc.term \ensuremath{\to} object.
cls_ctxtyp : list stlc.typ \ensuremath{\to} stlc.typ \ensuremath{\to} class.

%extend stlc.
aqopen : index \ensuremath{\to} list term \ensuremath{\to} term.
typeof (aqopen I ES) T \ensuremath{:\!-}
  classof_index I (cls_ctxtyp TS T), map typeof ES TS.
%end.

classof (obj_openterm XS_E) (cls_ctxtyp TS T) \ensuremath{:\!-}
  openmany XS_E (fun xs e \ensuremath{\Rightarrow} 
    assumemany stlc.typeof xs TS (stlc.typeof e T)).

subst_obj_cases I (obj_openterm XS_E) (stlc.aqopen I ES) Result \ensuremath{:\!-}
  subst_obj_aux I (obj_openterm XS_E) ES ES',
  applymany XS_E ES' Result.
\end{verbatim}

\heroSTUDENT{} I think that's all! This is exciting -- let me try it out:

\begin{verbatim}
(eq _TERM (letrec (bind (fun power \ensuremath{\Rightarrow} body ([
   lam onat (fun n \ensuremath{\Rightarrow}
   case_or_else n (patt_ozero)
     (vbody (liftobj (obj_openterm (bind (fun x \ensuremath{\Rightarrow}
       body (stlc.osucc stlc.ozero))))))
   (case_or_else n (patt_osucc patt_var)
     (vbind (fun n' \ensuremath{\Rightarrow} vbody (
       letobj (app power n') (fun i \ensuremath{\Rightarrow}
       liftobj (obj_openterm (bind (fun x \ensuremath{\Rightarrow}
         body (stlc.mult x (stlc.aqopen i [x])))))))))
   (liftobj (obj_openterm (bind (fun x \ensuremath{\Rightarrow} body stlc.ozero))))
   ))], app power (osucc (osucc ozero)))))),
  typeof _TERM T, eval _TERM V) ?
>> Yes:
>> T := liftclass (cls_ctxtyp (cons stlc.onat nil) stlc.onat),
>> V := liftobj (obj_openterm (bind (fun x \ensuremath{\Rightarrow} body (
          stlc.mult x (stlc.mult x (stlc.osucc stlc.ozero)))))).
\end{verbatim}

\noindent
It works! That's it! I cannot believe how easy this was!

\heroAUDIENCE{} We cannot possibly believe that you are claiming this was
easy!

\heroAUTHOR{} Still, try implementing something like this without a
metalanguage\ldots{} It takes a long time! As a result, it limits our
ability to experiment with and iterate on new language-design ideas.
That's why we started working on Makam. That took a few years, but now
we can at least show a type system like this in 28 pages of a
single-column PDF!

\heroADVISOR{} I wonder where all these voices are coming from?

\heroSTUDENT{} They somehow sound like the ghosts of people who left academia
for industry?

  
  \section{In which our hero Roza implements type generalization, tying
loose
ends}\label{in-which-our-hero-roza-implements-type-generalization-tying-loose-ends}

\begin{versy}
``We mentioned Hindley-Milner / we don't want you to be sad. \\
This paper's going to end soon / and it wasn't all that bad. \\
\hspace{1em}\vspace{-0.5em} \\
(Before we get to that, though / it's time to take a break. \\
If taksims seem monotonous / then put on some Nick Drake.) \\
\hspace{1em}\vspace{-0.5em} \\
We'll gather unif-variables / with structural recursion \\
and if you haven't guessed it yet / we'll get to use reflection.''
\end{versy}

\heroADVISOR{} Let me now show you how to implement type generalization for
polymorphic \texttt{let} in the style of
\citet{damas1984type,hindley1969principal,milner1978theory}. I've done
this
before\footnote{There is existing work that has considered the problem of ML type generalization
in the \lamprolog setting \citep{typgen-lamprolog-1,typgen-lamprolog-2}. Our presentation here follows a different approach based on reflective and generic predicates.},
and I need to leave for home soon, so bear with me for a bit. The gist
will be to reuse the unification support of our metalanguage, capturing
the \emph{metalevel unification variables} and generalizing over them.
That way we will have a very short implementation, and we won't have to
do a deep embedding of unification!

\heroSTUDENT{} So -- you're saying that in \lamprolog, other than reusing the
metalevel function type for implementing object-level substitution, we
can also reuse metalevel unification for the object level as well.

\identNormal

\heroADVISOR{} Exactly! First of all, the typing rule for a generalizing let
looks like this:

\vspace{-1.2em}\begin{mathpar}
\inferrule{\Gamma \vdash e : \tau \\ \vec{a} = \text{fv}(\tau) - \text{fv}(\Gamma) \\ \Gamma, x : \forall \vec{a}.\tau \vdash e' : \tau'}{\Gamma \vdash \text{let} \; x = e \; \text{in} \; e' : \tau'}
\end{mathpar}

We don't have any side-effectful operations, so there is no need for a
value restriction. Transcribing this to Makam is easy, if we assume a
predicate for generalizing the type, for now:

\begin{verbatim}
generalize : (Type: typ) (GeneralizedType: typ) \ensuremath{\to} prop.
let : term \ensuremath{\to} (term \ensuremath{\to} term) \ensuremath{\to} term.
\end{verbatim}

\importantCodeblock{}

\begin{verbatim}
typeof (let E F) T' \ensuremath{:\!-}
  typeof E T, generalize T Tgen,
  (x:term \ensuremath{\to} typeof x Tgen \ensuremath{\to} typeof (F x) T').
\end{verbatim}

\importantCodeblockEnd{}

Now, for generalization itself, we need the following ingredients based
on the typing rule:

\begin{itemize}
\tightlist
\item
  something that picks out free variables from a type, standing for the
  \(\text{fv}(\tau)\) part -- or, in our setting, this should really be
  read as uninstantiated unification variables. Those are the
  Makam-level unification variables that have not been forced to unify
  with concrete types because of the rest of the typing rules.
\item
  something that picks out free variables from the local context: the
  \(\text{fv}(\Gamma)\) part. Again, these are the uninstantiated
  unification variables rather than the free variables. In our case, the
  context \(\Gamma\) is represented by the local \texttt{typeof}
  assumptions that our typing rules add, so we'll have to look at those
  somehow.
\item
  a way to turn something that includes unification variables into a
  \(\forall\) type, corresponding to the \(\forall \vec{a}.\tau\) part.
  This essentially abstracts over a number of variables and uses them as
  the replacement for the unification variables inside \(\tau\).
\end{itemize}

All of those look like things that we should be able to do with our
generic recursion and with the reflective predicates we've been using!
However, to make the implementation simpler, we will generalize over one
variable at a time, instead of all at once -- but that should be
entirely equivalent to what's described in the typing rule.

First, we will define a \texttt{findunif} predicate that returns
\emph{one} unification variable \emph{of the right type} from a term, if
at least one such variable exists. To implement it, we will make use of
a generic operation in the Makam standard library, called
\texttt{generic\_fold}. It is quite similar to
\texttt{structural\_recursion}, but it does a fold through a term,
carrying an accumulator through. Pretty standard, really, and its code
is similar to what we did already for \texttt{structural\_recursion},
with no new surprises.

\begin{verbatim}
findunif_aux : [Any VarType]
  (Var: option VarType) (Current: Any) (Var': option VarType) \ensuremath{\to} prop.
findunif_aux (some Var) _ (some Var).
findunif_aux none (Current : CurrentType) (Result: option VarType) \ensuremath{:\!-}
  refl.isunif Current,
  if (dyn.eq Result (some Current)) then success
  else (eq Result none).
findunif_aux (In: option B) Current Out \ensuremath{:\!-}
  generic_fold @findunif_aux In Current Out.
findunif : [Any VarType] (Search: Any) (Found: VarType) \ensuremath{\to} prop.
findunif Search Found \ensuremath{:\!-} findunif_aux none Search (some Found).
\end{verbatim}

Here, the second rule of \texttt{findunif\_aux} is the important one --
it will only succeed when we encounter a unification variable of the
\emph{same type} \texttt{VarType} as the one we require. This rule
relies on the dynamic \texttt{typecase} aspect of the ad-hoc
polymorphism in \lamprolog, making use of the \texttt{dyn.eq}
standard-library predicate, which has a lax typing:

\begin{verbatim}
dyn.eq : [A B] A \ensuremath{\to} B \ensuremath{\to} prop.
dyn.eq X X.
\end{verbatim}

Though this predicate succeeds for the same case as the standard
\texttt{eq} does (when its two arguments are unifiable), the different
is that \texttt{dyn.eq} only forces the types \texttt{A} and \texttt{B}
to be unified at runtime, rather than statically too. Otherwise, our
rule would only apply when the type \texttt{CurrentType} of the current
unification variable we are visiting already matches the type that we
are searching for, \texttt{VarType}.

With \texttt{findunif} defined, we should already be able to find
\emph{one} (as opposed to all, as described above) uninstantiated
unification variable from a type. Here is an example of its use:

\begin{verbatim}
findunif (arrowmany TS T) (X: typ) ?
>> Yes:
>> X := T.
\end{verbatim}

Now we add a predicate \texttt{replaceunif} that, given a specific
unification variable and a specific term, replaces the variable's
occurrences with the term. This will be needed as part of the
\(\forall \vec{a}.\tau\) operation of the rule. Here I'll need another
reflective predicate, \texttt{refl.sameunif}, that succeeds when its two
arguments are the same exact unification variable; \texttt{eq} would
just unify them, which is not what we want.

\begin{verbatim}
replaceunif : [VarType Any]
  (Which: VarType) (ToWhat: VarType) (Where: Any) (Result: Any) \ensuremath{\to} prop.
replaceunif Which ToWhat Where ToWhat \ensuremath{:\!-}
  refl.isunif Where, refl.sameunif Which Where.
replaceunif Which ToWhat Where Where \ensuremath{:\!-}
  refl.isunif Where, not(refl.sameunif Which Where).
replaceunif Which ToWhat Where Result \ensuremath{:\!-} not(refl.isunif Where),
  structural_recursion @(replaceunif Which ToWhat) Where Result.
\end{verbatim}

Last, we'll need an auxiliary predicate that tells us whether a
unification variable exists within a term. This is easy; it's similar to
the above.

\begin{verbatim}
hasunif : [VarType Any] VarType \ensuremath{\to} bool \ensuremath{\to} Any \ensuremath{\to} bool \ensuremath{\to} prop.
hasunif _ true _ true.
hasunif X false Y true \ensuremath{:\!-} refl.sameunif X Y.
hasunif X In Y Out \ensuremath{:\!-} generic_fold @(hasunif X) In Y Out.

hasunif : [VarType Any] VarType \ensuremath{\to} Any \ensuremath{\to} prop.
hasunif Var Term \ensuremath{:\!-} hasunif Var false Term true.
\end{verbatim}

We are now mostly ready to implement \texttt{generalize}. We'll do this
recursively. The base case is when there are no unification variables
within a type left:

\importantCodeblock{}

\begin{verbatim}
generalize T T \ensuremath{:\!-} not(findunif T (X: typ)).
\end{verbatim}

\importantCodeblockEnd{}

For the recursive case, we will pick out the first unification variable
that we come upon using \texttt{findunif}. We will generalize over it
using \texttt{replaceunif} and then proceed to the rest. Still, there is
a last hurdle: we have to skip over the unification variables that are
in the \(\Gamma\) environment. For the time being, let's assume a
predicate that gives us all the types in the environment, so we can
write the recursive case down:

\importantCodeblock{}

\begin{verbatim}
get_types_in_environment : [A] A \ensuremath{\to} prop.
generalize T Res \ensuremath{:\!-}
  findunif T Var, get_types_in_environment GammaTypes,
  (x:typ \ensuremath{\to} (replaceunif Var x T (T' x), generalize (T' x) (T'' x))),
  if (hasunif Var GammaTypes) then (eq Res (T'' Var))
  else (eq Res (tforall T'')).
\end{verbatim}

\importantCodeblockEnd{}

\identDialog

\heroSTUDENT{} Oh, clever. But what should
\texttt{get\_types\_in\_environment} be? Don't we have to go back and
thread a list of types through our \texttt{typeof} predicate, that we
add a type \texttt{T} to every time we introduce a new
\texttt{typeof\ x\ T} assumption?

\heroADVISOR{} Well, we came this far without significantly rewriting our
rules, so it's a shame to do that now! Maybe we'll be excused to use yet
another reflective predicate that does what we want? There is a way to
get a list of all the local assumptions for a predicate, through the
reflexive predicate \texttt{refl.assume\_get}. It turns out that all the
rules and connectives we have been using are normal \lamprolog terms
like any other, so there's not really much magic to it. And those
assumptions will include all the types in \(\Gamma\)\ldots{}.

\importantCodeblock{}

\begin{verbatim}
get_types_in_environment Assumptions \ensuremath{:\!-}
  refl.assume_get typeof Assumptions.
\end{verbatim}

\importantCodeblockEnd{}

\heroSTUDENT{} Wait. It can't be.

\begin{verbatim}
typeof (let (lam _ (fun x \ensuremath{\Rightarrow} let x (fun y \ensuremath{\Rightarrow} y))) (fun id \ensuremath{\Rightarrow} id)) T ?
>> Yes:
>> T := tforall (fun a \ensuremath{\Rightarrow} arrow a a).
\end{verbatim}

\heroADVISOR{} And yet, it can.


  \section{In which our heroes summarize what they've done and our story
concludes before the credits start
rolling}\label{in-which-our-heroes-summarize-what-theyve-done-and-our-story-concludes-before-the-credits-start-rolling}

\heroSTUDENT{} I feel like we've done a lot here. And some of the things we
did I don't think I've seen in the literature before, but then again,
it's not clear to me what's Makam-specific and what isn't. In any case,
I think a lot of people would find that quickly prototyping their PL
research ideas using this style of higher-order logic programming is
very useful.

\heroADVISOR{} I agree, though it would be hard for somebody to publish a
paper on this. Some of it is novel, some of it is folklore, some of it,
we just did in a pleasant way; and we did also use a couple of
not-so-pleasant hacks. But let's make a list of what's what.

\vspace{-0.5em}

\begin{itemize}
\item
  We defined HOAS encodings of complicated binding forms, including
  mutually recursive definitions and patterns, while only having
  explicit support in our metalanguage for single-variable binding.
  These encodings seem to have been part of PL folklore, but we believe
  that a type like \texttt{bindmany} has never been shown as a reusable
  datatype that \lamprolog makes possible. We have made use of GADTs to
  encode some of these binding structures precisely, which we show are
  supported in \lamprolog through ad-hoc polymorphism. This we believe
  is a novel usage for this \lamprolog feature. Our binding
  constructions should be replicable in other \foreignlanguage{greek}{λ}Prolog implementations
  like Teyjus \citep{teyjus-main-reference,teyjus-2-implementation} and
  in ELPI \citep{elpi-main-reference}.
\item
  We defined a generic predicate to perform structural recursion using a
  very concise definition. It allows us to define structurally recursive
  predicates that only explicitly list out the important cases, in what
  we believe is a novel encoding for the \lamprolog
   setting. Any new definitions, such as constructors or datatypes we
  introduce later, do not need any special provision to be covered by
  the same predicates. They depend on a number of reflective predicates,
  most of which are available in other Prolog and \lamprolog dialects.
  These predicates are used to reflect on the structure of Makam terms
  and to get the list of local assumptions; for the most part, their use
  is limited to predicates that would be part of the standard library,
  not in user code.
\item
  The above encodings are reusable and have been made part of the Makam
  standard library. As a result, we were able to develop the type
  checker for quite an advanced type system, in very few lines of code
  specific to it, using rules that we believe do not, presentation-wise,
  stray far from their pen-and-paper versions. Our development includes
  mutually recursive definitions, polymorphism, datatypes, pattern
  matching, a conversion rule, Hindley-Milner type generalization, and
  staging constructs that allow the computation of contextually typed
  open terms of the simply typed lambda calculus. We are not aware of
  another metalinguistic framework that allows this level of
  expressivity and has been used to encode such type-system features
  with the same level of concision.
\item
  We have also shown that higher-order logic programming allows not just
  meta-level functions to be reused for encoding object-level binding;
  there are also cases where meta-level unification can also be reused
  to encode certain object-level features: for example, doing type
  generalization as in Algorithm W.
\end{itemize}

\heroSTUDENT{} Well, that was very interesting; thank you for working with me
on this!

\heroADVISOR{} I enjoyed this, too. Say, if you want to relax, there's a new
staging of the classic play by \citet{fischer2010play} downtown -- I saw
it yesterday, and it is really good!

\heroSTUDENT{} That's a great idea! You know, I wish there were more plays
like it\ldots{}. Well, good night, and see you on Monday!


  \section{In which are heroes are nowhere to be found, lost in a sea of
references to related
work}\label{in-which-are-heroes-are-nowhere-to-be-found-lost-in-a-sea-of-references-to-related-work}

\identNormal

The \textbf{Racket programming language} was designed to support
creation of new programming languages \citep{racket-manifesto} and has
been used to implement a very wide variety of DSLs serving specific
purposes, including typed languages such as Typed Racket by
\citet{typed-racket-main-reference}. We believe that one of the key
characteristics of the Racket approach to language implementation is the
ability to treat code as data. Makam is largely inspired by this
approach and follows along the same lines; this is not demonstrated in
the present work to a large extent, save for the use of
\texttt{refl.assume\_get}, which turns code (the local assumptions for a
predicate) into data (a reified list of the assumptions). Still, we
believe the presence of first-class binding support in the form of
higher-order abstract syntax makes the \lamprolog setting significantly
different from Racket.

The recent development of a methodology for developing \textbf{type
systems as macros} by \citet{racket-type-systems-as-macros} is a great
validation of the Racket approach and is especially relevant to our use
case, as it has been used to encode type systems similar to ML. The
integration that this methodology provides with the rest of the Racket
ecosystem offers a number of advantages, as does the
\rulename{Turnstile} DSL for writing typing rules close to the
pen-and-paper versions. We do believe that the higher-order logic
programming setting allows for more expressivity and genericity -- for
example, we have used the same techniques to define not only typing
rules but evaluation rules as well, and it is not immediately clear to
us whether examples such as our MetaML fragment would be as easy to
implement in \rulename{Turnstile}. We are exploring an approach similar
to \rulename{Turnstile} to implement a higher-level surface language for
writing typing rules using Makam itself.

Evaluation rules can be implemented using another DSL of the Racket
ecosystem, namely \textbf{PLT Redex} \citep{felleisen2009semantics}. We
believe that staying within the same framework for both aspects offers
other advantages, especially for encoding languages where the two
aspects are more linked, such as dependently typed languages with the
conversion rule. We give one small example of that in the form of the
type synonyms example. We also find that the presence of first-class
substitution support and the support for structural recursion in Makam
offers advantages over PLT Redex.

The \textbf{Spoofax language workbench} \citep{spoofax-main-reference}
offers a series of DSLs for implementing different aspects of a
language, such as parsing, binding, typing and dynamic semantics. We
have found that some of these DSLs have restrictions that would make the
implementation of type systems similar to the ones we demonstrate in the
present work challenging. Our intention with the design of Makam as a
language prototyping tool is for it to be a single expressive core,
where all different aspects of a language can be implemented. The
\textbf{K Framework} \citep{k-framework-main-reference} is a semantics
framework based on rewriting and has been used to implement the dynamic
semantics of a wide variety of languages. It has also been shown to be
effective for the implementation of type systems
\citep{k-framework-type-systems}, treating them as abstract machines
that compute types rather than values. The recent addition of a builtin
unification procedure has made this approach significantly more
effective, allowing the definition of ML type inference; however, the
fact that \lamprolog supports higher-order unification as well renders
it applicable in further situations such as dependently typed systems.
As future work, we are exploring the design of a core calculus to aid in
the bootstrapping of a language such as Makam, and we believe that a
connection with the rewriting-logic core of the K framework will prove
beneficial in this endeavor.


  \identNormal{}

  \section*{Acknowledgments}
  We thank Tej Chajed, Stephen Chang, Ben Greenman, Dale Miller, Gopalan Nadathur, and Cl{\'e}ment Pit-Claudel, as well as the anonymous reviewers of the present paper and of an earlier version, for their very helpful
  feedback and suggestions. This project was started while the first author was at MIT.

  This material is based upon work supported in part by the \grantsponsor{nsf}{NSF}{https://www.nsf.gov/} award
  \grantnum{nsf}{CCF-1217501}.  Any opinions, findings, and conclusions or
  recommendations expressed in this material are those of the authors
  and do not necessarily reflect the views of the National Science
  Foundation.

}

\bibliography{main}

\end{document}
