\section{In which our heroes add parentheses and discover how to do
multiple
binding}\label{in-which-our-heroes-add-parentheses-and-discover-how-to-do-multiple-binding}

\heroSTUDENT{} Still, I feel like we've been playing to the strengths of
\foreignlanguage{greek}{λ}Prolog\ldots{}. Yes, single-variable binding, substitutions, and so on
work nicely, but how about any other form of binding? Say, binding
multiple variables at the same time? We are definitely going to need
that for the language we have in mind. I was under the impression that
HOAS encodings do not work for that -- for example, I was reading
\citet{keuchel2016needle} recently and I remember something to that end.

\heroADVISOR{} That's not really true; having first-class support for
single-variable binders should be enough. But let's try it out, maybe
adding multiple-argument functions for example -- I mean uncurried ones.
Want to give it a try?

\heroSTUDENT{} Let me see. We want the terms to look roughly like this:

\begin{verbatim}
lammany (fun x \ensuremath{\Rightarrow} fun y \ensuremath{\Rightarrow} tuple [y, x])
\end{verbatim}

For the type of \texttt{lammany}, I want to write something like this,
but I know this is wrong.

\begin{verbatim}
lammany : (list term \ensuremath{\to} term) \ensuremath{\to} term.
\end{verbatim}

\heroADVISOR{} Yes, that doesn't quite work. It would introduce a fresh
variable for \texttt{list}s, not a number of fresh variables for
\texttt{term}s. HOAS functions are parametric, too, which means we
cannot even get to the potential elements of the fresh \texttt{list}
inside the \texttt{term}.

\heroSTUDENT{} Right. So I don't know, instead we want to use a type that
stands for \texttt{term\ \ensuremath{\to}\ term},
\texttt{term\ \ensuremath{\to}\ term\ \ensuremath{\to}\ term}, and so on.
Can we write \texttt{term\ \ensuremath{\to}\ ...\ \ensuremath{\to}\ term}?

\heroADVISOR{} Well, not quite, but we have already seen something similar, a
type that roughly stands for \texttt{term\ *\ ...\ *\ term}, and we did
not need anything special for that\ldots{}.

\heroSTUDENT{} You mean the \texttt{list} type?

\heroADVISOR{} Exactly. What do you think about this definition?

\begin{verbatim}
bindmanyterms : type.
bindnil : term \ensuremath{\to} bindmanyterms.
bindcons : (term \ensuremath{\to} bindmanyterms) \ensuremath{\to} bindmanyterms.
\end{verbatim}

\heroSTUDENT{} Hmm. That looks quite similar to lists; the parentheses in
\texttt{cons} are different. \texttt{nil} gets an extra \texttt{term}
argument, too\ldots{}.

\heroADVISOR{} Yes\ldots{} So what is happening here is that \texttt{bindcons}
takes a single argument, introducing a binder; and \texttt{bindnil} is
when we get to the body and don't need any more binders. Maybe we should
name them accordingly.

\heroSTUDENT{} Right, and could we generalize their types? Maybe that will
help me get a better grasp of it. How is this?

\begin{verbatim}
bindmany : type \ensuremath{\to} type \ensuremath{\to} type.
body : Body \ensuremath{\to} bindmany Variable Body.
bind : (Variable \ensuremath{\to} bindmany Variable Body) \ensuremath{\to} bindmany Variable Body.
\end{verbatim}

\heroADVISOR{} This looks great! That is exactly what's in the Makam standard
library, actually. And we can now define \texttt{lammany} using it --
and our example term from before.

\begin{verbatim}
lammany : bindmany term term \ensuremath{\to} term.
lammany (bind (fun x \ensuremath{\Rightarrow} bind (fun y \ensuremath{\Rightarrow} body (tuple [y,x]))))
\end{verbatim}

\heroSTUDENT{} I see. That is an interesting datatype. Is there some reference
about it?

\heroADVISOR{} Not that I know of, at least where it is called out as a
reusable datatype -- though the construction is definitely part of PL
folklore. After I started using this in Makam, I noticed similar
constructions in the wild, for example in MTac \citep{ziliani2013mtac},
for parametric HOAS implementation of telescopes in Coq.

\heroSTUDENT{} Interesting. So how do we work with \texttt{bindmany}? What's
the typing rule like?

\heroADVISOR{} The rule is written like this, and I'll explain what goes into
it:

\begin{verbatim}
arrowmany : list typ \ensuremath{\to} typ \ensuremath{\to} typ.
typeof (lammany F) (arrowmany TS T) \ensuremath{:\!-}
  openmany F (fun xs body \ensuremath{\Rightarrow} assumemany typeof xs TS (typeof body T)).
\end{verbatim}

\heroSTUDENT{} Let me see if I can read this\ldots{} \texttt{openmany} somehow
gives you fresh variables \texttt{xs} for the binders, plus the
\texttt{body} of the \texttt{lammany}; and then the
\texttt{assumemany\ typeof} part is what corresponds to extending the
\(\Gamma\) context with multiple typing assumptions?

\heroADVISOR{} Yes, and then we typecheck the \texttt{body} in that local
context that includes the fresh variables and the typing assumptions.
But let's do one step at a time. \texttt{openmany} is simple; we iterate
through the nested binders, introducing one fresh variable at a time. We
also substitute each bound variable for the current fresh variable, so
that when we get to the body, it only uses the fresh variables we
introduced.

\begin{verbatim}
openmany : bindmany A B \ensuremath{\to} (list A \ensuremath{\to} B \ensuremath{\to} prop) \ensuremath{\to} prop.
openmany (body Body) Q \ensuremath{:\!-} Q [] Body.
openmany (bind F) Q \ensuremath{:\!-} (x:A \ensuremath{\to} openmany (F x) (fun xs \ensuremath{\Rightarrow} Q (x :: xs))).
\end{verbatim}

\heroSTUDENT{} I see. I guess \texttt{assumemany} is similar, introducing one
assumption at a time?

\begin{verbatim}
assumemany : (A \ensuremath{\to} B \ensuremath{\to} prop) \ensuremath{\to} list A \ensuremath{\to} list B \ensuremath{\to} prop \ensuremath{\to} prop.
assumemany P [] [] Q \ensuremath{:\!-} Q.
assumemany P (X :: XS) (T :: TS) Q \ensuremath{:\!-} (P X T \ensuremath{\to} assumemany P XS TS Q).
\end{verbatim}

\heroADVISOR{} Yes, exactly! Just a note, though -- \lamprolog typically does
not allow the definition of \texttt{assumemany}, where a non-concrete
predicate like \texttt{P\ X\ Y} is used as an assumption, because of
logical reasons. Makam allows this form statically and so does ELPI
\citep{elpi-main-reference}, another \lamprolog implementation, though
there are instantiations of \texttt{P} that will fail at
run-time\footnote{The logical reason why this is not allowed in \lamprolog is that it violates the property of existence of uniform proofs; see \citet{assumemany-issue} for more information. An example of a goal that will fail at runtime is anything that includes a logical-or (denoted as ``\texttt{;}'') as an assumption, like ``\texttt{(typeof X T1; typeof X T2) \ensuremath{\to} ...}''.}.

\heroSTUDENT{} I see. But in that case we could just manually inline
\texttt{assumemany\ typeof} instead, so that's not a big problem, just
more verbose. But can I try our typing rule out?

\begin{verbatim}
typeof (lammany (bind (fun x \ensuremath{\Rightarrow} bind (fun y \ensuremath{\Rightarrow} body (tuple [y, x]))))) T ?
>> Yes:
>> T := arrowmany [T1, T2] (product [T2, T1]).
\end{verbatim}

\noindent
Great, I think I got the hang of this. Let me try to see if I can add a
multiple-argument application construct \texttt{appmany} and its
evaluation rules.

\begin{verbatim}
appmany : term \ensuremath{\to} list term \ensuremath{\to} term.
typeof (appmany E ES) T \ensuremath{:\!-}
  typeof E (arrowmany TS T), map typeof ES TS.
eval (appmany E ES) V \ensuremath{:\!-}
  eval E (lammany XS_E'), map eval ES VS, (...).
\end{verbatim}

\noindent
How can I do simultaneous substitution of all of the \texttt{XS} for
\texttt{VS}?

\heroADVISOR{} You'll need another standard-library predicate for
\texttt{bindmany}, which iteratively uses HOAS function application to
perform a number of substitutions:

\begin{verbatim}
applymany : bindmany A B \ensuremath{\to} list A \ensuremath{\to} B \ensuremath{\to} prop.
applymany (body B) [] B.
applymany (bind F) (X :: XS) B \ensuremath{:\!-} applymany (F X) XS B.
\end{verbatim}

\begin{verbatim}
eval (appmany E ES) V \ensuremath{:\!-}
  eval E (lammany XS_E'), map eval ES VS,
  applymany XS_E' VS E'', eval E'' V.
\end{verbatim}

\heroSTUDENT{} I see, that makes sense. Can I ask you something that worries
me, though -- all these fancy higher-order abstract binders, how do
we\ldots{} make them concrete? Say, how do we print them?

\heroADVISOR{} That's actually quite easy. We just add a concrete name to
them. A plain old \texttt{string}. Our typing rules etc. do not care
about it, but we could use it for parsing concrete syntax into our
abstract binding syntax, or for pretty-printing\ldots{}. All those are
stories for another time, though; let's just say that we could have
defined \texttt{bind} with an extra \texttt{string} argument,
representing the concrete name; and then \texttt{openmany} would just
ignore it.

\begin{verbatim}
bind : string \ensuremath{\to} (Var \ensuremath{\to} bindmany Var Body) \ensuremath{\to} bindmany Var Body.
\end{verbatim}

\heroSTUDENT{} Interesting. I would like to see more about this, but maybe
some other time. I thought of another thing that could be challenging:
mutually recursive \texttt{let\ rec}s?

\heroADVISOR{} Sure. Let's take this term for example:

\begin{verbatim}
let rec f = f_def and g = g_def in body
\end{verbatim}

\noindent
If we write this in a way where the binding structure is explicit, we
would bind \texttt{f} and \texttt{g} simultaneously and then write the
definitions and the body in that scope:

\begin{verbatim}
letrec (fun f \ensuremath{\Rightarrow} fun g \ensuremath{\Rightarrow} ([f_def, g_def], body))
\end{verbatim}

\heroSTUDENT{} I think I know how to do this then! How does this look?

\begin{verbatim}
letrec : bindmany term (list term * term) \ensuremath{\to} term.
\end{verbatim}

\heroADVISOR{} Exactly! Want to try writing the typing rules?

\heroSTUDENT{} Maybe something like this?

\begin{verbatim}
typeof (letrec XS_DefsBody) T' \ensuremath{:\!-}
  openmany XS_DefsBody (fun xs (defs, body) \ensuremath{\Rightarrow}
    assumemany typeof xs TS (map typeof defs TS),
    assumemany typeof xs TS (typeof body T')).
\end{verbatim}

\heroADVISOR{} Almost! You have used the syntax we use for writing rule
premises in the \texttt{fun} argument of \texttt{openmany}; the Makam
grammar only allows that with the syntactic form \texttt{pfun} instead,
which is used to write anonymous predicates. Since this \texttt{pfun}
argument will be a predicate and can thus perform computation, you are
also able to destructure parameters like you did here on
\texttt{(defs,\ body)} -- that doesn't work for normal functions in the
general case, since they need to treat arguments parametrically. This
works by performing unification of the parameter with the given term --
so \texttt{defs} and \texttt{body} need to be capitalized so that they
are understood to be unification variables.

\begin{verbatim}
typeof (letrec XS_DefsBody) T' \ensuremath{:\!-}
  openmany XS_DefsBody (pfun XS (Defs, Body) \ensuremath{\Rightarrow}
    assumemany typeof XS TS (map typeof Defs TS),
    assumemany typeof XS TS (typeof Body T')).
\end{verbatim}

\heroSTUDENT{} Ah, I
see\footnote{There is a subtlety here, having to do with the free variables that a unification variable
can capture. In \lamprolog, a unification variable is allowed to capture all the free variables in scope at the
point where it is introduced, as well as any variables it is explicitly applied to. \texttt{Defs} and \texttt{Body} are introduced as unification variables when we get to execute the \texttt{pfun}; otherwise, all unification variables used in a rule get introduced when we check whether the rule fires. As a result, \texttt{Defs} and \texttt{Body} can capture the \texttt{xs} variables that \texttt{openmany} introduces, whereas \texttt{T'} cannot. In \lamprolog terms, the \texttt{pfun} notation of Makam desugars to existential quantification of any (capitalized) unification variables that are mentioned while destructuring an argument, like the variables \texttt{Defs} and \texttt{Body}.}.
Let me ask you something though: one thing I noticed with our
representation of \texttt{letrec} is that we have to be careful so that
the number of binders matches the number of definitions we give. Our
typing rules disallow that, but I wonder if there's a way to have a more
accurate representation for \texttt{letrec} which includes that
requirement?

\heroADVISOR{} Funny you should ask that\ldots{} Let me tell you a story.
