%% For double-blind review submission, w/o CCS and ACM Reference (max submission space)
\documentclass[acmsmall,review,anonymous]{acmart}\settopmatter{printfolios=true,printccs=false,printacmref=false}
%% For double-blind review submission, w/ CCS and ACM Reference
%\documentclass[acmsmall,review,anonymous]{acmart}\settopmatter{printfolios=true}
%% For single-blind review submission, w/o CCS and ACM Reference (max submission space)
%\documentclass[acmsmall,review]{acmart}\settopmatter{printfolios=true,printccs=false,printacmref=false}
%% For single-blind review submission, w/ CCS and ACM Reference
%\documentclass[acmsmall,review]{acmart}\settopmatter{printfolios=true}
%% For final camera-ready submission, w/ required CCS and ACM Reference

\usepackage[utf8]{inputenc}
\usepackage[greek,english]{babel}
\usepackage{booktabs}
\usepackage{subcaption}
\usepackage{alltt}
\usepackage{xspace}
\usepackage{mathpartir}
\usepackage{indentfirst}
\usepackage{hyperref}
\usepackage{color}
\usepackage{fancyvrb}

\bibliographystyle{shared/ACM-Reference-Format}
\citestyle{acmauthoryear}   %% For author/year citations

%% Journal information
%% Supplied to authors by publisher for camera-ready submission;
%% use defaults for review submission.
\acmJournal{PACMPL}
\acmVolume{1}
\acmNumber{CONF} % CONF = POPL or ICFP or OOPSLA
\acmArticle{1}
\acmYear{2018}
\acmMonth{1}
\acmDOI{} % \acmDOI{10.1145/nnnnnnn.nnnnnnn}
\startPage{1}

%% Copyright information
%% Supplied to authors (based on authors' rights management selection;
%% see authors.acm.org) by publisher for camera-ready submission;
%% use 'none' for review submission.
\setcopyright{none}
%\setcopyright{acmcopyright}
%\setcopyright{acmlicensed}
%\setcopyright{rightsretained}
%\copyrightyear{2018}           %% If different from \acmYear

\begin{document}

\title{Prototyping a Functional Language using Higher-Order Logic Programming}
\subtitle{A Functional Pearl on learning the ways of \lamprolog/Makam}

\author{Antonis Stampoulis}
\affiliation{
  \institution{Originate Inc.}
  \city{New York}
  \state{New York}
}
\email{antonis.stampoulis@gmail.com}

\author{Adam Chlipala}
\affiliation{
  \department{CSAIL}
  \institution{MIT}
  \city{Cambridge}
  \state{Massachusetts}
}
\email{adamc@csail.mit.edu}

%% -- Macro definitions
\newcommand\TODO[0]{\textbf{TODO}}
\newcommand\lamprolog[0]{\foreignlanguage{greek}{λ}Prolog\xspace}
\newcommand\fomega[0]{F$\omega$\xspace}
\renewenvironment{verbatim}{\begin{quote}\begin{alltt}}{\end{alltt}\end{quote}}
\newenvironment{codequote}{\begin{quote}\begin{alltt}}{\end{alltt}\end{quote}}
\newcommand\hide[1]{}
\newcommand\tightlist[0]{\itemsep1pt\parskip0pt\parsep0pt}
\renewcommand\thesection{\textbf{CHAPTER \arabic{section}}}
\renewcommand\thesubsection{\textbf{SECTION \arabic{section}.\arabic{subsection}}}
\newcommand\hero[1]{\textit{#1}.}
\newcommand\heroTODO[0]{\textbf{\textcolor[rgb]{0.75,0.00,0.00}{TODO}}.}
\newcommand\heroSTUDENT[0]{\hero{HAGOP}}
\newcommand\heroADVISOR[0]{\hero{ROZA}}
\newcommand\heroAUDIENCE[0]{\hero{AUDIENCE}}
\newcommand\heroAUTHOR[0]{\hero{ANONYMOUS AUTHOR}}
\newenvironment{scenecomment}{\em\noindent}{}
\newcommand\identDialog[0]{\setlength{\leftskip}{1em}\setlength{\parindent}{-1em}}
\newcommand\identNormal[0]{\setlength{\leftskip}{0em}\setlength{\parindent}{0em}}
\newenvironment{normalident}{\identNormal}{\identDialog}

% =====
% used for highlighting from pandoc
\DefineVerbatimEnvironment{Highlighting}{Verbatim}{commandchars=\\\{\}}
\newenvironment{Shaded}{\begin{quote}}{\end{quote}}
\newcommand{\KeywordTok}[1]{\textcolor[rgb]{0.00,0.44,0.13}{\textbf{{#1}}}}
\newcommand{\DataTypeTok}[1]{\textcolor[rgb]{0.56,0.13,0.00}{{#1}}}
\newcommand{\DecValTok}[1]{\textcolor[rgb]{0.25,0.63,0.44}{{#1}}}
\newcommand{\BaseNTok}[1]{\textcolor[rgb]{0.25,0.63,0.44}{{#1}}}
\newcommand{\FloatTok}[1]{\textcolor[rgb]{0.25,0.63,0.44}{{#1}}}
\newcommand{\CharTok}[1]{\textcolor[rgb]{0.25,0.44,0.63}{{#1}}}
\newcommand{\StringTok}[1]{\textcolor[rgb]{0.25,0.44,0.63}{{#1}}}
\newcommand{\CommentTok}[1]{\textcolor[rgb]{0.38,0.63,0.69}{\textit{{#1}}}}
\newcommand{\OtherTok}[1]{\textcolor[rgb]{0.00,0.44,0.13}{{#1}}}
\newcommand{\AlertTok}[1]{\textcolor[rgb]{1.00,0.00,0.00}{\textbf{{#1}}}}
\newcommand{\FunctionTok}[1]{\textcolor[rgb]{0.02,0.16,0.49}{{#1}}}
\newcommand{\RegionMarkerTok}[1]{{#1}}
\newcommand{\ErrorTok}[1]{\textcolor[rgb]{1.00,0.00,0.00}{\textbf{{#1}}}}
\newcommand{\NormalTok}[1]{{#1}}
% =====

\begin{abstract}
We demonstrate how the framework of \emph{higher-order logic programming}, as exemplified
in the \lamprolog language design, is a prime vehicle for rapid prototyping of
implementations for programming languages with sophisticated type systems. We present the
literate development of a type checker for a language with a number of complicated
features, culminating in a standard ML-style core with polymorphic algebraic datatypes and
type generalization, extended with dependently typed constructs that are generic over a
separately defined language of dependent indices. We add each new feature in sequence,
without requiring changes to existing code; as a result, we are able to extend both the ML
core and the universe of dependent indices independently from each other. Scaling the
higher-order logic programming approach to this setting required us to develop novel
approaches to challenges like complex variable binding patterns in object languages,
performing generic structural traversals of code, and reusing the unification procedure of
the metalanguage as much as possible. For our development, we make use of Makam, a new
implementation of \lamprolog, which we introduce in tutorial style as part of our
literate development.
\end{abstract}

%% 2012 ACM Computing Classification System (CSS) concepts
%% Generate at 'http://dl.acm.org/ccs/ccs.cfm'.
\begin{CCSXML}
<ccs2012>
<concept>
<concept_id>10011007.10011006.10011008</concept_id>
<concept_desc>Software and its engineering~General programming languages</concept_desc>
<concept_significance>500</concept_significance>
</concept>
<concept>
<concept_id>10003456.10003457.10003521.10003525</concept_id>
<concept_desc>Social and professional topics~History of programming languages</concept_desc>
<concept_significance>300</concept_significance>
</concept>
</ccs2012>
\end{CCSXML}

\ccsdesc[500]{Software and its engineering~General programming languages}
\ccsdesc[300]{Social and professional topics~History of programming languages}
%% End of generated code

% \keywords{keyword1, keyword2, keyword3}  %% \keywords are mandatory in final camera-ready submission

\maketitle

{
  \setlength{\parskip}{3pt}
  \renewcommand{\labelitemi}{\textendash}

  % Hanging Indent
  \identDialog{}

  % No Indent
  % \identNormal{}
  
  \section{Where our heroes set out on a road to prototype a type
system}\label{where-our-heroes-set-out-on-a-road-to-prototype-a-type-system}

\hero{HAGOP (Student)} (\ldots{}) So yes, I think my next step should be
writing a toy implementation of the type system we have in mind, so that
we can try out some examples and see what works and what does not.

\hero{ROZA (Advisor)} Definitely -- trying out examples will help you
refine your ideas, too.

\heroSTUDENT{} Let's see, though; we have the simply typed \foreignlanguage{greek}{λ}-calculus, some ML
core features, a staging construct, and contextual types like in
\citet{nanevski2008contextual}\ldots{} I guess I will need a few weeks?

\heroADVISOR{} That sounds like a lot. Why don't you use some kind of
metalanguage to implement the prototype?

\heroSTUDENT{} You mean a tool like Racket \citep{racket-manifesto}, PLT Redex
\citep{felleisen2009semantics}, the K Framework
\citep{k-framework-main-reference} or Spoofax
\citep{spoofax-main-reference}?

\heroADVISOR{} Yes, all of those should be good choices. I was thinking though
that we could use higher-order logic programming\ldots{} it's a
formalism that is well-suited to what we want to do, since we will need
all sorts of different binding constructs, and the type system we are
thinking about is quite advanced.

\heroSTUDENT{} Oh, so you mean \foreignlanguage{greek}{λ}Prolog \citep{miller1988overview} or LF
\citep{lf-main-reference}.

\heroADVISOR{} Yes. Actually, a few years back a friend of mine worked on this
new implementation of \foreignlanguage{greek}{λ}Prolog just for this purpose -- rapid prototyping
of languages. It's called Makam. It should be able to handle what we
have in mind nicely, and we should not need to spend more than a few
hours on it!

\heroSTUDENT{} Sounds great! Anything I can read up on Makam then?

\heroADVISOR{} Not much, unfortunately\ldots{} But I know the language and its
standard library quite well, so let's work on this together; it'll be
fun. I'll show you how things work along the way!

\begin{scenecomment}
(Our heroes install Makam from --elided for blind reviewing-- and figure out how to run the REPL.)
\end{scenecomment}

  
  \section{Where our heroes get the easy stuff out of the
way}\label{where-our-heroes-get-the-easy-stuff-out-of-the-way}

\heroSTUDENT{} OK, let's just start with the simply typed lambda calculus to
see how this works. Let's define just the basics: application, lambda
abstraction, and the arrow type.

\heroADVISOR{} Right. We will first need to define the two meta-types for
these two sorts:

\begin{verbatim}
term : type.
typ : type.
\end{verbatim}

\heroSTUDENT{} Oh, so \texttt{type} is the reserved keyword for the meta-level
kind of types, and we'll use \texttt{typ} for our object-level types?

\heroADVISOR{} Exactly. And let's do the easy constructors first:

\begin{verbatim}
app : term \ensuremath{\to} term \ensuremath{\to} term.
arrow : typ \ensuremath{\to} typ \ensuremath{\to} typ.
\end{verbatim}

\heroSTUDENT{} So we add constructors to a type at any point; we do not list
them out when we define it like in Haskell. But how about lambdas? I
have heard that \foreignlanguage{greek}{λ}Prolog supports higher-order abstract syntax
\citep{hoas-standard-reference}, which should make those really easy to
add, too, right?

\heroADVISOR{} Yes, functions at the meta level are parametric, so they
correspond exactly to single-variable binding -- they cannot perform any
computation, and thus we do not have to worry about exotic terms. So
this works fine for Church-style lambdas:

\begin{verbatim}
lam : typ \ensuremath{\to} (term \ensuremath{\to} term) \ensuremath{\to} term.
\end{verbatim}

\heroSTUDENT{} I see. And how about the typing judgment,
\(\Gamma \vdash e : \tau\) ?

\heroADVISOR{} Let's add a predicate for that. Only, no \(\Gamma\), there is
an implicit context of assumptions that will serve the same purpose.

\begin{verbatim}
typeof : term \ensuremath{\to} typ \ensuremath{\to} prop.
\end{verbatim}

\heroSTUDENT{} Let me see if I can get the typing rule for application. I know
that in Prolog we write the conclusion of a rule first, and the premises
follow the \texttt{:-} sign. Does something like this work?

\begin{verbatim}
typeof (app E1 E2) T' :- typeof E1 (arrow T T'), typeof E2 T.
\end{verbatim}

\heroADVISOR{} Yes! That's exactly right. Makam uses capital letters for
unification variables.

\heroSTUDENT{} I will need help with the lambda typing rule, though. What's
the equivalent of extending the context as in \(\Gamma, \; x : \tau\) ?

\heroADVISOR{} Simple: we introduce a fresh constructor for terms and a new
typing rule for it:

\begin{verbatim}
typeof (lam T1 E) (arrow T1 T2) :- (x:term \ensuremath{\to} typeof x T1 \ensuremath{\to} typeof (E x) T2).
\end{verbatim}

\heroSTUDENT{} Hmm, so \texttt{x:term\ \ensuremath{\to}} introduces the fresh
constructor standing for the new variable, and
\texttt{typeof\ x\ T1\ \ensuremath{\to}} introduces the new assumption?
Oh, and we need to get to the body of the lambda function in order to
type-check it, so that's why you do \texttt{E\ x}.

\heroADVISOR{} Yes. Note that the introductions are locally scoped, so they
are only in effect for the recursive call.

\heroSTUDENT{} Makes sense. So do we have a type checker already? Can we run
queries?

\heroADVISOR{} We do! Observe:

\begin{verbatim}
typeof (lam _ (fun x \ensuremath{\Rightarrow} x)) T ?
>> Yes:
>> T := arrow T1 T1.
\end{verbatim}

\heroSTUDENT{} Cool! So underscores for unification variables we don't care
about and \texttt{?} for queries. But wait, last time I implemented
unification in my toy STLC implementation it was easy to make it go into
an infinite loop with \(\lambda x. x x\). How does that work here?

\heroADVISOR{} Well, you were missing the occurs-check. \foreignlanguage{greek}{λ}Prolog unification
includes it:

\begin{verbatim}
typeof (lam _ (fun x \ensuremath{\Rightarrow} app x x)) T' ?
>> Impossible.
\end{verbatim}

\heroSTUDENT{} Right. So let's see, what else can we do? How about adding
tuples to our language? Can we use something like a polymorphic list?

\heroADVISOR{} Sure, \foreignlanguage{greek}{λ}Prolog has polymorphic types and higher-order
predicates. Here's how lists are defined in the standard library:

\begin{verbatim}
list : type \ensuremath{\to} type.
nil : list A. cons : A \ensuremath{\to} list A \ensuremath{\to} list A.

map : (A \ensuremath{\to} B \ensuremath{\to} prop) \ensuremath{\to} list A \ensuremath{\to} list B \ensuremath{\to} prop.
map P nil nil.
map P (cons X XS) (cons Y YS) :- P X Y, map P XS YS.
\end{verbatim}

\heroSTUDENT{} Nice! I guess that's why you wanted to go with \foreignlanguage{greek}{λ}Prolog for
doing this instead of LF, since you cannot use polymorphism there?

\heroADVISOR{} Indeed. We will see, once we figure out what our language
should be, one thing we could do is transcribe our definitions to LF,
and then we could even use Beluga \citep{beluga-main-reference} to do
all of our metatheoretic proofs. Or maybe we could use Abella
\citep{abella-main-reference} directly.

\heroSTUDENT{} Sounds good. So, for tuples, this should work:

\begin{verbatim}
tuple : list term \ensuremath{\to} term. product : list typ \ensuremath{\to} typ.
typeof (tuple ES) (product TS) :- map typeof ES TS.
\end{verbatim}

\heroADVISOR{} Yes, and we can use syntactic sugar for \texttt{cons} and
\texttt{nil} too:

\begin{verbatim}
typeof (lam _ (fun x \ensuremath{\Rightarrow} lam _ (fun y \ensuremath{\Rightarrow} tuple [x, y]))) T ?
>> Yes:
>> T := arrow T1 (arrow T2 (product [T1, T2])).
\end{verbatim}

\heroSTUDENT{} So how about evaluation? Can we write the big-step semantics
too?

\heroADVISOR{} Why not? Let's add a predicate and do the two easy rules:

\begin{verbatim}
eval : term \ensuremath{\to} term \ensuremath{\to} prop.
eval (lam T F) (lam T F).
eval (tuple ES) (tuple VS) :- map eval ES VS.
\end{verbatim}

\heroSTUDENT{} OK, let me try my hand at the beta-redex case. I'll just do
call-by-value. I know that when using HOAS, function application is
exactly capture-avoiding substitution, so this should be fine:

\begin{verbatim}
eval (app E E') V'' :- eval E (lam _ F), eval E' V', eval (F V') V''.
\end{verbatim}

\heroADVISOR{} Exactly! See, I told you this would be easy!

  
  \section{Where our heroes add parentheses and discover how to do
multiple
binding}\label{where-our-heroes-add-parentheses-and-discover-how-to-do-multiple-binding}

\heroSTUDENT{} Still, I feel like we've been playing to the strengths of
\foreignlanguage{greek}{λ}Prolog\ldots{}. Yes, single-variable binding, substitutions, and so on
work nicely, but how about any other form of binding? Say, binding
multiple variables at the same time? We are definitely going to need
that for the language we have in mind. I was under the impression that
HOAS encodings do not work for that -- for example, I was reading
\citet{keuchel2016needle} recently and I remember something to that end.

\heroADVISOR{} That's not really true; having first-class support for
single-variable binders should be enough. But let's try it out, maybe
adding multiple-argument functions for example -- I mean uncurried ones.
Want to give it a try?

\heroSTUDENT{} Let me see. We want the terms to look roughly like this:

\begin{verbatim}
lammany (fun x \ensuremath{\Rightarrow} fun y \ensuremath{\Rightarrow} tuple [y, x])
\end{verbatim}

For the type of \texttt{lammany}, I want to write something like this,
but I know this is wrong.

\begin{verbatim}
lammany : (list term \ensuremath{\to} term) \ensuremath{\to} term.
\end{verbatim}

\heroADVISOR{} Yes, that doesn't quite work. It would introduce a fresh
variable for \texttt{list}s, not a number of fresh variables for
\texttt{term}s. HOAS functions are parametric, too, which means we
cannot even get to the potential elements of the fresh \texttt{list}
inside the \texttt{term}.

\heroSTUDENT{} Right. So I don't know, instead we want to use a type that
stands for \texttt{term\ \ensuremath{\to}\ term},
\texttt{term\ \ensuremath{\to}\ term\ \ensuremath{\to}\ term}, and so on.
Can we write \texttt{term\ \ensuremath{\to}\ ...\ \ensuremath{\to}\ term}?

\heroADVISOR{} Well, not quite, but we have already defined something similar,
a type that roughly stands for \texttt{term\ *\ ...\ *\ term}, and we
did not need anything special for that\ldots{}.

\heroSTUDENT{} You mean the \texttt{list} type?

\heroADVISOR{} Exactly. What do you think about this definition?

\begin{verbatim}
bindmanyterms : type.
bindnil : term \ensuremath{\to} bindmanyterms.
bindcons : (term \ensuremath{\to} bindmanyterms) \ensuremath{\to} bindmanyterms.
\end{verbatim}

\heroSTUDENT{} Hmm. That looks quite similar to lists; the parentheses in
\texttt{cons} are different. \texttt{nil} gets an extra \texttt{term}
argument, too\ldots{}.

\heroADVISOR{} Yes\ldots{} So what is happening here is that \texttt{bindcons}
takes a single argument, introducing a binder; and \texttt{bindnil} is
when we get to the body and don't need any more binders. Maybe we should
name them accordingly.

\heroSTUDENT{} Right, and could we generalize their types? Maybe that will
help me get a better grasp of it. How is this?

\begin{verbatim}
bindmany : type \ensuremath{\to} type \ensuremath{\to} type.
body : Body \ensuremath{\to} bindmany Variable Body.
bind : (Variable \ensuremath{\to} bindmany Variable Body) \ensuremath{\to} bindmany Variable Body.
\end{verbatim}

\heroADVISOR{} This looks great! That is exactly what's in the Makam standard
library, actually. And we can now define \texttt{lammany} using it --
and our example term from before.

\begin{verbatim}
lammany : bindmany term term \ensuremath{\to} term.
lammany (bind (fun x \ensuremath{\Rightarrow} bind (fun y \ensuremath{\Rightarrow} body (tuple [y,x]))))
\end{verbatim}

\heroSTUDENT{} I see. That is an interesting datatype. Is there some reference
about it?

\heroADVISOR{} Not that I know of, at least where it is called out as a
reusable datatype -- though the construction is definitely part of PL
folklore. After I started using this in Makam, I noticed similar
constructions in the wild, for example in MTac \citep{ziliani2013mtac},
for parametric HOAS implementation of telescopes in Coq.

\heroSTUDENT{} Interesting. So how do we work with \texttt{bindmany}? What's
the typing rule like?

\heroADVISOR{} The rule is written like this, and I'll explain what goes into
it:

\begin{verbatim}
arrowmany : list typ \ensuremath{\to} typ \ensuremath{\to} typ.
typeof (lammany F) (arrowmany TS T) :-
  openmany F (fun xs body \ensuremath{\Rightarrow}
    assumemany typeof xs TS (typeof body T)).
\end{verbatim}

\heroSTUDENT{} Let me see if I can read this\ldots{} \texttt{openmany} somehow
gives you fresh variables \texttt{xs} for the binders, plus the
\texttt{body} of the \texttt{lammany}; and then the
\texttt{assumemany\ typeof} part is what corresponds to extending the
\(\Gamma\) context with multiple typing assumptions?

\heroADVISOR{} Yes, and then we typecheck the \texttt{body} in that local
context that includes the fresh variables and the typing assumptions.
But let's do one step at a time. \texttt{openmany} is simple; we iterate
through the nested binders, introducing one fresh variable at a time. We
also substitute each bound variable for the current fresh variable, so
that when we get to the body, it only uses the fresh variables we
introduced.

\begin{verbatim}
openmany : bindmany A B \ensuremath{\to} (list A \ensuremath{\to} B \ensuremath{\to} prop) \ensuremath{\to} prop.
openmany (body Body) Q :- Q [] Body.
openmany (bind F) Q :-
  (x:A \ensuremath{\to} openmany (F x) (fun xs \ensuremath{\Rightarrow} Q (x :: xs))).
\end{verbatim}

\heroSTUDENT{} I see. I guess \texttt{assumemany} is similar, introducing one
assumption at a time?

\begin{verbatim}
assumemany : (A \ensuremath{\to} B \ensuremath{\to} prop) \ensuremath{\to} list A \ensuremath{\to} list B \ensuremath{\to} prop \ensuremath{\to} prop.
assumemany P [] [] Q :- Q.
assumemany P (X :: XS) (T :: TS) Q :- (P X T \ensuremath{\to} assumemany P XS TS Q).
\end{verbatim}

\heroADVISOR{} Yes, exactly! Just a note, though -- \lamprolog typically does
not allow the definition of \texttt{assumemany}, where a non-concrete
predicate like \texttt{P\ X\ Y} is used as an assumption, because of
logical reasons. Makam is more lax, and so is ELPI, another recent
\lamprolog implementation, and allows this form statically, though there
are instantiations of \texttt{P} that will fail at run-time.

\heroNEEDFEEDBACK{}
\todo{I am considering abandoning `assumemany` in favor of using a specialized `assume\_types\_of`. We could avoid the above explanation in that case; also this would make the development in this chapter work in other \lamprolog dialects, and also keep our \lamprolog experts from raising an issue at departing from the logical reading of \lamprolog so early in the text. The specialized predicate would look as follows.}

\begingroup\color{todo}

\begin{verbatim}
assume_types_of : list term \ensuremath{\to} list typ \ensuremath{\to} prop \ensuremath{\to} prop.
assume_types_of [] [] Q :- Q.
assume_types_of (X :: XS) (T :: TS) Q :- (typeof X T \ensuremath{\to} assume_types_of P XS TS Q).
\end{verbatim}

\endgroup

\heroSTUDENT{} I see. But we could just manually inline
\texttt{assumemany\ typeof} instead, so that's not a big problem, just
more verbose. But can I try our typing rule out?

\begin{verbatim}
typeof (lammany (bind (fun x \ensuremath{\Rightarrow} bind (fun y \ensuremath{\Rightarrow} body (tuple [y, x]))))) T ?
>> Yes:
>> T := arrowmany [T1, T2] (product [T2, T1]).
\end{verbatim}

Great, I think I got the hang of this. We could definitely add a
multiple-argument application construct \texttt{appmany} or define the
rules for \texttt{eval} for these. But that would be easy; we can do it
later. Something that worries me, though -- all these fancy higher-order
abstract binders, how do we \ldots{} make them concrete? Say, how do we
print them?

\heroADVISOR{} That's actually quite easy. We just add a concrete name to
them. A plain old \texttt{string}. Our typing rules etc. do not care
about it, but we could use it for parsing concrete syntax into our
abstract binding syntax, or for pretty-printing\ldots{}. Let's not get
into that for the time being, but let's just say that we could have
defined \texttt{bind} with an extra \texttt{string} argument,
representing the concrete name; and then \texttt{openmany} would just
ignore it.

\begin{verbatim}
bind : string \ensuremath{\to} (Var \ensuremath{\to} bindmany Var Body) \ensuremath{\to} bindmany Var Body.
\end{verbatim}

\heroSTUDENT{} Interesting. I would like to see more about this, but maybe
some other time. I thought of another thing that could be challenging:
mutually recursive \texttt{let\ rec}s?

\heroADVISOR{} Sure. Let's take this term for example:

\begin{verbatim}
let rec f = f_def and g = g_def in body
\end{verbatim}

If we write this in a way where the binding structure is explicit, we
would bind \texttt{f} and \texttt{g} simultaneously and then write the
definitions and the body in that scope:

\begin{verbatim}
letrec (fun f \ensuremath{\Rightarrow} fun g \ensuremath{\Rightarrow} ([f_def, g_def], body))
\end{verbatim}

\heroSTUDENT{} I think I know how to do this then! How does this look?

\begin{verbatim}
letrec : bindmany term (list term * term) \ensuremath{\to} term.
\end{verbatim}

\heroADVISOR{} Exactly! Want to try writing the typing rules?

\heroSTUDENT{} Maybe something like this?

\begin{verbatim}
typeof (letrec XS_DefsBody) T' :-
  openmany XS_DefsBody (fun xs (defs, body) \ensuremath{\Rightarrow}
    assumemany typeof xs TS (map typeof defs TS),
    assumemany typeof xs TS (typeof body T')).
\end{verbatim}

\heroADVISOR{} Almost! The parser isn't clever enough to tell that the
predicate argument to \texttt{openmany} is, in fact, a predicate, so we
can't use the normal predicate syntax for it. We can use the syntactic
form \texttt{pfun} for writing anonymous predicates instead. Since this
will be a predicate, you are also able to destructure parameters like
you did here on \texttt{(defs,\ body)} -- that doesn't work for normal
functions in the general case, since they need to treat arguments
parametrically. This works by performing unification of the parameter
with the given term -- so \texttt{defs} and \texttt{body} need to be
unification variables. So we have to write the rule like this:

\begin{verbatim}
typeof (letrec XS_DefsBody) T' :-
  openmany XS_DefsBody (pfun xs (Defs, Body) \ensuremath{\Rightarrow}
    assumemany typeof xs TS (map typeof Defs TS),
    assumemany typeof xs TS (typeof Body T')).
\end{verbatim}

\heroNEEDFEEDBACK{}
\todo{There is a subtle thing going on here, having to do with the free variables that a unification variable
can capture. In the above, `Defs` and `Body` can capture the free variables in `xs`; but `T'` cannot, since it is introduced
at the top-level, rather than nested inside the call to `openmany` as `Defs` and `Body` are. I previously had an explanation like
this: "A unification variable is allowed to capture all the free variables in scope at the
point where it is introduced. By default, all unification variables used in a rule get introduced when we
check whether the rule fires. But here certain unification variables are
introduced when `openmany` gets to use the `pfun` argument and has therefore introduced all the needed fresh variables,
so they can capture the free variables introduced by `openmany`." Too confusing/only needed for existing \lamprolog users?}

\heroSTUDENT{} Ah, I see. One thing I noticed with our representation of
\texttt{letrec} is that we have to be careful so that the number of
binders matches the number of definitions we give. Our typing rules
disallow that, but I wonder if there's a way to have a more accurate
representation for \texttt{letrec} which includes that requirement?

\heroADVISOR{} Funny you should ask that\ldots{} Let me tell you a story.

  
  \section{Where the legend of the GADTs and the Ad-Hoc Polymorphism is
recounted}\label{where-the-legend-of-the-gadts-and-the-ad-hoc-polymorphism-is-recounted}

\identNormal\it

Once upon a time, our republic lacked one of the natural wonders that it
is now well-known for, and which is now regularly enjoyed by tourists
and inhabitants alike. I am talking of course about the Great Arboretum
of Dangling Trees, known as GADTs for short. Then settlers from the
far-away land of the Dependency started coming to the republic, and
started speaking of Lists that Knew Their Length, of Terms that Knew
Their Types, of Collections of Elements that were Heterogeneous, and
about the other magical beings of their home. And they set out to build
a natural environment for these beings on the republic, namely the GADTs
that we know and love, to remind them of home a little. And their work
was good and was admired by many.

A long time passed, and dispatches from another far-away land came to
the republic, written by authors whose names are now lost in the sea of
anonymity, and I fear might forever remain so. And the dispatches went
something like this.

\rm

\heroAUTHOR{} \ldots{} In my land of \lamprolog that I speak of, the type
system is a subset of System F\(_\omega\) that should be familiar to you
-- the simply typed lambda calculus, plus prenex polymorphism, plus
simple type constructors of the form
\texttt{type\ *\ ...\ *\ type\ \ensuremath{\to}\ type}. There is also a
limited form of rank-2 polymorphism, allowing types of the form
\texttt{forall\ A\ T}, which are inhabited by unapplied polymorphic
constants through the notation \texttt{@foo}. There is a \texttt{prop}
sort for propositions, which is a normal type, but also a bit special:
its terms are not just values but are also computations, activated when
queried upon.

However, the language of this land has a distinguishing feature, called
Ad-Hoc Polymorphism. You see, the different rules that define a
predicate in our language can \emph{specialize} their type arguments.
This can be used to define polymorphic predicates that behave
differently for different types, like this, where we are essentially
doing a \texttt{typecase} and we choose a rule depending on the
\emph{type} of the argument (as opposed to its value):

\begin{verbatim}
print : [A] A \ensuremath{\to} prop.
print (I: int) \ensuremath{:\!-} (... code for printing integers ...)
print (S: string) \ensuremath{:\!-} (... code for printing strings ...)
\end{verbatim}

The local dialects Teyjus
\citep{teyjus-main-reference,teyjus-2-implementation} and Makam include
this feature, while it is not encountered in other dialects like ELPI
\citep{elpi-main-reference}. In the Makam dialect in particular, type
variables are understood to be parametric by default. In order to make
them ad-hoc and allow specializing them in rules, we need to denote them
using the \texttt{{[}A{]}} notation.

Of course, this feature has both to do with the statics as well as the
dynamics of our language: and while dynamically it means something akin
to a \texttt{typecase}, statically, it means that rules might specialize
their type variables, and this remains so for type-checking the whole
rule.

But alas! Is it not type specialization during pattern matching that is
an essential feature of the GADTs of your land? Maybe that means that we
can use Ad-Hoc Polymorphism not just to do \texttt{typecase} but also to
work with GADTs in our land? Consider this! The venerable List that
Knows Its Length:

\begin{verbatim}
zero : type. succ : type \ensuremath{\to} type.
vector : type \ensuremath{\to} type \ensuremath{\to} type.
vnil : vector A zero.
vcons : A \ensuremath{\to} vector A N \ensuremath{\to} vector A (succ N).
\end{verbatim}

And now for the essential \texttt{vmap}:

\begin{verbatim}
vmap : [N] (A \ensuremath{\to} B \ensuremath{\to} prop) \ensuremath{\to} vector A N \ensuremath{\to} vector B N \ensuremath{\to} prop.
vmap P vnil vnil.
vmap P (vcons X XS) (vcons Y YS) \ensuremath{:\!-} P X Y, vmap P XS YS.
\end{verbatim}

In each rule, the first argument already specializes the \texttt{N} type
-- in the first rule to \texttt{zero}, in the second, to
\texttt{succ\ N}. And so erroneous rules that do not respect this
specialization would not be accepted as well-typed sayings in our
language:

\begin{verbatim}
vmap P vnil (vcons X XS) \ensuremath{:\!-} ...
\end{verbatim}

And we should note that in this usage of Ad-Hoc Polymorphism for GADTs,
it is only the increased precision of the statics that we care about.
Dynamically, the rules for \texttt{vmap} can perform normal term-level
unification and only look at the constructors \texttt{vnil} and
\texttt{vcons} to see whether each rule applies, rather than relying on
the \texttt{typecase} aspects we spoke of before.

Coupling this with the binding constructs that I talked to you earlier
about, we can build new magical beings, like the \emph{Bind that Knows
Its Length}:

\begin{verbatim}
vbindmany : (Var: type) (N: type) (Body: type) \ensuremath{\to} type.
vbody : Body \ensuremath{\to} vbindmany Var zero Body.
vbind : (Var \ensuremath{\to} vbindmany Var N Body) \ensuremath{\to} vbindmany Var (succ N) Body.
\end{verbatim}

(Whereby I am using notation of the Makam dialect in my definition of
\texttt{vbindmany} that allows me to name parameters, purely for the
purposes of increased clarity.)

In the \texttt{openmany} version for \texttt{vbindmany}, the rules are
exactly as before, though the static type is more precise:

\begin{verbatim}
vopenmany : [N] vbindmany Var N Body \ensuremath{\to} (vector Var N \ensuremath{\to} Body \ensuremath{\to} prop) \ensuremath{\to} prop.
vopenmany (vbody Body) Q \ensuremath{:\!-} Q vnil Body.
vopenmany (vbind F) Q \ensuremath{:\!-}
  (x:A \ensuremath{\to} vopenmany (F x) (fun xs \ensuremath{\Rightarrow} Q (vcons x xs))).
\end{verbatim}

We can also showcase the \emph{Accurate Encoding of the Letrec}:

\begin{verbatim}
vletrec : vbindmany term N (vector term N * term) \ensuremath{\to} term.
\end{verbatim}

And that is the way that the land of \lamprolog supports GADTs, without
needing the addition of any feature, all thanks to the existing support
for Ad-Hoc Polymorphism.

\identDialog

  
  \section{Where our hero Hagop adds pattern matching on his
own}\label{where-our-hero-hagop-adds-pattern-matching-on-his-own}

\begin{scenecomment}
(Our hero Roza had a meeting with another student, so Hagop is back at his
office, trying to work out on his own how to encode patterns. He is fairly
confident at this point that having explicit support for single-variable
binding is enough to model most complicated forms of binding, especially when making use of
polymorphism and GADTs.)
\end{scenecomment}

\identNormal
\heroSTUDENT{} OK, so let's implement simple patterns and pattern-matching
like in ML\ldots{} First let's determine the right binding structure.
For a branch like:

\begin{verbatim}
| cons(hd, tl) \ensuremath{\to} ... hd .. tl ...
\end{verbatim}

the pattern introduces 2 variables, \texttt{hd} and \texttt{tl}, which
the body of the branch can refer to. But we can't really refer to those
variables in the pattern itself, at least for simple patterns\ldots{}
(There are cases where that's not the case, like or-patterns in some ML
dialects, or in dependent pattern matching, where consequent uses of the
same variable perform an exact match rather than unification, but let's
not worry about those cases right now, I am sure we could handle them
too if we needed to.) So basically, once we know what variables a
pattern introduces, we can bind them and give them names all at the same
time. So we need something like this:

\begin{verbatim}
branch(pattern, bind [# variables of P].body)
\end{verbatim}

Or, in Makam, the above branch will be something like:

\begin{verbatim}
branch(
  patt_cons patt_var patt_var,
  bind (fun hd \ensuremath{\Rightarrow} bind (fun tl \ensuremath{\Rightarrow} body (.. hd .. tl ..))))
\end{verbatim}

One way that I have started thinking about binding is that it is just a
way to introduce a notion of sharing into abstract syntax trees, so that
we can refer to the same thing a number of times. And basically for
these simple patterns, the sharing happens from the side of the pattern
into the branch body, not within the pattern itself.

With the above, I am thinking that the type of \texttt{branch} should be
something like:

\begin{verbatim}
branch : (Pattern: patt N) (Vars_Body: vbindmany term N term) \ensuremath{\to} ...
\end{verbatim}

Wait, before I get into the weeds let me just set up some things. First,
let's add a simple base type, say \texttt{nat}s, to have something to
work with as an example. I'll prefix their names with \texttt{o} for
``object language'', so as to avoid ambiguity. And I will also add a
\texttt{case\_or\_else} construct, standing for a single-branch
pattern-match construct. It should be easy to extend to a
multiple-branch construct, but I want to keep things as simple as
possible. I'll inline what I had written for \texttt{branch} above into
the definition of \texttt{case\_or\_else}.

\begin{verbatim}
onat : typ. ozero : term. osucc : term \ensuremath{\to} term.
typeof ozero onat. typeof (osucc N) onat :- typeof N onat.
eval ozero ozero. eval (osucc E) (osucc V) :- eval E V.
\end{verbatim}

\begin{verbatim}
case_or_else :
  (Scrutinee: term)
  (Patt: patt N) (Vars_Body: vbindmany term N term)
  (Else: term) \ensuremath{\to} term.
\end{verbatim}

Now for the typing rule -- it will be something like this:

\begin{verbatim}
typeof (case_or_else Scrutinee Pattern Vars_Body Else) BodyT :-
  typeof Scrutinee T,
  typeof_patt Pattern T VarTypes,
  vopenmany Vars_Body (pfun vars body \ensuremath{\Rightarrow}
    vassumemany typeof vars VarTypes (typeof body BodyT)),
  typeof Else BodyT.
\end{verbatim}

Right, so when checking a pattern, we'll have to determine both what
type of scrutinee it matches, as well as the types of the variables that
it contains. We will also need \texttt{vassumemany} that is just like
\texttt{assumemany} from before, but takes \texttt{vector} arguments
instead of \texttt{list}.

\begin{verbatim}
typeof_patt : [N] patt N \ensuremath{\to} typ \ensuremath{\to} vector typ N \ensuremath{\to} prop.
vassumemany : [N] (A \ensuremath{\to} B \ensuremath{\to} prop) \ensuremath{\to} vector A N \ensuremath{\to} vector B N \ensuremath{\to} prop \ensuremath{\to} prop.
(...)
\end{verbatim}

Now, I can just go ahead and define the patterns, together with their
typing relation, \texttt{typeof\_patt}.

Let me just work one by one for each pattern.

\begin{verbatim}
patt_var : patt (succ zero).
typeof_patt patt_var T (vcons T vnil).
\end{verbatim}

OK, that's how we'll write pattern variables, introducing a single
variable of a specific \texttt{typ} into the body of the branch. And the
following should be good for the \texttt{onat}s I defined earlier.

\begin{verbatim}
patt_ozero : patt zero.
typeof_patt patt_ozero onat vnil.

patt_osucc : patt N \ensuremath{\to} patt N.
typeof_patt (patt_osucc P) onat VarTypes :- typeof_patt P onat VarTypes.
\end{verbatim}

Wildcard patterns will match any value, and should not introduce a
variable into the body of the branch.

\begin{verbatim}
patt_wild : patt zero.
typeof_patt patt_wild T vnil.
\end{verbatim}

OK, and let's do patterns for our n-tuples\ldots{} I guess I'll need a
type for lists of patterns too.

\begin{verbatim}
patt_tuple : pattlist N \ensuremath{\to} patt N.
typeof_patt (patt_tuple PS) (product TS) VarTypes :-
  typeof_pattlist PS TS VarTypes.
pattlist : (N: type) \ensuremath{\to} type.
pnil : patt zero.
pcons : patt N \ensuremath{\to} pattlist N' \ensuremath{\to} pattlist (N + N').
\end{verbatim}

Uh-oh\ldots{} don't think I can do that
\texttt{N\ +\ N\textquotesingle{}} really. In this \texttt{pcons} case,
my pattern basically looks like \texttt{(P,\ ...PS)}; and I want the
overall pattern to have as many variables as \texttt{P} and \texttt{PS}
combined. But the GADTs support in \lamprolog seems to be quite basic, I
do not think there's any notion of type-level functions like
plus\ldots{}

However\ldots{} maybe I can work around that, if I add change
\texttt{patt} to include an ``accumulator'' argument, say
\texttt{NBefore}. Each constructor for patterns will now define how many
pattern variables it adds to that accumulator, yielding \texttt{NAfter},
rather than defining how many pattern variables it includes\ldots{} like
this:

\begin{verbatim}
patt, pattlist : (NBefore: type) (NAfter: type) \ensuremath{\to} type.
patt_var : patt N (succ N).
patt_ozero : patt N N.
patt_osucc : patt N N' \ensuremath{\to} patt N N'.
patt_wild : patt N N.
patt_tuple : pattlist N N' \ensuremath{\to} patt N N'.

pnil : pattlist N N.
pcons : patt N N' \ensuremath{\to} pattlist N' N'' \ensuremath{\to} pattlist N N''.
\end{verbatim}

Yes. I think that should work. I got a little editing to do in my
existing predicates to use this representation instead. For top-level
patterns, we should always start with the accumulator being
\texttt{zero}\ldots{}

\begin{verbatim}
case_or_else :
  (Scrutinee: term)
  (Patt: patt zero N) (Vars_Body: vbindmany term N term)
  (Else: term) \ensuremath{\to} term.
\end{verbatim}

I think I'll also have to change \texttt{typeof\_patt}, so that it
includes an accumulator argument of its own:

\begin{verbatim}
typeof_patt : [NBefore NAfter]
  patt NBefore NAfter \ensuremath{\to} typ \ensuremath{\to}
  vector typ NBefore \ensuremath{\to} vector typ NAfter \ensuremath{\to} prop.

typeof (case_or_else Scrutinee Pattern Vars_Body Else) BodyT :-
  typeof Scrutinee T,
  typeof_patt Pattern T vnil VarTypes,
  vopenmany Vars_Body (pfun vars body \ensuremath{\Rightarrow}
    vassumemany typeof vars VarTypes (typeof body BodyT)),
  typeof Else BodyT.
\end{verbatim}

Alright, let's proceed to the typing rules for patterns themselves:

\begin{verbatim}
typeof_patt patt_var T VarTypes VarTypes' :-
  vsnoc VarTypes T VarTypes'.
\end{verbatim}

OK, here I need \texttt{vsnoc} to add an element to the end of a vector.
That should yield the correct order for the types of pattern variables;
I am visiting the pattern left-to-right after all.

\begin{verbatim}
vsnoc : [N] vector A N \ensuremath{\to} A \ensuremath{\to} vector A (succ N) \ensuremath{\to} prop.
vsnoc vnil Y (vcons Y vnil).
vsnoc (vcons X XS) Y (vcons X XS_Y) :- vsnoc XS Y XS_Y.
\end{verbatim}

The rest should be easy to adapt\ldots{}

\begin{scenecomment}
(Our hero finishes adapting the rest of the rules for \texttt{typeof\_patt},
which are available in the unabridged version of this story. After
trying a few queries, he is convinced that his implementation of
pattern matching works well. The next day, he shows his work to Roza.)
\end{scenecomment}

\identDialog

\heroADVISOR{} That's great! I understand that this was a little tricky, but
still, it was not too bad, right? Actually, I know of one thing that is
quite simple to do: the evaluation rule. On paper we typically write
something roughly like:

\vspace{-1.5em}\begin{mathpar}
\inferrule{\texttt{match}(p, v) \leadsto \sigma \\ e[\sigma/xs] \Downarrow v'}
          {\texttt{case\_or\_else}(p, xs.e, e') \Downarrow v'}

\inferrule{\texttt{match}(p, v) \not\leadsto \\ e' \Downarrow v'}
          {\texttt{case\_or\_else}(v, p \mapsto xs.e, e') \Downarrow v'}
\end{mathpar}

So \texttt{match} tries to unify a pattern with a term, and yields a
substitution \(\sigma\) for the pattern variables if successful, which
is then applied to the body of the branch. If there is no \texttt{match}
to be found, then we use the \texttt{else} branch.

\heroSTUDENT{} Hmm\ldots{} so do we need two predicates, one for the case
where the \texttt{match} is successful, and one to check that a pattern
\emph{does not} match a scrutinee?

\heroADVISOR{} Actually we could have a single \texttt{match} predicate. And
we can use the logical \texttt{if-then-else} construct for the two
cases, which we have not seen so far. Let me write down the evaluation
rule, and I'll explain:

\begin{verbatim}
match : [NBefore NAfter]
  (Pattern: patt NBefore NAfter) (Scrutinee: term)
  (SubstBefore: vector term NBefore) (SubstAfter: vector term NAfter) \ensuremath{\to}
  prop.
eval (case_or_else Scrutinee Pattern Body Else) V' :-
  eval Scrutinee V,
  if (match Pattern V vnil Subst)
  then (vapplymany Body Subst Body', eval Body' V')
  else (eval Else V').
\end{verbatim}

The \texttt{if-then-else} construct behaves as follows: when there is at
least one way to prove the condition, it proceeds to the \texttt{then}
branch, otherwise it goes to the \texttt{else} branch. Pretty standard
really. It is one thing that the Prolog cut statement, \texttt{!}, is
useful for, but I find that using cut makes for less readable code.
\citet{kiselyov05backtracking} is worth reading for alternatives to the
cut statement and the semantics of
\texttt{if}-\texttt{then}-\texttt{else} and \texttt{not} in logic
programming, and Makam follows that paper closely.

\heroSTUDENT{} I see. Now, I noticed a \texttt{vapplymany} predicate -- what
is that?

\heroADVISOR{} That is a standard library predicate. It is used to perform
simultaneous substitution for all the variables in our multiple binding
type, \texttt{vbindmany}. Or another way to say it, it's the equivalent
of HOAS function application for \texttt{vbindmany}:

\begin{verbatim}
vapplymany : [N] vbindmany Var N Body \ensuremath{\to} vector Var N \ensuremath{\to} Body \ensuremath{\to} prop.
vapplymany (vbody Body) vnil Body.
vapplymany (vbind F) (vcons E ES) Body :- vapplymany (F E) ES Body.
\end{verbatim}

\heroSTUDENT{} I see\ldots{} OK, I think I know how to continue. I will write
a few of the \texttt{match} rules down.

\begin{verbatim}
match patt_var X Subst Subst' :- vsnoc Subst X Subst'.
match patt_wild X Subst Subst.
match patt_ozero ozero Subst Subst.
match (patt_osucc P) (osucc V) Subst Subst' :-
  match P V Subst Subst'.
\end{verbatim}

\begin{scenecomment}
(Our heroes also write down the rules for multiple patterns and tuples, which are
available in the unabridged version of this story.)
\end{scenecomment}

\heroNEEDFEEDBACK{}
\todo{I have switched to a more incremental presentation here, adding more explanation, since it seems that we lost a few reviewers here last time. However, this makes this section too long; it looks like the centerpiece of the paper, but it shouldn't be. (For example, there's a blog post with a similar encoding from a few years back.) Between this and algebraic datatypes, we will definitely need to cut something in order to have more in-depth explanations, but I'm not sure what. Thoughts?}


  \section{Where our heroes break into song and add more ML
features}\label{where-our-heroes-break-into-song-and-add-more-ml-features}

\begin{scenecomment}
(Our heroes need a small break, so they work on a couple of features while improvising on a makam\footnote{Makam is the system of melodic modes used in traditional Arabic and Turkish music and in the Greek rembetiko, comprised of a set of scales, patterns of melodic development, and rules for improvisation.}. Roza is singing, and Hagop is playing the oud.)
\end{scenecomment}

\begin{verse}
``Explicit System F polymorphism is easy, at some point we'll do Hindley-Milner too. \\
Types are well-formed by construction, an extra `$\vdash \tau \; \text{wf}$' judgment we won't do.''
\end{verse}

\begin{verbatim}
forall : (typ \ensuremath{\to} typ) \ensuremath{\to} typ.
lamt : (typ \ensuremath{\to} term) \ensuremath{\to} term.
appt : term \ensuremath{\to} typ \ensuremath{\to} term.
typeof (lamt E) (forall T) :- (a:typ \ensuremath{\to} typeof (E a) (T a)).
typeof (appt E T) (TF T) :- typeof E (forall TF).
\end{verbatim}

\begin{verse}
``We are now adding top-level programs, to get into datatype declarations. \\
We would rather do modules, but those would need quite a bit of deliberation. \\
And we still have contextual types to do, those will require our full attention.''
\end{verse}

\begin{verbatim}
program : type.
wfprogram : program \ensuremath{\to} prop.

let : term \ensuremath{\to} (term \ensuremath{\to} program) \ensuremath{\to} program.
wfprogram (let E P) :- typeof E T, (x:term \ensuremath{\to} typeof x T \ensuremath{\to} wfprogram (P x)).

main : term \ensuremath{\to} program.
wfprogram (main E) :- typeof E _.
\end{verbatim}

\begin{center}\rule{0.5\linewidth}{\linethickness}\end{center}

\heroADVISOR{} I think we are ready to do polymorphic algebraic datatypes now.
We'll add a type for type constructors, like \texttt{list}, dependent on
their arity; and a type for the constructors of a datatype. Also a type
for constructor declarations, dependent on the number of constructors
they introduce:

\begin{verbatim}
typeconstructor : type \ensuremath{\to} type.
constructor : type.

ctor_declaration : type \ensuremath{\to} type.
nil : ctor_declaration unit.
cons : list typ \ensuremath{\to} ctor_declaration T \ensuremath{\to} ctor_declaration (constructor * T).
\end{verbatim}

\heroSTUDENT{} Oh, so each constructor takes multiple arguments. Great. So
datatype declarations would be something like this:

\begin{verbatim}
datatype_declaration : type \ensuremath{\to} type \ensuremath{\to} type.
datatype_declaration : 
  (typeconstructor Arity \ensuremath{\to} dbind typ Arity (ctor_declaration Ctors)) \ensuremath{\to}
  datatype_declaration Arity Ctors.
datatype :
  datatype_declaration Arity Ctors \ensuremath{\to}
  (typeconstructor Arity \ensuremath{\to} dbind constructor Ctors program) \ensuremath{\to} program.
\end{verbatim}

\heroADVISOR{} Right, so when declaring a datatype, we introduce a
\texttt{typeconstructor} variable so that we can refer to the type
recursively when we declare our constructors. And we also have access to
the right number of polymorphic variables, matching the \texttt{Arity}
of the constructor. I like how you split out the declaration of the type
itself from the ``rest of the program'' part, since this could become
unwieldy otherwise.

\heroSTUDENT{} That's what I thought too. And I see why you made the type
constructors carry their arities -- to keep types well-formed by
construction. In order to be able to actually refer to the type
constructors, though, don't we need a type former:

\begin{verbatim}
tconstr : typeconstructor T \ensuremath{\to} subst typ T \ensuremath{\to} typ.
\end{verbatim}

\heroADVISOR{} We do. Also keep in mind that in a richer type system, we
probably would need an extra kind-checking predicate. But this will do
for now. Let's just make sure this is fine -- I'll write down the
declaration of binary trees, to make sure we're not missing anything,
and typecheck it with Makam.

\begin{verbatim}
%type (datatype_declaration
  (fun tree \ensuremath{\Rightarrow} dbindnext (fun a \ensuremath{\Rightarrow} dbindbase
    [ (* leaf *) [],
      (* node *) [tconstr tree [a], a, tconstr tree [a]] ]))).
>> (...) : datatype_declaration (typ * unit) (constructor * constructor * unit)
\end{verbatim}

\heroSTUDENT{} Looks good. Should we proceed to the actual well-formedness for
datatype declarations? I think we will need a predicate to keep track of
information about a constructor -- which datatype it belongs to and what
arguments it expects. That way we can carry that information in the
assumptions context.

\begin{verbatim}
constructor_info :
  typeconstructor Arity \ensuremath{\to} constructor \ensuremath{\to} dbind typ Arity (list typ) \ensuremath{\to} prop.
\end{verbatim}

\heroADVISOR{} Yes, and we are mostly ready otherwise:

\begin{verbatim}
wfprogram (datatype (datatype_declaration ConstructorDecls) Program') :-
  (dt:(typeconstructor T) \ensuremath{\to} ([PolyTypes]
    openmany (ConstructorDecls dt) (pfun tvars constructor_decls \ensuremath{\Rightarrow} (
      constructor_polytypes constructor_decls tvars PolyTypes)),
    openmany (Program' dt) (pfun constructors program' \ensuremath{\Rightarrow}
      assumemany (constructor_info dt) constructors PolyTypes
      (wfprogram program')))).
\end{verbatim}

\heroSTUDENT{} This is a tricky piece of code. Let me stare at it for a while.
(\ldots{}) What is this new \texttt{constructor\_polytypes} predicate?

\heroADVISOR{} I'm using that in order to re-abstract over the type
variables\ldots{}. See, in the constructor declaration, we've introduced
a number of type variables. We need to abstract over them, in order to
get the polymorphic type of each constructor for the rest of the
program. Note that \texttt{PolyTypes} can't capture the type variables
\texttt{tvars} we introduce.

\heroSTUDENT{} I think I got it. Let me try to implement it.

\heroADVISOR{} Here's a hint.

\begin{verbatim}
(x:typ \ensuremath{\to} y:typ \ensuremath{\to} applymany PolyType [x, y] (arrow y x)) ?
>> Yes:
>> PolyType := dbindnext (fun x \ensuremath{\Rightarrow} dbindnext (fun y \ensuremath{\Rightarrow} dbindbase (arrow y x))).
\end{verbatim}

\begin{scenecomment}
(After a few attempts, Hagop comes up with the following definition.)
\end{scenecomment}

\begin{verbatim}
constructor_polytypes : [Arity Ctors PolyTypes]
  ctor_declaration Ctors \ensuremath{\to} subst typ Arity \ensuremath{\to}
  subst (dbind typ Arity (list typ)) PolyTypes \ensuremath{\to} prop.

constructor_polytypes [] _ [].
constructor_polytypes (CtorType :: CtorTypes) TypVars (PolyType :: PolyTypes) :-
  applymany PolyType TypVars CtorType,
  constructor_polytypes CtorTypes TypVars PolyTypes.
\end{verbatim}

\heroSTUDENT{} I see what you were getting at. I think this is an interesting
use of \texttt{applymany}: we are using it in the opposite direction
than what we have used it so far. We are giving it \texttt{TypVars} and
\texttt{CtorType} as inputs, and then we get \texttt{PolyType}, with all
the needed binders, as an output. And since the way we're using it,
\texttt{PolyType} cannot capture the \texttt{TypVars}, it all works out
correctly!

\heroADVISOR{} Excellent! Let's add the term-level former for constructors,
too.

\heroSTUDENT{} That is easy, compared to what we just did.

\begin{verbatim}
constr : constructor \ensuremath{\to} list term \ensuremath{\to} term.
typeof (constr Constructor Args) (tconstr TypConstr TypArgs) :-
  constructor_info TypConstr Constructor PolyType,
  applymany PolyType TypArgs Typs, map typeof Args Typs.
\end{verbatim}

\heroADVISOR{} You're getting the hang of this. Let's do something actually
difficult, then; type synonyms.

  
  \section{Where our heroes reflect on structural
recursion}\label{where-our-heroes-reflect-on-structural-recursion}

\heroSTUDENT{} Type synonyms? Difficult? Why? Doesn't this work?

\begin{verbatim}
type_synonym : dbind typ T typ \ensuremath{\to} (typeconstructor T \ensuremath{\to} program) \ensuremath{\to} program.
type_synonym_info : typeconstructor T \ensuremath{\to} dbind typ T typ \ensuremath{\to} prop.
wfprogram (type_synonym Syn Program') :-
  (t:(typeconstructor T) \ensuremath{\to} type_synonym_info t Syn \ensuremath{\to} wfprogram (Program' t)).
\end{verbatim}

\heroADVISOR{} Sure, that works. How about the typing rule for them, then?
We'll need something like the conversion rule:

\begin{center}$\inferrule{\Gamma \vdash e : \tau \\ \tau =_{\delta} \tau'}{\Gamma \vdash e : \tau'}$\end{center}

\heroSTUDENT{} Right, \(=_{\delta}\) is equality up to expanding the type
synonyms.

\heroADVISOR{} Yes, we'll definitely need a type-equality predicate.

\begin{verbatim}
teq : typ \ensuremath{\to} typ \ensuremath{\to} prop.
\end{verbatim}

\heroSTUDENT{} OK. And then we do this?

\begin{verbatim}
typeof E T :- typeof E T', teq T T'.
\end{verbatim}

\heroADVISOR{} That would be nice, but we'll go into an infinite loop if that
rule gets used.

\heroSTUDENT{} Oh. Oh, right. There is a specific proof-finding strategy in
logic programming, and it can't always work\ldots{}. I guess we have to
switch our rules to an algorithmic type-system instead.

\heroADVISOR{} Precisely. Well, luckily, we can do that to a certain extent,
without rewriting everything. Consider this: we only need to use the
conversion rule in cases where we already know something about the type
\texttt{T} of the expression, but our typing rules do not match that
type.

\heroSTUDENT{} Oh. Do you mean that in bi-directional typing terms? So, doing
type analysis of an expression with a concrete type \texttt{T} might
fail, but synthesizing the type anew could work?

\heroADVISOR{} Exactly, and in that case we have to check that the two types
are equal, using \texttt{teq}. So we need to change the rule you wrote
to apply only in the case where \texttt{T} starts with a concrete
constructor, rather than when it is an uninstantiated unification
variable.

\heroSTUDENT{} Is that even possible? Is there a way in \foreignlanguage{greek}{λ}Prolog to tell
whether something is a unification variable?

\heroADVISOR{} There is! Most Prolog dialects have a predicate that does that
-- it's usually called \texttt{var}. In Makam it is called
\texttt{refl.isunif}, the \texttt{refl} namespace prefix standing for
\emph{reflective} predicates. So, here's a second attempt:

\begin{verbatim}
typeof E T :- not(refl.isunif T), typeof E T', teq T T'.
\end{verbatim}

\heroSTUDENT{} Interesting. But wouldn't this lead to an infinite loop, too?
After all, \texttt{teq} is reflexive -- so we could end up in the same
situation as before.

\heroADVISOR{} Correct: for every proof of
\texttt{typeof\ E\ T\textquotesingle{}} through the other rules, a new
proof using this rule will be discovered, which will lead to another
proof for it, etc. One fix is to make sure that this rule is only used
once at the end, if typing using the normal rules fails.

\heroSTUDENT{} So, something like this:

\begin{verbatim}
typeof, typeof_cases, typeof_conversion : term \ensuremath{\to} typ \ensuremath{\to} prop.
typeof E T :- if (typeof_cases E T) then success else (typeof_conversion E T).
typeof_cases (app E1 E2) T' :- typeof E1 (arrow T1 T2), typeof E2 T1.
...
typeof_conversion E T :- not(refl.isunif T), typeof_cases E T', teq T T'.
\end{verbatim}

\heroADVISOR{} Yes, but let's do a trick to side-step the issue for now. We
will force the rule to only fire once for each expression \texttt{E}, by
remembering that we have used the rule already:

\begin{verbatim}
already_in : [A] A \ensuremath{\to} prop.
typeof E T :- not(refl.isunif T), not(already_in (typeof E)),
              (already_in (typeof E) \ensuremath{\to} typeof E T'), teq T T'.
\end{verbatim}

\heroSTUDENT{} If we ever made a paper submission out of this, the reviewers
would not be happy about this rule. But sure. We still need to define
\texttt{teq} now. Oh, and we should add the conversion rule for
patterns, but that's almost identical as for terms. (\ldots{}) I'll do
\texttt{teq}\ldots{}.

\begin{verbatim}
teq (tconstr TC Args) T' :- type_synonym_info TC Syn, applymany Syn Args T, teq T T'.
teq T' (tconstr TC Args) :- type_synonym_info TC Syn, applymany Syn Args T, teq T' T.
teq (arrow T1 T2) (arrow T1' T2') :- teq T1 T1', teq T2 T2'.
teq (arrowmany TS T) (arrowmany TS' T') :- map teq TS TS', teq T T'.
...
\end{verbatim}

\heroADVISOR{} Writing boilerplate is not fun, is it?

\heroSTUDENT{} It is not. I wish we could just write the first two rules; they
are the important ones, after all. All the other ones just propagate the
structural recursion through. Also, whenever we add a new constructor
for types, we'll have to remember to add a \texttt{teq} rule for
it\ldots{}.

\heroADVISOR{} Why don't we reflect a bit on this? Ideally we would only write
a generic rule, to handle any concrete constructor applied to a number
of arguments. Something like:

\begin{verbatim}
teq (Constructor Arguments) (Constructor Arguments') :- map teq Arguments Arguments'.
\end{verbatim}

\heroSTUDENT{} Right, so in the example of the \texttt{arrow} type,
\texttt{Constructor} would match \texttt{arrow} and arguments would be
instantiated with \texttt{{[}T1,\ T2{]}}. But we are not really
guaranteed that all arguments are of \texttt{typ} type, or even that
they are all of the same type, right? Take \texttt{arrowmany} for
example! The argument list needs to be heterogeneous.

\heroADVISOR{} Glad you figured that out. We can do that using the existential
type -- let's call it \texttt{dyn} --, so the arguments can be of
\texttt{list\ dyn} type. And we'll need to make \texttt{teq}
polymorphic, handling any type that includes a \texttt{typ}:

\begin{verbatim}
dyn : type.  dyn : A \ensuremath{\to} dyn.
teq : [A] A \ensuremath{\to} A \ensuremath{\to} prop.
\end{verbatim}

\heroSTUDENT{} We'll also need a \texttt{map} for these heterogeneous lists,
too. I believe it will need a polymorphic function as an argument, so
that it can be used at different types for different elements of the
list. So something like this:

\begin{verbatim}
map : (forall A. [A] A \ensuremath{\to} A \ensuremath{\to} prop) \ensuremath{\to} list dyn \ensuremath{\to} list dyn \ensuremath{\to} prop.
\end{verbatim}

\heroADVISOR{} Right, we'd need higher-rank types here. There's a problem with
the alternative:

\begin{verbatim}
map : [A] (A \ensuremath{\to} A \ensuremath{\to} prop) \ensuremath{\to} list dyn \ensuremath{\to} list dyn \ensuremath{\to} prop
\end{verbatim}

\noindent
In the first \texttt{cons} cell, this would instantiate the type
\texttt{A} to the type of the first element of the list, making further
applications to different types impossible.

\heroSTUDENT{} Exactly. Does Makam support higher-rank polymorphism?

\heroADVISOR{} Unfortunately it does not right now, but it should. Nor do any
other \foreignlanguage{greek}{λ}Prolog implementations that I know of, though. Also, there is no
way to refer to a polymorphic constant without implicitly instantiating
it with new type variables. So we have to use a helper predicate right
now, called \texttt{dyn.call}, to avoid that issue:

\begin{verbatim}
dyn.call : [B] (A \ensuremath{\to} A \ensuremath{\to} prop) \ensuremath{\to} B \ensuremath{\to} B \ensuremath{\to} prop.
dyn.map : (A \ensuremath{\to} A \ensuremath{\to} prop) \ensuremath{\to} list dyn \ensuremath{\to} list dyn \ensuremath{\to} prop.
dyn.map P [] [].
dyn.map P (HD :: TL) (HD' :: TL') :- dyn.call P HD HD', dyn.map P TL TL'.
\end{verbatim}

\heroSTUDENT{} Fair enough. So, going back to our generic rule -- is there a
way to actually write it? Maybe there's another reflective predicate we
can use?

\heroADVISOR{} Exactly -- there is \texttt{refl.headargs}. If a term is
concrete, it decomposes it into a constructor and a list of
arguments\footnote{Other versions of Prolog have predicates toward the same effect; for example, SWI-Prolog \citep{wielemaker2012swi} provides `\texttt{compound\_{}name\_{}arguments}', which is quite similar.}.
This isn't a hack, though: we could define \texttt{refl.headargs}
without any special support, save for \texttt{refl.isunif}, if we
maintained a discipline whenever we add a new constructor:

\begin{verbatim}
arrowmany : list typ \ensuremath{\to} typ \ensuremath{\to} typ.
refl.headargs Term Head Args :- not(refl.isunif Term), eq Term (arrowmany TS T),
                                eq Head arrowmany, eq Args [dyn TS, dyn T].
\end{verbatim}

\heroSTUDENT{} So \texttt{refl.headargs} has this type:

\begin{verbatim}
refl.headargs : B \ensuremath{\to} A \ensuremath{\to} list dyn \ensuremath{\to} prop.
\end{verbatim}

\heroADVISOR{} Correct. We should now be able to proceed to defining the
boilerplate generically. Let's do it as a reusable higher-order
predicate for structural recursion. I'll give you the type; you fill in
the first case:

\begin{verbatim}
structural_recursion : [A B] (A \ensuremath{\to} A \ensuremath{\to} prop) \ensuremath{\to} B \ensuremath{\to} B \ensuremath{\to} prop.
\end{verbatim}

\heroSTUDENT{} Let me see. Oh, so, the first argument -- are we doing this in
open-recursion style? Maybe that's the predicate for recursive calls. I
need to deconstruct a term, apply the recursive call\ldots{}. How is
this?

\begin{verbatim}
structural_recursion Rec X Y :-
  refl.headargs X Constructor Arguments,
  dyn.map Rec Arguments Arguments',
  refl.headargs Y Constructor Arguments'.
\end{verbatim}

\heroAUDIENCE{} I'm sure this was not your first attempt in the unabridged
version of this story!

\heroADVISOR{} Wait, who said that? Anyway. That looks great! And you're right
about using \texttt{refl.headargs} in the other direction, to
reconstruct a new term with the same constructor and different
arguments.

\heroSTUDENT{} Are we done?

\heroADVISOR{} Almost there! We just need to handle the case of the meta-level
function type. It does not make sense to destructure functions using
\texttt{refl.headargs}; so, that fails for functions, and we have to
treat them specially:

\begin{verbatim}
structural_recursion Rec (X : A \ensuremath{\to} B) (Y : A \ensuremath{\to} B) :-
  (x:A \ensuremath{\to} structural_recursion Rec x x \ensuremath{\to} structural_recursion Rec (X x) (Y x)).
\end{verbatim}

\heroSTUDENT{} This is exciting; I hope it is part of the standard library of
Makam. I can do \texttt{teq} in a few lines now!

\begin{verbatim}
teq' : [A] A \ensuremath{\to} A \ensuremath{\to} prop.
teq T T' :- teq' T T'.
teq' (tconstr TC Args) T' :-
  type_synonym_info TC Synonym, applymany Synonym Args T, teq' T T'.
teq' T' (tconstr TC Args) :-
  type_synonym_info TC Synonym, applymany Synonym Args T, teq' T' T.
teq' T T' :- structural_recursion teq' T T'.
\end{verbatim}

\heroADVISOR{} That is exactly right! So, we've minimized the boilerplate, and
we won't need any adaptation when we add a new constructor -- even if we
make use of all sorts of new and complicated types.

\heroSTUDENT{} That's right: we did not do anything special for the binding
forms we defined\ldots{}. quite a payoff for a small amount of code!
But, wait, isn't \texttt{structural\_recursion} missing a case: that of
uninstantiated unification variables?

\heroADVISOR{} It is, but in my experience, it's better to define how to
handle unification variables as needed, in each new structurally
recursive predicate. In this case, we're only supposed to use
\texttt{teq} with ground terms, so it's fine if we fail when we
encounter a unification variable.

\begin{scenecomment}
(Our heroes try out a few examples and convince themselves that this works OK and no endless loops happen when things don't typecheck correctly.)
\end{scenecomment}

  
  \section{Where our heroes tackle dependencies, contexts, and a new level
of
meta}\label{where-our-heroes-tackle-dependencies-contexts-and-a-new-level-of-meta}

\heroSTUDENT{} I'm fairly confident by now that Makam should be able to handle
the research idea we want to try out. Shall we get to it?

\heroADVISOR{} Yes, it is time. So, what we are aiming to do is add a facility
for type-safe, heterogeneous meta-programming to our object language,
similar to MetaHaskell \citep{mainland2012explicitly}. This way we can
manipulate the terms of a separate object language in a type-safe
manner.

\heroSTUDENT{} Exactly. We'd like our object language to be a formal logic, so
our language will be similar to Beluga \citep{beluga-main-reference} or
VeriML \citep{stampoulis2013veriml}. We'll have to be able to pattern
match over the terms of the object language, too, so they are runtime
entities\ldots{}. But we don't need to do all of that; let's just do a
basic version for now, and I can do the rest on my own.

\heroADVISOR{} Sounds good. So, I think the fragment we should do is this: we
will have dependent functions over a distinguished language of
\emph{dependent indices}. We need the dependency so that, for example,
we can take an object-level type as an argument and return an
object-level term that uses that type.

\heroSTUDENT{} Exactly. Dependent products should be similar, but we can skip
them for now and just add a way to return an object-level term from the
meta-level.

\heroADVISOR{} Good idea. We are getting into many levels of meta -- there's
the meta-language we're using, Makam; there's the object language we are
encoding, which is a meta-language in itself, let's call that
Heterogeneous Meta ML Light (HMML?); and there's the ``object-object''
language that HMML is manipulating. And let's keep that last one simple:
the simply typed lambda calculus (STLC).

\heroSTUDENT{} Great. So, our dependent indices will be the types and terms of
STLC -- actually, the open terms of STLC.

\heroADVISOR{} It's a plan. So, let's get to it. Let's first add distinguished
sorts for dependent indices and dependent classifiers -- we'll use those
to type-check the indices, with an appropriate predicate. Let's also
have a distinguished type for \emph{dependent variables}, that is,
variables of dependent indices; and a way to substitute such a variable
for an object.

\begin{verbatim}
depindex, depclassifier, depvar : type.
depclassify : depindex \ensuremath{\to} depclassifier \ensuremath{\to} prop.
depclassify : depvar \ensuremath{\to} depclassifier \ensuremath{\to} prop.
depwf : depclassifier \ensuremath{\to} prop.
depsubst : [A] (depvar \ensuremath{\to} A) \ensuremath{\to} depindex \ensuremath{\to} A \ensuremath{\to} prop.
\end{verbatim}

\newcommand\dep[1]{\ensuremath{#1_{\text{d}}}}
\newcommand\lift[1]{\ensuremath{\langle#1\rangle}}

\heroSTUDENT{} Right, we might need to check that classifiers are well-formed.
And we might need to treat variables specially, so it's good that
they're a different type. So, that's why you made substitution a
predicate, rather than using the normal HOAS function application
\texttt{F\ X} directly, as we have been doing so far. I know that when
we add variables that stand for open STLC terms, there will be some
extra computation involved to substitute them for an open term, so the
normal application won't work as is.

\heroADVISOR{} Exactly; and that extra computation will be necessary in order
to maintain type-safety. Hopefully, we won't have to write any
unnecessary cases, though! Now, we have a few typing rules to add. I'll
use ``\(\dep{\cdot}\)'' to signify things that have to do with the
dependent indices.

\vspace{-1.5em}\begin{mathpar}
\inferrule{\dep{\Psi} \dep{\vdash} \dep{i} : \dep{c}}
          {\Gamma; \dep{\Psi} \vdash \lift{\dep{i}} : \lift{\dep{c}}}

\inferrule{\Gamma; \dep{\Psi}, \; \dep{v} : \dep{c} \vdash e : \tau \\ \dep{\Psi} \dep{\vdash} \dep{c} \; \text{wf}}
          {\Gamma; \dep{\Psi} \vdash \Lambda \dep{v} : \dep{c}.e : \Pi \dep{v} : \dep{c}.\tau}

\inferrule{\Gamma; \dep{\Psi} \vdash e : \Pi \dep{v} : \dep{c}.\tau \\ \dep{\Psi} \dep{\vdash} \dep{i} : \dep{c}}
          {\Gamma; \dep{\Psi} \vdash e @ \dep{i} : \dep{\text{subst}}(\tau, [\dep{i}/\dep{v}])}
\end{mathpar}

\heroSTUDENT{} Those are very easy to transcribe to Makam.

\begin{verbatim}
lamdep : depclassifier \ensuremath{\to} (depvar \ensuremath{\to} term) \ensuremath{\to} term.
appdep : term \ensuremath{\to} depindex \ensuremath{\to} term.
liftdep : depindex \ensuremath{\to} term. liftdep : depclassifier \ensuremath{\to} typ.
pidep : depclassifier \ensuremath{\to} (depvar \ensuremath{\to} typ) \ensuremath{\to} typ.
typeof (lamdep C EF) (pidep C TF) :-
  (v:depvar \ensuremath{\to} depclassify v C \ensuremath{\to} typeof (EF v) (TF v)), depwf C.
typeof (appdep E I) T' :- typeof E (pidep C TF), depclassify I C, depsubst TF I T'.
typeof (liftdep I) (liftdep C) :- depclassify I C.
\end{verbatim}

\heroADVISOR{} Looks nice. Just wanted to say, this framework is quite
general. We could instantiate dependent indices with a language of
natural numbers, equality predicates, and equality proofs, which would
be quite similar to the Dependent ML formulation of
\citet{licata2005formulation}. But let's go back to what we're trying to
do. I'll add the object language in a separate namespace prefix -- we
can use `\texttt{\%extend}' for going into a namespace -- and I'll just
copy-paste our STLC code from earlier on.

\begin{verbatim}
%extend object.
term : type. typ : type. typeof : term \ensuremath{\to} typ \ensuremath{\to} prop.
...
%end.
\end{verbatim}

\heroSTUDENT{} Great! I'll make these into dependent indices now, including
both types and terms.

\begin{verbatim}
iterm : object.term \ensuremath{\to} depindex.     ityp : object.typ \ensuremath{\to} depindex.
ctyp : object.typ \ensuremath{\to} depclassifier.  cext : depclassifier.
depclassify (iterm E) (ctyp T) :- object.typeof E T.
depclassify (ityp T) cext :- object.wftyp T.
depwf (ctyp T) :- object.wftyp T.
depwf cext.
\end{verbatim}

\heroADVISOR{} Right, we'll need to check that types are well-formed, too.
Right now, they are all well-formed by construction, but let's prepare
for any additions, by setting up a structurally recursive predicate. The
\texttt{wftyp\_cases} predicate will hold the important type-checking
cases, and we will have an extra predicate to say whether those cases
apply or not for a specific \texttt{typ}.

\begin{verbatim}
%extend object.
wftyp : typ \ensuremath{\to} prop. wftyp_aux : [A] A \ensuremath{\to} A \ensuremath{\to} prop.
wftyp_cases, wftyp_applies : [A] A \ensuremath{\to} prop.
wftyp T :- wftyp_aux T T.
wftyp_aux T T :- if (wftyp_applies T)
                 then (wftyp_cases T)
                 else (structural_recursion wftyp_aux T T).
%end.
\end{verbatim}

\heroSTUDENT{} I see -- if a type-checking rule applies, but fails, we don't
want to proceed to also try structural recursion; it would defeat the
purpose. Neat trick. I also see that your structural recursion just
needs to do a simple visit and it does not need to produce an output;
hence the repeat of the same \texttt{typ} argument. Let's prepare for
substitutions, too, in the same way.

\begin{verbatim}
depsubst_aux, depsubst_cases : [A] depvar \ensuremath{\to} depindex \ensuremath{\to} A \ensuremath{\to} A \ensuremath{\to} prop.
depsubst_applies : [A] depvar \ensuremath{\to} A \ensuremath{\to} prop.
depsubst F I Res :- (v:depvar \ensuremath{\to} depsubst_aux v I (F v) Res).
depsubst_aux Var Replace Where Res :-
  if (depsubst_applies Var Where)
  then (depsubst_cases Var Replace Where Res)
  else (structural_recursion (depsubst_aux Var Replace) Where Res).
\end{verbatim}

\heroADVISOR{} Great! We only have one thing missing: we need to close the
loop, being able to refer to a dependent variable from within an
object-level term or type.

\heroSTUDENT{} I got this.

\begin{verbatim}
%extend object.
varterm : depvar \ensuremath{\to} term.  vartyp : depvar \ensuremath{\to} typ.
typeof (varterm V) T :- depclassify V (ctyp T).
wftyp_applies (vartyp V). wftyp_cases (vartyp V) :- depclassify V cext.
%end.
depsubst_applies Var (object.varterm Var).
depsubst_cases Var (iterm Replace) (object.varterm Var) Replace.
depsubst_applies Var (object.vartyp Var).
depsubst_cases Var (ityp Replace)  (object.vartyp Var)  Replace.
\end{verbatim}

\heroADVISOR{} This is exciting; let me try it out! I'll do a function that
takes an object-level type and returns the object-level identity
function for it.

\begin{verbatim}
typeof (lamdep cext (fun t \ensuremath{\Rightarrow}
         (liftdep (iterm (object.lam (object.vartyp t) (fun x \ensuremath{\Rightarrow} x)))))) T ?
>> Yes!!!!!
>> T := pidep cext (fun t \ensuremath{\Rightarrow}
>>        liftdep (ctyp (object.arrow (object.vartyp t) (object.vartyp t))))
\end{verbatim}

\heroSTUDENT{} Look, even the Makam REPL is excited!

\heroADVISOR{} Wait until it sees what we have in store for it next: open STLC
terms in our indices!

\heroSTUDENT{} Good thing I've printed out the contextual types paper by
\citet{nanevski2008contextual}. (\ldots{}) OK, so it says here that we
can use contextual types to record, at the type level, the context that
open terms depend on. So let's say, an open \texttt{object.term} of type
\(\tau\) that mentions variables of a \(\Phi\) context would have a
contextual type of the form \([\Phi] \tau\). This is some sort of modal
typing, with a precise context.

\heroADVISOR{} Right. So in our case, open STLC terms depend on a number of
variables, and we will need to keep track of the STLC types of those
variables, in order to maintain type safety. So, let's add a new
dependent index for open STLC terms, and a dependent classifier for
their contextual types, which record the types of the variables that the
term depends on, as well as the actual type of the term itself.

\heroSTUDENT{} Let me see. I think something like this is what we want:

\begin{verbatim}
iopen_term : bindmany object.term object.term \ensuremath{\to} depindex.
cctx_typ : list object.typ \ensuremath{\to} object.typ \ensuremath{\to} depclassifier.
\end{verbatim}

\heroADVISOR{} That looks right to me. I can write the classification and
well-formedness rules for those.

\begin{verbatim}
depclassify (iopen_term XS_E) (cctx_typ TS T) :-
  openmany XS_E (pfun xs e \ensuremath{\Rightarrow}
    assumemany object.typeof xs TS (object.typeof e T),
    foreach object.wftyp TS).
depwf (cctx_typ TS T) :- foreach object.wftyp TS, object.wftyp T.
\end{verbatim}

\heroSTUDENT{} That makes a lot of sense. I see you are also checking
well-formedness for the types that the context introduces; and
\texttt{foreach} is exactly like \texttt{map}, but there's no output, so
it applies a single-argument predicate to each element of the list.

\heroADVISOR{} Right. We now get to the tricky part: referring to variables
that stand for open terms within other terms! You know what those are,
right? Those are Object-level Object-level Meta-variables.

\heroSTUDENT{} My head hurts; I'm getting
\href{https://en.wikipedia.org/wiki/Out_of_memory}{OOM} errors. Maybe
this is easier to implement in Makam than to talk about.

\heroADVISOR{} Maybe so. Well, let me just say this: those variables will
stand for open terms that depend on a specific context \(\Phi\), but we
might use them at a different context \(\Phi'\). We need a
\emph{substitution} \(\sigma\) to go from the context they were defined
into the current context.

\heroSTUDENT{} OK, and then we need to apply that substitution \(\sigma\) when
we substitute an actual open term for the metavariable. I know what to
do:

\vspace{-0.5em}

\begin{verbatim}
%extend object.
varmeta : depvar \ensuremath{\to} list term \ensuremath{\to} term.
typeof (varmeta V ES) T :- depclassify V (cctx_typ TS T), map object.typeof ES TS.
%end.
depsubst_applies Var (object.varmeta Var _).
depsubst_cases Var (iopen_term XS_E) (object.varmeta Var ES) Result :-
  applymany XS_E ES E', depsubst_aux Var (iopen_term XS_E) E' Result.
\end{verbatim}

\heroADVISOR{} That should be it; let's try this out! Let's do meta-level
application, maybe? So, take a ``function'' body that needs a single
argument, and an instantiation for that argument, and do the
substitution at the meta-level. This will be sort of like inlining. And
let's use unification variables wherever it makes sense, to push our
rules to infer what they can for themselves!

\begin{verbatim}
typeof (lamdep _ (fun t1 \ensuremath{\Rightarrow} (lamdep _ (fun t2 \ensuremath{\Rightarrow}
       (lamdep (cctx_typ [object.vartyp t1] (object.vartyp t2)) (fun f \ensuremath{\Rightarrow}
       (lamdep _ (fun a \ensuremath{\Rightarrow} (liftdep (iopen_term (bindbase (
         (object.varmeta f [object.varterm a]))))))))))))) T ?
>> Yes:
>> T := (pidep cext (fun t1 \ensuremath{\Rightarrow} pidep cext (fun t2 \ensuremath{\Rightarrow}
>>      (pidep (cctx_typ [object.vartyp t1] (object.vartyp t2)) (fun f \ensuremath{\Rightarrow}
>>      (pidep (ctyp (object.vartyp t1)) (fun a \ensuremath{\Rightarrow}
>>      (liftdep (cctx_typ [] (object.vartyp t2)))))))))).
\end{verbatim}

\begin{scenecomment}
(Our heroes try out a few more examples to convince themselves that this works.)
\end{scenecomment}

\heroSTUDENT{} That's it! That's it! I cannot believe how easy this was!

\heroAUDIENCE{} Neither can we believe that you thought this was easy!

\heroAUTHOR{} Trust me, you should have seen how many weeks it took me to
implement something like this in OCaml\ldots{}. it was enough to make me
start working on Makam. That took two years, but now we can at least
show it in 24 pages of a single-column PDF!

\heroADVISOR{} Where are all these voices coming from?

\heroSTUDENT{}
\textit{(Joke elided to avoid issues with double-blind submission.)}

  
  \section{Where our heroes implement type generalization, tying loose
ends}\label{where-our-heroes-implement-type-generalization-tying-loose-ends}

\begin{verse}
``We promised we'll do Hindley-Milner, we don't want you to be sad. \\
This paper is coming to an end soon, and it wasn't all that bad. \\
\hspace{1em}\vspace{-0.5em} \\
We'll gather all unification variables, using structural recursion. \\
And if you haven't guessed it yet, we'll use some term reflection.''
\end{verse}

\heroSTUDENT{} I have an idea for implementing type generalization for
polymorphic \texttt{let} in the style of
\citet{damas1984type,hindley1969principal,milner1978theory}. I remember
the typing rule looks like this:

\vspace{-1.2em}\begin{mathpar}
\inferrule{\Gamma \vdash e : \tau \\ \vec{a} = \text{fv}(\tau) - \text{fv}(\Gamma) \\ \Gamma, x : \forall \vec{a}.\tau \vdash e' : \tau'}{\Gamma \vdash \text{let} \; x = e \; \text{in} \; e' : \tau'}
\end{mathpar}

\heroADVISOR{} Right, and we don't have any side-effectful operations, so, no
need for a value restriction. Let's assume a predicate for generalizing
the type, for now; the rest of the rule is easy:

\begin{verbatim}
generalize : typ \ensuremath{\to} typ \ensuremath{\to} prop.
let : term \ensuremath{\to} (term \ensuremath{\to} term) \ensuremath{\to} term.
typeof (let E F) T' :-
  typeof E T, generalize T Tgen, (x:term \ensuremath{\to} typeof x Tgen \ensuremath{\to} typeof (F x) T').
\end{verbatim}

\heroSTUDENT{} Right, so for generalization, based on the typing rule, we need
the following ingredients:

\begin{itemize}
\tightlist
\item
  something that picks out free variables from a type -- or, in our
  setting, uninstantiated unification variables
\item
  something that picks out free variables from the local context
\item
  a way to turn something that includes unification variables into a
  \texttt{forall} type
\end{itemize}

\noindent
Those look like things that we should be able to do with our generic
recursion and with the reflective predicates we've been using!

\heroADVISOR{} Indeed! So, I've done this before, and I need to leave for home
soon, so bear with me for a bit. There's this generic operation in the
Makam standard library, called \texttt{generic.fold}. It is quite
similar to \texttt{structural\_recursion}, but it does a fold through a
term, carrying an accumulator through. Pretty standard, really, and its
code is similar to what we did already. I'll use it to define a
predicate that returns \emph{one} unification variable of the right type
from a term, if at least one exists.

\begin{verbatim}
findunif : [A B] option B \ensuremath{\to} A \ensuremath{\to} option B \ensuremath{\to} prop.
findunif (some X) _ (some X).
findunif none (X : B) (some (X : B)) :- refl.isunif X.
findunif In X Out :- generic.fold findunif In X Out.
findunif : [A B] A \ensuremath{\to} B \ensuremath{\to} prop.  findunif T X :- findunif none T (some X).
\end{verbatim}

\heroSTUDENT{} Oh, the second rule is the important one -- it will only match
when we encounter a unification variable of the same type as the one we
require, thanks to type specialization.

\heroADVISOR{} Exactly. Now we add a predicate that, given a specific
unification variable and a specific term, replaces its occurrences with
the term. I'll show you later why this operation is necessary. Here I'll
need another reflective predicate, \texttt{refl.sameunif}, that succeeds
when its two arguments are the same exact unification variable;
\texttt{eq} would just unify them, which is not what we want.

\begin{verbatim}
replaceunif : [A B] A \ensuremath{\to} A \ensuremath{\to} B \ensuremath{\to} B \ensuremath{\to} prop.
replaceunif Which ToWhat Where Result :- refl.isunif Where,
  if (refl.sameunif Which Where) then (eq (dyn Result) (dyn ToWhat))
  else (eq Result Where).
replaceunif Which ToWhat Where Result :- not(refl.isunif Where),
  structural_recursion (replaceunif Which ToWhat) Where Result.
\end{verbatim}

\heroADVISOR{} And last, we'll need an auxiliary predicate that tells us
whether a unification variable exists within a term. You can do that
yourself; it's similar to the above.

\heroSTUDENT{} Yes, I think I know how to do that.

\begin{verbatim}
hasunif : [A B] B \ensuremath{\to} bool \ensuremath{\to} A \ensuremath{\to} bool \ensuremath{\to} prop.
hasunif _ true _ true.
hasunif X false Y true :- refl.sameunif X Y.
hasunif X In Y Out :- generic.fold (hasunif X) In Y Out.
hasunif : [A B] A \ensuremath{\to} B \ensuremath{\to} prop. hasunif Term Var :- hasunif Var false Term true.
\end{verbatim}

\heroADVISOR{} OK, we are now mostly ready to implement \texttt{generalize}.
We'll do this recursively. The base case is when there are no
unification variables within a type left:

\begin{verbatim}
generalize T T :- not(findunif T X).
\end{verbatim}

\heroSTUDENT{} Ah, I see what you are getting at. For the recursive case, we
will pick out the first unification variable that we come upon using
\texttt{findunif}. We will generalize over it using \texttt{replaceunif}
and then proceed to the rest. But don't we have to skip over the
unification variables that are in the \(\Gamma\) environment?

\heroADVISOR{} Well, that's the last hurdle. Let's assume a predicate that
gives us all the types in the environment, and write the recursive case
down:

\begin{verbatim}
get_types_in_environment : [A] A \ensuremath{\to} prop.
generalize T Res :- 
  findunif T Var, get_types_in_environment GammaTypes,
  (x:typ \ensuremath{\to} (replaceunif Var x T (T' x), generalize (T' x) (T'' x))),
  if (hasunif GammaTypes Var) then (eq Res (T'' Var)) else (eq Res (forall T'')).
\end{verbatim}

\heroSTUDENT{} Oh, clever. But what should
\texttt{get\_types\_in\_environment} be? Don't we have to go back and
thread a list of types through our \texttt{typeof} predicate, every time
we introduce a new \texttt{typeof\ x\ T\ \ensuremath{\to}} assumption?

\heroADVISOR{} Well, we came this far without rewriting our rules, so it's a
shame to do that now! Maybe we'll be excused to use yet another
reflective predicate that does what we want? There is a way to get a
list of all the local assumptions for the \texttt{typeof} predicate; it
turns out that all the rules and connectives are normal \foreignlanguage{greek}{λ}Prolog terms
like any other, so there's not really much magic to it. And those
assumptions will include just the types in \(\Gamma\)\ldots{}.

\begin{verbatim}
get_types_in_environment Assumptions :-
  refl.assume_get (typeof : term \ensuremath{\to} typ \ensuremath{\to} prop) Assumptions.
\end{verbatim}

\heroSTUDENT{} Wait. It can't be.

\begin{verbatim}
typeof (let (lam _ (fun x \ensuremath{\Rightarrow} let x (fun y \ensuremath{\Rightarrow} y))) (fun id \ensuremath{\Rightarrow} id)) T ?
>> Yes:
>> T := forall (fun a \ensuremath{\Rightarrow} arrow a a).
\end{verbatim}

\heroADVISOR{} And yet, it can.


  \section{Where our heroes summarize what they've done and our story
concludes before the credits start
rolling}\label{where-our-heroes-summarize-what-theyve-done-and-our-story-concludes-before-the-credits-start-rolling}

\heroSTUDENT{} I feel like we've done a lot here. And, some of the things we
did I don't think I've seen in the literature before, but then again,
it's not clear to me what's Makam-specific and what isn't. In any case,
I think a lot of people would find that quickly prototyping their PL
research ideas using this style of higher-order logic programming is
very useful.

\heroADVISOR{} I agree, though it would be hard for somebody to publish a
paper on this. Some of it is novel, some of it is folklore, some of it,
we just did in a pleasant way; and we did also use a couple of
not-so-pleasant hacks. But let's make a list of what's what.

\vspace{-0.5em}

\begin{itemize}
\item
  We defined HOAS encodings of complicated binding forms, including
  mutually recursive definitions and patterns, while only having
  explicit support in our metalanguage for single-variable binding.
  These encodings are novel, as far as I know. We made use of GADTs for
  them, together with ad-hoc polymorphism, which seems to be a feature
  specific to the Makam implementation of \foreignlanguage{greek}{λ}Prolog. We also depended on a
  technicality for defining an essential
  predicate\footnote{\texttt{assumemany} uses what is technically referred to
    as a strong-hereditary Harrop formula, whereas Teyjus only supports weak-hereditary
    Harrop formulas \citep{teyjus-main-reference}.}. As a result, we
  have not been able to replicate these in the standard \foreignlanguage{greek}{λ}Prolog/Teyjus
  implementation \citep{teyjus-main-reference}, though all the features
  we need are part of the original \foreignlanguage{greek}{λ}Prolog language design
  \citep{miller1988overview}.
\item
  We defined a generic predicate to perform structural recursion using a
  very concise definition. It allows us to define structurally recursive
  predicates that only explicitly list out the important cases, in what
  we believe is a novel encoding for the \foreignlanguage{greek}{λ}Prolog setting. Any new
  definitions, such as constructors or datatypes we introduce later, do
  not need any special provision to be covered by the same predicates.
  They depend on a number of reflective predicates, which are available
  in other Prolog dialects; however, we are not aware of a published
  example that makes use of them in the \foreignlanguage{greek}{λ}Prolog setting. These
  predicates are used to reflect on the structure of Makam terms, and to
  get the list of local assumptions; for the most part, their use is
  limited to predicates that would be part of the standard library, not
  in user code.
\item
  The above encodings are reusable and can be made part of the Makam
  standard library. As a result, we were able to develop the type
  checker for quite an advanced type system, in very few lines of code
  specific to it, all the while allowing for the parts of terms and
  types that can be determined from the context to be left implicit. Our
  development includes mutually recursive definitions, polymorphism,
  polymorphic datatypes, pattern-matching, a conversion rule,
  Hindley-Milner type generalization, constructs for dependent types
  over a separately specified language of dependent indices, and an
  example of such indices that use contextually typed open terms of the
  simply typed lambda calculus. We are not aware of another
  metalinguistic framework that allows this level of expressivity and
  has been used to encode such type-system features.
\item
  We have also shown that higher-order logic programming allows not just
  meta-level functions to be reused for encoding object-level binding;
  there are also cases where meta-level unification can also be reused
  to encode certain object-level features: matching a pattern against a
  scrutinee and doing type generalization as in Algorithm W.
\end{itemize}

\heroSTUDENT{} Well, that was very interesting; thank you for working with me
on this!

\heroADVISOR{} I enjoyed this, too. Say, if you want to relax, there's a new
staging of the classic play by \citet{fischer2010play} downtown -- I saw
it yesterday, and it is really good!

\heroSTUDENT{} That's a great idea, one of my favorites. I wish there were
more plays like it\ldots{}. Well, good night, and see you on Monday!


  \identNormal{}
}

\bibliography{main}

\end{document}
