\section{In which our hero Hagop adds pattern matching on his
own}\label{in-which-our-hero-hagop-adds-pattern-matching-on-his-own}

\begin{scenecomment}
(Roza had a meeting with another student, so Hagop took a small break and is now back at his
office. He is trying to work out on his own how to encode patterns. He is fairly
confident at this point that having explicit support for single-variable
binding is enough to model most complicated forms of binding, especially when making use of
polymorphism and GADTs.)
\end{scenecomment}

\identNormal
\heroSTUDENT{} OK, so let's implement simple patterns and pattern-matching
like in ML\ldots{} First let's determine the right binding structure.
For a branch like:

\begin{verbatim}
| cons(hd, tl) \ensuremath{\to} ... hd .. tl ...
\end{verbatim}

the pattern introduces 2 variables, \texttt{hd} and \texttt{tl}, which
the body of the branch can refer to. But we can't really refer to those
variables in the pattern itself, at least for simple
patterns\footnote{There are counterexamples, like for or-patterns in some ML dialects, or for dependent pattern matching, where consequent uses of the same variable perform exact matches rather than unification. We choose to omit the handling of cases like those in the present work for presentation purposes.}\ldots{}.
So there's no binding going on really within the pattern; instead, once
we figure out how many variables a pattern introduces, we can do the
actual binding all at once, when we get to the body of the branch:

\begin{verbatim}
branch(pattern, bind [# of variables in pattern].body)
\end{verbatim}

So we could write the above branch in Makam like this:

\begin{verbatim}
\tokconst{branch} (\tokconst{patt_cons} \tokobjconst{patt_var} \tokobjconst{patt_var})
       (\tokstdconst{bind} (\tokkeyword{fun} \tokconst{hd} \tokkeyword{\ensuremath{\Rightarrow}} \tokstdconst{bind} (\tokkeyword{fun} \tokconst{tl} \tokkeyword{\ensuremath{\Rightarrow}} \tokstdconst{body} (.. \tokconst{hd} .. \tokconst{tl} ..))))
\end{verbatim}

We do have to keep the order of variables consistent somehow, so
\texttt{hd} here should refer to the first occurrence of
\texttt{patt\_var}, and \texttt{tl} to the second. Based on these, I am
thinking that the type of \texttt{branch} should be something like:

\begin{verbatim}
\tokconst{branch} : (\tokmetavariable{Pattern}: \toktypeid{patt} \tokmetavariable{N}) (\tokmetavariable{Vars_Body}: \tokstdtypeid{vbindmany} \toktypeid{term} \tokmetavariable{N} \toktypeid{term}) \tokarrowtype{\ensuremath{\to}} ...
\end{verbatim}

Wait, before I get into the weeds let me just set up some things. First,
let's add a simple base type, say \texttt{nat}s, to have something to
work with as an example. I'll prefix their names with \texttt{o} for
``object language,'' so as to avoid ambiguity. And I will also add
\texttt{case\_or\_else}, standing for a single-branch pattern-match
construct. It should be easy to extend to a multiple-branch construct,
but I want to keep things as simple as possible. I'll inline what I had
written for \texttt{branch} above into the definition of
\texttt{case\_or\_else}.

\begin{verbatim}
\tokobjconst{onat} : \toktypeid{typ}. \tokobjconst{ozero} : \toktypeid{term}. \tokobjconst{osucc} : \toktypeid{term} \tokarrowtype{\ensuremath{\to}} \toktypeid{term}.
\tokpropconst{typeof} \tokobjconst{ozero} \tokobjconst{onat}. \tokpropconst{typeof} (\tokobjconst{osucc} \tokmetavariable{N}) \tokobjconst{onat} \toksymbol{\ensuremath{:\!-}} \tokpropconst{typeof} \tokmetavariable{N} \tokobjconst{onat}.
\tokpropconst{eval} \tokobjconst{ozero} \tokobjconst{ozero}. \tokpropconst{eval} (\tokobjconst{osucc} \tokmetavariable{E}) (\tokobjconst{osucc} \tokmetavariable{V}) \toksymbol{\ensuremath{:\!-}} \tokpropconst{eval} \tokmetavariable{E} \tokmetavariable{V}.
\end{verbatim}

\begin{verbatim}
\tokobjconst{case_or_else} : (\tokmetavariable{Scrutinee}: \toktypeid{term})
  (\tokmetavariable{Patt}: \toktypeid{patt} \tokmetavariable{N}) (\tokmetavariable{Vars_Body}: \tokstdtypeid{vbindmany} \toktypeid{term} \tokmetavariable{N} \toktypeid{term})
  (\tokmetavariable{Else}: \toktypeid{term}) \tokarrowtype{\ensuremath{\to}} \toktypeid{term}.
\end{verbatim}

Now for the typing rule -- it will be something like this:

\begin{verbatim}
\tokpropconst{typeof} (\tokobjconst{case_or_else} \tokmetavariable{Scrutinee} \tokmetavariable{Pattern} \tokmetavariable{Vars_Body} \tokmetavariable{Else}) \tokmetavariable{BodyT} \toksymbol{\ensuremath{:\!-}}
  \tokpropconst{typeof} \tokmetavariable{Scrutinee} \tokmetavariable{T}, \tokpropconst{typeof_patt} \tokmetavariable{Pattern} \tokmetavariable{T} \tokmetavariable{VarTypes},
  \tokstdconst{vopenmany} \tokmetavariable{Vars_Body} (\tokkeyword{pfun} \tokmetavariable{Vars} \tokmetavariable{Body} \tokkeyword{\ensuremath{\Rightarrow}}
    \tokstdconst{vassumemany} \tokpropconst{typeof} \tokmetavariable{Vars} \tokmetavariable{VarTypes} (\tokpropconst{typeof} \tokmetavariable{Body} \tokmetavariable{BodyT})),
  \tokpropconst{typeof} \tokmetavariable{Else} \tokmetavariable{BodyT}.
\end{verbatim}

Right, so when checking a pattern, we'll have to determine both what
type of scrutinee it matches, as well as the types of the variables that
it contains. We will also need \texttt{vassumemany} that is just like
\texttt{assumemany} from before but which takes \texttt{vector}
arguments instead of \texttt{list}.

\begin{verbatim}
\tokpropconst{typeof_patt} : [\tokmetavariable{N}] \toktypeid{patt} \tokmetavariable{N} \tokarrowtype{\ensuremath{\to}} \toktypeid{typ} \tokarrowtype{\ensuremath{\to}} \tokstdtypeid{vector} \toktypeid{typ} \tokmetavariable{N} \tokarrowtype{\ensuremath{\to}} \tokbuiltintype{prop}.
\tokstdconst{vassumemany} : [\tokmetavariable{N}] (\tokmetavariable{A} \tokarrowtype{\ensuremath{\to}} \tokmetavariable{B} \tokarrowtype{\ensuremath{\to}} \tokbuiltintype{prop}) \tokarrowtype{\ensuremath{\to}} \tokstdtypeid{vector} \tokmetavariable{A} \tokmetavariable{N} \tokarrowtype{\ensuremath{\to}} \tokstdtypeid{vector} \tokmetavariable{B} \tokmetavariable{N} \tokarrowtype{\ensuremath{\to}} \tokbuiltintype{prop} \tokarrowtype{\ensuremath{\to}} \tokbuiltintype{prop}.
(...)
\end{verbatim}

Now, I can just go ahead and define the patterns, together with their
typing relation, \texttt{typeof\_patt}.

Let me just work one by one for each pattern.

\begin{verbatim}
\tokobjconst{patt_var} : \toktypeid{patt} (\tokstdtypeid{succ} \tokstdtypeid{zero}).
\tokpropconst{typeof_patt} \tokobjconst{patt_var} \tokmetavariable{T} (\tokstdconst{vcons} \tokmetavariable{T} \tokstdconst{vnil}).
\end{verbatim}

OK, that's how we'll write pattern variables, introducing a single
variable of a specific \texttt{typ} into the body of the branch. And the
following should be good for the \texttt{onat}s I defined earlier.

\begin{verbatim}
\tokobjconst{patt_ozero} : \toktypeid{patt} \tokstdtypeid{zero}.
\tokpropconst{typeof_patt} \tokobjconst{patt_ozero} \tokobjconst{onat} \tokstdconst{vnil}.

\tokobjconst{patt_osucc} : \toktypeid{patt} \tokmetavariable{N} \tokarrowtype{\ensuremath{\to}} \toktypeid{patt} \tokmetavariable{N}.
\tokpropconst{typeof_patt} (\tokobjconst{patt_osucc} \tokmetavariable{P}) \tokobjconst{onat} \tokmetavariable{VarTypes} \toksymbol{\ensuremath{:\!-}} \tokpropconst{typeof_patt} \tokmetavariable{P} \tokobjconst{onat} \tokmetavariable{VarTypes}.
\end{verbatim}

A wildcard pattern will match any value and should not introduce a
variable into the body of the branch.

\begin{verbatim}
\tokobjconst{patt_wild} : \toktypeid{patt} \tokstdtypeid{zero}.
\tokpropconst{typeof_patt} \tokobjconst{patt_wild} \tokmetavariable{T} \tokstdconst{vnil}.
\end{verbatim}

OK, and let's do patterns for our n-tuples\ldots{}. I guess I'll need a
type for lists of patterns too.

\begin{verbatim}
\tokobjconst{patt_tuple} : \toktypeid{pattlist} \tokmetavariable{N} \tokarrowtype{\ensuremath{\to}} \toktypeid{patt} \tokmetavariable{N}.
\tokpropconst{typeof_patt} (\tokobjconst{patt_tuple} \tokmetavariable{PS}) (\tokobjconst{product} \tokmetavariable{TS}) \tokmetavariable{VarTypes} \toksymbol{\ensuremath{:\!-}}
  \tokpropconst{typeof_pattlist} \tokmetavariable{PS} \tokmetavariable{TS} \tokmetavariable{VarTypes}.
\tokconst{pattlist} : (\tokmetavariable{N}: \tokbuiltintype{type}) \tokarrowtype{\ensuremath{\to}} \tokbuiltintype{type}.
\tokobjconst{pnil} : \toktypeid{patt} \tokstdtypeid{zero}.
\tokobjconst{pcons} : \toktypeid{patt} \tokmetavariable{N} \tokarrowtype{\ensuremath{\to}} \toktypeid{pattlist} \tokmetavariable{N'} \tokarrowtype{\ensuremath{\to}} \toktypeid{pattlist} (\tokmetavariable{N} + \tokmetavariable{N'}).
\end{verbatim}

Uh-oh\ldots{} don't think I can do that
\texttt{N\ +\ N\textquotesingle{}}, really. In this \texttt{pcons} case,
my pattern basically looks like \texttt{(P,\ ...PS)}; and I want the
overall pattern to have as many variables as \texttt{P} and \texttt{PS}
combined. But the GADTs support in \lamprolog seems to be quite basic. I
do not think there's any notion of type-level functions like
plus\footnote{Since GADTs in \foreignlanguage{greek}{λ}Prolog have not been considered in the past, we only present what is already supported by the existing language design and by many \foreignlanguage{greek}{λ}Prolog implementations in the present work. We are exploring extensions to \foreignlanguage{greek}{λ}Prolog to support type-level computation as part of future work.}\ldots{}.

However\ldots{} maybe I can work around that, if I change \texttt{patt}
to include an ``accumulator'' argument, say \texttt{NBefore}. Each
constructor for patterns will now define how many pattern variables it
adds to that accumulator, yielding \texttt{NAfter}, rather than defining
how many pattern variables it includes\ldots{} like this:

\begin{verbatim}
\tokconst{patt}, \tokconst{pattlist} : (\tokmetavariable{NBefore}: \tokbuiltintype{type}) (\tokmetavariable{NAfter}: \tokbuiltintype{type}) \tokarrowtype{\ensuremath{\to}} \tokbuiltintype{type}.
\tokobjconst{patt_var} : \toktypeid{patt} \tokmetavariable{N} (\tokstdtypeid{succ} \tokmetavariable{N}).
\tokobjconst{patt_ozero} : \toktypeid{patt} \tokmetavariable{N} \tokmetavariable{N}.
\tokobjconst{patt_osucc} : \toktypeid{patt} \tokmetavariable{N} \tokmetavariable{N'} \tokarrowtype{\ensuremath{\to}} \toktypeid{patt} \tokmetavariable{N} \tokmetavariable{N'}.
\tokobjconst{patt_wild} : \toktypeid{patt} \tokmetavariable{N} \tokmetavariable{N}.
\tokobjconst{patt_tuple} : \toktypeid{pattlist} \tokmetavariable{N} \tokmetavariable{N'} \tokarrowtype{\ensuremath{\to}} \toktypeid{patt} \tokmetavariable{N} \tokmetavariable{N'}.
\tokobjconst{pnil} : \toktypeid{pattlist} \tokmetavariable{N} \tokmetavariable{N}.
\tokobjconst{pcons} : \toktypeid{patt} \tokmetavariable{N} \tokmetavariable{N'} \tokarrowtype{\ensuremath{\to}} \toktypeid{pattlist} \tokmetavariable{N'} \tokmetavariable{N''} \tokarrowtype{\ensuremath{\to}} \toktypeid{pattlist} \tokmetavariable{N} \tokmetavariable{N''}.
\end{verbatim}

Yes, I think that should work. I have a little editing to do in my
existing predicates to use this representation instead. For top-level
patterns, we should always start with the accumulator being
\texttt{zero}\ldots{}

\begin{verbatim}
\tokobjconst{case_or_else} : (\tokmetavariable{Scrutinee}: \toktypeid{term})
  (\tokmetavariable{Patt}: \toktypeid{patt} \tokstdtypeid{zero} \tokmetavariable{N}) (\tokmetavariable{Vars_Body}: \tokstdtypeid{vbindmany} \toktypeid{term} \tokmetavariable{N} \toktypeid{term})
  (\tokmetavariable{Else}: \toktypeid{term}) \tokarrowtype{\ensuremath{\to}} \toktypeid{term}.
\end{verbatim}

I also have to change \texttt{typeof\_patt}, so that it includes an
accumulator argument of its own:

\importantCodeblock{}

\begin{verbatim}
\tokpropconst{typeof_patt} : [\tokmetavariable{NBefore} \tokmetavariable{NAfter}] \toktypeid{patt} \tokmetavariable{NBefore} \tokmetavariable{NAfter} \tokarrowtype{\ensuremath{\to}} \toktypeid{typ} \tokarrowtype{\ensuremath{\to}}
  \tokstdtypeid{vector} \toktypeid{typ} \tokmetavariable{NBefore} \tokarrowtype{\ensuremath{\to}} \tokstdtypeid{vector} \toktypeid{typ} \tokmetavariable{NAfter} \tokarrowtype{\ensuremath{\to}} \tokbuiltintype{prop}.
\end{verbatim}

\importantCodeblock{}

\begin{verbatim}
\tokpropconst{typeof} (\tokobjconst{case_or_else} \tokmetavariable{Scrutinee} \tokmetavariable{Pattern} \tokmetavariable{Vars_Body} \tokmetavariable{Else}) \tokmetavariable{BodyT} \toksymbol{\ensuremath{:\!-}}
  \tokpropconst{typeof} \tokmetavariable{Scrutinee} \tokmetavariable{T}, \tokpropconst{typeof_patt} \tokmetavariable{Pattern} \tokmetavariable{T} \tokstdconst{vnil} \tokmetavariable{VarTypes},
  \tokstdconst{vopenmany} \tokmetavariable{Vars_Body} (\tokkeyword{pfun} \tokmetavariable{Vars} \tokmetavariable{Body} \tokkeyword{\ensuremath{\Rightarrow}}
    \tokstdconst{vassumemany} \tokpropconst{typeof} \tokmetavariable{Vars} \tokmetavariable{VarTypes} (\tokpropconst{typeof} \tokmetavariable{Body} \tokmetavariable{BodyT})),
  \tokpropconst{typeof} \tokmetavariable{Else} \tokmetavariable{BodyT}.
\end{verbatim}

\importantCodeblockEnd{}

All right, let's proceed to the typing rules for patterns themselves:

\importantCodeblock{}

\begin{verbatim}
\tokpropconst{typeof_patt} \tokobjconst{patt_var} \tokmetavariable{T} \tokmetavariable{VarTypes} \tokmetavariable{VarTypes'} \toksymbol{\ensuremath{:\!-}}
  \tokstdconst{vsnoc} \tokmetavariable{VarTypes} \tokmetavariable{T} \tokmetavariable{VarTypes'}.
\end{verbatim}

\importantCodeblockEnd{}

OK, here I need \texttt{vsnoc} to add an element to the end of a vector.
That should yield the correct order for the types of pattern variables;
I am visiting the pattern left-to-right after all.

\begin{verbatim}
\tokstdconst{vsnoc} : [\tokmetavariable{N}] \tokstdtypeid{vector} \tokmetavariable{A} \tokmetavariable{N} \tokarrowtype{\ensuremath{\to}} \tokmetavariable{A} \tokarrowtype{\ensuremath{\to}} \tokstdtypeid{vector} \tokmetavariable{A} (\tokstdtypeid{succ} \tokmetavariable{N}) \tokarrowtype{\ensuremath{\to}} \tokbuiltintype{prop}.
\tokstdconst{vsnoc} \tokstdconst{vnil} \tokmetavariable{Y} (\tokstdconst{vcons} \tokmetavariable{Y} \tokstdconst{vnil}).
\tokstdconst{vsnoc} (\tokstdconst{vcons} \tokmetavariable{X} \tokmetavariable{XS}) \tokmetavariable{Y} (\tokstdconst{vcons} \tokmetavariable{X} \tokmetavariable{XS_Y}) \toksymbol{\ensuremath{:\!-}} \tokstdconst{vsnoc} \tokmetavariable{XS} \tokmetavariable{Y} \tokmetavariable{XS_Y}.
\end{verbatim}

The rest is easy to adapt\ldots{}.

\begin{scenecomment}
(Our hero finishes adapting the rest of the rules for \texttt{typeof\_patt},
which are available in the unabridged version of this story.)
\end{scenecomment}

Let me see if this works! I'll try out the predecessor function:

\begin{verbatim}
\tokpropconst{typeof} (\tokobjconst{lam} \tokmetavariable{_} (\tokkeyword{fun} \tokconst{n} \tokkeyword{\ensuremath{\Rightarrow}} \tokobjconst{case_or_else} \tokconst{n}
  (\tokobjconst{patt_osucc} \tokobjconst{patt_var}) (\tokstdconst{vbind} (\tokkeyword{fun} \tokconst{pred} \tokkeyword{\ensuremath{\Rightarrow}} \tokstdconst{vbody} \tokconst{pred}))
  \tokobjconst{ozero})) \tokmetavariable{T} ?
\tokquery{>>} \tokquery{Yes:}
\tokquery{>>} \tokmetavariable{T} \toksymbol{:=} \tokobjconst{arrow} \tokobjconst{onat} \tokobjconst{onat}.
\end{verbatim}

Great! Time to show this to Roza.

\identDialog
