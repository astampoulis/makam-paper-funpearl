\section{Where our heroes summarize what they've done and our story
concludes before the credits start
rolling}\label{where-our-heroes-summarize-what-theyve-done-and-our-story-concludes-before-the-credits-start-rolling}

\heroSTUDENT{} I feel like we've done a lot here. And some of the things we
did I don't think I've seen in the literature before, but then again,
it's not clear to me what's Makam-specific and what isn't. In any case,
I think a lot of people would find that quickly prototyping their PL
research ideas using this style of higher-order logic programming is
very useful.

\heroADVISOR{} I agree, though it would be hard for somebody to publish a
paper on this. Some of it is novel, some of it is folklore, some of it,
we just did in a pleasant way; and we did also use a couple of
not-so-pleasant hacks. But let's make a list of what's what.

\vspace{-0.5em}

\begin{itemize}
\item
  We defined HOAS encodings of complicated binding forms, including
  mutually recursive definitions and patterns, while only having
  explicit support in our metalanguage for single-variable binding.
  These encodings seem to have been part of PL folklore, but we believe
  that a type like \texttt{bindmany} has never been shown as a reusable
  datatype that \lamprolog makes possible. We have made use of GADTs to
  encode some of these binding structures precisely, which we show are
  supported in \lamprolog through ad-hoc polymorphism. This we believe
  is a novel usage for this \lamprolog feature. Our binding
  constructions should be replicable in the standard \foreignlanguage{greek}{λ}Prolog/Teyjus
  implementation \citep{teyjus-main-reference} and in ELPI
  \citep{elpi-main-reference}.
\item
  We defined a generic predicate to perform structural recursion using a
  very concise definition. It allows us to define structurally recursive
  predicates that only explicitly list out the important cases, in what
  we believe is a novel encoding for the \lamprolog
   setting. Any new definitions, such as constructors or datatypes we
  introduce later, do not need any special provision to be covered by
  the same predicates. They depend on a number of reflective predicates,
  which are available in other Prolog dialects; however, we are not
  aware of a published example that makes use of them in the \lamprolog
   setting. These predicates are used to reflect on the structure of
  Makam terms and to get the list of local assumptions; for the most
  part, their use is limited to predicates that would be part of the
  standard library, not in user code.
\item
  The above encodings are reusable and have been made part of the Makam
  standard library. As a result, we were able to develop the type
  checker for quite an advanced type system, in very few lines of code
  specific to it, using rules that we believe do not, presentation-wise,
  stray far from their pen-and-paper versions. Our development includes
  mutually recursive definitions, polymorphism, datatypes, pattern
  matching, a conversion rule, Hindley-Milner type generalization, and
  staging constructs that allow the computation of contextually typed
  open terms of the simply typed lambda calculus. We are not aware of
  another metalinguistic framework that allows this level of
  expressivity and has been used to encode such type-system features
  with the same level of concision.
\item
  We have also shown that higher-order logic programming allows not just
  meta-level functions to be reused for encoding object-level binding;
  there are also cases where meta-level unification can also be reused
  to encode certain object-level features: for example, doing type
  generalization as in Algorithm W.
\end{itemize}

\heroSTUDENT{} Well, that was very interesting; thank you for working with me
on this!

\heroADVISOR{} I enjoyed this, too. Say, if you want to relax, there's a new
staging of the classic play by \citet{fischer2010play} downtown -- I saw
it yesterday, and it is really good!

\heroSTUDENT{} That's a great idea! You know, I wish there were more plays
like it\ldots{}. Well, good night, and see you on Monday!
