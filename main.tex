%% For double-blind review submission, w/o CCS and ACM Reference (max submission space)
\documentclass[acmsmall,review,anonymous]{acmart}\settopmatter{printfolios=true,printccs=false,printacmref=false}
%% For double-blind review submission, w/ CCS and ACM Reference
%\documentclass[acmsmall,review,anonymous]{acmart}\settopmatter{printfolios=true}
%% For single-blind review submission, w/o CCS and ACM Reference (max submission space)
%\documentclass[acmsmall,review]{acmart}\settopmatter{printfolios=true,printccs=false,printacmref=false}
%% For single-blind review submission, w/ CCS and ACM Reference
%\documentclass[acmsmall,review]{acmart}\settopmatter{printfolios=true}
%% For final camera-ready submission, w/ required CCS and ACM Reference

\usepackage[utf8]{inputenc}
\usepackage[greek,english]{babel}
\usepackage{booktabs}
\usepackage{subcaption}
\usepackage{alltt}
\usepackage{xspace}
\usepackage{mathpartir}
\usepackage{indentfirst}
\usepackage{hyperref}
\usepackage{color}
\usepackage{fancyvrb}

\bibliographystyle{shared/ACM-Reference-Format}
\citestyle{acmauthoryear}   %% For author/year citations

%% Journal information
%% Supplied to authors by publisher for camera-ready submission;
%% use defaults for review submission.
\acmJournal{PACMPL}
\acmVolume{1}
\acmNumber{CONF} % CONF = POPL or ICFP or OOPSLA
\acmArticle{1}
\acmYear{2018}
\acmMonth{1}
\acmDOI{} % \acmDOI{10.1145/nnnnnnn.nnnnnnn}
\startPage{1}

%% Copyright information
%% Supplied to authors (based on authors' rights management selection;
%% see authors.acm.org) by publisher for camera-ready submission;
%% use 'none' for review submission.
\setcopyright{none}
%\setcopyright{acmcopyright}
%\setcopyright{acmlicensed}
%\setcopyright{rightsretained}
%\copyrightyear{2018}           %% If different from \acmYear

\begin{document}

\title{Prototyping a Functional Language using Higher-Order Logic Programming}
\subtitle{A Functional Pearl on learning the ways of \lamprolog/Makam}

\author{Antonis Stampoulis}
\affiliation{
  \institution{Originate Inc.}
  \city{New York}
  \state{New York}
}
\email{antonis.stampoulis@gmail.com}

\author{Adam Chlipala}
\affiliation{
  \department{CSAIL}
  \institution{MIT}
  \city{Cambridge}
  \state{Massachusetts}
}
\email{adamc@csail.mit.edu}

%% -- Macro definitions
\newcommand\TODO[0]{\textbf{TODO}}
\newcommand\lamprolog[0]{\foreignlanguage{greek}{λ}Prolog\xspace}
\newcommand\fomega[0]{F$\omega$\xspace}
\renewenvironment{verbatim}{\begin{quote}\begin{alltt}}{\end{alltt}\end{quote}}
\newenvironment{codequote}{\begin{quote}\begin{alltt}}{\end{alltt}\end{quote}}
\newcommand\hide[1]{}
\newcommand\tightlist[0]{\itemsep1pt\parskip0pt\parsep0pt}
\renewcommand\thesection{\textbf{CHAPTER \arabic{section}}}
\renewcommand\thesubsection{\textbf{SECTION \arabic{section}.\arabic{subsection}}}
\newcommand\hero[1]{\textit{#1}.}
\newcommand\heroSTUDENT[0]{\hero{HAGOP}}
\newcommand\heroADVISOR[0]{\hero{ROZA}}
\newcommand\heroAUDIENCE[0]{\hero{AUDIENCE}}
\newcommand\heroAUTHOR[0]{\hero{ANONYMOUS AUTHOR}}
\newenvironment{scenecomment}{\em\noindent}{}
\newcommand\identDialog[0]{\setlength{\leftskip}{1em}\setlength{\parindent}{-1em}}
\newcommand\identNormal[0]{\setlength{\leftskip}{0em}\setlength{\parindent}{0em}}
\newenvironment{normalident}{\identNormal}{\identDialog}

\definecolor{todo}{rgb}{0.75,0.00,0.00}
\newcommand\todo[1]{\textcolor{todo}{#1}}
\newcommand\heroTODO[0]{\textbf{\todo{TODO}}.}
\newcommand\heroNEEDFEEDBACK[0]{\textbf{\todo{Need Feedback}}:}

% =====
% used for highlighting from pandoc
\DefineVerbatimEnvironment{Highlighting}{Verbatim}{commandchars=\\\{\}}
\newenvironment{Shaded}{\begin{quote}}{\end{quote}}
\newcommand{\KeywordTok}[1]{\textcolor[rgb]{0.00,0.44,0.13}{\textbf{{#1}}}}
\newcommand{\DataTypeTok}[1]{\textcolor[rgb]{0.56,0.13,0.00}{{#1}}}
\newcommand{\DecValTok}[1]{\textcolor[rgb]{0.25,0.63,0.44}{{#1}}}
\newcommand{\BaseNTok}[1]{\textcolor[rgb]{0.25,0.63,0.44}{{#1}}}
\newcommand{\FloatTok}[1]{\textcolor[rgb]{0.25,0.63,0.44}{{#1}}}
\newcommand{\CharTok}[1]{\textcolor[rgb]{0.25,0.44,0.63}{{#1}}}
\newcommand{\StringTok}[1]{\textcolor[rgb]{0.25,0.44,0.63}{{#1}}}
\newcommand{\CommentTok}[1]{\textcolor[rgb]{0.38,0.63,0.69}{\textit{{#1}}}}
\newcommand{\OtherTok}[1]{\textcolor[rgb]{0.00,0.44,0.13}{{#1}}}
\newcommand{\AlertTok}[1]{\textcolor[rgb]{1.00,0.00,0.00}{\textbf{{#1}}}}
\newcommand{\FunctionTok}[1]{\textcolor[rgb]{0.02,0.16,0.49}{{#1}}}
\newcommand{\RegionMarkerTok}[1]{{#1}}
\newcommand{\ErrorTok}[1]{\textcolor[rgb]{1.00,0.00,0.00}{\textbf{{#1}}}}
\newcommand{\NormalTok}[1]{{#1}}
% =====

\begin{abstract}
We demonstrate how the framework of \emph{higher-order logic programming}, as exemplified
in the \lamprolog language design, is a prime vehicle for rapid prototyping of
implementations for programming languages with sophisticated type systems. We present the
literate development of a type checker for a language with a number of complicated
features, culminating in a standard ML-style core with polymorphic algebraic datatypes and
type generalization, extended with dependently typed constructs that are generic over a
separately defined language of dependent indices. We add each new feature in sequence,
without requiring changes to existing code; as a result, we are able to extend both the ML
core and the universe of dependent indices independently from each other. Scaling the
higher-order logic programming approach to this setting required us to develop novel
approaches to challenges like complex variable binding patterns in object languages,
performing generic structural traversals of code, and reusing the unification procedure of
the metalanguage as much as possible. For our development, we make use of Makam, a new
implementation of \lamprolog, which we introduce in tutorial style as part of our
literate development.
\end{abstract}

%% 2012 ACM Computing Classification System (CSS) concepts
%% Generate at 'http://dl.acm.org/ccs/ccs.cfm'.
\begin{CCSXML}
<ccs2012>
<concept>
<concept_id>10011007.10011006.10011008</concept_id>
<concept_desc>Software and its engineering~General programming languages</concept_desc>
<concept_significance>500</concept_significance>
</concept>
<concept>
<concept_id>10003456.10003457.10003521.10003525</concept_id>
<concept_desc>Social and professional topics~History of programming languages</concept_desc>
<concept_significance>300</concept_significance>
</concept>
</ccs2012>
\end{CCSXML}

\ccsdesc[500]{Software and its engineering~General programming languages}
\ccsdesc[300]{Social and professional topics~History of programming languages}
%% End of generated code

% \keywords{keyword1, keyword2, keyword3}  %% \keywords are mandatory in final camera-ready submission

\maketitle

{
  \setlength{\parskip}{3pt}
  \renewcommand{\labelitemi}{\textendash}

  % Hanging Indent
  \identDialog{}

  % No Indent
  % \identNormal{}

  {\large \todo{Version identifier: \textbf{\input{generated/randomword}}}}
  
  \input{generated/01-introduction}
  
  \input{generated/02-stlc}
  
  \input{generated/03-bindmany}
  
  \section{Where the legend of the GADTs and the Ad-Hoc Polymorphism is
recounted}\label{where-the-legend-of-the-gadts-and-the-ad-hoc-polymorphism-is-recounted}

\identNormal\it

Once upon a time, our republic lacked one of the natural wonders that it
is now well-known for, and which is now regularly enjoyed by tourists
and inhabitants alike. I am talking of course about the Great Arboretum
of Dangling Trees, known as GADTs for short. Then settlers from the
far-away land of the Dependency started coming to the republic, and
started speaking of Lists that Knew Their Length, of Terms that Knew
Their Types, of Collections of Elements that were Heterogeneous, and
about the other magical beings of their home. And they set out to build
a natural environment for these beings on the republic, namely the GADTs
that we know and love, to remind them of home a little. And their work
was good and was admired by many.

A long time passed, and dispatches from another far-away land came to
the republic, written by authors whose names are now lost in the sea of
anonymity, and I fear might forever remain so. And the dispatches went
something like this.

\rm

\heroAUTHOR{} \ldots{} In my land of \lamprolog that I speak of, the type
system is a subset of System F\(_\omega\) that should be familiar to you
-- the simply typed lambda calculus, plus prenex polymorphism, plus
simple type constructors of the form
\texttt{type\ *\ ...\ *\ type\ \ensuremath{\to}\ type}. There is also a
limited form of rank-2 polymorphism, allowing types of the form
\texttt{forall\ A\ T}, which are inhabited by unapplied polymorphic
constants through the notation \texttt{@foo}. There is a \texttt{prop}
sort for propositions, which is a normal type, but also a bit special:
its terms are not just values but are also computations, activated when
queried upon.

However, the language of this land has a distinguishing feature, called
Ad-Hoc Polymorphism. You see, the different rules that define a
predicate in our language can \emph{specialize} their type arguments.
This can be used to define polymorphic predicates that behave
differently for different types, like this, where we are essentially
doing a \texttt{typecase} and we choose a rule depending on the
\emph{type} of the argument (as opposed to its value):

\begin{verbatim}
print : [A] A \ensuremath{\to} prop.
print (I: int) :- (... code for printing integers ...)
print (S: string) :- (... code for printing strings ...)
\end{verbatim}

The local dialects Teyjus
\citep{teyjus-main-reference,teyjus-2-implementation} and Makam include
this feature, while it is not encountered in other dialects like ELPI
\citep{elpi-main-reference}. In the Makam dialect in particular, type
variables are understood to be parametric by default. In order to make
them ad-hoc and allow specializing them in rules, we need to denote them
using the \texttt{{[}A{]}} notation.

Of course, this feature has both to do with the statics as well as the
dynamics of our language: and while dynamically it means something akin
to a \texttt{typecase}, statically, it means that rules might specialize
their type variables, and this remains so for type-checking the whole
rule.

But alas! Is it not type specialization during pattern matching that is
an essential feature of the GADTs of your land? Maybe that means that we
can use Ad-Hoc Polymorphism not just to do \texttt{typecase} but also to
work with GADTs in our land? Consider this! The venerable List that
Knows Its Length:

\begin{verbatim}
zero : type. succ : type \ensuremath{\to} type.
vector : type \ensuremath{\to} type \ensuremath{\to} type.
vnil : vector A zero.
vcons : A \ensuremath{\to} vector A N \ensuremath{\to} vector A (succ N).
\end{verbatim}

And now for the essential \texttt{vmap}:

\begin{verbatim}
vmap : [N] (A \ensuremath{\to} B \ensuremath{\to} prop) \ensuremath{\to} vector A N \ensuremath{\to} vector B N \ensuremath{\to} prop.
vmap P vnil vnil.
vmap P (vcons X XS) (vcons Y YS) :- P X Y, vmap P XS YS.
\end{verbatim}

In each rule, the first argument already specializes the \texttt{N} type
-- in the first rule to \texttt{zero}, in the second, to
\texttt{succ\ N}. And so erroneous rules that do not respect this
specialization would not be accepted as well-typed sayings in our
language:

\begin{verbatim}
vmap P vnil (vcons X XS) :- ...
\end{verbatim}

And we should note that in this usage of Ad-Hoc Polymorphism for GADTs,
it is only the increased precision of the statics that we care about.
Dynamically, the rules for \texttt{vmap} can perform normal term-level
unification and only look at the constructors \texttt{vnil} and
\texttt{vcons} to see whether each rule applies, rather than relying on
the \texttt{typecase} aspects we spoke of before.

Coupling this with the binding constructs that I talked to you earlier
about, we can build new magical beings, like the \emph{Bind that Knows
Its Length}:

\begin{verbatim}
vbindmany : (Var: type) (N: type) (Body: type) \ensuremath{\to} type.
vbody : Body \ensuremath{\to} vbindmany Var zero Body.
vbind : (Var \ensuremath{\to} vbindmany Var N Body) \ensuremath{\to} vbindmany Var (succ N) Body.
\end{verbatim}

(Whereby I am using notation of the Makam dialect in my definition of
\texttt{vbindmany} that allows me to name parameters, purely for the
purposes of increased clarity.)

In the \texttt{openmany} version for \texttt{vbindmany}, the rules are
exactly as before, though the static type is more precise:

\begin{verbatim}
vopenmany : [N] vbindmany Var N Body \ensuremath{\to} (vector Var N \ensuremath{\to} Body \ensuremath{\to} prop) \ensuremath{\to} prop.
vopenmany (vbody Body) Q :- Q vnil Body.
vopenmany (vbind F) Q :-
  (x:A \ensuremath{\to} vopenmany (F x) (fun xs \ensuremath{\Rightarrow} Q (vcons x xs))).
\end{verbatim}

We can also showcase the \emph{Accurate Encoding of the Letrec}:

\begin{verbatim}
vletrec : vbindmany term N (vector term N * term) \ensuremath{\to} term.
\end{verbatim}

And that is the way that the land of \lamprolog supports GADTs, without
needing the addition of any feature, all thanks to the existing support
for Ad-Hoc Polymorphism.

\identDialog

  
  \section{Where our hero Hagop adds pattern matching on his
own}\label{where-our-hero-hagop-adds-pattern-matching-on-his-own}

\begin{scenecomment}
(Roza had a meeting with another student, so Hagop took a small break, and is now back at his
office. He is trying to work out on his own how to encode patterns. He is fairly
confident at this point that having explicit support for single-variable
binding is enough to model most complicated forms of binding, especially when making use of
polymorphism and GADTs.)
\end{scenecomment}

\identNormal
\heroSTUDENT{} OK, so let's implement simple patterns and pattern-matching
like in ML\ldots{} First let's determine the right binding structure.
For a branch like:

\begin{verbatim}
| cons(hd, tl) \ensuremath{\to} ... hd .. tl ...
\end{verbatim}

the pattern introduces 2 variables, \texttt{hd} and \texttt{tl}, which
the body of the branch can refer to. But we can't really refer to those
variables in the pattern itself, at least for simple
patterns\footnote{There are counterexamples, like for or-patterns in some ML dialects, or for dependent pattern matching, where consequent uses of the same variable perform exact matches rather than unification. We choose to omit the handling of cases like those in the present work for presentation purposes.}\ldots{}.
So there's no binding going on really within the pattern; instead, once
we figure out how many variables a pattern introduces, we can do the
actual binding all at once, when we get to the body of the branch:

\begin{verbatim}
branch(pattern, bind [# of variables in pattern].body)
\end{verbatim}

So we could write the above branch in Makam like this:

\begin{verbatim}
branch(patt_cons patt_var patt_var,
       bind (fun hd \ensuremath{\Rightarrow} bind (fun tl \ensuremath{\Rightarrow} body (.. hd .. tl ..))))
\end{verbatim}

We do have to keep the order of variables consistent somehow, so
\texttt{hd} here should refer to the first occurrence of
\texttt{patt\_var}, and \texttt{tl} to the second. Based on these, I am
thinking that the type of \texttt{branch} should be something like:

\begin{verbatim}
branch : (Pattern: patt N) (Vars_Body: vbindmany term N term) \ensuremath{\to} ...
\end{verbatim}

Wait, before I get into the weeds let me just set up some things. First,
let's add a simple base type, say \texttt{nat}s, to have something to
work with as an example. I'll prefix their names with \texttt{o} for
``object language,'' so as to avoid ambiguity. And I will also add a
\texttt{case\_or\_else} construct, standing for a single-branch
pattern-match construct. It should be easy to extend to a
multiple-branch construct, but I want to keep things as simple as
possible. I'll inline what I had written for \texttt{branch} above into
the definition of \texttt{case\_or\_else}.

\begin{verbatim}
onat : typ. ozero : term. osucc : term \ensuremath{\to} term.
typeof ozero onat. typeof (osucc N) onat \ensuremath{:\!-} typeof N onat.
eval ozero ozero. eval (osucc E) (osucc V) \ensuremath{:\!-} eval E V.
\end{verbatim}

\begin{verbatim}
case_or_else : (Scrutinee: term)
  (Patt: patt N) (Vars_Body: vbindmany term N term)
  (Else: term) \ensuremath{\to} term.
\end{verbatim}

Now for the typing rule -- it will be something like this:

\begin{verbatim}
typeof (case_or_else Scrutinee Pattern Vars_Body Else) BodyT \ensuremath{:\!-}
  typeof Scrutinee T, typeof_patt Pattern T VarTypes,
  vopenmany Vars_Body (pfun vars body \ensuremath{\Rightarrow}
    vassumemany typeof vars VarTypes (typeof body BodyT)),
  typeof Else BodyT.
\end{verbatim}

Right, so when checking a pattern, we'll have to determine both what
type of scrutinee it matches, as well as the types of the variables that
it contains. We will also need \texttt{vassumemany} that is just like
\texttt{assumemany} from before but which takes \texttt{vector}
arguments instead of \texttt{list}.

\begin{verbatim}
typeof_patt : [N] patt N \ensuremath{\to} typ \ensuremath{\to} vector typ N \ensuremath{\to} prop.
vassumemany : [N] (A \ensuremath{\to} B \ensuremath{\to} prop) \ensuremath{\to} vector A N \ensuremath{\to} vector B N \ensuremath{\to} prop \ensuremath{\to} prop.
(...)
\end{verbatim}

Now, I can just go ahead and define the patterns, together with their
typing relation, \texttt{typeof\_patt}.

Let me just work one by one for each pattern.

\begin{verbatim}
patt_var : patt (succ zero).
typeof_patt patt_var T (vcons T vnil).
\end{verbatim}

OK, that's how we'll write pattern variables, introducing a single
variable of a specific \texttt{typ} into the body of the branch. And the
following should be good for the \texttt{onat}s I defined earlier.

\begin{verbatim}
patt_ozero : patt zero.
typeof_patt patt_ozero onat vnil.

patt_osucc : patt N \ensuremath{\to} patt N.
typeof_patt (patt_osucc P) onat VarTypes \ensuremath{:\!-} typeof_patt P onat VarTypes.
\end{verbatim}

A wildcard pattern will match any value and should not introduce a
variable into the body of the branch.

\begin{verbatim}
patt_wild : patt zero.
typeof_patt patt_wild T vnil.
\end{verbatim}

OK, and let's do patterns for our n-tuples\ldots{}. I guess I'll need a
type for lists of patterns too.

\begin{verbatim}
patt_tuple : pattlist N \ensuremath{\to} patt N.
typeof_patt (patt_tuple PS) (product TS) VarTypes \ensuremath{:\!-}
  typeof_pattlist PS TS VarTypes.
pattlist : (N: type) \ensuremath{\to} type.
pnil : patt zero.
pcons : patt N \ensuremath{\to} pattlist N' \ensuremath{\to} pattlist (N + N').
\end{verbatim}

Uh-oh\ldots{} don't think I can do that
\texttt{N\ +\ N\textquotesingle{}} really. In this \texttt{pcons} case,
my pattern basically looks like \texttt{(P,\ ...PS)}; and I want the
overall pattern to have as many variables as \texttt{P} and \texttt{PS}
combined. But the GADTs support in \lamprolog seems to be quite basic. I
do not think there's any notion of type-level functions like
plus\ldots{}.

However\ldots{} maybe I can work around that, if I change \texttt{patt}
to include an ``accumulator'' argument, say \texttt{NBefore}. Each
constructor for patterns will now define how many pattern variables it
adds to that accumulator, yielding \texttt{NAfter}, rather than defining
how many pattern variables it includes\ldots{} like this:

\begin{verbatim}
patt, pattlist : (NBefore: type) (NAfter: type) \ensuremath{\to} type.
patt_var : patt N (succ N).
patt_ozero : patt N N.
patt_osucc : patt N N' \ensuremath{\to} patt N N'.
patt_wild : patt N N.
patt_tuple : pattlist N N' \ensuremath{\to} patt N N'.

pnil : pattlist N N.
pcons : patt N N' \ensuremath{\to} pattlist N' N'' \ensuremath{\to} pattlist N N''.
\end{verbatim}

Yes, I think that should work. I have a little editing to do in my
existing predicates to use this representation instead. For top-level
patterns, we should always start with the accumulator being
\texttt{zero}\ldots{}

\begin{verbatim}
case_or_else : (Scrutinee: term)
  (Patt: patt zero N) (Vars_Body: vbindmany term N term)
  (Else: term) \ensuremath{\to} term.
\end{verbatim}

I think I'll also have to change \texttt{typeof\_patt}, so that it
includes an accumulator argument of its own:

\begin{verbatim}
typeof_patt : [NBefore NAfter] patt NBefore NAfter \ensuremath{\to} typ \ensuremath{\to}
  vector typ NBefore \ensuremath{\to} vector typ NAfter \ensuremath{\to} prop.

typeof (case_or_else Scrutinee Pattern Vars_Body Else) BodyT \ensuremath{:\!-}
  typeof Scrutinee T, typeof_patt Pattern T vnil VarTypes,
  vopenmany Vars_Body (pfun vars body \ensuremath{\Rightarrow}
    vassumemany typeof vars VarTypes (typeof body BodyT)),
  typeof Else BodyT.
\end{verbatim}

All right, let's proceed to the typing rules for patterns themselves:

\begin{verbatim}
typeof_patt patt_var T VarTypes VarTypes' \ensuremath{:\!-}
  vsnoc VarTypes T VarTypes'.
\end{verbatim}

OK, here I need \texttt{vsnoc} to add an element to the end of a vector.
That should yield the correct order for the types of pattern variables;
I am visiting the pattern left-to-right after all.

\begin{verbatim}
vsnoc : [N] vector A N \ensuremath{\to} A \ensuremath{\to} vector A (succ N) \ensuremath{\to} prop.
vsnoc vnil Y (vcons Y vnil).
vsnoc (vcons X XS) Y (vcons X XS_Y) \ensuremath{:\!-} vsnoc XS Y XS_Y.
\end{verbatim}

The rest is easy to adapt\ldots{}.

\begin{scenecomment}
(Our hero finishes adapting the rest of the rules for \texttt{typeof\_patt},
which are available in the unabridged version of this story.)
\end{scenecomment}

Let me see if this works! I'll try out the predecessor function:

\begin{verbatim}
typeof (lam _ (fun n \ensuremath{\Rightarrow} case_or_else n
  (patt_osucc patt_var) (vbind (fun pred \ensuremath{\Rightarrow} vbody pred))
  ozero)) T ?
>> Yes:
>> T := arrow onat onat.
\end{verbatim}

Great! Time to show this to Roza.

\identDialog


  \input{generated/06-miniml}
  
  \input{generated/07-synonyms}
  
  \section{Where our heroes tackle a new level of meta, contexts and
substitutions}\label{where-our-heroes-tackle-a-new-level-of-meta-contexts-and-substitutions}

\heroSTUDENT{} I'm fairly confident by now that Makam should be able to handle
the research idea we want to try out. Shall we get to it?

\heroADVISOR{} Yes, it is time. So, what we are aiming to do is add a facility
for type-safe, heterogeneous meta-programming to our object language,
similar to MetaHaskell \citep{mainland2012explicitly}. This way we can
manipulate the terms of a \emph{separate} object language in a type-safe
manner.

\heroSTUDENT{} Exactly. For the research language we have in mind, we aim for
our object language to be a formal logic, so our language will be
similar to Beluga \citep{beluga-main-reference} or VeriML
\citep{veriml-main-reference}. We will also need dependent functions and
pattern-matching over the object language\ldots{} But we don't need to
do all of that; let's just do a basic version for now, and I can do the
rest on my own.

\newcommand\dep[1]{\ensuremath{#1}}
\newcommand\lift[1]{\ensuremath{\langle#1\rangle}}
\newcommand\odash[0]{\ensuremath{\vdash_{\text{o}}}}
\newcommand\wf[0]{\ensuremath{\; \text{wf}}}
\newcommand\aq[1]{\ensuremath{\texttt{aq}(#1)}}
\newcommand\aqopen[1]{\ensuremath{\texttt{aqopen}(#1)}}

\heroADVISOR{} Sounds good. First, let's agree on some terminology, because a
lot of words are getting overloaded a lot. Let us call \emph{objects}
\(o\) any sorts of terms of the object language that we will be
manipulating. And, for a lack of a better word, let us call
\emph{classes} \(c\) the ``types'' that characterize those objects
through a typing relation of the form \(\Psi \odash o : c\). It is
unfortunate that these names suggest object-orientation, but this is not
the intent.

\heroSTUDENT{} I see what you are saying. Let's keep the objects simple -- to
start, let's just do the terms of the simply typed lambda calculus
(STLC). In that cases our classes will just be the types of STLC. The
objects are run-time entities: essentially, our programs will be able to
``compute'' objects. So we need a way to return (or ``lift'') an object
\(o\) as a meta-level value \(\lift{o}\).

\heroADVISOR{} Good idea. We are getting into many levels of meta -- there's
the metalanguage we're using, Makam; there's the object language we are
encoding, which is now becoming a metalanguage in itself, let's call
that Heterogeneous Meta ML Light (HMML?); and there's the
``object-object'' language that HMML is manipulating. One option would
be to have the object-object language be the full HMML metalanguage
itself, which would lead us to a homogeneous, multi-stage language like
MetaML \citep{metaml-main-reference}. But, I agree, we should keep the
object-object language simple: the STLC will suffice.

\heroSTUDENT{} Great. How about we try to do the standard example of a staged
\texttt{power} function? Here's a rough sketch, where I'm using
\texttt{\textasciitilde{}I} for antiquotation:

\begin{verbatim}
let power (n: onat): < stlc.arrow stlc.onat stlc.onat > =
  match n with
    0 \ensuremath{\Rightarrow} < stlc.lam (fun x \ensuremath{\Rightarrow} 1) >
  | S n' \ensuremath{\Rightarrow} letobj I = power n' in
      < stlc.lam (fun x \ensuremath{\Rightarrow} stlc.mult (stlc.app ~I x) x) >
\end{verbatim}

\heroADVISOR{} It's a plan. So, let's get to it. Should we write some of the
system down on paper first?

\heroSTUDENT{} Yes, that would be very useful. For this example, we will need
the lifting construct \(\lift{\cdot}\) and the \texttt{letobj} typing
rules. Here are their typing rules, which depend on an appropriately
defined typing judgment \(\Psi \odash o : c\) for objects. In our case,
this will initially match the \(\Psi; \Delta \vdash t : e\) typing
judgment for STLC. We use \(\dep{i}\) for variables standing for
objects, which we will call \emph{indices}. And we will need a way to
antiquote indices inside STLC terms, which means that we will have to
\emph{extend} the STLC terms as well as their typing judgment
accordingly. Last, I'll also write down their evaluation rules, as they
are quite simple.

\newcommand\stlce[0]{\hat{e}}
\newcommand\stlct[0]{\hat{t}}
\newcommand\stlc[1]{\hat{#1}}

\vspace{-1.5em}\begin{mathpar}
\begin{array}{ll}
\rulename{Ob-Ob-Syntax}                                                   & \rulename{HMML-Syntax} \\
\stlce  ::= \lambda x:\stlct.\stlce \; | \; \stlce_1 \; \stlce_2 \; | \; x \; | \; n \; | \; \stlce_1 * \stlce_2 \; | \; \textbf{\aq{i}} & e ::= \text{...} \; | \; \lift{\dep{o}} \; | \; \texttt{letobj} \; \dep{i} = \dep{o} \; \texttt{in} \; e \\
\stlct  ::= \stlct_1 \to \stlct_2 \; | \; \stlc{\text{nat}} & \tau ::= \text{...} \; | \; \lift{\dep{c}} \\
\dep{o} ::= \stlce \hspace{1.5em} \dep{c} ::= \stlct &
\end{array} \\

\inferrule[Typeof-LiftObj]
          {\dep{\Psi} \odash \dep{o} : \dep{c}}
          {\Gamma; \dep{\Psi} \vdash \lift{\dep{o}} : \lift{\dep{c}}}

\inferrule[Typeof-LetObj]
          {\Gamma; \dep{\Psi} \vdash e : \lift{\dep{c}} \\ \Gamma; \dep{\Psi}, \; \dep{i} : \dep{c} \vdash e : \tau \\ i \not\in \text{fv}(\tau)}
          {\Gamma; \dep{\Psi} \vdash \texttt{letobj} \; \dep{i} = e \; \texttt{in} \; e' : \tau}

\inferrule[STLC-Typeof-Antiquote]
          {\dep{i} : \stlct \in \Psi}
          {\Psi; \Delta \vdash \aq{\dep{i}} : \stlct}
          
\inferrule[Eval-LiftObj]
          {\hspace{1em}}{\lift{\dep{o}} \Downarrow \lift{\dep{o}}}

\inferrule[Eval-LetObj]
          {e \Downarrow \lift{\dep{o}} \\ e'[\dep{o}/\dep{i}] \Downarrow v}
          {\texttt{letobj} \; \dep{i} = e \; \texttt{in} \; e' \Downarrow v}

\inferrule[SubstObj]{}{
  e[\dep{o}/\dep{i}] = e' \; \text{defined by structural recursion, save for:} \; {\aq{\dep{i}}[\stlce/\dep{i}] = \stlce}
}
\end{mathpar}

The typing rules should be quite simple to transcribe to Makam:

\begin{verbatim}
object, class, index : type.
classof : object \ensuremath{\to} class \ensuremath{\to} prop.
classof_index : index \ensuremath{\to} class \ensuremath{\to} prop.
subst_obj : (I_E: index \ensuremath{\to} term) (O: object) (E_O'I: term) \ensuremath{\to} prop.

liftobj : object \ensuremath{\to} term. liftclass : class \ensuremath{\to} typ.
typeof (liftobj O) (liftclass C) :- classof O C.

letobj : term \ensuremath{\to} (index \ensuremath{\to} term) \ensuremath{\to} term.
typeof (letobj E EF') T :-
  typeof E (liftclass C), (i:index \ensuremath{\to} classof_index i C \ensuremath{\to} typeof (EF' i) T).

eval (liftobj O) (liftobj O).
eval (letobj E I_E') V :-
  eval E (liftobj O), subst_obj I_E' O E', eval E' V.
\end{verbatim}

\heroADVISOR{} Great. I'll add the object language in a separate namespace
prefix -- we can use `\texttt{\%extend}' for going into a namespace --
and I'll just copy-paste our STLC code from earlier on. Let me also add
our new antiquote as a new STLC term constructor!

\begin{verbatim}
%extend stlc.
term : type. typ : type. typeof : term \ensuremath{\to} typ \ensuremath{\to} prop.
...
aq : index \ensuremath{\to} term.
%end.
\end{verbatim}

\heroSTUDENT{} Time to add STLC terms as \texttt{object}s, and their types as
\texttt{class}es. We can then give the corresponding rule for
\texttt{classof}. And I think that's it for the typing rules!

\begin{verbatim}
obj_term : stlc.term \ensuremath{\to} object. cls_typ : stlc.typ \ensuremath{\to} class.
classof (obj_term E) (cls_typ T) :- stlc.typeof E T.
stlc.typeof (stlc.aq I) T :- classof_index I (cls_typ T).
\end{verbatim}

\begin{scenecomment}
(Hagop transcribes the example from before. Writing out the term takes several lines, so he finds himself
wishing that Makam supported some way to write terms of object languages in their native syntax;
quite curiously, he also finds himself wishing that he had a stack of blank pages.
Unbeknownst to him, his first wish has already been granted, but his second wish hasn't, so he
will have to learn about it at some later time in the future.)
\end{scenecomment}

\begin{verbatim}
typeof (letrec (bind (fun power \ensuremath{\Rightarrow} body ([ ..long term.. ], power)))) T ?
>> Yes:
>> T := arrow onat (liftclass (cls_typ (stlc.arrow stlc.onat stlc.onat))).
\end{verbatim}

\heroADVISOR{} That's great! Only thing missing to try out an evaluation
example too is implementing \texttt{subst\_obj}. Thanks to
\texttt{structural\_recursion} though, that is very easy:

\begin{verbatim}
subst_obj_aux, subst_obj_cases : [Any]
  (Var: index) (Replace: object) (Where: Any) (Result: Any) \ensuremath{\to} prop.
subst_obj I_Term O Term_O'I :-
  (i:index \ensuremath{\to} subst_obj_aux i O (I_Term i) Term_O'I).

subst_obj_aux Var Replace Where Result :-
  if (subst_obj_cases Var Replace Where Result)
  then success
  else (structural_recursion @(subst_obj_aux Var Replace) Where Result).
subst_obj_cases Var (obj_term Replace) (stlc.aq Var) Replace.
\end{verbatim}

\noindent
My definition here is quite subtle, so let me walk you through it.
First, we extend the \texttt{subst\_obj} predicate to work on any type
-- that's what \texttt{subst\_obj\_aux} is for. We set up the structural
recursion, by attempting to see whether the ``essential'' cases actually
apply -- those are captured in the \texttt{subst\_obj\_cases} predicate.
If they don't, that means we should proceed by structural recursion. I
did not mention it before, but the \texttt{@} notation that we used to
treat a polymorphic constant as a term of type \texttt{forall\ A\ T},
can be used with an arbitrary term as well, to assign it such a type if
possible. Finally, the essential case itself is a direct transcription
of the pen-and-paper version.

\heroSTUDENT{} Let me go and re-read that a little. (\ldots{}) I think it
makes sense now. Well, is that all? Are we done?

\begin{verbatim}
eval (letrec (bind (fun power \ensuremath{\Rightarrow} body ([ ..long term.. ],
        app power (osucc (osucc ozero)))))) V ?
>> Yes!!!
>> V := < obj_term (\foreignlanguage{greek}{λ}x.x * ((\foreignlanguage{greek}{λ}a.a * (\foreignlanguage{greek}{λ}b.1) a) x)) >.
\end{verbatim}

\heroADVISOR{} See, even the Makam REPL is
excited\footnote{We have taken the liberty here to transcribe the result to more meaningful syntax to make it easier to verify.}!
That looks correct, even though there are a lot of administrative
redeces. We should be able to fix that with the next kind of object in
our check-list though: open STLC terms! That way, instead of having
\texttt{power} return an object containing a lambda function, it can
return an open term. Here's how I would write the same example from
before:

\begin{verbatim}
let power_aux (n: onat): < [ stlc.onat ] stlc.onat > =
  match n with
    0 \ensuremath{\Rightarrow} < [x]. 1 >
  | S n' \ensuremath{\Rightarrow} letobj I = power_aux n' in
      < [x]. stlc.mult ~(I/[x]) x >
\end{verbatim}

\noindent
We have to list out explicitly the variables that an open term depends
on, so that's the \texttt{{[}x{]}.} notation I use. Then, we can use
contextual types \citep{nanevski2008contextual} for the type of those
open terms.

\heroSTUDENT{} Good thing I've already printed the paper out. (\ldots{}) OK,
so it says here that we can use contextual types to record, at the type
level, the context that open terms depend on. So let's say, an open
\texttt{stlc.term} of type \(t\) that mentions variables of a \(\Phi\)
context would have a contextual type of the form \([\Phi] t\). This is
some sort of modal typing, with a precise context.

\heroADVISOR{} Right. We now get to the tricky part: referring to variables
that stand for open terms within other terms! You know what those are,
right? Those are Object-level Object-level Meta-variables.

\heroSTUDENT{} My head hurts; I'm getting
\href{https://en.wikipedia.org/wiki/Out_of_memory}{OOM} errors. Maybe
this is easier to implement in Makam than to talk about.

\heroADVISOR{} Maybe so. Well, let me just say this: those variables will
stand for open terms that depend on a specific context \(\Phi\), but we
might use them at a different context \(\Phi'\). We need a
\emph{substitution} \(\sigma\) to go from the context they were defined
into the current context. I think writing down the rules on paper will
help:

\begin{mathpar}
\begin{array}{l}
\rulename{Ob-Ob-Syntax} \\
\dep{o} ::= \text{...} \; | \; [x_1, \text{...}, x_n]. \stlce \\
\dep{c} ::= \text{...} \; | \; [\stlct_1, \text{...}, \stlct_n] \stlct \\
\stlce ::= \text{...} | \; \aqopen{i}/\sigma \\
\sigma ::= [\stlce_1, \text{...}, \stlce_n]
\end{array}

\inferrule[Classof-OpenTerm]
          {\Psi; x_1 : \stlct_1, \text{...}, x_n : \stlct_n \vdash \stlce : \stlct}
          {\Psi \odash [x_1, \text{...}, x_n]. \stlce : [\stlct_1, \text{...}, \stlct_n] \stlct}

\inferrule[STLC-TypeOf-AntiquoteOpen]
          {\dep{i} : [\stlct_1, \text{...}, \stlct_n] \stlct \in \Psi \\
           \forall i.\Psi \odash \stlce_i : \stlct_i}
          {\Psi; \Delta \vdash \aqopen{\dep{i}}/[\stlce_1, \text{...}, \stlce_n] : \stlct}
          
\inferrule[SubstObj]{}{
  (\aqopen{\dep{i}}/\sigma)[[x_1, \text{...}, x_n]. \stlce / i] = \stlce[\stlce_1/x_1, \text{...}, \stlce_n/x_n] \text{ if } \sigma[[x_1, \text{...}, x_n]. \stlce / i] = [\stlce_1, \text{...}, \stlce_n]
}
\end{mathpar}

\heroSTUDENT{} I've seen that rule for \rulename{SubstObj} before, and it is
still tricky\ldots{} We need to replace the open variables in \(e\)
through the substitution
\(\sigma = [\stlce^*_1, \text{...}, \stlce^*_n]\). However, the terms
\(\stlce^*_1\) through \(\stlce^*_n\) might mention the \(i\) index
themselves, so we first need to apply the top level substitution to
\(\sigma\) itself! After that, we do replace the open variables in
\(\stlce\).

\heroADVISOR{} I feel that we are getting to the point where it's easier to
write things down in Makam rather than on paper:

\begin{verbatim}
obj_openterm : bindmany stlc.term stlc.term \ensuremath{\to} object.
cls_ctxtyp : list stlc.typ \ensuremath{\to} stlc.typ \ensuremath{\to} class.

%extend stlc.
aqopen : index \ensuremath{\to} list term \ensuremath{\to} term.
%end.

stlc.typeof (stlc.aqopen I ES) T :-
  classof_index I (cls_ctxtyp TS T),
  map stlc.typeof ES TS.

classof (obj_openterm XS_E) (cls_ctxtyp TS T) :-
  openmany XS_E (fun xs e \ensuremath{\Rightarrow}
    assumemany stlc.typeof xs TS (stlc.typeof e T)).

subst_obj_cases Var (obj_openterm Replace) (stlc.aqopen Var Subst) Result :-
  applymany Replace Subst Intermediate,
  subst_obj_aux Var (obj_openterm Replace) Intermediate Result.
\end{verbatim}

\heroSTUDENT{} I think that's all! This is exciting -- let me try it out:

\begin{verbatim}
(eq _TERM (letrec (bind (fun power \ensuremath{\Rightarrow} body ([
    lam onat (fun n \ensuremath{\Rightarrow}
    case_or_else n
      (patt_ozero) (* |\ensuremath{\to} *)
        (vbody (liftobj (obj_openterm (bind (fun x \ensuremath{\Rightarrow}
          body (stlc.osucc stlc.ozero))))))
    (case_or_else n
      (patt_osucc patt_var) (* |\ensuremath{\to} *) (vbind (fun n' \ensuremath{\Rightarrow} vbody (
         letobj (app power n') (fun i \ensuremath{\Rightarrow}
         liftobj (obj_openterm (bind (fun x \ensuremath{\Rightarrow}
           body (stlc.mult x (stlc.aqopen i [x])))))))))
      (liftobj (obj_openterm (bind (fun x \ensuremath{\Rightarrow} body stlc.ozero))))
    ))], app power (osucc (osucc ozero)))))),
  typeof _TERM T, eval _TERM V) ?
>> Yes:
>> T := liftclass (cls_ctxtyp (cons stlc.onat nil) stlc.onat),
>> V := liftobj (obj_openterm (bind (fun x \ensuremath{\Rightarrow} body (
          stlc.mult x (stlc.mult x (stlc.osucc stlc.ozero)))))).
\end{verbatim}

\noindent
It works! That's it! I cannot believe how easy this was!

\heroAUDIENCE{} We cannot possibly believe that you are claiming this was
easy!

\heroAUTHOR{} Still, try implementing something like this without a
metalanguage\ldots{} It takes a long time! As a result, it limits our
ability to experiment with and iterate on new language design ideas.
That's why I started working on Makam. That took a few years, but now we
can at least show a type system like this in 27 pages of a single-column
PDF!

\heroADVISOR{} I wonder where all these voices are coming from?

\heroSTUDENT{} Somehow, they sound like the ghosts of people who left academia
for industry.

\heroTODO{}
\textit{(Joke will be elided to avoid issues with double-blind submission.)}

  
  \section{Where our hero Roza implements type generalization, tying loose
ends}\label{where-our-hero-roza-implements-type-generalization-tying-loose-ends}

\begin{verse}
``We mentioned Hindley-Milner / we don't want you to be sad. \\
This paper's going to end soon / and it wasn't all that bad. \\
\hspace{1em}\vspace{-0.5em} \\
(Before we get to that though / it's time to get a break. \\
If taksims seem monotonous / then put on some Nick Drake.) \\
\hspace{1em}\vspace{-0.5em} \\
We'll gather unif-variables / with structural recursion \\
and if you haven't guessed it yet / we'll get to use reflection.''
\end{verse}

\heroADVISOR{} Let me now show you how to implement type generalization for
polymorphic \texttt{let} in the style of
\citet{damas1984type,hindley1969principal,milner1978theory}. I've done
this before, and I need to leave for home soon, so bear with me for a
bit. The gist of this will be to reuse the unification support of our
metalanguage, capturing the \emph{metalevel unification variables} and
generalizing over them. That way we will have a very short
implementation and we won't have to do a deep embedding of unification!

\heroSTUDENT{} So -- you're saying that in \lamprolog, other than reusing the
metalevel function type for implementing object level substitution, we
can also reuse metalevel unification for the object level as well.

\identNormal

\heroADVISOR{} Exactly! First of all, the typing rule for a generalizing let
looks like this:

\vspace{-1.2em}\begin{mathpar}
\inferrule{\Gamma \vdash e : \tau \\ \vec{a} = \text{fv}(\tau) - \text{fv}(\Gamma) \\ \Gamma, x : \forall \vec{a}.\tau \vdash e' : \tau'}{\Gamma \vdash \text{let} \; x = e \; \text{in} \; e' : \tau'}
\end{mathpar}

We don't have any side-effectful operations, so, there is no need for a
value restriction. Transcribing this to Makam is easy, if we assume a
predicate for generalizing the type, for now:

\begin{verbatim}
generalize : (Type: typ) (GeneralizedType: typ) \ensuremath{\to} prop.
let : term \ensuremath{\to} (term \ensuremath{\to} term) \ensuremath{\to} term.
typeof (let E F) T' :-
  typeof E T, generalize T Tgen, (x:term \ensuremath{\to} typeof x Tgen \ensuremath{\to} typeof (F x) T').
\end{verbatim}

Now, for generalization itself, we need the following ingredients based
on the typing rule:

\begin{itemize}
\tightlist
\item
  something that picks out free variables from a type, standing for the
  \(\text{fv}(\tau)\) part -- or, in our setting, this should really be
  read as uninstantiated unification variables. Those are the
  Makam-level unification variables that have not been forced to unify
  with a concrete type because of the rest of the typing rules.
\item
  something that picks out free variables from the local context: the
  \(\text{fv}(\Gamma)\) part. Again, these are the uninstantiated
  unification variables rather than the free variables. In our case, the
  context \(\Gamma\) is represented by the local \texttt{typeof}
  assumptions that our typing rules add, so we'll have to look at those
  somehow.
\item
  a way to turn something that includes unification variables into a
  \(\forall\) type, corresponding to the \(\forall \vec{a}.\tau\) part.
  This essentially abstracts over a number of variables and uses them as
  the replacement for the ones inside \(\tau\).
\end{itemize}

All of those look like things that we should be able to do with our
generic recursion and with the reflective predicates we've been using!
However, to make the implementation simpler, we will generalize over one
variable at a time, instead of all at once -- but that should be
entirely equivalent to what's described in the typing rule.

First, we will define a \texttt{findunif} predicate that returns
\emph{one} unification variable \emph{of the right type} from a term, if
at least one such variable exists. To implement this, we will make use
of a generic operation in the Makam standard library, called
\texttt{generic\_fold}. It is quite similar to
\texttt{structural\_recursion}, but it does a fold through a term,
carrying an accumulator through. Pretty standard, really, and its code
is similar to what we did already for \texttt{structural\_recursion},
with no new surprises.

\begin{verbatim}
findunif_aux : [Any VarType]
  (Var: option VarType) (Current: Any) (Var': option VarType) \ensuremath{\to} prop.
findunif_aux (some Var) _ (some Var).
findunif_aux none (Current : VarType) (some (Current : VarType)) :-
  refl.isunif Current.
findunif_aux In Current Out :- generic_fold @findunif_aux In Current Out.

findunif : [Any VarType] (Search: Any) (Found: VarType) \ensuremath{\to} prop.
findunif Search Found :- findunif_aux none Search (some Found).
\end{verbatim}

Here, the second rule of \texttt{findunif\_aux} is the important one --
it will only match when we encounter a unification variable of the same
type as the one we require. So this rule uses the dynamic
\texttt{typecase} aspect of the ad-hoc polymorphism in \lamprolog. With
this, we should be already able to find \emph{one} (as opposed to all,
as described above) uninstantiated unification variable from a type.
Here is an example of its use:

Now we add a predicate \texttt{replaceunif} that, given a specific
unification variable and a specific term, replaces its occurrences with
the term. This will be needed as part of the \(\forall \vec{a}.\tau\)
operation of the rule. Here I'll need another reflective predicate,
\texttt{refl.sameunif}, that succeeds when its two arguments are the
same exact unification variable; \texttt{eq} would just unify them,
which is not what we want.

\begin{verbatim}
replaceunif : [VarType Any]
  (Which: VarType) (ToWhat: VarType) (Where: Any) (Result: Any) \ensuremath{\to} prop.
replaceunif Which ToWhat Where ToWhat :-
  refl.isunif Where, refl.sameunif Which Where.
replaceunif Which ToWhat Where Where :-
  refl.isunif Where, not(refl.sameunif Which Where).
replaceunif Which ToWhat Where Result :- not(refl.isunif Where),
  structural_recursion @(replaceunif Which ToWhat) Where Result.
\end{verbatim}

Last, we'll need an auxiliary predicate that tells us whether a
unification variable exists within a term. This is easy; it's similar to
the above.

\begin{verbatim}
hasunif : [VarType Any] VarType \ensuremath{\to} bool \ensuremath{\to} Any \ensuremath{\to} bool \ensuremath{\to} prop.
hasunif _ true _ true.
hasunif X false Y true :- refl.sameunif X Y.
hasunif X In Y Out :- generic_fold @(hasunif X) In Y Out.

hasunif : [VarType Any] VarType \ensuremath{\to} Any \ensuremath{\to} prop.
hasunif Var Term :- hasunif Var false Term true.
\end{verbatim}

We are now mostly ready to implement \texttt{generalize}. We'll do this
recursively. The base case is when there are no unification variables
within a type left:

\begin{verbatim}
generalize T T :- not(findunif T X).
\end{verbatim}

For the recursive case, we will pick out the first unification variable
that we come upon using \texttt{findunif}. We will generalize over it
using \texttt{replaceunif} and then proceed to the rest. Still, there is
a last hurdle: we have to skip over the unification variables that are
in the \(\Gamma\) environment. For the time being, let's assume a
predicate that gives us all the types in the environment, and write the
recursive case down:

\begin{verbatim}
get_types_in_environment : [A] A \ensuremath{\to} prop.
generalize T Res :-
  findunif T Var, get_types_in_environment GammaTypes,
  (x:typ \ensuremath{\to} (replaceunif Var x T (T' x), generalize (T' x) (T'' x))),
  if (hasunif Var GammaTypes)
  then (eq Res (T'' Var))
  else (eq Res (tforall T'')).
\end{verbatim}

\identDialog

\heroSTUDENT{} Oh, clever. But what should
\texttt{get\_types\_in\_environment} be? Don't we have to go back and
thread a list of types through our \texttt{typeof} predicate, every time
we introduce a new \texttt{typeof\ x\ T\ \ensuremath{\to}} assumption?

\heroADVISOR{} Well, we came this far without significantly rewriting our
rules, so it's a shame to do that now! Maybe we'll be excused to use yet
another reflective predicate that does what we want? There is a way to
get a list of all the local assumptions for the \texttt{typeof}
predicate; it turns out that all the rules and connectives are normal
\lamprolog terms like any other, so there's not really much magic to it.
And those assumptions will include just the types in \(\Gamma\)\ldots{}.

\begin{verbatim}
get_types_in_environment Assumptions :-
  refl.assume_get (typeof : term \ensuremath{\to} typ \ensuremath{\to} prop) Assumptions.
\end{verbatim}

\heroSTUDENT{} Wait. It can't be.

\begin{verbatim}
typeof (let (lam _ (fun x \ensuremath{\Rightarrow} let x (fun y \ensuremath{\Rightarrow} y))) (fun id \ensuremath{\Rightarrow} id)) T ?
>> Yes:
>> T := tforall (fun a \ensuremath{\Rightarrow} arrow a a).
\end{verbatim}

\heroADVISOR{} And yet, it can.


  \input{generated/10-summary}

  \identNormal{}
}

\bibliography{main}

\end{document}
