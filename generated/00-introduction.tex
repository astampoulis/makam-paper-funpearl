\section{Where our heroes set on a road to prototype a type
system}\label{where-our-heroes-set-on-a-road-to-prototype-a-type-system}

\hero{HAGOP} (\ldots{}) So yes, I think my next step should be writing a toy
implementation of the type system we have in mind, so that we can try
out some examples and see what works and what does not.

\hero{ROZA} Definitely -- trying out examples will help us refine our ideas
too.

\hero{HAGOP} Let's see though, we have an ML core, dependently typed
constructs, and contextual types like in
\citet{nanevski2008contextual}\ldots{} I guess I will need a few months?

\hero{ROZA} That sounds like a lot. Why don't you use some kind of
metalanguage to implement the prototype?

\hero{HAGOP} You mean a tool like PLT Redex \citep{felleisen2009semantics},
the K Framework \citep{rosu2010overview,ellison2009rewriting}, Spoofax
\citep{kats2010spoofax}, or CRSX \citep{rose2011crsx}?

\hero{ROZA} Sure, all fine choices. Though I do not think these frameworks
have been used to implement a type system as advanced as the one we have
in mind, or can handle all the binding constructs we will need\ldots{} I
was actually thinking we should use higher-order logic programming.

\hero{HAGOP} Oh, so \lamprolog \citep{miller1988overview} or LF
\citep{pfenning1999system}.

\hero{ROZA} Yes. Actually a few years back I worked on this new implementation
of \lamprolog
just for this purpose -- rapid prototyping of languages. It's called
Makam. It should be able to handle what we have in mind, and we won't
need more than a few hours.

\hero{HAGOP} Sounds great! Anything I can read up on it?

\hero{ROZA} Not much unfortunately\ldots{} Let's just work on this together,
it'll be fun.

(Our heroes install Makam from
\if@ACM@anonymous\{\url{http://github.com/astampoulis/makam}\}\else{\em--elided for blind
reviewing--}\fi\xspace
and figure out how to run the REPL.)
