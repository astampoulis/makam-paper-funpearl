\section{Where our hero Roza implements type generalization, tying loose
ends}\label{where-our-hero-roza-implements-type-generalization-tying-loose-ends}

\begin{verse}
``We mentioned Hindley-Milner / we don't want you to be sad. \\
This paper's going to end soon / and it wasn't all that bad. \\
\hspace{1em}\vspace{-0.5em} \\
(Before we get to that though / it's time to get a break. \\
If taksims seem monotonous / then put on some Nick Drake.) \\
\hspace{1em}\vspace{-0.5em} \\
We'll gather unif-variables / with structural recursion \\
and if you haven't guessed it yet / we'll get to use reflection.''
\end{verse}

\heroSTUDENT{} I have an idea for implementing type generalization for
polymorphic \texttt{let} in the style of
\citet{damas1984type,hindley1969principal,milner1978theory}. I remember
the typing rule looks like this:

\vspace{-1.2em}\begin{mathpar}
\inferrule{\Gamma \vdash e : \tau \\ \vec{a} = \text{fv}(\tau) - \text{fv}(\Gamma) \\ \Gamma, x : \forall \vec{a}.\tau \vdash e' : \tau'}{\Gamma \vdash \text{let} \; x = e \; \text{in} \; e' : \tau'}
\end{mathpar}

\heroADVISOR{} Right, and we don't have any side-effectful operations, so, no
need for a value restriction. Let's assume a predicate for generalizing
the type, for now; the rest of the rule is easy:

\begin{verbatim}
generalize : typ \ensuremath{\to} typ \ensuremath{\to} prop.
let : term \ensuremath{\to} (term \ensuremath{\to} term) \ensuremath{\to} term.
typeof (let E F) T' :-
  typeof E T, generalize T Tgen, (x:term \ensuremath{\to} typeof x Tgen \ensuremath{\to} typeof (F x) T').
\end{verbatim}

\heroSTUDENT{} Right, so for generalization, based on the typing rule, we need
the following ingredients:

\begin{itemize}
\tightlist
\item
  something that picks out free variables from a type -- or, in our
  setting, uninstantiated unification variables
\item
  something that picks out free variables from the local context
\item
  a way to turn something that includes unification variables into a
  \texttt{forall} type
\end{itemize}

\noindent
Those look like things that we should be able to do with our generic
recursion and with the reflective predicates we've been using!

\heroADVISOR{} Indeed! So, I've done this before, and I need to leave for home
soon, so bear with me for a bit. There's this generic operation in the
Makam standard library, called \texttt{generic.fold}. It is quite
similar to \texttt{structural\_recursion}, but it does a fold through a
term, carrying an accumulator through. Pretty standard, really, and its
code is similar to what we did already. I'll use it to define a
predicate that returns \emph{one} unification variable of the right type
from a term, if at least one exists.

\begin{verbatim}
findunif : [A B] option B \ensuremath{\to} A \ensuremath{\to} option B \ensuremath{\to} prop.
findunif (some X) _ (some X).
findunif none (X : B) (some (X : B)) :- refl.isunif X.
findunif In X Out :- generic.fold findunif In X Out.
findunif : [A B] A \ensuremath{\to} B \ensuremath{\to} prop.  findunif T X :- findunif none T (some X).
\end{verbatim}

\heroSTUDENT{} Oh, the second rule is the important one -- it will only match
when we encounter a unification variable of the same type as the one we
require, thanks to type specialization.

\heroADVISOR{} Exactly. Now we add a predicate that, given a specific
unification variable and a specific term, replaces its occurrences with
the term. I'll show you later why this operation is necessary. Here I'll
need another reflective predicate, \texttt{refl.sameunif}, that succeeds
when its two arguments are the same exact unification variable;
\texttt{eq} would just unify them, which is not what we want.

\begin{verbatim}
replaceunif : [A B] A \ensuremath{\to} A \ensuremath{\to} B \ensuremath{\to} B \ensuremath{\to} prop.
replaceunif Which ToWhat Where Result :- refl.isunif Where,
  if (refl.sameunif Which Where) then (eq (dyn Result) (dyn ToWhat))
  else (eq Result Where).
replaceunif Which ToWhat Where Result :- not(refl.isunif Where),
  structural_recursion @(replaceunif Which ToWhat) Where Result.
\end{verbatim}

\heroADVISOR{} And last, we'll need an auxiliary predicate that tells us
whether a unification variable exists within a term. You can do that
yourself; it's similar to the above.

\heroSTUDENT{} Yes, I think I know how to do that.

\begin{verbatim}
hasunif : [A B] B \ensuremath{\to} bool \ensuremath{\to} A \ensuremath{\to} bool \ensuremath{\to} prop.
hasunif _ true _ true.
hasunif X false Y true :- refl.sameunif X Y.
hasunif X In Y Out :- generic.fold (hasunif X) In Y Out.
hasunif : [A B] A \ensuremath{\to} B \ensuremath{\to} prop. hasunif Term Var :- hasunif Var false Term true.
\end{verbatim}

\heroADVISOR{} OK, we are now mostly ready to implement \texttt{generalize}.
We'll do this recursively. The base case is when there are no
unification variables within a type left:

\begin{verbatim}
generalize T T :- not(findunif T X).
\end{verbatim}

\heroSTUDENT{} Ah, I see what you are getting at. For the recursive case, we
will pick out the first unification variable that we come upon using
\texttt{findunif}. We will generalize over it using \texttt{replaceunif}
and then proceed to the rest. But don't we have to skip over the
unification variables that are in the \(\Gamma\) environment?

\heroADVISOR{} Well, that's the last hurdle. Let's assume a predicate that
gives us all the types in the environment, and write the recursive case
down:

\begin{verbatim}
get_types_in_environment : [A] A \ensuremath{\to} prop.
generalize T Res :- 
  findunif T Var, get_types_in_environment GammaTypes,
  (x:typ \ensuremath{\to} (replaceunif Var x T (T' x), generalize (T' x) (T'' x))),
  if (hasunif GammaTypes Var) then (eq Res (T'' Var)) else (eq Res (tforall T'')).
\end{verbatim}

\heroSTUDENT{} Oh, clever. But what should
\texttt{get\_types\_in\_environment} be? Don't we have to go back and
thread a list of types through our \texttt{typeof} predicate, every time
we introduce a new \texttt{typeof\ x\ T\ \ensuremath{\to}} assumption?

\heroADVISOR{} Well, we came this far without rewriting our rules, so it's a
shame to do that now! Maybe we'll be excused to use yet another
reflective predicate that does what we want? There is a way to get a
list of all the local assumptions for the \texttt{typeof} predicate; it
turns out that all the rules and connectives are normal \foreignlanguage{greek}{λ}Prolog terms
like any other, so there's not really much magic to it. And those
assumptions will include just the types in \(\Gamma\)\ldots{}.

\begin{verbatim}
get_types_in_environment Assumptions :-
  refl.assume_get (typeof : term \ensuremath{\to} typ \ensuremath{\to} prop) Assumptions.
\end{verbatim}

\heroSTUDENT{} Wait. It can't be.

\begin{verbatim}
typeof (let (lam _ (fun x \ensuremath{\Rightarrow} let x (fun y \ensuremath{\Rightarrow} y))) (fun id \ensuremath{\Rightarrow} id)) T ?
>> Yes:
>> T := tforall (fun a \ensuremath{\Rightarrow} arrow a a).
\end{verbatim}

\heroADVISOR{} And yet, it can.
